\documentclass[12pt]{article}
\usepackage[margin=1in]{geometry}
\usepackage{amsmath,amssymb}
\begin{document}

\section*{Problem 8}
Odnos binomnih koeficijenata uz stepen $x^{1007}$, $(x \in (0, +\infty))$ u razvojima binoma $(1+x)^{2014}$ i $(1+x)^{2013}$ redom, iznosi:
\begin{enumerate}
    \item[(A)] $\frac{1007}{1006}$
    \item[(B)] $2$
    \item[(C)] $\frac{3}{2}$
    \item[(D)] $\frac{1}{2014}$
    \item[(E)] $\frac{1}{2015}$
    \item[(N)] Ne znam
\end{enumerate}

\subsection*{Solution}
The term with $x^k$ in $(1+x)^n$ has the coefficient $\binom{n}{k}$.
We need the ratio of the coefficient of $x^{1007}$ in $(1+x)^{2014}$ to the coefficient of $x^{1007}$ in $(1+x)^{2013}$.
Ratio = $\frac{\binom{2014}{1007}}{\binom{2013}{1007}}$.
Using the property $\binom{n}{k} = \frac{n}{n-k}\binom{n-1}{k}$:
\[
\binom{2014}{1007} = \frac{2014}{2014-1007} \binom{2013}{1007} = \frac{2014}{1007} \binom{2013}{1007} = 2 \binom{2013}{1007}
\]
Thus, the ratio is:
\[
\frac{2 \binom{2013}{1007}}{\binom{2013}{1007}} = 2
\]

\subsection*{Answer}
$2$ (option \textbf{(B)}).

\end{document}