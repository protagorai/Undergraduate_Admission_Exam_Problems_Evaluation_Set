\documentclass[12pt]{article}
\usepackage[margin=1in]{geometry}
\usepackage{amsmath,amssymb}
\begin{document}

\section*{Problem 5}
Ako je $x+|x| = \frac{x}{|x|}$, ($x \in \mathbf{R} \setminus \{0\}$), tada $x$ pripada skupu:
\begin{enumerate}
    \item[(A)] $(0, 1)$
    \item[(B)] $(-1, 0)$
    \item[(C)] $(1, 3)$
    \item[(D)] $(2, +\infty)$
    \item[(E)] $(-\infty, 0)$
    \item[(N)] Ne znam
\end{enumerate}

\subsection*{Solution}
We distinguish two cases based on the sign of $x$.
Case 1: $x > 0$.
Then $|x| = x$. The equation becomes:
\[
x + x = \frac{x}{x} \implies 2x = 1 \implies x = \frac{1}{2}
\]
Since $x=1/2 > 0$, this is a valid solution.

Case 2: $x < 0$.
Then $|x| = -x$. The equation becomes:
\[
x - x = \frac{x}{-x} \implies 0 = -1
\]
This is impossible.

Thus, the only solution is $x = 1/2$.
The interval $(0, 1)$ contains $1/2$.

\subsection*{Answer}
$(0, 1)$ (option \textbf{(A)}).

\end{document}