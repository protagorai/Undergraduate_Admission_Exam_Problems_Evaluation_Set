\documentclass[12pt]{article}
\usepackage[margin=1in]{geometry}
\usepackage{amsmath,amssymb}
\begin{document}

\section*{Problem 15}
Dat je polinom $P(x) = a_0 x^4 + a_1 x^3 + a_2 x^2 + a_3 x + a_4$ ($a_0, a_1, a_2, a_3, a_4 \in \mathbf{R}, a_0 \neq 0$), takav da je $P(0) = P(1) = P(2) = P(-1) = 0$ i $P(-2) = 12$. Tada je $P(3)$ jednako:
\begin{enumerate}
    \item[(A)] $\frac{1}{3}$
    \item[(B)] $-\frac{1}{2}$
    \item[(C)] $1$
    \item[(D)] $2$
    \item[(E)] $12$
    \item[(N)] Ne znam
\end{enumerate}

\subsection*{Solution}
Since $0, 1, 2, -1$ are roots of $P(x)$, we can write:
\[
P(x) = a_0 x(x-1)(x-2)(x+1) = a_0 (x^2-1)(x^2-2x)
\]
Use $P(-2) = 12$ to find $a_0$:
\[
P(-2) = a_0 (-2)((-2)-1)((-2)-2)((-2)+1)
\]
\[
P(-2) = a_0 (-2)(-3)(-4)(-1) = a_0 (24)
\]
$24a_0 = 12 \implies a_0 = \frac{1}{2}$.
Now find $P(3)$:
\[
P(3) = \frac{1}{2} (3)(3-1)(3-2)(3+1)
\]
\[
P(3) = \frac{1}{2} (3)(2)(1)(4) = \frac{24}{2} = 12
\]

\subsection*{Answer}
$12$ (option \textbf{(E)}).

\end{document}