\documentclass[12pt]{article}
\usepackage[margin=1in]{geometry}
\usepackage{amsmath,amssymb}
\begin{document}

\section*{Problem 11}
Proizvod svih realnih rešenja jednačine $\frac{2013x}{2014} = 2013^{\log_x 2014}$ pripada skupu:
\begin{enumerate}
    \item[(A)] $(0, 1]$
    \item[(B)] $(1, 2]$
    \item[(C)] $(2, 3]$
    \item[(D)] $(3, 4]$
    \item[(E)] $(4, +\infty)$
    \item[(N)] Ne znam
\end{enumerate}

\subsection*{Solution}
Let $a = 2013$ and $b = 2014$. The equation is:
\[
\frac{ax}{b} = a^{\log_x b}
\]
Take $\log_x$ of both sides (assuming $x>0, x \neq 1$):
\[
\log_x \left(\frac{ax}{b}\right) = \log_x (a^{\log_x b})
\]
\[
\log_x a + \log_x x - \log_x b = (\log_x b)(\log_x a)
\]
\[
\log_x a + 1 - \log_x b = \log_x a \cdot \log_x b
\]
Let $u = \log_x a$ and $v = \log_x b$.
\[
u + 1 - v = uv \implies uv - u + v - 1 = 0
\]
\[
u(v-1) + (v-1) = 0 \implies (u+1)(v-1) = 0
\]
So either $u = -1$ or $v = 1$.
Case 1: $u = -1 \implies \log_x a = -1 \implies x^{-1} = a \implies x = \frac{1}{a} = \frac{1}{2013}$.
Case 2: $v = 1 \implies \log_x b = 1 \implies x^1 = b \implies x = b = 2014$.
The solutions are $x_1 = \frac{1}{2013}$ and $x_2 = 2014$.
The product of solutions is:
\[
P = x_1 \cdot x_2 = \frac{2014}{2013} = 1 + \frac{1}{2013}
\]
Since $1 < 1 + \frac{1}{2013} \le 2$, the product belongs to $(1, 2]$.

\subsection*{Answer}
$(1, 2]$ (option \textbf{(B)}).

\end{document}