\documentclass[12pt]{article}
\usepackage[margin=1in]{geometry}
\usepackage{amsmath,amssymb}
\begin{document}

\section*{Problem 20}
Iz skupa od $10$ studenata, među kojima su samo jedan student elektrotehnike i samo jedan student matematike,
biramo komisiju od $6$ članova tako da ako je u komisiji student elektrotehnike mora u toj komisiji biti i student matematike.
Koliko se takvih komisija može obrazovati?
\[
\text{(A) }210\quad
\text{(B) }98\quad
\text{(C) }126\quad
\text{(D) }154\quad
\text{(E) }165.
\]

\subsection*{Solution}
Ukupan broj komisija od $6$ članova iz $10$ studenata je
\[
\binom{10}{6}=210.
\]
Neispravne komisije su one koje sadrže studenta elektrotehnike, ali ne sadrže studenta matematike.
Tada je student elektrotehnike fiksiran u komisiji, student matematike izbačen, a preostalih $5$ članova biramo od preostalih $8$ studenata:
\[
\binom{8}{5}=56.
\]
Zato je broj ispravnih komisija
\[
210-56=154.
\]

\subsection*{Answer}
$154$ (option \textbf{D}).

\end{document}

