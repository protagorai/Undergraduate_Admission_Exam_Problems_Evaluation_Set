\documentclass[12pt]{article}
\usepackage[margin=1in]{geometry}
\usepackage{amsmath,amssymb}
\begin{document}

\section*{Problem 19}
Od lista hartije kružnog oblika izrezan je kružni isečak od koga je napravljen konusni levak najveće zapremine.
Centralni ugao tog kružnog isečka u radijanima je:
\[
\text{(A) }\frac{\pi}{3}\quad
\text{(B) }\frac{2\pi\sqrt6}{3}\quad
\text{(C) }\frac{2\pi}{3}\quad
\text{(D) }\frac{2\pi}{\sqrt3}\quad
\text{(E) }\frac{\pi\sqrt6}{2}.
\]

\subsection*{Solution}
Neka je poluprečnik početnog kruga $L$; to je izvodnica kupe (kosina) dobijene savijanjem isečka.
Ako je centralni ugao isečka $\theta$, onda je dužina luka $L\theta$, a to postaje obim osnove kupe:
\[
2\pi r=L\theta \ \Longrightarrow\ r=\frac{L\theta}{2\pi}.
\]
Visina kupe je
\[
h=\sqrt{L^2-r^2}=L\sqrt{1-\frac{\theta^2}{4\pi^2}}.
\]
Zapremina je
\[
V=\frac13\pi r^2h
=\frac13\pi\left(\frac{L^2\theta^2}{4\pi^2}\right)\cdot
L\sqrt{1-\frac{\theta^2}{4\pi^2}}
\,=\,\frac{L^3}{12\pi}\,\theta^2\sqrt{1-\frac{\theta^2}{4\pi^2}}.
\]
Pošto je $L$ konstanta, maksimizujemo funkciju u promenljivoj $u=\frac{\theta}{2\pi}\in(0,1]$:
\[
f(u)=u^2\sqrt{1-u^2}.
\]
Maksimizujemo $f(u)^2=u^4(1-u^2)=u^4-u^6$:
\[
\frac{d}{du}(u^4-u^6)=4u^3-6u^5=2u^3(2-3u^2).
\]
Za maksimum (uz $u>0$) dobijamo $u^2=\frac{2}{3}$, tj.\ $u=\sqrt{\frac{2}{3}}=\frac{\sqrt6}{3}$.
Zato je
\[
\theta=2\pi u=2\pi\cdot\frac{\sqrt6}{3}=\frac{2\pi\sqrt6}{3}.
\]

\subsection*{Answer}
$\dfrac{2\pi\sqrt6}{3}$ (option \textbf{B}).

\end{document}

