\documentclass[12pt]{article}
\usepackage[margin=1in]{geometry}
\usepackage{amsmath,amssymb}
\begin{document}

\section*{Problem 11}
Proizvod svih realnih rešenja jednačine
\[
\frac{2013x}{2014}=2013^{\log_x 2014}
\]
pripada skupu:
\[
\text{(A) }(0,1)\quad
\text{(B) }(1,2]\quad
\text{(C) }(2,3)\quad
\text{(D) }(3,4]\quad
\text{(E) }(4,+\infty).
\]

\subsection*{Solution}
Zbog $\log_x 2014$ važi $x>0$ i $x\neq 1$.
Uvedimo smenu $x=2014^{a}$ $(a\neq 0)$. Tada je
\[
\log_x 2014=\log_{2014^{a}}2014=\frac{1}{a}.
\]
Jednačina postaje
\[
\frac{2013}{2014}\,2014^{a}=2013^{1/a}
\quad\Longleftrightarrow\quad
2013\cdot 2014^{a-1}=2013^{1/a}.
\]
Ako je $a=1$, dobijamo rešenje $x=2014$.

Ako je $a\neq 1$, logaritmovanjem sledi
\[
\ln 2013+(a-1)\ln 2014=\frac{1}{a}\ln 2013
\ \Longleftrightarrow\
(a-1)\ln 2014=-\frac{a-1}{a}\ln 2013,
\]
pa (deljenjem sa $a-1$) dobijamo
\[
a=-\frac{\ln 2013}{\ln 2014}.
\]
Otuda je drugo rešenje
\[
x=2014^{a}=2014^{-\ln 2013/\ln 2014}=e^{-\ln 2013}=\frac{1}{2013}.
\]
Proizvod rešenja je
\[
2014\cdot \frac{1}{2013}=\frac{2014}{2013}\in(1,2].
\]

\subsection*{Answer}
$(1,2]$ (option \textbf{B}), jer je proizvod $\dfrac{2014}{2013}$.

\end{document}

