\documentclass[12pt]{article}
\usepackage[margin=1in]{geometry}
\usepackage{amsmath,amssymb}
\begin{document}

\section*{Problem 18}
Neka je $S$ skup svih realnih rešenja nejednačine
\[
\tan x\,(1-\tan^2 x)\,(1-3\tan^2 x)\,\bigl(1+\tan^2 x\cdot \tan 3x\bigr)>0
\]
i neka je $S_1\subset S$. Tada skup $S_1$ može biti:
\[
\text{(A) }\left(-\frac{\pi}{2},\frac{\pi}{2}\right)\quad
\text{(B) }\left(\frac{\pi}{3},\frac{\pi}{2}\right)\quad
\text{(C) }\left(\frac{3\pi}{4},\pi\right)\quad
\text{(D) }\left(\frac{7\pi}{6},\frac{3\pi}{2}\right)\quad
\text{(E) }\left(\frac{\pi}{2},\frac{5\pi}{6}\right).
\]

\subsection*{Solution}
Radimo u promenljivoj $t=\tan x$ (uz uslov da su $\tan x$ i $\tan 3x$ definisani).
Koristimo formulu
\[
\tan 3x=\frac{3t-t^3}{1-3t^2}\qquad (1-3t^2\neq 0).
\]
Tada
\begin{align*}
(1-3t^2)\bigl(1+t^2\tan 3x\bigr)
&=(1-3t^2)+t^2(3t-t^3)\\
&=1-3t^2+3t^3-t^5.
\end{align*}
Zato se proizvod u nejednačini svodi na
\[
t(1-t^2)\bigl(1-3t^2+3t^3-t^5\bigr)>0.
\]
Može se proveriti da je $t=1$ nultočka polinoma $1-3t^2+3t^3-t^5$, pa se posle faktorisanja dobija
\[
t(t-1)^2(t+1)\bigl(t^2+at+1\bigr)\bigl(t^2+bt+1\bigr)>0,
\]
gde su $a=\frac{1+\sqrt{17}}{2}>0$ i $b=\frac{1-\sqrt{17}}{2}<0$.
Kvadrat $ (t-1)^2$ ne utiče na znak, a kvadratni faktor $t^2+bt+1$ je uvek pozitivan (diskriminanta je negativna).

Za $x\in\left(\frac{\pi}{3},\frac{\pi}{2}\right)$ važi $t=\tan x>\tan\frac{\pi}{3}=\sqrt3>0$,
pa su i $t>0$ i $t+1>0$, a $t^2+at+1>0$ (jer je $t>0$).
Tako je ceo proizvod strogo pozitivan na tom intervalu.
Takođe, u tom intervalu nema tačaka gde je $\tan x$ ili $\tan 3x$ nedefinisan (granice su isključene).
Zato je
\[
\left(\frac{\pi}{3},\frac{\pi}{2}\right)\subset S,
\]
pa $S_1$ može biti baš taj interval.

\subsection*{Answer}
$\left(\dfrac{\pi}{3},\dfrac{\pi}{2}\right)$ (option \textbf{B}).

\end{document}

