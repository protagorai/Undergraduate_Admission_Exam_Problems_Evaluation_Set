\documentclass[12pt]{article}
\usepackage[margin=1in]{geometry}
\usepackage{amsmath,amssymb}
\begin{document}

\section*{Problem 13}
Sva realna rešenja jednačine
\[
\frac{1}{\sqrt{x}+\sqrt{x-2}}+\frac{1}{\sqrt{x+2}+\sqrt{x}}=\frac{1}{4}
\]
pripadaju skupu:
\[
\text{(A) }[2,6)\quad
\text{(B) }[6,10)\quad
\text{(C) }[10,14)\quad
\text{(D) }[14,18)\quad
\text{(E) }[18,+\infty).
\]

\subsection*{Solution}
Domen je $x\ge 2$.
Racionalisanjem dobijamo:
\[
\frac{1}{\sqrt{x}+\sqrt{x-2}}=\frac{\sqrt{x}-\sqrt{x-2}}{x-(x-2)}=\frac{\sqrt{x}-\sqrt{x-2}}{2},
\]
\[
\frac{1}{\sqrt{x+2}+\sqrt{x}}=\frac{\sqrt{x+2}-\sqrt{x}}{(x+2)-x}=\frac{\sqrt{x+2}-\sqrt{x}}{2}.
\]
Zbir je
\[
\frac{\sqrt{x}-\sqrt{x-2}+\sqrt{x+2}-\sqrt{x}}{2}
=\frac{\sqrt{x+2}-\sqrt{x-2}}{2}.
\]
Jednačina postaje
\[
\frac{\sqrt{x+2}-\sqrt{x-2}}{2}=\frac{1}{4}
\ \Longleftrightarrow\
\sqrt{x+2}-\sqrt{x-2}=\frac{1}{2}.
\]
Neka su $a=\sqrt{x+2}$ i $b=\sqrt{x-2}$. Tada je $a-b=\frac12$ i
\[
a^2-b^2=4=(a-b)(a+b)=\frac12(a+b),
\]
pa $a+b=8$. Rešavanjem sistema dobijamo
\[
a=\frac{8+\frac12}{2}=\frac{17}{4},\qquad b=\frac{8-\frac12}{2}=\frac{15}{4}.
\]
Otuda je
\[
x+2=a^2=\frac{289}{16}\ \Longrightarrow\ x=\frac{289}{16}-2=\frac{257}{16}\approx 16.06.
\]
Dakle rešenje pripada intervalu $[14,18)$.

\subsection*{Answer}
$[14,18)$ (option \textbf{D}), jer je $x=\dfrac{257}{16}$.

\end{document}

