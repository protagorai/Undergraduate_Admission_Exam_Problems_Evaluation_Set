\documentclass[12pt]{article}
\usepackage[margin=1in]{geometry}
\usepackage{amsmath,amssymb}
\begin{document}

\section*{Problem 12}
Krug sadrži tri tačke čije su koordinate $(0,6)$, $(0,10)$ i $(8,0)$. Apscisa druge tačke u kojoj dati krug seče $x$-osu, jednaka je:
\[
\text{(A) }7\quad
\text{(B) }7{,}25\quad
\text{(C) }7{,}5\quad
\text{(D) }7{,}75\quad
\text{(E) }9.
\]

\subsection*{Solution}
Tačke $(0,6)$ i $(0,10)$ imaju istu apscisu, pa je središte njihove duži $(0,8)$, a simetrala te tetive je prava $y=8$.
Dakle, centar kruga je oblika $C(h,8)$.

Poluprečnik je isti do tačaka $(0,6)$ i $(8,0)$, pa:
\[
h^2+(8-6)^2=(8-h)^2+(0-8)^2.
\]
Odavde
\[
h^2+4=(8-h)^2+64=h^2-16h+128
\ \Longrightarrow\ -16h=-124\ \Longrightarrow\ h=\frac{31}{4}.
\]
Dakle $C\!\left(\frac{31}{4},8\right)$ i
\[
r^2=h^2+4=\frac{961}{16}+\frac{64}{16}=\frac{1025}{16}.
\]
Jednačina kruga je
\[
\left(x-\frac{31}{4}\right)^2+(y-8)^2=\frac{1025}{16}.
\]
Za presek sa $x$-osom stavimo $y=0$:
\[
\left(x-\frac{31}{4}\right)^2+64=\frac{1025}{16}
\ \Longrightarrow\
\left(x-\frac{31}{4}\right)^2=\frac{1}{16}.
\]
Zato je $x=\frac{31\pm 1}{4}$, tj.\ $x=8$ ili $x=\frac{30}{4}=7.5$.
Pošto je jedna tačka preseka već $(8,0)$, druga ima apscisu $7.5$.

\subsection*{Answer}
$7{,}5$ (option \textbf{C}).

\end{document}

