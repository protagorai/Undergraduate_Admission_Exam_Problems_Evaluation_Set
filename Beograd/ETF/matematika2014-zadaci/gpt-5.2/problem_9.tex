\documentclass[12pt]{article}
\usepackage[margin=1in]{geometry}
\usepackage{amsmath,amssymb}
\begin{document}

\section*{Problem 9}
Data je kvadratna funkcija $f(x)=x^2+bx+c$ $(b,c\in\mathbb R)$ takva da je $f(f(1))=f(f(2))=0$, pri čemu je $f(1)\neq f(2)$. Vrednost $f(0)$ jednaka je:
\[
\text{(A) }-6\quad
\text{(B) }-\frac{2}{3}\quad
\text{(C) }-\frac{3}{2}\quad
\text{(D) }\frac{1}{4}\quad
\text{(E) }-2.
\]

\subsection*{Solution}
Označimo $u=f(1)$ i $v=f(2)$. Uslovi $f(u)=0$ i $f(v)=0$ i činjenica $u\neq v$ znače da su $u$ i $v$ dva različita nultočka funkcije $f$, pa
\[
f(x)=(x-u)(x-v)=x^2-(u+v)x+uv.
\]
Dakle $b=-(u+v)$ i $c=uv$.

S druge strane,
\[
u=f(1)=1+b+c=1-(u+v)+uv,
\]
\[
v=f(2)=4+2b+c=4-2(u+v)+uv.
\]
Preuređivanjem dobijamo
\[
2u+v=1+uv,\qquad 2u+3v=4+uv.
\]
Oduzimanjem sledi $2v=3$, tj.\ $v=\frac{3}{2}$, a zatim iz prve jednačine
\[
2u+\frac{3}{2}=1+\frac{3}{2}u \ \Rightarrow\ \frac12 u=-\frac12 \ \Rightarrow\ u=-1.
\]
Zato je
\[
f(0)=c=uv=-1\cdot\frac32=-\frac32.
\]

\subsection*{Answer}
$-\dfrac{3}{2}$ (option \textbf{C}).

\end{document}

