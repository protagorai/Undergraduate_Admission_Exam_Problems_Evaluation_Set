\documentclass[12pt]{article}
\usepackage[margin=1in]{geometry}
\usepackage{amsmath,amssymb}
\begin{document}

\section*{Problem 15}
Dat je polinom $P(x) = a_0x^4 + a_1x^3 + a_2x^2 + a_3x + a_4$ ($a_0, a_1, a_2, a_3, a_4 \in \mathbb{R}, a_0 \neq 0$), takav da je $P(0) = P(1) = P(2) = P(-1) = 0$ i $P(-2) = 12$. Tada je $P(3)$ jednako:

(A) $\frac{1}{3}$ \quad (B) $-\frac{1}{2}$ \quad (C) $1$ \quad (D) $2$ \quad (E) $12$ \quad (N) Ne znam

\subsection*{Solution}
Pošto je $P(x)$ polinom četvrtog stepena i ima nule u tačkama $0, 1, 2, -1$, možemo ga zapisati kao:
\[
P(x) = a_0 \cdot x(x-1)(x-2)(x+1) = a_0 \cdot x(x+1)(x-1)(x-2)
\]
\[
= a_0 \cdot x(x^2-1)(x-2) = a_0 \cdot x(x-2)(x^2-1)
\]
\[
= a_0 \cdot x(x-2)(x^2-1) = a_0 \cdot (x^2-2x)(x^2-1)
\]
\[
= a_0 \cdot (x^4 - x^2 - 2x^3 + 2x) = a_0 \cdot (x^4 - 2x^3 - x^2 + 2x)
\]

Dakle:
\[
P(x) = a_0(x^4 - 2x^3 - x^2 + 2x)
\]

Koristeći uslov $P(-2) = 12$:
\[
P(-2) = a_0((-2)^4 - 2(-2)^3 - (-2)^2 + 2(-2))
\]
\[
= a_0(16 - 2(-8) - 4 + 2(-2)) = a_0(16 + 16 - 4 - 4) = a_0 \cdot 24 = 12
\]

Dakle, $a_0 = \frac{12}{24} = \frac{1}{2}$.

Sada možemo naći $P(3)$:
\[
P(3) = \frac{1}{2}(3^4 - 2 \cdot 3^3 - 3^2 + 2 \cdot 3)
\]
\[
= \frac{1}{2}(81 - 2 \cdot 27 - 9 + 6) = \frac{1}{2}(81 - 54 - 9 + 6) = \frac{1}{2} \cdot 24 = 12
\]

\subsection*{Answer}
$12$ (option \textbf{E}).

\end{document}