\documentclass[12pt]{article}
\usepackage[margin=1in]{geometry}
\usepackage{amsmath,amssymb}
\begin{document}

\section*{Problem 20}
Iz skupa od 10 studenata, među kojima su samo jedan student elektrotehnike i samo jedan student matematike, biramo komisiju od 6 članova, ali tako da ako je u komisiji student elektrotehnike mora u toj komisiji biti i student matematike. Koliko se takvih komisija može obrazovati?

(A) 210 \quad (B) 98 \quad (C) 126 \quad (D) 154 \quad (E) 165 \quad (N) Ne znam

\subsection*{Solution}
Imamo 10 studenata: 1 student elektrotehnike (E), 1 student matematike (M), i 8 ostalih studenata.

Trebamo da biramo komisiju od 6 članova sa uslovom: ako je E u komisiji, tada mora biti i M.

Ovo je ekvivalentno sa: E nije u komisiji ILI su i E i M u komisiji.

\textbf{Slučaj 1:} E nije u komisiji
Biramo 6 članova iz preostalih 9 studenata (M + 8 ostalih):
\[
\binom{9}{6} = \binom{9}{3} = \frac{9!}{3! \cdot 6!} = \frac{9 \cdot 8 \cdot 7}{3 \cdot 2 \cdot 1} = \frac{504}{6} = 84
\]

\textbf{Slučaj 2:} I E i M su u komisiji
Biramo preostala 4 člana iz 8 ostalih studenata:
\[
\binom{8}{4} = \frac{8!}{4! \cdot 4!} = \frac{8 \cdot 7 \cdot 6 \cdot 5}{4 \cdot 3 \cdot 2 \cdot 1} = \frac{1680}{24} = 70
\]

Ukupan broj komisija je:
\[
84 + 70 = 154
\]

\subsection*{Answer}
154 (option \textbf{D}).

\end{document}