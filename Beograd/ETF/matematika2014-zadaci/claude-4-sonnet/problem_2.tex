\documentclass[12pt]{article}
\usepackage[margin=1in]{geometry}
\usepackage{amsmath,amssymb}
\begin{document}

\section*{Problem 2}
Pojeftinjenje neke robe najpre za 10\%, a zatim za 20\%, jednako je pojeftinjenju iste robe za:

(A) 30\% \quad (B) 25\% \quad (C) 32\% \quad (D) 28\% \quad (E) 19\% \quad (N) Ne znam

\subsection*{Solution}
Neka je početna cena robe $C$.

Nakon prvog pojeftinjenja za 10\%:
\[
C_1 = C \cdot (1 - 0.10) = 0.9C
\]

Nakon drugog pojeftinjenja za 20\%:
\[
C_2 = C_1 \cdot (1 - 0.20) = 0.9C \cdot 0.8 = 0.72C
\]

Ukupno pojeftinjenje je:
\[
\frac{C - C_2}{C} = \frac{C - 0.72C}{C} = \frac{0.28C}{C} = 0.28 = 28\%
\]

\subsection*{Answer}
28\% (option \textbf{D}).

\end{document}