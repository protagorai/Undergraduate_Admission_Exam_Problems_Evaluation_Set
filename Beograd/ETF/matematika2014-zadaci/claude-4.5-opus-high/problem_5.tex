\documentclass[12pt]{article}
\usepackage[margin=1in]{geometry}
\usepackage{amsmath,amssymb}
\begin{document}

\section*{Problem 5}
Ako je $x + |x| = \frac{x}{|x|}$, $(x \in \mathbf{R} \setminus \{0\})$, tada $x$ pripada skupu:

(A) $(0, 1)$ \quad (B) $(-1, 0)$ \quad (C) $(1, 3)$ \quad (D) $(2, +\infty)$ \quad (E) $(-\infty, 0)$ \quad (N) Ne znam

\subsection*{Solution}
We consider two cases based on the sign of $x$.

\textbf{Case 1:} $x > 0$

Then $|x| = x$, so:
\[
x + x = \frac{x}{x} \implies 2x = 1 \implies x = \frac{1}{2}
\]

Since $\frac{1}{2} > 0$, this is valid. And $\frac{1}{2} \in (0, 1)$.

\textbf{Case 2:} $x < 0$

Then $|x| = -x$, so:
\[
x + (-x) = \frac{x}{-x} \implies 0 = -1
\]

This is a contradiction, so there is no solution for $x < 0$.

Therefore, the only solution is $x = \frac{1}{2}$, which belongs to the interval $(0, 1)$.

\subsection*{Answer}
$(0, 1)$ (option \textbf{A}).

\end{document}
