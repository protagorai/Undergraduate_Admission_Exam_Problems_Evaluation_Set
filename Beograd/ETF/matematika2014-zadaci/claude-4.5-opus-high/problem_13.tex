\documentclass[12pt]{article}
\usepackage[margin=1in]{geometry}
\usepackage{amsmath,amssymb}
\begin{document}

\section*{Problem 13}
Sva realna rešenja iracionalne jednačine $\displaystyle\frac{1}{\sqrt{x}+\sqrt{x-2}} + \frac{1}{\sqrt{x+2}+\sqrt{x}} = \frac{1}{4}$ pripadaju skupu:

(A) $[2, 6)$ \quad (B) $[6, 10)$ \quad (C) $[10, 14)$ \quad (D) $[14, 18)$ \quad (E) $[18, +\infty)$ \quad (N) Ne znam

\subsection*{Solution}
First, rationalize each fraction:

\[
\frac{1}{\sqrt{x}+\sqrt{x-2}} = \frac{\sqrt{x}-\sqrt{x-2}}{(\sqrt{x}+\sqrt{x-2})(\sqrt{x}-\sqrt{x-2})} = \frac{\sqrt{x}-\sqrt{x-2}}{x-(x-2)} = \frac{\sqrt{x}-\sqrt{x-2}}{2}
\]

\[
\frac{1}{\sqrt{x+2}+\sqrt{x}} = \frac{\sqrt{x+2}-\sqrt{x}}{(\sqrt{x+2}+\sqrt{x})(\sqrt{x+2}-\sqrt{x})} = \frac{\sqrt{x+2}-\sqrt{x}}{(x+2)-x} = \frac{\sqrt{x+2}-\sqrt{x}}{2}
\]

Adding these:
\[
\frac{\sqrt{x}-\sqrt{x-2}}{2} + \frac{\sqrt{x+2}-\sqrt{x}}{2} = \frac{\sqrt{x+2}-\sqrt{x-2}}{2} = \frac{1}{4}
\]

Therefore:
\[
\sqrt{x+2}-\sqrt{x-2} = \frac{1}{2}
\]

Let $u = \sqrt{x+2}$ and $v = \sqrt{x-2}$. Then:
\[
u - v = \frac{1}{2}
\]
\[
u^2 - v^2 = (x+2) - (x-2) = 4
\]

Since $u^2 - v^2 = (u-v)(u+v)$:
\[
\frac{1}{2}(u+v) = 4 \implies u + v = 8
\]

From $u - v = \frac{1}{2}$ and $u + v = 8$:
\[
2u = 8.5 \implies u = 4.25 = \frac{17}{4}
\]

So $\sqrt{x+2} = \frac{17}{4} \implies x + 2 = \frac{289}{16} \implies x = \frac{289}{16} - 2 = \frac{289 - 32}{16} = \frac{257}{16} = 16.0625$

Since $16.0625 \in [14, 18)$, the answer is option (D).

\subsection*{Answer}
$[14, 18)$ (option \textbf{D}).

\end{document}
