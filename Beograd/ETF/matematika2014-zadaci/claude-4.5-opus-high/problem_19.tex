\documentclass[12pt]{article}
\usepackage[margin=1in]{geometry}
\usepackage{amsmath,amssymb}
\begin{document}

\section*{Problem 19}
Od lista hartije kružnog oblika izrezan je kružni isečak od koga je napravljen konusni levak najveće zapremine. Centralni ugao tog kružnog isečka u radijanima je:

(A) $\displaystyle\frac{\pi}{3}$ \quad (B) $\displaystyle\frac{2\pi}{3}\sqrt{6}$ \quad (C) $\displaystyle\frac{2\pi}{3}$ \quad (D) $\displaystyle\frac{2\pi}{\sqrt{3}}$ \quad (E) $\displaystyle\frac{\pi\sqrt{6}}{2}$ \quad (N) Ne znam

\subsection*{Solution}
Let the radius of the original circular paper be $R$ and the central angle of the sector be $\theta$.

When the sector is formed into a cone:
\begin{itemize}
    \item The arc length of the sector becomes the circumference of the base: $R\theta = 2\pi r$, where $r$ is the radius of the cone's base.
    \item The radius of the sector becomes the slant height of the cone: $l = R$.
\end{itemize}

From the first condition: $r = \frac{R\theta}{2\pi}$

The height of the cone: $h = \sqrt{l^2 - r^2} = \sqrt{R^2 - r^2} = \sqrt{R^2 - \frac{R^2\theta^2}{4\pi^2}} = R\sqrt{1 - \frac{\theta^2}{4\pi^2}}$

Volume of the cone:
\[
V = \frac{1}{3}\pi r^2 h = \frac{1}{3}\pi \cdot \frac{R^2\theta^2}{4\pi^2} \cdot R\sqrt{1 - \frac{\theta^2}{4\pi^2}}
\]
\[
= \frac{R^3\theta^2}{12\pi}\sqrt{1 - \frac{\theta^2}{4\pi^2}}
\]

Let $u = \frac{\theta^2}{4\pi^2}$, so $\theta^2 = 4\pi^2 u$.

\[
V = \frac{R^3 \cdot 4\pi^2 u}{12\pi}\sqrt{1-u} = \frac{R^3\pi u}{3}\sqrt{1-u}
\]

To maximize, take derivative with respect to $u$:
\[
\frac{dV}{du} = \frac{R^3\pi}{3}\left(\sqrt{1-u} + u \cdot \frac{-1}{2\sqrt{1-u}}\right) = \frac{R^3\pi}{3} \cdot \frac{2(1-u) - u}{2\sqrt{1-u}}
\]
\[
= \frac{R^3\pi}{3} \cdot \frac{2 - 3u}{2\sqrt{1-u}}
\]

Setting $\frac{dV}{du} = 0$: $2 - 3u = 0 \implies u = \frac{2}{3}$

So $\frac{\theta^2}{4\pi^2} = \frac{2}{3} \implies \theta^2 = \frac{8\pi^2}{3} \implies \theta = \frac{2\pi\sqrt{2}}{\sqrt{3}} = \frac{2\pi\sqrt{6}}{3}$

\subsection*{Answer}
$\displaystyle\frac{2\pi}{3}\sqrt{6}$ (option \textbf{B}).

\end{document}
