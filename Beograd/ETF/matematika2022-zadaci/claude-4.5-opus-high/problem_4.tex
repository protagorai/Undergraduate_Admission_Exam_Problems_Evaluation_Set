\documentclass[12pt]{article}
\usepackage[margin=1in]{geometry}
\usepackage{amsmath,amssymb}
\begin{document}

\section*{Problem 4}
Find the set of values of parameter $m \in \mathbb{R}$ for which the equation $(1-m)x^2 + (m-2)x + 1 = 0$ has two distinct real solutions (of the form $-\infty < a < b < \infty$).

(A) $\mathbb{R}\setminus\{a\}$ \quad (B) $(a,b)\cup(b,\infty)$ \quad (C) $[a,\infty)$ \quad (D) $\mathbb{R}\setminus\{a,b\}$ \quad (E) $(a,\infty)$

\subsection*{Solution}
For the equation to have two distinct real solutions, we need to consider two cases.

\textbf{Case 1:} $m \neq 1$ (quadratic equation)

The discriminant must be positive:
\[
D = (m-2)^2 - 4(1-m)(1) > 0
\]
\[
D = m^2 - 4m + 4 - 4 + 4m = m^2 > 0
\]

This gives $m \neq 0$.

\textbf{Case 2:} $m = 1$ (linear equation)

The equation becomes:
\[
(1-1)x^2 + (1-2)x + 1 = 0 \Rightarrow -x + 1 = 0 \Rightarrow x = 1
\]

This gives only one solution, so $m = 1$ is excluded.

\textbf{Verification of $m = 0$:}

When $m = 0$:
\[
x^2 - 2x + 1 = 0 \Rightarrow (x-1)^2 = 0
\]

This gives a repeated root, so $m = 0$ is excluded.

\textbf{Conclusion:}

The equation has two distinct real solutions when $m \in \mathbb{R} \setminus \{0, 1\}$.

With $a = 0$ and $b = 1$, the answer is $\mathbb{R}\setminus\{a,b\}$.

\subsection*{Answer}
$\mathbb{R}\setminus\{a,b\}$ where $a = 0, b = 1$ (option \textbf{D}).

\end{document}
