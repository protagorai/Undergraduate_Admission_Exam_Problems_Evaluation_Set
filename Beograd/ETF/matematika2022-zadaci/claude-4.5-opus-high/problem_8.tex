\documentclass[12pt]{article}
\usepackage[margin=1in]{geometry}
\usepackage{amsmath,amssymb}
\begin{document}

\section*{Problem 8}
Find the number of distinct natural numbers that can be written using digits 1, 2, 3, 4, 5 such that each digit is used at most once.

(A) 325 \quad (B) 120 \quad (C) 5 \quad (D) 3125 \quad (E) 450

\subsection*{Solution}
We need to count all possible natural numbers formed by using the digits 1, 2, 3, 4, 5 where each digit appears at most once in each number.

\textbf{1-digit numbers:}
We can choose any one of the 5 digits.
\[
\text{Count} = 5
\]

\textbf{2-digit numbers:}
Choose 2 digits from 5 and arrange them.
\[
\text{Count} = P(5,2) = 5 \times 4 = 20
\]

\textbf{3-digit numbers:}
Choose 3 digits from 5 and arrange them.
\[
\text{Count} = P(5,3) = 5 \times 4 \times 3 = 60
\]

\textbf{4-digit numbers:}
Choose 4 digits from 5 and arrange them.
\[
\text{Count} = P(5,4) = 5 \times 4 \times 3 \times 2 = 120
\]

\textbf{5-digit numbers:}
Use all 5 digits and arrange them.
\[
\text{Count} = P(5,5) = 5! = 120
\]

\textbf{Total:}
\[
5 + 20 + 60 + 120 + 120 = 325
\]

\subsection*{Answer}
$325$ (option \textbf{A}).

\end{document}
