\documentclass[12pt]{article}
\usepackage[margin=1in]{geometry}
\usepackage{amsmath,amssymb}
\begin{document}

\section*{Problem 17}
Ako je odnos binomnih koeficijenata četvrtog i trećeg člana u razvoju binoma $\left(\sqrt[4]{5} - \sqrt[4]{3}\right)^n$ $(n \in \mathbb{N}, n \geq 3)$ jednak 10, onda je broj racionalnih članova u ovom razvoju jednak:

\subsection*{Solution}
U razvoju binoma $(a + b)^n$, četvrti član je $\binom{n}{3}a^{n-3}b^3$, a treći član je $\binom{n}{2}a^{n-2}b^2$.

Odnos binomnih koeficijenata:
\[
\frac{\binom{n}{3}}{\binom{n}{2}} = 10
\]

\[
\frac{\frac{n!}{3!(n-3)!}}{\frac{n!}{2!(n-2)!}} = \frac{2!(n-2)!}{3!(n-3)!} = \frac{2 \cdot (n-2)!}{6 \cdot (n-3)!} = \frac{n-2}{3} = 10
\]

Dakle:
\[
n - 2 = 30 \Rightarrow n = 32
\]

Sada tražimo racionalne članove u razvoju $\left(\sqrt[4]{5} - \sqrt[4]{3}\right)^{32}$.

Opšti član je:
\[
T_{k+1} = \binom{32}{k} (\sqrt[4]{5})^{32-k} (-\sqrt[4]{3})^k = \binom{32}{k} (-1)^k 5^{\frac{32-k}{4}} 3^{\frac{k}{4}}
\]

Da bi član bio racionalan, potrebno je da budu $\frac{32-k}{4}$ i $\frac{k}{4}$ celi brojevi.

To znači da $k$ i $32-k$ moraju biti deljivi sa 4.

Pošto je $32-k \equiv -k \pmod{4}$, potrebno je $k \equiv 0 \pmod{4}$.

Za $k = 0, 4, 8, 12, 16, 20, 24, 28, 32$, imamo 9 vrednosti.

\subsection*{Answer}
Više od 4 (option \textbf{E}).

\end{document}