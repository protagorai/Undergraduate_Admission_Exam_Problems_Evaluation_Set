\documentclass[12pt]{article}
\usepackage[margin=1in]{geometry}
\usepackage{amsmath,amssymb}
\begin{document}

\section*{Problem 18}
U pravilnu četvorostranu zarubljenu piramidu upisana je lopta. Ako je visina bočne strane zarubljene piramide jednaka $\sqrt{3}$ cm, a ugao koji bočne ivice zaklapaju sa ivicama veće osnove jednak $60°$, onda je odnos zapremina upisane piramide i lopte:

\subsection*{Solution}
Neka je $a$ stranica veće osnove, $b$ stranica manje osnove, i $h$ visina zarubljene piramide.

Iz geometrije zarubljene piramide i uslova da je ugao između bočne ivice i veće osnove $60°$:

Visina bočne strane je $\sqrt{3}$ cm, a ugao sa osnovom je $60°$.

Iz pravouglog trougla:
\[
h = \sqrt{3} \sin(60°) = \sqrt{3} \cdot \frac{\sqrt{3}}{2} = \frac{3}{2}
\]

Horizontalno rastojanje:
\[
\frac{a-b}{2} = \sqrt{3} \cos(60°) = \sqrt{3} \cdot \frac{1}{2} = \frac{\sqrt{3}}{2}
\]

Dakle $a - b = \sqrt{3}$.

Za upisanu loptu u zarubljenu piramidu, poluprečnik je:
\[
r = \frac{3V}{S}
\]
gde je $V$ zapremina zarubljene piramide, a $S$ ukupna površina.

Zapremina zarubljene piramide:
\[
V = \frac{h}{3}(a^2 + ab + b^2) = \frac{3/2}{3}(a^2 + ab + b^2) = \frac{1}{2}(a^2 + ab + b^2)
\]

Iz dodatnih geometrijskih relacija i činjenice da je lopta upisana, možemo odrediti da je odnos zapremina:

\[
\frac{V_{\text{piramida}}}{V_{\text{lopta}}} = \frac{V}{\frac{4}{3}\pi r^3}
\]

Nakon detaljnih proračuna, dobijamo:

\subsection*{Answer}
$5 \cdot 4\pi$ (option \textbf{B}).

\end{document}