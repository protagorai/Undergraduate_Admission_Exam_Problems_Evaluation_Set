\documentclass[12pt]{article}
\usepackage[margin=1in]{geometry}
\usepackage{amsmath,amssymb}
\begin{document}

\section*{Problem 8}
Broj različitih prirodnih brojeva napisanih pomoću cifara 1, 2, 3, 4, 5 tako da se svaka cifra koristi najviše jednom je:

\subsection*{Solution}
Trebamo da prebrojimo sve prirodne brojeve koji se mogu formirati koristeći cifre 1, 2, 3, 4, 5, gde se svaka cifra koristi najviše jednom.

Ovo uključuje:
- Jednocifrene brojeve: 5 brojeva (1, 2, 3, 4, 5)
- Dvocifrene brojeve: $5 \times 4 = 20$ brojeva
- Trocifrene brojeve: $5 \times 4 \times 3 = 60$ brojeva  
- Četvorocifrene brojeve: $5 \times 4 \times 3 \times 2 = 120$ brojeva
- Petocifrene brojeve: $5 \times 4 \times 3 \times 2 \times 1 = 120$ brojeva

Ukupno:
\[
5 + 20 + 60 + 120 + 120 = 325
\]

\subsection*{Answer}
325 (option \textbf{A}).

\end{document}