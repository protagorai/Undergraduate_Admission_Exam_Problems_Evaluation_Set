\documentclass[12pt]{article}
\usepackage[margin=1in]{geometry}
\usepackage{amsmath,amssymb}
\begin{document}

\section*{Problem 5}
Ostatak pri deljenju polinoma $P(x) = x^{2022} - 2x^{2021} + x^{2019} - x^2 + 2x + 1$ polinomom $Q(x) = x^2 + 1$ iznosi:

\subsection*{Solution}
Kada delimo polinom $P(x)$ sa $Q(x) = x^2 + 1$, ostatak je polinom stepena manjeg od 2, tj. $R(x) = ax + b$.

Imamo: $P(x) = Q(x) \cdot S(x) + R(x)$

Iz $x^2 + 1 = 0$ dobijamo $x^2 = -1$, pa su koreni $x = i$ i $x = -i$.

Za $x = i$:
$P(i) = R(i)$

Računamo $P(i)$:
- $i^2 = -1$
- $i^{2019} = i^{4 \cdot 504 + 3} = (i^4)^{504} \cdot i^3 = 1^{504} \cdot (-i) = -i$
- $i^{2021} = i^{4 \cdot 505 + 1} = i$
- $i^{2022} = i^{4 \cdot 505 + 2} = i^2 = -1$

\[
P(i) = (-1) - 2(i) + (-i) - (-1) + 2(i) + 1 = -1 - 2i - i + 1 + 2i + 1 = 1 - i
\]

Za $x = -i$:
- $(-i)^2 = -1$
- $(-i)^{2019} = (-1)^{2019} \cdot i^{2019} = (-1) \cdot (-i) = i$
- $(-i)^{2021} = (-1)^{2021} \cdot i^{2021} = (-1) \cdot i = -i$
- $(-i)^{2022} = (-1)^{2022} \cdot i^{2022} = 1 \cdot (-1) = -1$

\[
P(-i) = (-1) - 2(-i) + i - (-1) + 2(-i) + 1 = -1 + 2i + i + 1 - 2i + 1 = 1 + i
\]

Iz sistema:
\[
R(i) = ai + b = 1 - i
\]
\[
R(-i) = -ai + b = 1 + i
\]

Oduzimanjem: $2ai = -2i \Rightarrow a = -1$
Sabiranjem: $2b = 2 \Rightarrow b = 1$

\subsection*{Answer}
$R(x) = -x + 1$ (option \textbf{E}).

\end{document}