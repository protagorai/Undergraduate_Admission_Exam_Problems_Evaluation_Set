\documentclass[12pt]{article}
\usepackage[margin=1in]{geometry}
\usepackage{amsmath,amssymb}
\begin{document}

\section*{Problem 19}
Najveći broj realnih rešenja jednačine $|2 - |x - x^2|| = p$, gde je $p$ realni parametar, je:

\subsection*{Solution}
Analizirajmo funkciju $f(x) = |2 - |x - x^2||$ po koracima.

Prvo, $g(x) = x - x^2 = x(1-x)$.
- $g(x) \geq 0$ za $x \in [0, 1]$
- $g(x) < 0$ za $x \in (-\infty, 0) \cup (1, \infty)$

Dakle:
\[
|x - x^2| = \begin{cases}
x - x^2 & \text{za } x \in [0, 1] \\
x^2 - x & \text{za } x \in (-\infty, 0) \cup (1, \infty)
\end{cases}
\]

Sada analiziramo $h(x) = 2 - |x - x^2|$:

Za $x \in [0, 1]$: $h(x) = 2 - (x - x^2) = 2 - x + x^2$
Za $x \in (-\infty, 0) \cup (1, \infty)$: $h(x) = 2 - (x^2 - x) = 2 - x^2 + x$

Tražimo gde je $h(x) = 0$:
- Za $x \in [0, 1]$: $2 - x + x^2 = 0 \Rightarrow x^2 - x + 2 = 0$ (nema realnih rešenja)
- Za $x \in (-\infty, 0) \cup (1, \infty)$: $2 - x^2 + x = 0 \Rightarrow x^2 - x - 2 = 0 \Rightarrow (x-2)(x+1) = 0$

Dakle $h(x) = 0$ za $x = -1$ i $x = 2$.

Analizirajmo ponašanje $h(x)$:
- $h(x) > 0$ za $x \in (-1, 2)$
- $h(x) < 0$ za $x \in (-\infty, -1) \cup (2, \infty)$

Konačno:
\[
f(x) = |h(x)| = \begin{cases}
h(x) & \text{za } x \in (-1, 2) \\
-h(x) & \text{za } x \in (-\infty, -1) \cup (2, \infty)
\end{cases}
\]

Analizirajući grafik funkcije $f(x)$, vidimo da:
- Za $p = 0$: 2 rešenja ($x = -1, 2$)
- Za male pozitivne $p$: 6 rešenja
- Za veće $p$: manje rešenja

Maksimalni broj rešenja je 6.

\subsection*{Answer}
6 (option \textbf{B}).

\end{document}