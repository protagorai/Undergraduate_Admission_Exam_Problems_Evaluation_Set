\documentclass[12pt]{article}
\usepackage[margin=1in]{geometry}
\usepackage{amsmath,amssymb}
\begin{document}

\section*{Problem 18}
A sphere is inscribed in a regular truncated square pyramid (a right square frustum).
The slant height of a lateral face is $\sqrt3$ cm, and the angle between a lateral edge and an edge of the larger base is $60^\circ$.
Find the ratio of volumes (frustum : sphere).

\subsection*{Solution}
Let the side lengths of the larger and smaller bases be $a$ and $b$ ($a>b$), and let the frustum height be $h$.

\medskip
\noindent\textbf{Step 1: Use the $60^\circ$ condition.}
Consider a lateral edge from a vertex of the larger base to the corresponding vertex of the smaller base.
Its horizontal displacement in the base plane has components $\frac{a-b}{2}$ in two perpendicular directions, so the horizontal length is
\[
d=\sqrt{\left(\frac{a-b}{2}\right)^2+\left(\frac{a-b}{2}\right)^2}=\frac{a-b}{\sqrt2}.
\]
Thus the lateral edge length is
\[
s=\sqrt{h^2+d^2}=\sqrt{h^2+\frac{(a-b)^2}{2}}.
\]
The angle between this edge and a base edge (direction along one axis) is $60^\circ$, so
\[
\cos 60^\circ=\frac{\text{(absolute value of x-component)}}{\text{edge length}}
=\frac{\frac{a-b}{2}}{s}=\frac12,
\]
which gives $s=a-b$. Squaring:
\[
(a-b)^2=h^2+\frac{(a-b)^2}{2}
\quad\Longrightarrow\quad
h^2=\frac{(a-b)^2}{2}
\quad\Longrightarrow\quad
h=\frac{a-b}{\sqrt2}.
\]

\medskip
\noindent\textbf{Step 2: Use the slant height $l=\sqrt3$.}
The slant height of a lateral face (distance between midpoints of corresponding edges) satisfies
\[
l^2=h^2+\left(\frac{a-b}{2}\right)^2
=\frac{(a-b)^2}{2}+\frac{(a-b)^2}{4}
=\frac{3}{4}(a-b)^2.
\]
So
\[
l=\frac{\sqrt3}{2}(a-b)=\sqrt3 \quad\Longrightarrow\quad a-b=2.
\]
Hence
\[
h=\frac{2}{\sqrt2}=\sqrt2.
\]

\medskip
\noindent\textbf{Step 3: Condition for an inscribed sphere.}
Take the axial cross-section through the axis and midpoints of two opposite base edges; it is an isosceles trapezoid with bases $a$ and $b$ and legs equal to the slant height $l$.
Such a trapezoid has an incircle (and corresponds to an insphere here) iff the sum of bases equals the sum of legs:
\[
a+b=2l=2\sqrt3.
\]
Together with $a-b=2$:
\[
a=\sqrt3+1,\qquad b=\sqrt3-1.
\]

\medskip
\noindent\textbf{Step 4: Volumes.}
Frustum volume:
\[
V_f=\frac{h}{3}\left(a^2+b^2+\sqrt{a^2b^2}\right).
\]
Compute:
\[
a^2=(\sqrt3+1)^2=4+2\sqrt3,\quad b^2=(\sqrt3-1)^2=4-2\sqrt3,
\]
\[
\sqrt{a^2b^2}=\sqrt{(4+2\sqrt3)(4-2\sqrt3)}=\sqrt{16-12}=2.
\]
Thus
\[
V_f=\frac{\sqrt2}{3}\,( (4+2\sqrt3)+(4-2\sqrt3)+2)=\frac{\sqrt2}{3}\cdot 10=\frac{10\sqrt2}{3}.
\]
Since the sphere is tangent to both bases, its diameter equals the frustum height, so
\[
2r=h\Rightarrow r=\frac{h}{2}=\frac{\sqrt2}{2}.
\]
Sphere volume:
\[
V_s=\frac{4}{3}\pi r^3=\frac{4}{3}\pi\left(\frac{\sqrt2}{2}\right)^3
=\frac{\sqrt2}{3}\pi.
\]
Therefore
\[
V_f:V_s=\frac{10\sqrt2}{3}:\frac{\sqrt2}{3}\pi=10:\pi.
\]

\subsection*{Answer}
$10:\pi$ (option \textbf{D}).

\end{document}

