\documentclass[12pt]{article}
\usepackage[margin=1in]{geometry}
\usepackage{amsmath,amssymb}
\begin{document}

\section*{Problem 17}
In the binomial expansion of
\[
\left(\sqrt[3]{5}-\sqrt[5]{3}\right)^n \qquad (n\in\mathbb{N},\ n\ge 3),
\]
the ratio of the binomial coefficients of the 4th and 3rd term equals $10$.
Find the number of rational terms in this expansion.

\subsection*{Solution}
In $(a-b)^n$, the $k$-th term (counting from $1$) has binomial coefficient $\binom{n}{k-1}$.
Thus:
\[
\text{3rd term coefficient}=\binom{n}{2},\qquad
\text{4th term coefficient}=\binom{n}{3}.
\]
Given
\[
\frac{\binom{n}{3}}{\binom{n}{2}}=10.
\]
Compute:
\[
\frac{\binom{n}{3}}{\binom{n}{2}}
=\frac{\frac{n(n-1)(n-2)}{6}}{\frac{n(n-1)}{2}}
=\frac{n-2}{3}.
\]
So $\dfrac{n-2}{3}=10\Rightarrow n=32$.

\medskip
Now consider a general term:
\[
\binom{32}{k}\left(5^{1/3}\right)^{32-k}\left(3^{1/5}\right)^k
=\binom{32}{k}\,5^{\frac{32-k}{3}}\,3^{\frac{k}{5}}
\]
(up to a sign).
This term is rational iff both exponents are integers:
\[
\frac{k}{5}\in\mathbb{Z}\quad\text{and}\quad \frac{32-k}{3}\in\mathbb{Z}.
\]
So $k=5m$. Then $32-5m\equiv 0\pmod 3$.
Since $32\equiv 2$ and $5\equiv 2\pmod 3$, this gives
\[
2-2m\equiv 0\pmod 3 \;\Longrightarrow\; 1-m\equiv 0\pmod 3 \;\Longrightarrow\; m\equiv 1\pmod 3.
\]
With $0\le k\le 32$ we have $0\le m\le 6$, so $m\in\{1,4\}$, i.e.
\[
k\in\{5,20\}.
\]
Therefore there are exactly $2$ rational terms.

\subsection*{Answer}
$2$ (option \textbf{A}).

\end{document}

