\documentclass[12pt]{article}
\usepackage[margin=1in]{geometry}
\usepackage{amsmath,amssymb}
\begin{document}

\section*{Problem 19}
For real parameter $p$, consider
\[
\left|2-\left|x-x^2\right|\right|=p.
\]
What is the maximum possible number of real solutions (over all $p\in\mathbb{R}$)?

\subsection*{Solution}
Let $g(x)=|x-x^2|=|x(1-x)|\ge 0$. Then the equation is
\[
|2-g(x)|=p.
\]
Necessarily $p\ge 0$. Also
\[
|2-g|=p \quad\Longleftrightarrow\quad g=2+p \ \ \text{or}\ \ g=2-p.
\]
The second value $2-p$ is valid only if $2-p\ge 0$.

\medskip
Fix a constant $c>0$ and solve $g(x)=c$, i.e.
\[
|x-x^2|=c.
\]
This splits into two quadratics:
\[
x-x^2=c \quad\Longleftrightarrow\quad x^2-x+c=0,
\]
\[
x-x^2=-c \quad\Longleftrightarrow\quad x^2-x-c=0.
\]
The second quadratic has discriminant $1+4c>0$ for all $c>0$, so it always contributes \emph{two} real roots.
The first has discriminant $1-4c$, so it contributes:
\[
\begin{cases}
2 \text{ roots}, & 0<c<\frac14,\\
1 \text{ root}, & c=\frac14,\\
0 \text{ roots}, & c>\frac14.
\end{cases}
\]
Therefore
\[
\#\{x: g(x)=c\}=
\begin{cases}
4, & 0<c<\frac14,\\
3, & c=\frac14,\\
2, & c>\frac14.
\end{cases}
\]

\medskip
Now take $c_1=2+p$ (always $> \frac14$), so $g(x)=c_1$ gives exactly $2$ solutions.
If $0\le p\le 2$, also $c_2=2-p\ge 0$.
To maximize the total number of solutions of $|2-g|=p$, we want $c_2$ in $(0,\frac14)$ so that $g(x)=c_2$ contributes $4$ solutions.
This happens when
\[
0<2-p<\frac14 \quad\Longleftrightarrow\quad \frac74<p<2.
\]
For such $p$, the two sets (levels $c_1$ and $c_2$) are disjoint because $c_1\ne c_2$, so the total number of solutions is
\[
2+4=6.
\]
No larger total is possible, since at most two level values are involved.

\subsection*{Answer}
The maximum possible number of real solutions is $6$ (option \textbf{B}).

\end{document}

