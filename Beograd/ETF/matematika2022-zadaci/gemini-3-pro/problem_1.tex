\documentclass[12pt]{article}
\usepackage[margin=1in]{geometry}
\usepackage{amsmath,amssymb}
\usepackage[utf8]{inputenc}

\begin{document}

\section*{Problem 1}
Cena računara bila je 100000 dinara, pa je onda podignuta za 25 procenata. Zatim je ta nova cena na akciji snižena za 20 procenata i iznosi:
\begin{itemize}
    \item[(A)] 90000
    \item[(B)] 96000
    \item[(C)] 120000
    \item[(D)] 105000
    \item[(E)] 100000
    \item[(N)] Ne znam
\end{itemize}

\subsection*{Solution}
Neka je početna cena $C_0 = 100000$.
Nakon poskupljenja od 25\%, nova cena $C_1$ je:
\[ C_1 = C_0 \cdot \left(1 + \frac{25}{100}\right) = 100000 \cdot 1.25 = 125000 \]
Nakon sniženja od 20\%, konačna cena $C_2$ je:
\[ C_2 = C_1 \cdot \left(1 - \frac{20}{100}\right) = 125000 \cdot 0.80 \]
\[ C_2 = 12500 \cdot 8 = 100000 \]

\subsection*{Answer}
100000 (option \textbf{E}).

\end{document}
