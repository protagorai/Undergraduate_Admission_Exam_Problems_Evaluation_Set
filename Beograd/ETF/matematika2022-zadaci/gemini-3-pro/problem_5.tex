\documentclass[12pt]{article}
\usepackage[margin=1in]{geometry}
\usepackage{amsmath,amssymb}
\usepackage[utf8]{inputenc}

\begin{document}

\section*{Problem 5}
Ostatak pri deljenju polinoma $P(x) = x^{2022} - 2x^{2021} + x^{2019} - x^2 + 2x + 1$ polinomom $Q(x) = x^2 + 1$ iznosi:
\begin{itemize}
    \item[(A)] $-2x+1$
    \item[(B)] 1
    \item[(C)] $-x-1$
    \item[(D)] $x-1$
    \item[(E)] $-x+1$
    \item[(N)] Ne znam
\end{itemize}

\subsection*{Solution}
Nule polinoma $Q(x) = x^2 + 1$ su $x = i$ i $x = -i$.
Prema Bezujevom stavu, ostatak $R(x)$ pri deljenju sa $x^2 + 1$ je linearni polinom $ax + b$ takav da je $P(i) = R(i)$.
Računamo $P(i)$:
\[ P(i) = i^{2022} - 2i^{2021} + i^{2019} - i^2 + 2i + 1 \]
Koristimo osobine stepena imaginarne jedinice $i$:
$i^1 = i, i^2 = -1, i^3 = -i, i^4 = 1$.
Eksponenti modulo 4:
$2022 \equiv 2 \pmod 4 \implies i^{2022} = -1$
$2021 \equiv 1 \pmod 4 \implies i^{2021} = i$
$2019 \equiv 3 \pmod 4 \implies i^{2019} = -i$
Zamenom dobijamo:
\[ P(i) = (-1) - 2(i) + (-i) - (-1) + 2i + 1 \]
\[ P(i) = -1 - 2i - i + 1 + 2i + 1 = 1 - i \]
Ostatak je oblika $R(x) = ax + b$, pa je $R(i) = ai + b$.
Izjednačavanjem:
\[ ai + b = 1 - i \]
Poređenjem realnih i imaginarnih delova:
$b = 1$
$a = -1$
Dakle, ostatak je $R(x) = -x + 1$.

\subsection*{Answer}
$-x + 1$ (option \textbf{E}).

\end{document}
