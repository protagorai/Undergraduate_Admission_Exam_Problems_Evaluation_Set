\documentclass[12pt]{article}
\usepackage[margin=1in]{geometry}
\usepackage{amsmath,amssymb}
\usepackage[utf8]{inputenc}

\begin{document}

\section*{Problem 18}
U pravilnu četvorostranu zarubljenu piramidu upisana je lopta. Ako je visina bočne strane zarubljene piramide jednaka $\sqrt{3} cm$, a ugao koji bočne ivice zaklapaju sa ivicama veće osnove jednak $60^0$, onda je odnos zapremina zarubljene piramide i lopte:
\begin{itemize}
    \item[(A)] $10\sqrt{2}:\pi$
    \item[(B)] $5:4\pi$
    \item[(C)] $5:6\pi$
    \item[(D)] $10:\pi$
    \item[(E)] $20:\pi$
    \item[(N)] Ne znam
\end{itemize}

\subsection*{Solution}
Neka su osnovice $a$ i $b$ ($a > b$), a visina bočne strane (apotema) $h = \sqrt{3}$.
Bočna strana je jednakokraki trapez. Ugao na većoj osnovici je $60^\circ$.
Visina trapeza je $h = \frac{a-b}{2} \tan 60^\circ = \frac{a-b}{2} \sqrt{3}$.
Iz $h = \sqrt{3}$ sledi $\frac{a-b}{2} = 1 \implies a-b = 2$.
Uslov da se u zarubljenu piramidu može upisati lopta povlači da se u osni presek (koji je takođe jednakokraki trapez sa visinom piramide $H$ i krakom $h$) može upisati krug (poluprečnika $r = H/2$).
Takođe, za postojanje upisane lopte, potreban je i uslov tangencijalnosti bočnih strana?
Zapravo, uslov za loptu je da je visina $H$ jednaka prečniku upisane lopte, i da se u poprečni presek (kroz sredine stranica osnova) može upisati krug.
Taj presek je trapez sa osnovicama $a$ i $b$ i kracima $h$.
Uslov tangentnosti četvorougla: $a+b = 2h = 2\sqrt{3}$.
Rešavamo sistem:
$a-b = 2$
$a+b = 2\sqrt{3}$
Sledi $2a = 2+2\sqrt{3} \implies a = 1+\sqrt{3}$.
$2b = 2\sqrt{3}-2 \implies b = \sqrt{3}-1$.
Visina piramide $H$ je prečnik upisane lopte.
$H^2 = h^2 - (\frac{a-b}{2})^2 = (\sqrt{3})^2 - 1^2 = 2$.
$H = \sqrt{2}$. Poluprečnik lopte $r = \frac{\sqrt{2}}{2}$.
Zapremina lopte $V_L = \frac{4}{3} \pi r^3 = \frac{4}{3} \pi \frac{2\sqrt{2}}{8} = \frac{\pi\sqrt{2}}{3}$.
Zapremina piramide $V_P = \frac{H}{3} (a^2 + ab + b^2)$.
$a^2 + b^2 = (\sqrt{3}+1)^2 + (\sqrt{3}-1)^2 = 8$.
$ab = (\sqrt{3}+1)(\sqrt{3}-1) = 2$.
$V_P = \frac{\sqrt{2}}{3} (8 + 2) = \frac{10\sqrt{2}}{3}$.
Odnos $V_P : V_L = \frac{10\sqrt{2}}{3} : \frac{\pi\sqrt{2}}{3} = 10 : \pi$.

\subsection*{Answer}
$10:\pi$ (option \textbf{D}).

\end{document}
