\documentclass[12pt]{article}
\usepackage[margin=1in]{geometry}
\usepackage{amsmath,amssymb}
\usepackage[utf8]{inputenc}

\begin{document}

\section*{Problem 8}
Broj različitih prirodnih brojeva napisanih pomoću cifara 1, 2, 3, 4, 5 tako da se svaka cifra koristiti najviše jednom je:
\begin{itemize}
    \item[(A)] 325
    \item[(B)] 120
    \item[(C)] 5
    \item[(D)] 3125
    \item[(E)] 450
    \item[(N)] Ne znam
\end{itemize}

\subsection*{Solution}
Cifre su $\{1, 2, 3, 4, 5\}$. Svaka se koristi najviše jednom. To znači da možemo formirati jednocifrene, dvocifrene, trocifrene, četvorocifrene i petocifrene brojeve bez ponavljanja cifara.
Broj varijacija bez ponavljanja od $n=5$ elemenata klase $k$:
1. Jednocifreni: $V_5^1 = 5$
2. Dvocifreni: $V_5^2 = 5 \cdot 4 = 20$
3. Trocifreni: $V_5^3 = 5 \cdot 4 \cdot 3 = 60$
4. Četvorocifreni: $V_5^4 = 5 \cdot 4 \cdot 3 \cdot 2 = 120$
5. Petocifreni: $V_5^5 = 5 \cdot 4 \cdot 3 \cdot 2 \cdot 1 = 120$

Ukupan broj je zbir ovih vrednosti:
\[ S = 5 + 20 + 60 + 120 + 120 = 325 \]

\subsection*{Answer}
325 (option \textbf{A}).

\end{document}
