\documentclass[12pt]{article}
\usepackage[margin=1in]{geometry}
\usepackage{amsmath,amssymb}
\usepackage[utf8]{inputenc}
\usepackage[T1]{fontenc}

\begin{document}

\section*{Zadatak 11}
Tačke $A(-2, 2)$ i $B(2, -2)$ su temena trougla $ABC$, a $N(1, 2)$ je presek visina tog trougla. Zbir koordinata temena $C$ jednak je:

\subsection*{Rešenje}
Neka je $C(x, y)$.
Prava $AB$ ima koeficijent pravca $k_{AB} = \frac{-2-2}{2-(-2)} = -1$.
Visina iz $C$ je normalna na $AB$ i prolazi kroz $N(1, 2)$. Njen koeficijent pravca je $k_h = -1/(-1) = 1$.
Jednačina visine $h_c$: $y - 2 = 1(x - 1) \implies y = x + 1$. Tačka $C$ pripada ovoj pravoj.
Visina iz $A$ je normalna na $BC$ i prolazi kroz $N$. Prava $AN$: $y=2$ (jer $A$ i $N$ imaju istu y-koordinatu).
Dakle, visina iz $A$ je horizontalna, što znači da je stranica $BC$ vertikalna.
Kako $B$ ima koordinate $(2, -2)$, prava $BC$ je $x = 2$.
Tačka $C$ je presek prave $x=2$ i $y=x+1$.
$x=2 \implies y = 2+1 = 3$.
$C(2, 3)$.
Zbir koordinata: $2 + 3 = 5$.

\subsection*{Odgovor}
5 (opcija \textbf{B}).

\end{document}
