\documentclass[12pt]{article}
\usepackage[margin=1in]{geometry}
\usepackage{amsmath,amssymb}
\usepackage[utf8]{inputenc}
\usepackage[T1]{fontenc}

\begin{document}

\section*{Zadatak 16}
Broj svih rešenja sistema jednačina
\[
\sin^2 x + \sin^2 y = \frac{3}{4}
\]
\[
x + y = \frac{5\pi}{12}
\]
takvih da je $x \in (-\pi, 2\pi)$ i $y \in (-2\pi, \pi)$ jeste:

\subsection*{Rešenje}
Koristimo $\sin^2 x = \frac{1-\cos 2x}{2}$.
$\frac{1-\cos 2x}{2} + \frac{1-\cos 2y}{2} = \frac{3}{4} \implies 2 - (\cos 2x + \cos 2y) = \frac{3}{2}$.
$\cos 2x + \cos 2y = \frac{1}{2}$.
$2 \cos(x+y) \cos(x-y) = \frac{1}{2}$.
Zamenom $x+y = \frac{5\pi}{12}$:
$2 \cos\frac{5\pi}{12} \cos(x-y) = \frac{1}{2}$.
$\cos\frac{5\pi}{12} = \sin\frac{\pi}{12} = \frac{\sqrt{6}-\sqrt{2}}{4}$.
$2 \frac{\sqrt{6}-\sqrt{2}}{4} \cos(x-y) = \frac{1}{2} \implies \frac{\sqrt{6}-\sqrt{2}}{2} \cos(x-y) = \frac{1}{2}$.
$\cos(x-y) = \frac{1}{\sqrt{6}-\sqrt{2}} = \frac{\sqrt{6}+\sqrt{2}}{4} = \cos\frac{\pi}{12}$.
$x - y = \pm \frac{\pi}{12} + 2k\pi$.
Sistem: $x+y = \frac{5\pi}{12}$, $x-y = \pm \frac{\pi}{12} + 2k\pi$.

Slučaj 1: $x-y = \frac{\pi}{12} + 2k\pi$.
$2x = \frac{6\pi}{12} + 2k\pi = \frac{\pi}{2} + 2k\pi \implies x = \frac{\pi}{4} + k\pi$.
$2y = \frac{4\pi}{12} - 2k\pi = \frac{\pi}{3} - 2k\pi \implies y = \frac{\pi}{6} - k\pi$.
Rešenja u intervalima $x \in (-\pi, 2\pi)$, $y \in (-2\pi, \pi)$:
$k=0: x=\pi/4, y=\pi/6$. (OK)
$k=1: x=5\pi/4, y=-5\pi/6$. (OK)
$k=-1: x=-3\pi/4, y=7\pi/6$. ($y > \pi$, nije rešenje)
$k=2: x=9\pi/4 > 2\pi$.
Ostali $k$ ne daju rešenja. (2 rešenja).

Slučaj 2: $x-y = -\frac{\pi}{12} + 2k\pi$.
$2x = \frac{4\pi}{12} + 2k\pi \implies x = \frac{\pi}{6} + k\pi$.
$2y = \frac{6\pi}{12} - 2k\pi \implies y = \frac{\pi}{4} - k\pi$.
Rešenja:
$k=0: x=\pi/6, y=\pi/4$. (OK)
$k=1: x=7\pi/6, y=-3\pi/4$. (OK)
$k=-1: x=-5\pi/6, y=5\pi/4 > \pi$.
$k=2: x=13\pi/6 > 2\pi$.
Ukupno 4 rešenja.

\subsection*{Odgovor}
4 (opcija \textbf{D}).

\end{document}
