\documentclass[12pt]{article}
\usepackage[margin=1in]{geometry}
\usepackage{amsmath,amssymb}
\usepackage[utf8]{inputenc}
\usepackage[T1]{fontenc}

\begin{document}

\section*{Zadatak 17}
Dvocifreni broj koji je jednak proizvodu zbira svojih cifara i apsolutne vrednosti razlike pripada intervalu:

\subsection*{Rešenje}
Broj je $\overline{xy} = 10x+y$.
Jednačina: $10x+y = (x+y)|x-y|$.
Slučaj $x > y$: $10x+y = (x+y)(x-y) = x^2-y^2$. Nema rešenja u skupu cifara.
Slučaj $x < y$: $10x+y = (x+y)(y-x) = y^2-x^2$.
$y^2 - y - (x^2+10x) = 0$.
Diskriminanta $D = 1 + 4(x^2+10x) = (2x+10)^2 - 99$.
Da bi $D$ bio kvadrat $k^2$, mora biti $(2x+10)^2 - k^2 = 99$.
Rešavanjem u skupu celih brojeva dobija se $x=4$.
Za $x=4$, $y^2-y-56=0 \implies (y-8)(y+7)=0 \implies y=8$.
Broj je 48.
Provera: $(4+8)|4-8| = 12 \cdot 4 = 48$. Tačno.
Broj 48 pripada intervalu $[31, 50]$.

\subsection*{Odgovor}
$[31, 50]$ (opcija \textbf{B}).

\end{document}
