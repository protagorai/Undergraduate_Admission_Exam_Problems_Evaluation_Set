\documentclass[12pt]{article}
\usepackage[margin=1in]{geometry}
\usepackage{amsmath,amssymb}
\usepackage[utf8]{inputenc}
\usepackage[T1]{fontenc}

\begin{document}

\section*{Zadatak 9}
Vrednost izraza $\cos\frac{\pi}{7} - \cos\frac{2\pi}{7} + \cos\frac{3\pi}{7}$ jeste:

\subsection*{Rešenje}
Neka je $S = \cos\frac{\pi}{7} - \cos\frac{2\pi}{7} + \cos\frac{3\pi}{7}$.
Koristimo identitet $\cos\frac{2\pi}{7} = -\cos\frac{5\pi}{7}$.
$S = \cos\frac{\pi}{7} + \cos\frac{5\pi}{7} + \cos\frac{3\pi}{7}$.
Ovo je suma kosinusa uglova u aritmetičkoj progresiji sa razlikom $2\pi/7$.
Možemo pomnožiti izraz sa $2\sin\frac{\pi}{7}$:
\[
2\sin\frac{\pi}{7} S = 2\sin\frac{\pi}{7}\cos\frac{\pi}{7} - 2\sin\frac{\pi}{7}\cos\frac{2\pi}{7} + 2\sin\frac{\pi}{7}\cos\frac{3\pi}{7}
\]
Koristeći $2\sin\alpha\cos\beta = \sin(\alpha+\beta) + \sin(\alpha-\beta)$:
\[
= \sin\frac{2\pi}{7} - (\sin\frac{3\pi}{7} - \sin\frac{\pi}{7}) + (\sin\frac{4\pi}{7} - \sin\frac{2\pi}{7})
\]
\[
= \sin\frac{2\pi}{7} - \sin\frac{3\pi}{7} + \sin\frac{\pi}{7} + \sin\frac{3\pi}{7} - \sin\frac{2\pi}{7} \quad (\text{jer } \sin\frac{4\pi}{7} = \sin\frac{3\pi}{7})
\]
\[
= \sin\frac{\pi}{7}
\]
Dakle, $2\sin\frac{\pi}{7} S = \sin\frac{\pi}{7}$. Kako je $\sin\frac{\pi}{7} \neq 0$, sledi $S = 1/2$.

\subsection*{Odgovor}
1/2 (opcija \textbf{B}).

\end{document}
