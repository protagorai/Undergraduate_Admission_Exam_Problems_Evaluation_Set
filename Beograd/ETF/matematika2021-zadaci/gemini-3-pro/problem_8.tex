\documentclass[12pt]{article}
\usepackage[margin=1in]{geometry}
\usepackage{amsmath,amssymb}
\usepackage[utf8]{inputenc}
\usepackage[T1]{fontenc}

\begin{document}

\section*{Zadatak 8}
Koliko jednakih članova imaju aritmetičke progresije $2, 7, 12, 17, \dots$ i $2, 5, 8, 11, \dots$ ako svaka od njih ima 121 član?

\subsection*{Rešenje}
Prva progresija: $a_n = 2 + 5(n-1)$, $1 \le n \le 121$. Maksimalni član $2 + 5 \cdot 120 = 602$.
Druga progresija: $b_m = 2 + 3(m-1)$, $1 \le m \le 121$. Maksimalni član $2 + 3 \cdot 120 = 362$.
Zajednički članovi formiraju aritmetičku progresiju sa prvim članom 2 i korakom $NZS(5, 3) = 15$.
Opšti član preseka: $c_k = 2 + 15k$.
Uslov: $c_k \le \min(602, 362) = 362$.
$2 + 15k \le 362 \implies 15k \le 360 \implies k \le 24$.
Kako $k$ kreće od 0 (za prvi član 2), broj članova je $24 - 0 + 1 = 25$.

\subsection*{Odgovor}
25 (opcija \textbf{D}).

\end{document}
