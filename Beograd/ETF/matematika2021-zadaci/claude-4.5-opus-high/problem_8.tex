\documentclass[12pt]{article}
\usepackage[margin=1in]{geometry}
\usepackage{amsmath,amssymb}
\begin{document}

\section*{Problem 8}
Koliko jednakih članova imaju aritmetičke progresije $2, 7, 12, 17, \ldots$ i $2, 5, 8, 11, \ldots$ ako svaka od njih ima 121 član?

\subsection*{Solution}
Prva aritmetička progresija: $a_n = 2 + (n-1) \cdot 5 = 5n - 3$, za $n = 1, 2, \ldots, 121$.

Članovi: $2, 7, 12, 17, \ldots$, poslednji član je $a_{121} = 5 \cdot 121 - 3 = 602$.

Druga aritmetička progresija: $b_m = 2 + (m-1) \cdot 3 = 3m - 1$, za $m = 1, 2, \ldots, 121$.

Članovi: $2, 5, 8, 11, \ldots$, poslednji član je $b_{121} = 3 \cdot 121 - 1 = 362$.

Tražimo zajedničke članove, tj. rešenja sistema:
\[
5n - 3 = 3m - 1
\]
\[
5n - 3m = 2
\]

Opšte rešenje: $n = 3k + 1$, $m = 5k + 1$ za $k = 0, 1, 2, \ldots$

Provera: $5(3k+1) - 3(5k+1) = 15k + 5 - 15k - 3 = 2$ ✓

Uslovi:
\[
1 \leq n \leq 121 \Rightarrow 1 \leq 3k + 1 \leq 121 \Rightarrow 0 \leq k \leq 40
\]
\[
1 \leq m \leq 121 \Rightarrow 1 \leq 5k + 1 \leq 121 \Rightarrow 0 \leq k \leq 24
\]

Presek: $0 \leq k \leq 24$.

Broj zajedničkih članova: $k \in \{0, 1, 2, \ldots, 24\}$, dakle $25$ članova.

\subsection*{Answer}
$25$ (option \textbf{D}).

\end{document}
