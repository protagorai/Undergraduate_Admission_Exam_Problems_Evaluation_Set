\documentclass[12pt]{article}
\usepackage[margin=1in]{geometry}
\usepackage{amsmath,amssymb}
\begin{document}

\section*{Problem 18}
Ostatak pri deljenju polinoma $P(x) = x^{2024} + x^{2023} + x^{2022} + x^{2021} + x^{20} + 1$ polinomom $Q(x) = x^3 + x^2 + x + 1$ jeste:

\subsection*{Solution}
Primetimo da je $Q(x) = x^3 + x^2 + x + 1 = \frac{x^4 - 1}{x - 1}$ za $x \neq 1$.

Koreni od $Q(x)$ su $x = -1, i, -i$ (primitivni četvrti koreni iz jedinice, osim 1).

Ostatak pri deljenju je polinom stepena najviše 2: $R(x) = ax^2 + bx + c$.

Koristimo činjenicu da je $P(x) \equiv R(x) \pmod{Q(x)}$, pa je $P(\omega) = R(\omega)$ za svaki koren $\omega$ od $Q(x)$.

\textbf{Za $x = -1$:}
\[
P(-1) = (-1)^{2024} + (-1)^{2023} + (-1)^{2022} + (-1)^{2021} + (-1)^{20} + 1
\]
\[
= 1 - 1 + 1 - 1 + 1 + 1 = 2
\]
\[
R(-1) = a - b + c = 2
\]

\textbf{Za $x = i$:}
\[
i^{2024} = i^{4 \cdot 506} = 1
\]
\[
i^{2023} = i^{2024-1} = i^{-1} = -i
\]
\[
i^{2022} = i^{-2} = -1
\]
\[
i^{2021} = i^{-3} = i
\]
\[
i^{20} = i^{4 \cdot 5} = 1
\]
\[
P(i) = 1 + (-i) + (-1) + i + 1 + 1 = 2
\]
\[
R(i) = a \cdot (-1) + b \cdot i + c = -a + c + bi = 2
\]
Dakle: $-a + c = 2$ i $b = 0$.

\textbf{Za $x = -i$:}
\[
(-i)^{2024} = i^{2024} = 1
\]
\[
(-i)^{2023} = -(-i)^{2024}/(-i) = -1/(-i) = -i \cdot (-1) = i
\]

Zapravo: $(-i)^{2023} = (-1)^{2023} \cdot i^{2023} = -1 \cdot (-i) = i$
\[
(-i)^{2022} = (-1)^{2022} \cdot i^{2022} = 1 \cdot (-1) = -1
\]
\[
(-i)^{2021} = (-1)^{2021} \cdot i^{2021} = -1 \cdot i = -i
\]
\[
(-i)^{20} = 1
\]
\[
P(-i) = 1 + i + (-1) + (-i) + 1 + 1 = 2
\]
\[
R(-i) = a \cdot (-1) + b \cdot (-i) + c = -a + c - bi = 2
\]
Potvrđuje: $-a + c = 2$ i $b = 0$.

Iz $a - b + c = 2$ i $b = 0$: $a + c = 2$.
Iz $-a + c = 2$: $c - a = 2$.

Rešavamo sistem:
\[
a + c = 2
\]
\[
-a + c = 2
\]
Sabiranjem: $2c = 4$, pa $c = 2$.
Oduzimanjem: $2a = 0$, pa $a = 0$.

Dakle $R(x) = 2$.

\subsection*{Answer}
$R(x) = 2$, odnosno ostatak je konstanta $2$ (option \textbf{B}).

\end{document}
