\documentclass[12pt]{article}
\usepackage[margin=1in]{geometry}
\usepackage{amsmath,amssymb}
\begin{document}

\section*{Problem 17}
Dvocifreni broj koji je jednak proizvodu zbira svojih cifara i apsolutne vrednosti razlike pripada intervalu:

\subsection*{Solution}
Neka je dvocifreni broj $\overline{ab} = 10a + b$, gde je $a \in \{1, 2, \ldots, 9\}$ i $b \in \{0, 1, \ldots, 9\}$.

Zbir cifara: $a + b$

Apsolutna vrednost razlike: $|a - b|$

Uslov: $10a + b = (a + b) \cdot |a - b|$

\textbf{Slučaj 1:} $a \geq b$, pa je $|a - b| = a - b$.
\[
10a + b = (a + b)(a - b) = a^2 - b^2
\]
\[
a^2 - b^2 - 10a - b = 0
\]
\[
a^2 - 10a = b^2 + b
\]
\[
a(a - 10) = b(b + 1)
\]

Za $a = 1$: $1 \cdot (-9) = -9 = b(b+1)$. Nema rešenja za $b \geq 0$.

Za $a = 2$: $2 \cdot (-8) = -16 = b(b+1)$. Nema rešenja.

...

Za $a = 9$: $9 \cdot (-1) = -9 = b(b+1)$. Nema rešenja.

\textbf{Slučaj 2:} $a < b$, pa je $|a - b| = b - a$.
\[
10a + b = (a + b)(b - a) = b^2 - a^2
\]
\[
b^2 - a^2 - 10a - b = 0
\]
\[
b^2 - b = a^2 + 10a
\]
\[
b(b - 1) = a(a + 10)
\]

Za $a = 1$: $b(b-1) = 1 \cdot 11 = 11$. $b^2 - b - 11 = 0$, $b = \frac{1 + \sqrt{45}}{2} \approx 3.85$. Nije ceo broj.

Za $a = 2$: $b(b-1) = 2 \cdot 12 = 24$. $b^2 - b - 24 = 0$, $b = \frac{1 + \sqrt{97}}{2} \approx 5.42$. Nije ceo broj.

Za $a = 3$: $b(b-1) = 3 \cdot 13 = 39$. $b^2 - b - 39 = 0$, $b = \frac{1 + \sqrt{157}}{2} \approx 6.76$. Nije ceo broj.

Za $a = 4$: $b(b-1) = 4 \cdot 14 = 56$. $b^2 - b - 56 = 0$, $b = \frac{1 + \sqrt{225}}{2} = \frac{1 + 15}{2} = 8$. ✓

Provera: $a = 4$, $b = 8$, broj je 48.
$(4 + 8) \cdot |4 - 8| = 12 \cdot 4 = 48$ ✓

Za $a = 5$: $b(b-1) = 5 \cdot 15 = 75$. $b^2 - b - 75 = 0$, $b = \frac{1 + \sqrt{301}}{2} \approx 9.17$. Nije ceo broj.

Za $a = 6$: $b(b-1) = 6 \cdot 16 = 96$. $b^2 - b - 96 = 0$, $b = \frac{1 + \sqrt{385}}{2} \approx 10.3 > 9$. Van opsega.

Jedino rešenje je $48$, koje pripada intervalu $[31, 50]$.

\subsection*{Answer}
$[31, 50]$ (option \textbf{B}).

\end{document}
