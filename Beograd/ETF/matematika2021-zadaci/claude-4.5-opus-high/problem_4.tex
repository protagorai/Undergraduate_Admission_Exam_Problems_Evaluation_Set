\documentclass[12pt]{article}
\usepackage[margin=1in]{geometry}
\usepackage{amsmath,amssymb}
\begin{document}

\section*{Problem 4}
Broj sabiraka u razvijenom obliku izraza $(a + b + c)^{10}$ jeste:

\subsection*{Solution}
Razvoj $(a + b + c)^{10}$ korišćenjem multinomijalne teoreme daje članove oblika:
\[
\frac{10!}{i! \cdot j! \cdot k!} a^i b^j c^k
\]
gde je $i + j + k = 10$ i $i, j, k \geq 0$.

Broj različitih sabiraka jednak je broju načina da se izaberu nenegativni celi brojevi $i, j, k$ takvi da je $i + j + k = 10$.

Ovo je klasičan kombinatorni problem -- broj načina da se $n$ identičnih objekata rasporedi u $r$ različitih kutija, što je dato formulom:
\[
\binom{n + r - 1}{r - 1} = \binom{10 + 3 - 1}{3 - 1} = \binom{12}{2}
\]

Izračunajmo:
\[
\binom{12}{2} = \frac{12 \cdot 11}{2} = 66
\]

\subsection*{Answer}
$66$ (option \textbf{D}).

\end{document}
