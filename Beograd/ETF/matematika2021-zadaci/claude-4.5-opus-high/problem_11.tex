\documentclass[12pt]{article}
\usepackage[margin=1in]{geometry}
\usepackage{amsmath,amssymb}
\begin{document}

\section*{Problem 11}
Tačke $A(-2, 2)$ i $B(2, -2)$ su temena trougla $ABC$, a $N(1, 2)$ je presek visina tog trougla. Zbir koordinata temena $C$ jednak je:

\subsection*{Solution}
Neka je $C(x_C, y_C)$.

Ortocentar (presek visina) trougla je tačka $N(1, 2)$.

Visina iz $A$ je normalna na stranicu $BC$. Vektor $\overrightarrow{BC} = (x_C - 2, y_C + 2)$.

Visina iz $A$ prolazi kroz $A(-2, 2)$ i $N(1, 2)$. Vektor $\overrightarrow{AN} = (3, 0)$.

Pošto je $\overrightarrow{AN} \perp \overrightarrow{BC}$:
\[
\overrightarrow{AN} \cdot \overrightarrow{BC} = 0
\]
\[
3(x_C - 2) + 0 \cdot (y_C + 2) = 0
\]
\[
x_C = 2
\]

Visina iz $B$ je normalna na stranicu $AC$. Vektor $\overrightarrow{AC} = (x_C + 2, y_C - 2) = (4, y_C - 2)$.

Visina iz $B$ prolazi kroz $B(2, -2)$ i $N(1, 2)$. Vektor $\overrightarrow{BN} = (-1, 4)$.

Pošto je $\overrightarrow{BN} \perp \overrightarrow{AC}$:
\[
\overrightarrow{BN} \cdot \overrightarrow{AC} = 0
\]
\[
(-1) \cdot 4 + 4 \cdot (y_C - 2) = 0
\]
\[
-4 + 4y_C - 8 = 0
\]
\[
4y_C = 12
\]
\[
y_C = 3
\]

Dakle $C(2, 3)$ i zbir koordinata je $2 + 3 = 5$.

\subsection*{Answer}
$5$ (option \textbf{B}).

\end{document}
