\documentclass[12pt]{article}
\usepackage[margin=1in]{geometry}
\usepackage{amsmath,amssymb}
\begin{document}

\section*{Problem 7}
Skup svih realnih resenja nejednacine
\[
\ln(x+1)>x^2+3x+3
\]
je oblika (za neke realne brojeve $a,b$ takve da $-1<a<b<+\infty$):
\[
\text{(A) }(-1,a)\cup(b,+\infty)\quad
\text{(B) }(-1,a)\quad
\text{(C) }(a,b)\quad
\text{(D) }(a,+\infty)\quad
\text{(E) nijedan od ponudjenih.}
\]

\subsection*{Solution}
Definisimo funkciju na domenu $x>-1$:
\[
g(x)=\ln(x+1)-(x^2+3x+3).
\]
Trazimo gde je $g(x)>0$.
Derivat je
\[
g'(x)=\frac{1}{x+1}-(2x+3).
\]
Jednacina $g'(x)=0$ daje
\[
\frac{1}{x+1}=2x+3\ \Longleftrightarrow\ 1=(2x+3)(x+1)=2x^2+5x+3,
\]
odnosno
\[
2x^2+5x+2=0 \ \Longrightarrow\ x=-2,\ -\frac12.
\]
Od toga je u domenu samo $x=-\frac12$.
Dalje,
\[
g''(x)=-\frac{1}{(x+1)^2}-2<0,
\]
pa je $x=-\frac12$ globalni maksimum funkcije $g$ na $(-1,+\infty)$.
Racunamo
\[
g\!\left(-\frac12\right)=\ln\!\left(\frac12\right)-\left(\frac14-\frac32+3\right)
 =-\ln 2-\frac74<0.
\]
Dakle je i maksimum negativan, pa je $g(x)\le g(-\tfrac12)<0$ za sve $x>-1$.
Zato nejednacina nema realnih resenja.

\subsection*{Answer}
Prazan skup (option \textbf{E}).

\end{document}

