\documentclass[12pt]{article}
\usepackage[margin=1in]{geometry}
\usepackage{amsmath,amssymb}
\begin{document}

\section*{Problem 8}
Koliko jednakih clanova imaju aritmeticke progresije
$2,7,12,17,\dots$ i $2,5,8,11,\dots$ ako svaka od njih ima $121$ clan?

\subsection*{Solution}
Prva progresija ima razliku $5$, pa je
\[
a_n=2+5(n-1)=5n-3,\qquad n=1,2,\dots,121.
\]
Druga progresija ima razliku $3$, pa je
\[
b_m=2+3(m-1)=3m-1,\qquad m=1,2,\dots,121.
\]
Zajednicki clanovi zadovoljavaju
\[
5n-3=3m-1 \ \Longleftrightarrow\ 5n=3m+2.
\]
Iz uslova deljivosti po $3$ dobijamo $5n\equiv 2\pmod 3$, tj.\ $2n\equiv 2\pmod 3$, pa je
\[
n\equiv 1\pmod 3 \ \Longrightarrow\ n=3k+1.
\]
Tada
\[
m=\frac{5n-2}{3}=\frac{5(3k+1)-2}{3}=\frac{15k+3}{3}=5k+1.
\]
Ogranicenja $1\le n\le 121$ daju $3k+1\le 121\Rightarrow k\le 40$, dok $1\le m\le 121$
daje $5k+1\le 121\Rightarrow k\le 24$.
Dakle $k=0,1,2,\dots,24$, ukupno $25$ vrednosti.

\subsection*{Answer}
$25$ (option \textbf{D}).

\end{document}

