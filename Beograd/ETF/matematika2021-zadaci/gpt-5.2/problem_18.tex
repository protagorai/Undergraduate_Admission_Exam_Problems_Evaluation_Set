\documentclass[12pt]{article}
\usepackage[margin=1in]{geometry}
\usepackage{amsmath,amssymb}
\begin{document}

\section*{Problem 18}
Ostatak pri deljenju polinoma
\[
P(x)=x^{2024}+x^{2023}+x^{2022}+x^{2021}+x^{20}+1
\]
polinomom
\[
Q(x)=x^3+x^2+x+1
\]
jeste:
(A) $x^2+1$ \ (B) $2$ \ (C) $0$ \ (D) $1$ \ (E) $x^2-1$.

\subsection*{Solution}
Primecujemo
\[
Q(x)=x^3+x^2+x+1=(x+1)(x^2+1).
\]
Dalje,
\[
x^4-1=(x^2-1)(x^2+1)=(x-1)(x+1)(x^2+1),
\]
pa $Q(x)\mid (x^4-1)$, odnosno u aritmetici po modulu $Q$ vazi
\[
x^4\equiv 1 \pmod{Q(x)}.
\]
Zato redukujemo stepene po modulu $4$:
\[
x^{2024}\equiv x^0=1,\quad
x^{2023}\equiv x^3,\quad
x^{2022}\equiv x^2,\quad
x^{2021}\equiv x,\quad
x^{20}\equiv x^0=1.
\]
Otuda
\[
P(x)\equiv 1+x^3+x^2+x+1+1 = x^3+x^2+x+3.
\]
Ali $x^3+x^2+x+1\equiv 0$, pa je $x^3+x^2+x\equiv -1$, te
\[
P(x)\equiv (-1)+3=2.
\]
Dakle, ostatak je konstanta $2$.

\subsection*{Answer}
$2$ (option \textbf{B}).

\end{document}

