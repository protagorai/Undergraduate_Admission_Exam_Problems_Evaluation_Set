\documentclass[12pt]{article}
\usepackage[margin=1in]{geometry}
\usepackage{amsmath,amssymb}
\begin{document}

\section*{Problem 2}
Koje od sledecih nejednakosti su tacne?
\begin{align*}
&(i)\ \sqrt[3]{3}<\sqrt2, \\
&(ii)\ \sqrt[3]{3+\sqrt3}<\sqrt{2+\sqrt2}, \\
&(iii)\ 3\sqrt5<5\sqrt3, \\
&(iv)\ \sqrt{12+3\sqrt{75}}<2\sqrt{48}+5\sqrt3, \\
&(v)\ \sqrt{9+4\sqrt5}<\sqrt{19-8\sqrt3}.
\end{align*}

\subsection*{Solution}
\textbf{(i)} Obe strane su pozitivne, pa mozemo povisiti na stepen $6$:
\[
(\sqrt[3]{3})^6=3^2=9,\qquad (\sqrt2)^6=2^3=8,
\]
pa je $9<8$ netacno. Dakle, (i) je \emph{netacna}.

\medskip
\textbf{(ii)} Povisimo na stepen $6$:
\[
(\sqrt[3]{3+\sqrt3})^6=(3+\sqrt3)^2=12+6\sqrt3,
\]
\[
(\sqrt{2+\sqrt2})^6=(2+\sqrt2)^3
=(2+\sqrt2)(6+4\sqrt2)=20+14\sqrt2.
\]
Posto je $12+6\sqrt3<20+14\sqrt2$ (npr.\ numericki), (ii) je \emph{tacna}.

\medskip
\textbf{(iii)} Kvadriranjem dobijamo
\[
(3\sqrt5)^2=45<75=(5\sqrt3)^2,
\]
pa je (iii) \emph{tacna}.

\medskip
\textbf{(iv)} Prvo $\sqrt{75}=5\sqrt3$, pa je
\[
\sqrt{12+3\sqrt{75}}=\sqrt{12+15\sqrt3}>0.
\]
Desna strana je
\[
2\sqrt{48}+5\sqrt3=2\cdot4\sqrt3+5\sqrt3=13\sqrt3.
\]
Kako je $12+15\sqrt3< (13\sqrt3)^2=507$, sledi
\[
\sqrt{12+15\sqrt3}<13\sqrt3,
\]
pa je (iv) \emph{tacna}.

\medskip
\textbf{(v)} Uporedimo kvadrate:
\[
9+4\sqrt5 \stackrel{?}{<} 19-8\sqrt3
\iff 4\sqrt5+8\sqrt3<10,
\]
sto je netacno (leva strana je znatno veca od $10$). Dakle, (v) je \emph{netacna}.

\medskip
Tacne su (ii), (iii) i (iv).

\subsection*{Answer}
(ii), (iii) i (iv) (option \textbf{C}).

\end{document}

