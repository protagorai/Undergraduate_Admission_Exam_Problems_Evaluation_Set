\documentclass[12pt]{article}
\usepackage[margin=1in]{geometry}
\usepackage{amsmath,amssymb}
\begin{document}

\section*{Problem 11}
Tacke $A(-2,2)$ i $B(2,-2)$ su temena trougla $ABC$, a $N(1,2)$ je presek visina tog trougla.
Zbir koordinata temena $C$ jednak je:
(A) 3 \ (B) 5 \ (C) 7 \ (D) 9 \ (E) 11.

\subsection*{Solution}
Neka je $C=(x,y)$ i neka je $N=(1,2)$ ortocentar.
Vazi:
\begin{itemize}
\item visina iz $A$ je normalna na $BC$, pa je $(B-C)\cdot(N-A)=0$,
\item visina iz $B$ je normalna na $AC$, pa je $(A-C)\cdot(N-B)=0$.
\end{itemize}
Racunamo
\[
N-A=(1+2,\ 2-2)=(3,0),\qquad B-C=(2-x,\ -2-y).
\]
Prvi uslov:
\[
(B-C)\cdot(N-A)=3(2-x)+0(-2-y)=0 \ \Longrightarrow\ x=2.
\]
Dalje,
\[
N-B=(1-2,\ 2+2)=(-1,4),\qquad A-C=(-2-2,\ 2-y)=(-4,2-y).
\]
Drugi uslov:
\[
(A-C)\cdot(N-B)=(-4)(-1)+(2-y)\cdot 4=4+8-4y=0 \ \Longrightarrow\ y=3.
\]
Dakle $C=(2,3)$, pa je zbir koordinata $2+3=5$.

\subsection*{Answer}
$5$ (option \textbf{B}).

\end{document}

