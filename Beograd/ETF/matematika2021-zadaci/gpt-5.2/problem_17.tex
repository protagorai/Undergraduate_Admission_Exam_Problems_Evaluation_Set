\documentclass[12pt]{article}
\usepackage[margin=1in]{geometry}
\usepackage{amsmath,amssymb}
\begin{document}

\section*{Problem 17}
Dvocifreni broj koji je jednak proizvodu zbira svojih cifara i apsolutne vrednosti razlike
pripada intervalu:
\[
\text{(A) }[10,30]\quad
\text{(B) }[31,50]\quad
\text{(C) }[51,70]\quad
\text{(D) }[71,80]\quad
\text{(E) }[81,99].
\]

\subsection*{Solution}
Neka je dvocifreni broj $10a+b$, gde su $a\in\{1,\dots,9\}$ i $b\in\{0,\dots,9\}$.
Uslov zadatka je
\[
10a+b=(a+b)\,|a-b|.
\]
Razmatramo slucajeve.

\medskip
\textbf{1) $a\ge b$.} Tada je $|a-b|=a-b$ i
\[
10a+b=(a+b)(a-b)=a^2-b^2.
\]
Prebacivanjem dobijamo
\[
b^2+b=a^2-10a=a(a-10).
\]
Leva strana je $\ge 0$, dok je desna strana negativna za $a\in\{1,\dots,9\}$, sto je nemoguce.
Dakle, ovaj slucaj otpada.

\medskip
\textbf{2) $b>a$.} Tada je $|a-b|=b-a$ i
\[
10a+b=(a+b)(b-a)=b^2-a^2.
\]
Odavde
\[
b^2-b=a^2+10a=a(a+10).
\]
Sada ispitamo moguce vrednosti $a(a+10)$:
\[
\begin{array}{c|ccccccccc}
a&1&2&3&4&5&6&7&8&9\\ \hline
a(a+10)&11&24&39&56&75&96&119&144&171
\end{array}
\]
Dok su moguce vrednosti $b^2-b=b(b-1)$ za $b\in\{0,\dots,9\}$:
\[
0,0,2,6,12,20,30,42,56,72.
\]
Jedino poklapanje je $56$, pa je $a=4$ i $b=8$.
Dakle, broj je $48$, i on pripada intervalu $[31,50]$.

\subsection*{Answer}
$48\in[31,50]$ (option \textbf{B}).

\end{document}

