\documentclass[12pt]{article}
\usepackage[margin=1in]{geometry}
\usepackage{amsmath,amssymb}
\begin{document}

\section*{Problem 12}
Dat je konveksan cetvorougao $ABCD$ u kojem je $\angle ABD=50^\circ$, $\angle ADB=80^\circ$,
$\angle ACB=40^\circ$ i $\angle DBC=\angle BDC+30^\circ$.
Tada je $\angle DBC$ jednak:
(A) $40^\circ$ \ (B) $45^\circ$ \ (C) $55^\circ$ \ (D) $65^\circ$ \ (E) $70^\circ$.

\subsection*{Solution}
Oznacimo
\[
x=\angle DBC,\qquad y=\angle BDC,
\]
pa je dato $x=y+30^\circ$.

U trouglu $ABD$ znamo $\angle ABD=50^\circ$ i $\angle ADB=80^\circ$, pa je
\[
\angle BAD=180^\circ-50^\circ-80^\circ=50^\circ.
\]
Posmatrajmo sada uglove cetvorougla kod temena $B$ i $D$.
Kako je dijagonala $BD$ unutar konveksnog cetvorougla, vazi
\[
\angle ABC=\angle ABD+\angle DBC=50^\circ+x,
\]
\[
\angle ADC=\angle ADB+\angle BDC=80^\circ+y.
\]
Koristeci $x=y+30^\circ$ dobijamo
\[
\angle ADC=80^\circ+y=80^\circ+(x-30^\circ)=50^\circ+x=\angle ABC.
\]
Dakle, uglovi $\angle ABC$ i $\angle ADC$ su jednaki i obe ``vide'' duz $AC$.
To znaci da tacke $A,B,C,D$ leze na istoj kruznici (cetvorougao je ciklican).

U ciklicnom cetvorouglu su suprotni uglovi suplementni, pa je
\[
\angle ABC+\angle ADC=180^\circ.
\]
Ali $\angle ABC=\angle ADC$, pa mora biti
\[
2\angle ABC=180^\circ \ \Longrightarrow\ \angle ABC=90^\circ.
\]
Zato
\[
50^\circ+x=\angle ABC=90^\circ \ \Longrightarrow\ x=40^\circ.
\]

\subsection*{Answer}
$40^\circ$ (option \textbf{A}).

\end{document}

