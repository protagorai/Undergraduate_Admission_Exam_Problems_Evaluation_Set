\documentclass[12pt]{article}
\usepackage[margin=1in]{geometry}
\usepackage{amsmath,amssymb}
\begin{document}

\section*{Problem 9}
Vrednost izraza
\[
\cos\frac{\pi}{7}-\cos\frac{2\pi}{7}+\cos\frac{3\pi}{7}
\]
jeste:
(A) $1$ \ (B) $\frac12$ \ (C) $-\frac12$ \ (D) $-\frac14$ \ (E) $0$.

\subsection*{Solution}
Oznacimo $\theta=\frac{\pi}{7}$. Tada je izraz
\[
S=\cos\theta-\cos 2\theta+\cos 3\theta.
\]
Uvedimo kompleksan broj
\[
\omega=e^{i\theta}=e^{i\pi/7}.
\]
Tada je $\omega^7=e^{i\pi}=-1$, pa su resenja jednacine $z^7=-1$ upravo
\[
\omega^1,\ \omega^3,\ \omega^5,\ \omega^7,\ \omega^9,\ \omega^{11},\ \omega^{13}.
\]
To su koreni polinoma $z^7+1$, pa je njihov zbir jednak $0$ (jer je koeficijent uz $z^6$ jednak $0$):
\[
\omega+\omega^3+\omega^5+\omega^7+\omega^9+\omega^{11}+\omega^{13}=0.
\]
Uparimo kompleksno konjugovane clanove (jer je $\overline{\omega^k}=\omega^{-k}=\omega^{14-k}$):
\[
(\omega+\omega^{13})+(\omega^3+\omega^{11})+(\omega^5+\omega^9)+\omega^7=0.
\]
Kako je $\omega^7=-1$ i $\omega^k+\omega^{-k}=2\cos(k\theta)$, dobijamo
\[
2\cos\theta+2\cos 3\theta+2\cos 5\theta-1=0
\ \Longrightarrow\ 
\cos\theta+\cos 3\theta+\cos 5\theta=\frac12.
\]
Ali je $5\theta=\frac{5\pi}{7}=\pi-\frac{2\pi}{7}$, pa je
\[
\cos 5\theta=\cos\!\left(\pi-2\theta\right)=-\cos 2\theta.
\]
Zato
\[
S=\cos\theta-\cos 2\theta+\cos 3\theta
=\cos\theta+\cos 3\theta+\cos 5\theta
=\frac12.
\]

\subsection*{Answer}
$\dfrac12$ (option \textbf{B}).

\end{document}

