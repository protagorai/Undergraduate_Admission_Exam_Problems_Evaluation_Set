\documentclass[12pt]{article}
\usepackage[margin=1in]{geometry}
\usepackage{amsmath,amssymb}
\begin{document}

\section*{Problem 8}
How many equal terms do the arithmetic progressions 2, 7, 12, 17, ... and 2, 5, 8, 11, ... have, if each progression has 121 terms?

\subsection*{Solution}
First progression: $a_n = 2 + (n-1) \cdot 5 = 5n - 3$ for $n = 1, 2, ..., 121$
Second progression: $b_m = 2 + (m-1) \cdot 3 = 3m - 1$ for $m = 1, 2, ..., 121$

We need to find when $a_n = b_m$:
\[
5n - 3 = 3m - 1
\]
\[
5n - 2 = 3m
\]
\[
m = \frac{5n - 2}{3}
\]

For $m$ to be a positive integer, we need $5n - 2 \equiv 0 \pmod{3}$.
Since $5 \equiv 2 \pmod{3}$, we have:
\[
2n - 2 \equiv 0 \pmod{3}
\]
\[
2n \equiv 2 \pmod{3}
\]
\[
n \equiv 1 \pmod{3}
\]

So $n = 1, 4, 7, 10, 13, ...$, i.e., $n = 3k + 1$ for $k = 0, 1, 2, ...$

We need $1 \leq n \leq 121$ and $1 \leq m \leq 121$.

From $n = 3k + 1 \leq 121$: $k \leq 40$
From $m = \frac{5n - 2}{3} = \frac{5(3k + 1) - 2}{3} = \frac{15k + 3}{3} = 5k + 1 \leq 121$: $k \leq 24$

Therefore, $k$ can take values $0, 1, 2, ..., 24$, giving us 25 equal terms.

\subsection*{Answer}
25 (option \textbf{D}).

\end{document}