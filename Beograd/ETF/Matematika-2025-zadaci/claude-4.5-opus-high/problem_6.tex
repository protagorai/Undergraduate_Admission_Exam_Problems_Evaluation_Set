\documentclass[12pt]{article}
\usepackage[margin=1in]{geometry}
\usepackage{amsmath,amssymb}
\begin{document}

\section*{Problem 6}
Ako je duži krak pravouglog trapeza jednak 10, a poluprečnik kružnice upisane u taj trapez jednak 4, onda je površina tog trapeza jednaka:

\subsection*{Solution}
Neka je pravougli trapez sa osnovicama $a$ (donja) i $b$ (gornja), visinom $h$, i dužim krakom (kosim) dužine 10.

Za upisanu kružnicu poluprečnika $r = 4$, važi formula:
\[
r = \frac{P}{s}
\]
gde je $P$ površina trapeza, a $s$ poluobim.

Za pravougli trapez, visina je jednaka kraćem kraku. Pošto je kružnica upisana, poluprečnik je jednak visini podeljeno sa 2? Ne, to nije tačno.

Za trapez sa upisanom kružnicom važi:
\[
a + b = c + d
\]
gde su $c$ i $d$ kraci trapeza.

Za pravougli trapez, jedan krak je jednak visini $h$, a drugi (kosi) krak je 10.

Dakle: $a + b = h + 10$

Takođe, za upisanu kružnicu u trapez:
\[
r = \frac{h}{2} \cdot \frac{2}{1} = h \cdot \frac{a - b}{a + b - c - d}
\]

Zapravo, za trapez sa upisanom kružnicom, poluprečnik je:
\[
r = \frac{h}{2}
\]
ako i samo ako je trapez tangencijalni (što jeste, jer ima upisanu kružnicu).

Dakle:
\[
4 = \frac{h}{2} \Rightarrow h = 8
\]

Sada, za pravougli trapez, kosi krak zadovoljava:
\[
\text{kosi krak}^2 = h^2 + (a - b)^2
\]
\[
10^2 = 8^2 + (a - b)^2
\]
\[
100 = 64 + (a - b)^2
\]
\[
(a - b)^2 = 36
\]
\[
a - b = 6
\]

Iz uslova za tangencijalni trapez:
\[
a + b = h + 10 = 8 + 10 = 18
\]

Rešavamo sistem:
\[
a + b = 18
\]
\[
a - b = 6
\]

Sabiranjem: $2a = 24 \Rightarrow a = 12$

Oduzimanjem: $2b = 12 \Rightarrow b = 6$

Površina trapeza:
\[
P = \frac{(a + b) \cdot h}{2} = \frac{18 \cdot 8}{2} = \frac{144}{2} = 72
\]

\subsection*{Answer}
$72$ (opcija \textbf{E}).

\end{document}
