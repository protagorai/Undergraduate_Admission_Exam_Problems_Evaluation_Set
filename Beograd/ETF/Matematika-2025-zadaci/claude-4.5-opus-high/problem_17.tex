\documentclass[12pt]{article}
\usepackage[margin=1in]{geometry}
\usepackage{amsmath,amssymb}
\begin{document}

\section*{Problem 17}
Ako polinom $P_n(x)$ ($n \geq 3$) pri deljenju sa $x$ daje ostatak 3, a pri deljenju sa $x^2 + 1$ daje ostatak $-x + 2$, onda polinom $(P_n(x))^2 + P_n(x) + 1$ pri deljenju sa $x^3 + x$ daje ostatak:

\subsection*{Solution}
Iz uslova zadatka:
\begin{itemize}
    \item $P_n(0) = 3$ (ostatak pri deljenju sa $x$)
    \item $P_n(x) \equiv -x + 2 \pmod{x^2 + 1}$
\end{itemize}

Iz drugog uslova, za $x = i$ (gde je $i^2 = -1$):
\[
P_n(i) = -i + 2
\]
\[
P_n(-i) = i + 2
\]

Treba naći ostatak pri deljenju $(P_n(x))^2 + P_n(x) + 1$ sa $x^3 + x = x(x^2 + 1)$.

Ostatak će biti polinom stepena najviše 2: $R(x) = ax^2 + bx + c$.

Moramo imati:
\[
(P_n(x))^2 + P_n(x) + 1 \equiv ax^2 + bx + c \pmod{x(x^2 + 1)}
\]

Ovo znači:
\begin{enumerate}
    \item Za $x = 0$: $(P_n(0))^2 + P_n(0) + 1 = c$
    \item Za $x = i$: $(P_n(i))^2 + P_n(i) + 1 = ai^2 + bi + c = -a + bi + c$
    \item Za $x = -i$: $(P_n(-i))^2 + P_n(-i) + 1 = a(-i)^2 + b(-i) + c = -a - bi + c$
\end{enumerate}

Računamo:

\textbf{Za $x = 0$:}
\[
c = 3^2 + 3 + 1 = 9 + 3 + 1 = 13
\]

\textbf{Za $x = i$:}
\[
P_n(i) = -i + 2
\]
\[
(P_n(i))^2 = (-i + 2)^2 = i^2 - 4i + 4 = -1 - 4i + 4 = 3 - 4i
\]
\[
(P_n(i))^2 + P_n(i) + 1 = (3 - 4i) + (-i + 2) + 1 = 6 - 5i
\]

Dakle: $-a + bi + c = 6 - 5i$

Iz realnog dela: $-a + c = 6 \Rightarrow -a + 13 = 6 \Rightarrow a = 7$

Iz imaginarnog dela: $b = -5$

\textbf{Provera za $x = -i$:}
\[
P_n(-i) = i + 2
\]
\[
(P_n(-i))^2 = (i + 2)^2 = i^2 + 4i + 4 = -1 + 4i + 4 = 3 + 4i
\]
\[
(P_n(-i))^2 + P_n(-i) + 1 = (3 + 4i) + (i + 2) + 1 = 6 + 5i
\]

Proveravamo: $-a - bi + c = -7 - (-5)i + 13 = 6 + 5i$ ✓

Dakle, ostatak je:
\[
R(x) = 7x^2 - 5x + 13
\]

\subsection*{Answer}
$7x^2 - 5x + 13$ (opcija \textbf{A}).

\end{document}
