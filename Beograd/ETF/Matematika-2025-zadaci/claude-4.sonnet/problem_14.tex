\documentclass[12pt]{article}
\usepackage[margin=1in]{geometry}
\usepackage{amsmath,amssymb}
\usepackage[utf8]{inputenc}
\begin{document}

\section*{Problem 14}
Find the angle between the ellipse $\frac{x^2}{9} + \frac{y^2}{4} = 1$ and the hyperbola $\frac{x^2}{4} - y^2 = 1$ at their intersection point in the first quadrant.

\subsection*{Solution}
First, let's find the intersection points by solving the system:
\begin{align}
\frac{x^2}{9} + \frac{y^2}{4} &= 1 \\
\frac{x^2}{4} - y^2 &= 1
\end{align}

From equation (2): $y^2 = \frac{x^2}{4} - 1$

Substituting into equation (1):
\[
\frac{x^2}{9} + \frac{\frac{x^2}{4} - 1}{4} = 1
\]
\[
\frac{x^2}{9} + \frac{x^2}{16} - \frac{1}{4} = 1
\]
\[
\frac{x^2}{9} + \frac{x^2}{16} = \frac{5}{4}
\]

Finding a common denominator:
\[
\frac{16x^2 + 9x^2}{144} = \frac{5}{4}
\]
\[
\frac{25x^2}{144} = \frac{5}{4}
\]
\[
25x^2 = \frac{5 \cdot 144}{4} = 180
\]
\[
x^2 = \frac{180}{25} = \frac{36}{5}
\]
\[
x = \frac{6}{\sqrt{5}} = \frac{6\sqrt{5}}{5}
\]

From $y^2 = \frac{x^2}{4} - 1$:
\[
y^2 = \frac{36/5}{4} - 1 = \frac{9}{5} - 1 = \frac{4}{5}
\]
\[
y = \frac{2}{\sqrt{5}} = \frac{2\sqrt{5}}{5}
\]

So the intersection point in the first quadrant is $\left(\frac{6\sqrt{5}}{5}, \frac{2\sqrt{5}}{5}\right)$.

Now we need to find the slopes of the tangent lines to both curves at this point.

For the ellipse $\frac{x^2}{9} + \frac{y^2}{4} = 1$, implicit differentiation gives:
\[
\frac{2x}{9} + \frac{2y}{4}\frac{dy}{dx} = 0
\]
\[
\frac{dy}{dx} = -\frac{4x}{9y}
\]

At the intersection point:
\[
m_1 = -\frac{4 \cdot \frac{6\sqrt{5}}{5}}{9 \cdot \frac{2\sqrt{5}}{5}} = -\frac{\frac{24\sqrt{5}}{5}}{\frac{18\sqrt{5}}{5}} = -\frac{24}{18} = -\frac{4}{3}
\]

For the hyperbola $\frac{x^2}{4} - y^2 = 1$, implicit differentiation gives:
\[
\frac{2x}{4} - 2y\frac{dy}{dx} = 0
\]
\[
\frac{dy}{dx} = \frac{x}{4y}
\]

At the intersection point:
\[
m_2 = \frac{\frac{6\sqrt{5}}{5}}{4 \cdot \frac{2\sqrt{5}}{5}} = \frac{\frac{6\sqrt{5}}{5}}{\frac{8\sqrt{5}}{5}} = \frac{6}{8} = \frac{3}{4}
\]

The angle $\theta$ between two lines with slopes $m_1$ and $m_2$ is given by:
\[
\tan \theta = \left|\frac{m_1 - m_2}{1 + m_1 m_2}\right|
\]

\[
\tan \theta = \left|\frac{-\frac{4}{3} - \frac{3}{4}}{1 + \left(-\frac{4}{3}\right)\left(\frac{3}{4}\right)}\right| = \left|\frac{-\frac{16 + 9}{12}}{1 - 1}\right| = \left|\frac{-\frac{25}{12}}{0}\right|
\]

Since the denominator is 0, this means the curves are perpendicular, so $\theta = 90°$.

\subsection*{Answer}
$90°$ (option \textbf{D}).

\end{document}

