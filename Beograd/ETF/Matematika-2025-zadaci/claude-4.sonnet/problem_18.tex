\documentclass[12pt]{article}
\usepackage[margin=1in]{geometry}
\usepackage{amsmath,amssymb}
\usepackage[utf8]{inputenc}
\begin{document}

\section*{Problem 18}
Find the set of all real solutions to the inequality $(2\sin x)^{\sin x - \sin 2x} \geq 1$ on the interval $(0, \pi)$.

\subsection*{Solution}
For the expression $(2\sin x)^{\sin x - \sin 2x} \geq 1$ to make sense, we need $2\sin x > 0$, which means $\sin x > 0$. Since we're on the interval $(0, \pi)$, this condition is satisfied.

Now, let's analyze when $(2\sin x)^{\sin x - \sin 2x} \geq 1$.

We know that $\sin 2x = 2\sin x \cos x$, so:
\[
\sin x - \sin 2x = \sin x - 2\sin x \cos x = \sin x(1 - 2\cos x)
\]

So our inequality becomes:
\[
(2\sin x)^{\sin x(1 - 2\cos x)} \geq 1
\]

For $a^b \geq 1$ where $a > 0$, we have the following cases:
1. If $a \geq 1$ and $b \geq 0$, then $a^b \geq 1$
2. If $a \geq 1$ and $b < 0$, then $a^b \leq 1$, so we need $a^b = 1$ (i.e., $a = 1$ or $b = 0$)
3. If $0 < a < 1$ and $b \leq 0$, then $a^b \geq 1$
4. If $0 < a < 1$ and $b > 0$, then $a^b < 1$

Let's consider when $2\sin x \geq 1$ and when $2\sin x < 1$:

\textbf{Case 1:} $2\sin x \geq 1$, i.e., $\sin x \geq \frac{1}{2}$
This occurs when $x \in \left[\frac{\pi}{6}, \frac{5\pi}{6}\right]$.

For the inequality to hold, we need $\sin x(1 - 2\cos x) \geq 0$.
Since $\sin x > 0$ in $(0, \pi)$, we need $1 - 2\cos x \geq 0$, i.e., $\cos x \leq \frac{1}{2}$.
This occurs when $x \in \left[\frac{\pi}{3}, \frac{2\pi}{3}\right]$.

The intersection is $x \in \left[\frac{\pi}{3}, \frac{2\pi}{3}\right]$.

\textbf{Case 2:} $2\sin x < 1$, i.e., $\sin x < \frac{1}{2}$
This occurs when $x \in \left(0, \frac{\pi}{6}\right) \cup \left(\frac{5\pi}{6}, \pi\right)$.

For the inequality to hold, we need $\sin x(1 - 2\cos x) \leq 0$.
Since $\sin x > 0$, we need $1 - 2\cos x \leq 0$, i.e., $\cos x \geq \frac{1}{2}$.
This occurs when $x \in \left[0, \frac{\pi}{3}\right] \cup \left[\frac{2\pi}{3}, \pi\right]$.

The intersection is $x \in \left(0, \frac{\pi}{6}\right] \cup \left[\frac{5\pi}{6}, \pi\right)$.

Wait, let me be more careful about the boundary points and the open interval $(0, \pi)$.

Actually, we also need to consider the case where the exponent is zero:
$\sin x(1 - 2\cos x) = 0$

This happens when:
1. $\sin x = 0$ (but this doesn't occur in $(0, \pi)$ except at the boundaries)
2. $1 - 2\cos x = 0$, i.e., $\cos x = \frac{1}{2}$, which gives $x = \frac{\pi}{3}$ or $x = \frac{5\pi}{3}$

Since we're in $(0, \pi)$, we have $x = \frac{\pi}{3}$.

Combining all cases, the solution set is:
\[
\left(0, \frac{\pi}{6}\right] \cup \left[\frac{\pi}{3}, \frac{2\pi}{3}\right] \cup \left[\frac{5\pi}{6}, \pi\right)
\]

Looking at the answer choices, this corresponds to option (D).

\subsection*{Answer}
$\left(0, \frac{\pi}{6}\right] \cup \left[\frac{\pi}{3}, \frac{5\pi}{6}\right]$ (option \textbf{D}).

\end{document}

