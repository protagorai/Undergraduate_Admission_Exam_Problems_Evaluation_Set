\documentclass[12pt]{article}
\usepackage[margin=1in]{geometry}
\usepackage{amsmath,amssymb}
\usepackage[utf8]{inputenc}
\begin{document}

\section*{Problem 6}
If the long side of a right trapezoid is 10, and the radius of the inscribed circle is 4, find the area of the trapezoid.

\subsection*{Solution}
Let's denote the right trapezoid with parallel sides $a$ and $b$ (where $a > b$), height $h$, and the non-parallel side of length 10.

For a trapezoid with an inscribed circle:
1. The sum of opposite sides must be equal: $a + b = h + 10$
2. The radius of the inscribed circle is $r = \frac{\text{Area}}{\text{semi-perimeter}} = \frac{A}{s}$

Since it's a right trapezoid, the height $h$ is also the length of one of the non-parallel sides.

The area of the trapezoid is:
\[
A = \frac{1}{2}(a + b) \cdot h
\]

The semi-perimeter is:
\[
s = \frac{a + b + h + 10}{2}
\]

Using the inscribed circle property:
\[
r = \frac{A}{s} \Rightarrow 4 = \frac{\frac{1}{2}(a + b) \cdot h}{\frac{a + b + h + 10}{2}}
\]

Simplifying:
\[
4 = \frac{(a + b) \cdot h}{a + b + h + 10}
\]

From the property of tangential quadrilaterals: $a + b = h + 10$.

Substituting:
\[
4 = \frac{(h + 10) \cdot h}{(h + 10) + h + 10} = \frac{h(h + 10)}{2h + 20} = \frac{h(h + 10)}{2(h + 10)} = \frac{h}{2}
\]

Therefore: $h = 8$.

Now we can find $a + b$:
\[
a + b = h + 10 = 8 + 10 = 18
\]

The area is:
\[
A = \frac{1}{2}(a + b) \cdot h = \frac{1}{2} \cdot 18 \cdot 8 = 72
\]

\subsection*{Answer}
$72$ (option \textbf{E}).

\end{document}

