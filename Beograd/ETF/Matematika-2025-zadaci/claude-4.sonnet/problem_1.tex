\documentclass[12pt]{article}
\usepackage[margin=1in]{geometry}
\usepackage{amsmath,amssymb}
\usepackage[utf8]{inputenc}
\begin{document}

\section*{Problem 1}
If $x = 2^{p/q}$ solves the equation $\sqrt{x\sqrt{x\sqrt{x\sqrt{x}}}} = 2$, find $p + q$ for some natural numbers $p$ and $q$.

\subsection*{Solution}
Let's denote the nested radical as $y = \sqrt{x\sqrt{x\sqrt{x\sqrt{x}}}}$.

We can write this as:
\[
y = \sqrt{x \cdot \sqrt{x \cdot \sqrt{x \cdot \sqrt{x}}}}
\]

Let's work from the inside out. If we assume the pattern continues infinitely, we get:
\[
y = \sqrt{x \cdot y}
\]

Squaring both sides:
\[
y^2 = x \cdot y
\]

Since $y \neq 0$ (as $y = 2$), we can divide by $y$:
\[
y = x
\]

But we're given that $y = 2$, so $x = 2$.

Now, if $x = 2^{p/q} = 2$, then:
\[
2^{p/q} = 2^1
\]

This means $\frac{p}{q} = 1$, so $p = q$.

The simplest natural numbers satisfying this are $p = 1$ and $q = 1$.

Therefore, $p + q = 1 + 1 = 2$.

Let's verify: If $x = 2$, then $\sqrt{x\sqrt{x\sqrt{x\sqrt{x}}}} = \sqrt{2\sqrt{2\sqrt{2\sqrt{2}}}}$.

Working from inside: $\sqrt{2} \approx 1.414$, $\sqrt{2\sqrt{2}} \approx 1.682$, $\sqrt{2\sqrt{2\sqrt{2}}} \approx 1.834$, $\sqrt{2\sqrt{2\sqrt{2\sqrt{2}}}} \approx 1.915$.

Actually, let me reconsider this more carefully. The expression has exactly 4 nested radicals, not infinite.

Let $u = \sqrt{x}$, so $x = u^2$.
Then: $\sqrt{x\sqrt{x\sqrt{x\sqrt{x}}}} = \sqrt{u^2\sqrt{u^2\sqrt{u^2\sqrt{u^2}}}} = \sqrt{u^2\sqrt{u^2\sqrt{u^2 \cdot u}}} = \sqrt{u^2\sqrt{u^2\sqrt{u^3}}}$

Let me try a different approach. Since $x = 2^{p/q}$, let's substitute and solve:
$\sqrt{x\sqrt{x\sqrt{x\sqrt{x}}}} = 2$

Let $x = 2^t$ where $t = p/q$. Then:
$\sqrt{2^t\sqrt{2^t\sqrt{2^t\sqrt{2^t}}}} = 2$

Working from inside:
- $\sqrt{2^t} = 2^{t/2}$
- $\sqrt{2^t \cdot 2^{t/2}} = \sqrt{2^{t + t/2}} = \sqrt{2^{3t/2}} = 2^{3t/4}$
- $\sqrt{2^t \cdot 2^{3t/4}} = \sqrt{2^{t + 3t/4}} = \sqrt{2^{7t/4}} = 2^{7t/8}$
- $\sqrt{2^t \cdot 2^{7t/8}} = \sqrt{2^{t + 7t/8}} = \sqrt{2^{15t/8}} = 2^{15t/16}$

So we need: $2^{15t/16} = 2^1$

Therefore: $\frac{15t}{16} = 1$, which gives $t = \frac{16}{15}$.

Since $t = \frac{p}{q} = \frac{16}{15}$, we have $p = 16$ and $q = 15$.

Therefore, $p + q = 16 + 15 = 31$.

\subsection*{Answer}
$31$ (option \textbf{N} - not among the given options, but this is the correct mathematical result).

\end{document}

