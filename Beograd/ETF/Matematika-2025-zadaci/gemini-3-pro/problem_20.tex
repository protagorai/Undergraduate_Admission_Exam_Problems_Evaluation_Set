\documentclass[12pt]{article}
\usepackage[margin=1in]{geometry}
\usepackage{amsmath,amssymb}
\begin{document}

\section*{Problem 20}
Skup vrednosti realnog parametra $p$ za koje jednačina $|x-p| + |x-1| = 1$ ima tačno dva realna i različita rešenja je oblika (za neke $a,b,c,d \in R$ takve da je $-\infty < a < b < c < d < +\infty$):

\subsection*{Solution}

Let $f(x) = |x-p| + |x-1|$.
The minimum value of $f(x)$ is $|p-1|$.
If $|p-1| > 1$, no solution.
If $|p-1| = 1$, infinite solutions (the segment between $p$ and 1).
If $|p-1| < 1$, min value is $< 1$. The function has V-shape characteristics outside the interval $[p,1]$ (or $[1,p]$). The value 1 is strictly greater than the minimum $|p-1|$. Since $f(x) \to \infty$ as $x \to \pm \infty$, the line $y=1$ intersects the graph at exactly two points.
So we need $|p-1| < 1 \implies -1 < p-1 < 1 \implies 0 < p < 2$.
However, we must ensure the solutions are distinct. If min value $< 1$, the intersection happens on the sloped parts, so solutions are distinct.
The set is $(0, 2)$.
This corresponds to interval $(a, b)$.


\subsection*{Answer}
(a, b) (option \textbf{B}).

\end{document}
