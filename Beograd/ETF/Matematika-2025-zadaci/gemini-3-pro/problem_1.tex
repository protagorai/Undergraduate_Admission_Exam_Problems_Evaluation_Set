\documentclass[12pt]{article}
\usepackage[margin=1in]{geometry}
\usepackage{amsmath,amssymb}
\begin{document}

\section*{Problem 1}
Ako je $x = 2^{p/q}$ rešenje jednačine $\sqrt{x\sqrt{x\sqrt{x\sqrt{x\sqrt{x}}}}} = 2$, za neke uzajamno proste prirodne brojeve $p$ i $q$, tada je $p+q$ jednako:

\subsection*{Solution}

The given equation is
\[
\sqrt{x\sqrt{x\sqrt{x\sqrt{x\sqrt{x}}}}} = 2.
\]
The exponent of $x$ on the left hand side is
\[
\frac{1}{2} + \frac{1}{4} + \frac{1}{8} + \frac{1}{16} + \frac{1}{32} = \frac{16+8+4+2+1}{32} = \frac{31}{32}.
\]
Thus, $x^{31/32} = 2$.
Substituting $x = 2^{p/q}$, we get
\[
(2^{p/q})^{31/32} = 2^1 \implies 2^{\frac{31p}{32q}} = 2^1.
\]
So $\frac{31p}{32q} = 1$, which means $\frac{p}{q} = \frac{32}{31}$.
Since 32 and 31 are coprime natural numbers, we have $p=32$ and $q=31$.
We need to find $p+q$:
\[
p+q = 32 + 31 = 63.
\]


\subsection*{Answer}
63 (option \textbf{E}).

\end{document}
