\documentclass[12pt]{article}
\usepackage[margin=1in]{geometry}
\usepackage{amsmath,amssymb}
\begin{document}

\section*{Problem 3}
Ako je $\log_a b = 6$, za $a, b \in (0, 1) \cup (1, +\infty)$, onda je $\log_b(a^2b) + \log_{\sqrt{a}}\left(\frac{a}{\sqrt[3]{b}}\right)$ jednako:

\subsection*{Solution}

We are given $\log_a b = 6$.
We calculate the first term:
\[
\log_b(a^2b) = \log_b a^2 + \log_b b = 2\log_b a + 1 = \frac{2}{\log_a b} + 1 = \frac{2}{6} + 1 = \frac{1}{3} + 1 = \frac{4}{3}.
\]
We calculate the second term:
\[
\log_{\sqrt{a}}\left(\frac{a}{\sqrt[3]{b}}\right) = \log_{a^{1/2}}(a \cdot b^{-1/3}) = \frac{1}{1/2} \log_a (a \cdot b^{-1/3}) = 2 (\log_a a + \log_a b^{-1/3}).
\]
\[
= 2 (1 - \frac{1}{3}\log_a b) = 2 (1 - \frac{1}{3} \cdot 6) = 2 (1 - 2) = 2(-1) = -2.
\]
The sum is:
\[
\frac{4}{3} + (-2) = \frac{4}{3} - \frac{6}{3} = -\frac{2}{3}.
\]


\subsection*{Answer}
-\frac{2}{3} (option \textbf{A}).

\end{document}
