\documentclass[12pt]{article}
\usepackage[margin=1in]{geometry}
\usepackage{amsmath,amssymb}
\usepackage[utf8]{inputenc}
\begin{document}

\section*{Problem 17}
Ako polinom $P_n(x)$ $(n\ge 3)$ pri deljenju sa $x$ daje ostatak $3$, a pri deljenju sa $x^2+1$ daje ostatak $-x+2$,
onda polinom $(P_n(x))^2+P_n(x)+1$ pri deljenju sa $x^3+x$ daje ostatak.

\subsection*{Solution}
Po uslovu deljenja sa $x$ važi $P_n(0)=3$.
Zato za $Q(x)=P_n(x)^2+P_n(x)+1$ dobijamo
\[
Q(0)=3^2+3+1=13,
\]
odnosno
\[
Q(x)\equiv 13 \pmod{x}.
\]

Dalje, iz uslova $P_n(x)\equiv -x+2\pmod{x^2+1}$ sledi
\[
Q(x)\equiv (-x+2)^2+(-x+2)+1=x^2-5x+7\pmod{x^2+1}.
\]
Pošto je $x^2\equiv -1\pmod{x^2+1}$, dobijamo
\[
Q(x)\equiv (-1)-5x+7=6-5x\pmod{x^2+1}.
\]

Tražimo ostatak $R(x)=ax^2+bx+c$ pri deljenju sa $x(x^2+1)=x^3+x$ (stepen $<3$) takav da:
\[
R(x)\equiv 13 \pmod{x}\Rightarrow c=13,
\]
i
\[
R(x)\equiv 6-5x\pmod{x^2+1}.
\]
Sa $c=13$ pišemo
\[
R(x)-(6-5x)=ax^2+(b+5)x+7.
\]
Da bi ovo bilo deljivo sa $x^2+1$ (stepen $\le 2$), mora biti jednako $k(x^2+1)$ za neki konstantan $k$.
Otuda $a=k$, $b+5=0\Rightarrow b=-5$, i $7=k\Rightarrow a=7$.

Dakle
\[
R(x)=7x^2-5x+13.
\]

\subsection*{Answer}
$7x^2-5x+13$ (option \textbf{A}).

\end{document}


