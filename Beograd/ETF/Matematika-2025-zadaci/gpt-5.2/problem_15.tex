\documentclass[12pt]{article}
\usepackage[margin=1in]{geometry}
\usepackage{amsmath,amssymb}
\usepackage[utf8]{inputenc}
\begin{document}

\section*{Problem 15}
Broj realnih i različitih rešenja jednačine
\[
\tan^2 x=\frac{1-\cos x}{1-\sin x}
\]
na segmentu $[0,2\pi]$ je.

\subsection*{Solution}
Zapišimo jednačinu preko $\sin x$ i $\cos x$:
\[
\frac{\sin^2 x}{\cos^2 x}=\frac{1-\cos x}{1-\sin x},
\]
pri čemu mora važiti $\cos x\ne 0$ i $\sin x\ne 1$.
Prebacivanjem dobijamo
\[
\sin^2 x(1-\sin x)=\cos^2 x(1-\cos x).
\]
To je ekvivalentno
\[
\sin^2 x-\cos^2 x=\sin^3 x-\cos^3 x.
\]
Faktorišemo:
\[
(\sin x-\cos x)(\sin x+\cos x)=(\sin x-\cos x)(\sin^2 x+\sin x\cos x+\cos^2 x).
\]
Dakle, ili je $\sin x=\cos x$, ili
\[
\sin x+\cos x=\sin^2 x+\sin x\cos x+\cos^2 x=1+\sin x\cos x.
\]
\textbf{1) $\sin x=\cos x$:} tada je $\tan x=1$, pa
\[
x=\frac{\pi}{4},\ \frac{5\pi}{4}\ \text{na }[0,2\pi].
\]

\medskip
\noindent\textbf{2) $\sin x+\cos x=1+\sin x\cos x$:}
Neka su $p=\sin x+\cos x$, $q=\sin x\cos x$. Tada je $p=1+q$.
Takođe,
\[
p^2=\sin^2 x+\cos^2 x+2\sin x\cos x=1+2q.
\]
Zamenom $q=p-1$ dobijamo $p^2=1+2(p-1)=2p-1$, tj. $(p-1)^2=0$, pa je $p=1$ i $q=0$.
Dakle $\sin x+\cos x=1$ i $\sin x\cos x=0$.
Pošto $\cos x=0$ nije dozvoljeno, mora biti $\sin x=0$ i $\cos x=1$, pa je $x=0$ ili $x=2\pi$.

Ukupno su rešenja: $0,\ 2\pi,\ \pi/4,\ 5\pi/4$, ukupno $4$.

\subsection*{Answer}
$4$ (option \textbf{B}).

\end{document}


