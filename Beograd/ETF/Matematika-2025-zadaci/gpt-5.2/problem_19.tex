\documentclass[12pt]{article}
\usepackage[margin=1in]{geometry}
\usepackage{amsmath,amssymb}
\usepackage[utf8]{inputenc}
\begin{document}

\section*{Problem 19}
Oko pravog valjka čiji je poluprečnik osnove jednak $r$, a visina jednaka $3r$, opisana je prava kupa tako da se centri donjih osnova valjka i kupe poklapaju, a gornja osnova valjka dodiruje omotač kupe. Minimalna zapremina takve kupe iznosi.

\subsection*{Solution}
Neka kupa ima visinu $H$ i poluprečnik osnove $R$.
Poluprečnik preseka kupe na visini $z$ od osnove (mereno duž ose) opada linearno:
\[
\rho(z)=R\left(1-\frac{z}{H}\right).
\]
Uslov ``gornja osnova valjka dodiruje omotač kupe'' znači da na visini $z=3r$ poluprečnik kupe tačno iznosi $r$:
\[
\rho(3r)=R\left(1-\frac{3r}{H}\right)=r.
\]
Otuda
\[
R=\frac{r}{1-3r/H}=\frac{rH}{H-3r},\qquad H>3r.
\]
Zapremina kupe je
\[
V=\frac{1}{3}\pi R^2H=\frac{1}{3}\pi\left(\frac{rH}{H-3r}\right)^2H
=\frac{1}{3}\pi r^2\frac{H^3}{(H-3r)^2}.
\]
Uvedimo $t=\frac{H}{r}>3$. Tada
\[
V=\frac{1}{3}\pi r^3\cdot \frac{t^3}{(t-3)^2}.
\]
Minimizujemo $f(t)=\frac{t^3}{(t-3)^2}$ za $t>3$:
\[
\ln f=3\ln t-2\ln(t-3)\Rightarrow
\frac{f'}{f}=\frac{3}{t}-\frac{2}{t-3}.
\]
Postavimo $f'=0$:
\[
\frac{3}{t}=\frac{2}{t-3}\Rightarrow 3(t-3)=2t\Rightarrow t=9.
\]
Tada je
\[
f(9)=\frac{9^3}{6^2}=\frac{729}{36}=\frac{81}{4},
\]
pa je minimalna zapremina
\[
V_{\min}=\frac{1}{3}\pi r^3\cdot \frac{81}{4}=\frac{27\pi r^3}{4}.
\]

\subsection*{Answer}
$\dfrac{27\pi r^3}{4}$ (option \textbf{C}).

\end{document}


