\documentclass[12pt]{article}
\usepackage[margin=1in]{geometry}
\usepackage{amsmath,amssymb}
\usepackage[utf8]{inputenc}
\begin{document}

\section*{Problem 6}
Ako je duži krak pravouglog trapeza jednak $10$, a poluprečnik kružnice upisane u taj trapez jednak $4$, onda je površina tog trapeza jednaka.

\subsection*{Solution}
Pošto je kružnica upisana, trapez je tangencijalni, pa važi:
\[
\text{(zbir osnovica)}=\text{(zbir krakova)}.
\]
Kružnica je tangentna na obe paralelne osnovice, pa je rastojanje osnovica jednako prečniku:
\[
h=2r=8.
\]
Dakle jedan krak (normalan) je $h=8$, a duži (kosi) krak je $10$.
U pravouglom trapezu razlika osnovica je horizontalna projekcija kosog kraka:
\[
a-b=\sqrt{10^2-8^2}=\sqrt{36}=6.
\]
Iz tangencijalnog uslova:
\[
a+b = 8+10=18.
\]
Rešavanjem sistema $a+b=18$, $a-b=6$ dobijamo $a=12$, $b=6$.
Površina je
\[
P=\frac{a+b}{2}\,h=\frac{18}{2}\cdot 8=72.
\]

\subsection*{Answer}
$72$ (option \textbf{E}).

\end{document}


