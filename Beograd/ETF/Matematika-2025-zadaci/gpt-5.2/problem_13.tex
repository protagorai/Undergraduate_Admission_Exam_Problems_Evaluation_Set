\documentclass[12pt]{article}
\usepackage[margin=1in]{geometry}
\usepackage{amsmath,amssymb}
\usepackage[utf8]{inputenc}
\begin{document}

\section*{Problem 13}
Granična vrednost
\[
\lim_{x\to 2}\frac{x\sqrt{x}-\sqrt{2}\,x-2\sqrt{x}+2\sqrt{2}}{x^3-4x^2+4x}
\]
iznosi.

\subsection*{Solution}
Brojilac grupišemo:
\[
x\sqrt{x}-2\sqrt{x}=\sqrt{x}(x-2),\qquad
-\sqrt{2}\,x+2\sqrt{2}=-\sqrt{2}(x-2),
\]
pa je
\[
x\sqrt{x}-\sqrt{2}\,x-2\sqrt{x}+2\sqrt{2}=(x-2)(\sqrt{x}-\sqrt{2}).
\]
Imenilac:
\[
x^3-4x^2+4x=x(x^2-4x+4)=x(x-2)^2.
\]
Zato je razlomak
\[
\frac{(x-2)(\sqrt{x}-\sqrt{2})}{x(x-2)^2}=\frac{\sqrt{x}-\sqrt{2}}{x(x-2)}.
\]
Racionalizujemo:
\[
\sqrt{x}-\sqrt{2}=\frac{x-2}{\sqrt{x}+\sqrt{2}},
\]
pa dobijamo
\[
\lim_{x\to 2}\frac{1}{x(\sqrt{x}+\sqrt{2})}
=\frac{1}{2( \sqrt{2}+\sqrt{2})}=\frac{1}{4\sqrt{2}}.
\]

\subsection*{Answer}
$\dfrac{1}{4\sqrt2}$ (option \textbf{A}).

\end{document}


