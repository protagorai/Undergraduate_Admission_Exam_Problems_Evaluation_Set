\documentclass[12pt]{article}
\usepackage[margin=1in]{geometry}
\usepackage{amsmath,amssymb}
\usepackage[utf8]{inputenc}
\begin{document}

\section*{Problem 10}
Ako je zbir prva tri člana rastućeg aritmetičkog niza jednak $15$ i njihov proizvod jednak $105$,
a zbir prvih $n$ članova tog niza jednak $1023$, onda je $n$ jednako.

\subsection*{Solution}
Neka su prva tri člana $a,\,a+d,\,a+2d$ sa $d>0$.
Uslov zbira daje:
\[
a+(a+d)+(a+2d)=3a+3d=15 \Rightarrow a+d=5.
\]
Dakle srednji član je $5$, pa su članovi $5-d,\ 5,\ 5+d$.
Uslov proizvoda:
\[
(5-d)\cdot 5\cdot (5+d)=5(25-d^2)=105 \Rightarrow 25-d^2=21 \Rightarrow d^2=4 \Rightarrow d=2.
\]
Dakle niz je $3,5,7,\dots$ sa $a_1=3$ i $d=2$.
Zbir prvih $n$ članova:
\[
S_n=\frac{n}{2}\bigl(2a_1+(n-1)d\bigr)
=\frac{n}{2}\bigl(6+2n-2\bigr)=\frac{n}{2}(2n+4)=n(n+2).
\]
Uslov $S_n=1023$ daje
\[
n(n+2)=1023 \Rightarrow n^2+2n-1023=0.
\]
\[
\Delta=4+4092=4096=64^2 \Rightarrow n=\frac{-2+64}{2}=31.
\]

\subsection*{Answer}
$31$ (option \textbf{C}).

\end{document}


