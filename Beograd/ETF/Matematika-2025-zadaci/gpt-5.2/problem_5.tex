\documentclass[12pt]{article}
\usepackage[margin=1in]{geometry}
\usepackage{amsmath,amssymb}
\usepackage[utf8]{inputenc}
\begin{document}

\section*{Problem 5}
Ako je $\sin\frac{\alpha}{2}=\frac{1}{\sqrt{5}}$, za $\alpha\in(\pi,2\pi)$, onda je $\tan\alpha$ jednako.

\subsection*{Solution}
Pošto je $\alpha\in(\pi,2\pi)$, sledi $\alpha/2\in(\pi/2,\pi)$, pa je $\cos(\alpha/2)<0$.
Zato
\[
\cos\frac{\alpha}{2}=-\sqrt{1-\sin^2\frac{\alpha}{2}}=-\sqrt{1-\frac{1}{5}}=-\frac{2}{\sqrt{5}}.
\]
Tada je
\[
\tan\frac{\alpha}{2}=\frac{\sin(\alpha/2)}{\cos(\alpha/2)}=\frac{1/\sqrt{5}}{-2/\sqrt{5}}=-\frac{1}{2}.
\]
Formula za tangens duplog ugla daje:
\[
\tan\alpha=\frac{2\tan(\alpha/2)}{1-\tan^2(\alpha/2)}
=\frac{2(-1/2)}{1-1/4}
=\frac{-1}{3/4}=-\frac{4}{3}.
\]

\subsection*{Answer}
$-\dfrac{4}{3}$ (option \textbf{B}).

\end{document}


