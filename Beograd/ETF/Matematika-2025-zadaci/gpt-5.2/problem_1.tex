\documentclass[12pt]{article}
\usepackage[margin=1in]{geometry}
\usepackage{amsmath,amssymb}
\usepackage[utf8]{inputenc}
\begin{document}

\section*{Problem 1}
Ako je $x = 2^{p/q}$ rešenje jednačine
\[
\sqrt{x\sqrt{x\sqrt{x\sqrt{x\sqrt{x}}}}} = 2,
\]
za neke uzajamno proste prirodne brojeve $p$ i $q$, tada je $p+q$ jednako.

\subsection*{Solution}
Neka je izraz sa leve strane oblika $x^{\alpha}$. Krenimo iznutra:
\[
\sqrt{x}=x^{1/2},\quad
\sqrt{x\sqrt{x}}=\sqrt{x\cdot x^{1/2}}=x^{3/4},
\]
\[
\sqrt{x\sqrt{x\sqrt{x}}}= \sqrt{x\cdot x^{3/4}}=x^{7/8},\quad
\sqrt{x\sqrt{x\sqrt{x\sqrt{x}}}}=x^{15/16},
\]
\[
\sqrt{x\sqrt{x\sqrt{x\sqrt{x\sqrt{x}}}}}=x^{31/32}.
\]
Jednačina postaje $x^{31/32}=2$, pa je
\[
x=2^{32/31}.
\]
Dakle $(p,q)=(32,31)$, i
\[
p+q=32+31=63.
\]

\subsection*{Answer}
$63$ (option \textbf{E}).

\end{document}


