\documentclass[12pt]{article}
\usepackage[margin=1in]{geometry}
\usepackage{amsmath,amssymb}
\usepackage[utf8]{inputenc}
\begin{document}

\section*{Problem 1}
Ako je $x = 2^{p/q}$ rešenje jednačine
\[
\sqrt{x\sqrt{x\sqrt{x\sqrt{x\sqrt{x}}}}}=2,
\]
za neke uzajamno proste prirodne brojeve $p$ i $q$, pronaći $p+q$.

\subsection*{Solution}
Uočimo da je izraz pod korenom ponovljeno gniježdenje kvadratnih korena. Posmatrajmo eksponent:
\[
\sqrt{x}=x^{1/2},\;\sqrt{x\sqrt{x}}=\sqrt{x^{1+1/2}}=x^{3/4},\;\sqrt{x\sqrt{x\sqrt{x}}}=x^{7/8},\;\sqrt{x\sqrt{x\sqrt{x\sqrt{x}}}}=x^{15/16},\;\sqrt{x\sqrt{x\sqrt{x\sqrt{x\sqrt{x}}}}}=x^{31/32}.
\]
Dakle leva strana je $x^{31/32}$, pa jednačina postaje
\[
 x^{31/32}=2 \quad\Longrightarrow\quad x=2^{32/31}.
\]
Po definiciji $x=2^{p/q}$ sa uzajamno prostim $p,q$, dobijamo $(p,q)=(32,31)$ te
\[
p+q=32+31=63.
\]

\subsection*{Answer}
$63$ (option \textbf{E}).

\end{document}