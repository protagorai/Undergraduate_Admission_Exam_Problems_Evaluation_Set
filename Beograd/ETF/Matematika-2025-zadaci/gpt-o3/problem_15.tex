\documentclass[12pt]{article}
\usepackage[margin=1in]{geometry}
\usepackage{amsmath,amssymb}
\usepackage[utf8]{inputenc}
\begin{document}

\section*{Problem 15}
Koliko realnih \emph{različitih} rešenja ima jednačina
\[
\tan^{2}x = \frac{1-\cos x}{1-\sin x},\qquad x\in[0,2\pi] ?
\]

\subsection*{Solution}
Obeležimo $s=\sin x$, $c=\cos x$.
Jednačina je definisana samo ako $c\ne0$ (da bi $\tan x$ postojao) i $s\ne1$ (da se izbegne nula u imeniocu desne strane).

\paragraph{1. Prebacivanje na $s$ i $c$.}
\[
\frac{s^{2}}{c^{2}} = \frac{1-c}{1-s}\quad\Longrightarrow\quad s^{2}(1-s)=c^{2}(1-c).\tag{1}
\]
Koristimo $c^{2}=1-s^{2}$ pa desnu stranu pišemo kao $(1-s^{2})(1-c)$. Ipak, pogodnije je primetiti da se u oba člana pojavljuje faktor $(s-c)$.

Dodajmo i oduzmimo tako da izdvojimo taj faktor:
\[
s^{2}(1-s)-c^{2}(1-c)=0\ \Longrightarrow\ (s-c)\bigl(s+c -s c -1\bigr)=0.\tag{2}
\]
Dakle su dve mogućnosti:
\[
\boxed{s=c}\quad\text{ili}\quad\boxed{s+c=1+sc}.\tag{A}
\]

\paragraph{2. Prvi slučaj $\;\sin x = \cos x$.}
Tada je $\tan x = 1$ i, na zadatom intervalu, dobijamo
\[
x=\frac{\pi}{4},\;\frac{5\pi}{4}.
\]
Napomena: $x=\frac{\pi}{4}$ je regularna tačka (ne anulira imenioce), a $x=\tfrac{5\pi}{4}$ takođe – oba zadovoljavaju jednačinu.

\paragraph{3. Drugi slučaj $\;s+c=1+sc$.}
Stavimo $p=s+c$ i $q=sc$. Увео је класичну замену; тада услов постаје $p=1+q$. Са друге стране
\[
p^{2}=s^{2}+c^{2}+2sc=1+2q.\tag{3}
\]
Уметањем $q=p-1$ у (3):
\[p^{2}=1+2(p-1)=2p-1\ \Longrightarrow\ (p-1)^{2}=0 \ \Longrightarrow\ p=1\ (\text{па}\ q=0).\]
Зато $s+c=1$ и $sc=0$. Производ нула значи или $s=0$ или $c=0$; али $c=0$ је искључено, па је $s=0$ и $c=1$ – једино могуће кад је $x=0$ или $x=2\pi$.

\paragraph{4. Преглед решења.}
\[
 x\in\left\{0,\ 2\pi,\ \tfrac{\pi}{4},\ \tfrac{5\pi}{4}\right\}.
\]
Сва четири вредности задовољавају иницијалне доменске услове, па су решења различита.

\subsection*{Answer}
Једначина има \textbf{4} реална и различита решења (опција \textbf{B}).

\end{document}