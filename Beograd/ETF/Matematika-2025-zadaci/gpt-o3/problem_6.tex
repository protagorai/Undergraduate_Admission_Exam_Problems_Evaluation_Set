\documentclass[12pt]{article}
\usepackage[margin=1in]{geometry}
\usepackage{amsmath,amssymb}
\usepackage[utf8]{inputenc}
\begin{document}

\section*{Problem 6}
U pravouglom trapezu duži krak iznosi $10$, a poluprečnik kružnice upisane u trapez je $4$. Odrediti površinu trapeza.

\subsection*{Solution}
Pošto je kružnica upisana, trapez je tangencijalan; rastojanje osnovica jednako je prečniku upisane kružnice:
\[
h=2r=8.
\]
Kraći krak (visina) je $h=8$, dok je duži kosi krak $10$. Razlika osnovica jednaka je horizontalnoj projekciji kosog kraka:
\[
a-b=\sqrt{10^2-8^2}=6.
\]
Usled tangencijalnog uslova važi $(a+b)=(h+10)=18$. Rešavajući sistem
\[
\begin{cases}a+b=18\\ a-b=6\end{cases}\Longrightarrow a=12,\;b=6.
\]
Površina trapeza:
\[
P=\frac{a+b}{2}\,h=\frac{18}{2}\cdot 8=72.
\]

\subsection*{Answer}
$72$ (option \textbf{E}).

\end{document}