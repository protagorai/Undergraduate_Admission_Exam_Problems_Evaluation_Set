\documentclass[12pt]{article}
\usepackage[margin=1in]{geometry}
\usepackage{amsmath,amssymb}
\usepackage[utf8]{inputenc}
\begin{document}

\section*{Problem 17}
Za polinom $P_n(x)$ ($n\ge3$) važi
\[
P_n(x) \equiv 3 \pmod{x},\qquad P_n(x) \equiv -x+2 \pmod{x^{2}+1}.
\]
Odrediti ostatak pri deljenju polinoma
\[
Q(x)=P_n(x)^{2}+P_n(x)+1
\]
sa $x^{3}+x=x(x^{2}+1)$.

\subsection*{Solution}
Ostatak tražimo u obliku
\[
R(x)=ax^{2}+bx+c\qquad(\deg R<3).
\]
\paragraph{1. Uslov modulo $x$.}  Iz $P_n(0)=3$ sledi
\[Q(0)=3^{2}+3+1=13 \;\Longrightarrow\;c=13.\]

\paragraph{2. Uslov modulo $x^{2}+1$.}  Budući da je $P_n(x)\equiv -x+2 \pmod{x^{2}+1}$,
\[
Q(x)\equiv (-x+2)^{2}+(-x+2)+1 = x^{2}-5x+7 \pmod{x^{2}+1}.
\]
Pošto je $x^{2}\equiv -1$ modulo $x^{2}+1$, dobijamo
\[
Q(x)\equiv -1-5x+7 = 6-5x \pmod{x^{2}+1}.
\]
Ostatak $R(x)$ mora imati istu kongruenciju, pa
\[
R(x)- (6-5x)\equiv0\pmod{x^{2}+1}.
\]
Sa $c=13$ pišemo
\[R(x)=ax^{2}+bx+13.
\]
Tada
\[ax^{2}+bx+13-(6-5x)=ax^{2}+(b+5)x+7,
\]
što treba da bude višekratnik $x^{2}+1$, recimo $k(x^{2}+1)$.  Izjednačavanjem koeficijenata dobijamo $a=k$, $b+5=0$, $7=k$. Otuda $a=7$, $b=-5$.

\paragraph{3. Ostatak.}
\[
R(x)=7x^{2}-5x+13.
\]

\subsection*{Answer}
$7x^{2}-5x+13$ (opcija \textbf{A}).

\end{document}