\documentclass[12pt]{article}
\usepackage[margin=1in]{geometry}
\usepackage{amsmath,amssymb}
\usepackage[utf8]{inputenc}
\begin{document}

\section*{Problem 13}
Odrediti graničnu vrednost
\[
\lim_{x\to 2}\frac{x\sqrt{x}-\sqrt{2}\,x-2\sqrt{x}+2\sqrt{2}}{x^{3}-4x^{2}+4x}.
\]

\subsection*{Solution}
Primetite da se u brojiocu pojavljuju parovi koji sadrže faktor $(x-2)$.
\[
x\sqrt{x}-2\sqrt{x}=\sqrt{x}(x-2), \qquad -\sqrt{2}\,x+2\sqrt{2}=-\sqrt{2}(x-2).
\]
Zbir je
\[
(x-2)(\sqrt{x}-\sqrt{2}).
\]
Imenilac se faktoriše
\[
x^{3}-4x^{2}+4x= x(x^{2}-4x+4)=x(x-2)^{2}.
\]
Dakle
\[
\frac{(x-2)(\sqrt{x}-\sqrt{2})}{x(x-2)^{2}}=\frac{\sqrt{x}-\sqrt{2}}{x(x-2)}.
\]
Racionalizujmo brojilac: \(\sqrt{x}-\sqrt{2}=\dfrac{x-2}{\sqrt{x}+\sqrt{2}}\), pa izraz postaje
\[
\frac{1}{x\,(\sqrt{x}+\sqrt{2})}.
\]
Sada je prelaz na graničnu vrednost neposredan:
\[
\lim_{x\to 2}\frac{1}{x(\sqrt{x}+\sqrt{2})}=\frac{1}{2\,(\sqrt{2}+\sqrt{2})}=\frac{1}{4\sqrt{2}}.
\]

\subsection*{Answer}
$\displaystyle \frac{1}{4\sqrt{2}}$ (option \textbf{A}).

\end{document}