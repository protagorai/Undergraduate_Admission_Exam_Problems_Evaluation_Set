\documentclass[12pt]{article}
\usepackage[margin=1in]{geometry}
\usepackage{amsmath,amssymb}
\begin{document}

\section*{Problem 17}
Zbir svih rešenja jednačine $\left(\sqrt{\log_7(-2x-1)} - \log_7\sqrt{4x^2+4x+1} - 1\right) \cdot \log_7|x+7| = -3\log_7\sqrt[3]{x+7}$ iznosi:

\subsection*{Solution}
First, let's simplify the expressions.

Note that $4x^2 + 4x + 1 = (2x + 1)^2$, so:
\[
\log_7\sqrt{4x^2+4x+1} = \log_7|2x+1| = \frac{1}{2}\log_7(2x+1)^2
\]

For $\sqrt{\log_7(-2x-1)}$ to be defined, we need $-2x - 1 > 0$ (so $x < -\frac{1}{2}$) and $\log_7(-2x-1) \geq 0$ (so $-2x-1 \geq 1$, meaning $x \leq -1$).

Since $x < -\frac{1}{2}$, we have $2x + 1 < 0$, so $|2x+1| = -(2x+1) = -2x-1$.

Thus $\log_7\sqrt{(2x+1)^2} = \log_7|2x+1| = \log_7(-2x-1)$.

Let $u = \log_7(-2x-1) \geq 0$. Then $\sqrt{u} - u - 1$.

Also, $\sqrt[3]{x+7} = (x+7)^{1/3}$, so $\log_7\sqrt[3]{x+7} = \frac{1}{3}\log_7(x+7)$ (need $x > -7$).

Let $v = \log_7|x+7|$. The equation becomes:
\[
(\sqrt{u} - u - 1) \cdot v = -3 \cdot \frac{1}{3}v = -v
\]
\[
(\sqrt{u} - u - 1) \cdot v + v = 0
\]
\[
v(\sqrt{u} - u) = 0
\]

\textbf{Case 1:} $v = 0 \Rightarrow |x + 7| = 1 \Rightarrow x = -6$ or $x = -8$

Check $x = -6$: Need $x \leq -1$ ✓, $x > -7$ ✓. $-2(-6)-1 = 11 > 0$ ✓
Check $x = -8$: Need $x > -7$ ✗

So $x = -6$ is valid.

\textbf{Case 2:} $\sqrt{u} - u = 0 \Rightarrow \sqrt{u}(1 - \sqrt{u}) = 0$

$\sqrt{u} = 0 \Rightarrow u = 0 \Rightarrow -2x - 1 = 1 \Rightarrow x = -1$
$\sqrt{u} = 1 \Rightarrow u = 1 \Rightarrow -2x - 1 = 7 \Rightarrow x = -4$

Check $x = -1$: $x \leq -1$ ✓, $x > -7$ ✓
Check $x = -4$: $x \leq -1$ ✓, $x > -7$ ✓

Both valid.

Sum of solutions: $(-6) + (-1) + (-4) = -11$

\subsection*{Answer}
$-11$ (option \textbf{A}).

\end{document}
