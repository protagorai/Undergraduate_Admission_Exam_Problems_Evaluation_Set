\documentclass[12pt]{article}
\usepackage[margin=1in]{geometry}
\usepackage{amsmath,amssymb}
\begin{document}

\section*{Problem 5}
Trougao je presečen na dva dela jednakih površina pravom koja je paralelna osnovici. Ako je $a$ osnovica trougla, tada je osnovica manjeg trougla koja leži na datoj pravoj jednaka:

\subsection*{Solution}
Let the original triangle have base $a$ and height $h$.

The area of the original triangle is $S = \dfrac{1}{2}ah$.

Let the smaller triangle (formed by the cut parallel to the base) have base $b$ and height $h_1$.

Since the cutting line is parallel to the base, the smaller triangle is similar to the original triangle. The ratio of corresponding sides equals the ratio of heights:
\[
\frac{b}{a} = \frac{h_1}{h}
\]

The area of the smaller triangle is:
\[
S_1 = \frac{1}{2}bh_1
\]

Since the two parts have equal areas, the smaller triangle has area equal to half of the original:
\[
S_1 = \frac{S}{2} = \frac{1}{4}ah
\]

For similar triangles, the ratio of areas equals the square of the ratio of corresponding sides:
\[
\frac{S_1}{S} = \left(\frac{b}{a}\right)^2
\]

We have $\dfrac{S_1}{S} = \dfrac{1}{2}$, so:
\[
\left(\frac{b}{a}\right)^2 = \frac{1}{2}
\]
\[
\frac{b}{a} = \frac{1}{\sqrt{2}} = \frac{\sqrt{2}}{2}
\]
\[
b = \frac{a\sqrt{2}}{2} = \frac{a}{\sqrt{2}}
\]

This can also be written as $\dfrac{a\sqrt{2}}{2}$.

\subsection*{Answer}
$\dfrac{a\sqrt{2}}{2}$ (option \textbf{D}).

\end{document}
