\documentclass[12pt]{article}
\usepackage[margin=1in]{geometry}
\usepackage{amsmath,amssymb}
\begin{document}

\section*{Problem 16}
Ako je zbir svih binomnih koeficijenata u razvoju binoma
\[
\big(\sqrt[5]{3}+\sqrt[3]{5}\big)^n
\]
jednak $4^{52}$, odrediti broj racionalnih članova u ovom razvoju.

\subsection*{Solution}
\textbf{1) Određivanje $n$.}
Zbir binomnih koeficijenata u razvoju $(a+b)^n$ je
\[
\sum_{k=0}^n \binom{n}{k} = 2^n.
\]
Dato je $2^n=4^{52}=(2^2)^{52}=2^{104}$, pa je
\[
n=104.
\]

\medskip
\textbf{2) Opšti član i uslov racionalnosti.}
Opšti član razvoja je
\[
\binom{104}{k}\,(\sqrt[5]{3})^{104-k}\,(\sqrt[3]{5})^{k}
\;=\;\binom{104}{k}\,3^{\frac{104-k}{5}}\,5^{\frac{k}{3}}.
\]
Pošto su $3$ i $5$ različiti prosti brojevi, član je racionalan ako i samo ako su oba eksponenta celi brojevi:
\[
\frac{104-k}{5}\in\mathbb{Z}\quad\text{i}\quad \frac{k}{3}\in\mathbb{Z}.
\]
To znači
\[
k\equiv 104\pmod 5 \;\Longleftrightarrow\; k\equiv 4\pmod 5,
\qquad k\equiv 0\pmod 3.
\]
Tražimo rešenje modulo $15$. Brojevi deljivi sa $3$ su $0,3,6,9,12$ (mod $15$), a jedini koji je $4$ (mod $5$)
je $9$. Dakle
\[
k\equiv 9\pmod{15}.
\]
Zato je $k=9+15m$. U opsegu $0\le k\le 104$ dobijamo:
\[
9+15m\le 104 \Rightarrow m\le 6,
\]
pa je $m=0,1,2,3,4,5,6$, ukupno $7$ vrednosti $k$.

\subsection*{Answer}
$7$ (option \textbf{A}).

\end{document}

