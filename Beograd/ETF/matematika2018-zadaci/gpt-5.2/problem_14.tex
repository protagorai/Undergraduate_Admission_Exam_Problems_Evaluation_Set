\documentclass[12pt]{article}
\usepackage[margin=1in]{geometry}
\usepackage{amsmath,amssymb}
\begin{document}

\section*{Problem 14}
Prva dva člana rastuće geometrijske progresije su rešenja jednačine
\[
\frac{2^{2\sin x}}{1+2^{2\sin x}} = 1-\frac{3-2^{2\sin x}}{5-2^{2\sin x}}
\]
na intervalu $(0,\pi)$. Ako je zbir ove progresije $651\pi$, odrediti ukupan broj članova.

\subsection*{Solution}
Označimo $t=2^{2\sin x}>0$. Desna strana jednačine je
\[
1-\frac{3-t}{5-t}=\frac{5-t-(3-t)}{5-t}=\frac{2}{5-t}.
\]
Jednačina postaje
\[
\frac{t}{1+t}=\frac{2}{5-t}.
\]
Množenjem dobijamo
\[
t(5-t)=2(1+t)\;\Longleftrightarrow\; 5t-t^2=2+2t
\;\Longleftrightarrow\; t^2-3t+2=0
\;\Longleftrightarrow\; (t-1)(t-2)=0.
\]
Slučaj $t=1$ daje $2^{2\sin x}=1\Rightarrow \sin x=0$, tj. $x\in\{0,\pi\}$, što nije u $(0,\pi)$.
Zato mora biti $t=2$, pa
\[
2^{2\sin x}=2 \;\Longleftrightarrow\; 2\sin x=1 \;\Longleftrightarrow\; \sin x=\frac12.
\]
Na $(0,\pi)$ rešenja su
\[
x_1=\frac{\pi}{6},\qquad x_2=\frac{5\pi}{6}.
\]
To su prva dva člana geometrijske progresije, pa je količnik
\[
q=\frac{x_2}{x_1}=\frac{5\pi/6}{\pi/6}=5,\qquad a_1=\frac{\pi}{6}.
\]
Zbir prvih $n$ članova je
\[
S_n=a_1\frac{q^n-1}{q-1}=\frac{\pi}{6}\cdot\frac{5^n-1}{4}=\pi\cdot\frac{5^n-1}{24}.
\]
Po uslovu $S_n=651\pi$, pa
\[
\frac{5^n-1}{24}=651 \;\Longleftrightarrow\; 5^n-1=15624 \;\Longleftrightarrow\; 5^n=15625=5^6,
\]
odakle je $n=6$.

\subsection*{Answer}
$6$ (option \textbf{B}).

\end{document}

