\documentclass[12pt]{article}
\usepackage[margin=1in]{geometry}
\usepackage{amsmath,amssymb}
\begin{document}

\section*{Problem 5}
Trougao je presečen pravom paralelnom osnovici na dva dela jednakih površina.
Ako je $a$ osnovica trougla, odrediti osnovicu manjeg trougla.

\subsection*{Solution}
Prava paralelna osnovici određuje manji trougao sličan polaznom trouglu.
Ako je koeficijent sličnosti (od manjeg ka većem) jednak $k$, tada se površine odnose kao $k^2$.
Po uslovu, površina manjeg trougla je polovina površine većeg, pa važi
\[
k^2=\frac12 \quad\Rightarrow\quad k=\frac1{\sqrt2}.
\]
Osnovice se odnose linearno, pa je osnovica manjeg trougla
\[
a\cdot k=\frac{a}{\sqrt2}=\frac{a\sqrt2}{2}.
\]

\subsection*{Answer}
$\dfrac{a\sqrt2}{2}$ (option \textbf{D}).

\end{document}

