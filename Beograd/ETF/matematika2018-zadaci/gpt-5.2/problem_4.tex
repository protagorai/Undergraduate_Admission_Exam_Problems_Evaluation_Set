\documentclass[12pt]{article}
\usepackage[margin=1in]{geometry}
\usepackage{amsmath,amssymb}
\begin{document}

\section*{Problem 4}
Uprostiti izraz
\[
\left(\frac{x\sqrt{x}-y\sqrt{y}}{\sqrt{x}-\sqrt{y}}+\sqrt{xy}\right)
:\left(\frac{\sqrt{x}-\sqrt{y}}{x-y}\right)^{-2},
\]
za sve realne $x,y$ za koje je definisan.

\subsection*{Solution}
Postavimo $u=\sqrt{x}$ i $v=\sqrt{y}$ (tada je $x=u^2$, $y=v^2$). Tada
\[
\frac{x\sqrt{x}-y\sqrt{y}}{\sqrt{x}-\sqrt{y}}
\;=\;\frac{u^3-v^3}{u-v}=u^2+uv+v^2=x+\sqrt{xy}+y.
\]
Zato je prvi zagradjeni deo
\[
x+\sqrt{xy}+y+\sqrt{xy}=x+y+2\sqrt{xy}=(\sqrt{x}+\sqrt{y})^2=(u+v)^2.
\]
Dalje, pošto je $x-y=(u-v)(u+v)$,
\[
\frac{\sqrt{x}-\sqrt{y}}{x-y}=\frac{u-v}{(u-v)(u+v)}=\frac1{u+v},
\]
pa
\[
\left(\frac{\sqrt{x}-\sqrt{y}}{x-y}\right)^{-2}=\left(\frac1{u+v}\right)^{-2}=(u+v)^2.
\]
Dakle ceo izraz je
\[
\frac{(u+v)^2}{(u+v)^2}=1
\]
(za dozvoljene $x,y$ gde su imenitelji nenula).

\subsection*{Answer}
$1$ (option \textbf{A}).

\end{document}

