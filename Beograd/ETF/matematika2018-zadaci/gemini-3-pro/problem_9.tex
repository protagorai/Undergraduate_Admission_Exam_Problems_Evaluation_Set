\documentclass[12pt]{article}
\usepackage[margin=1in]{geometry}
\usepackage{amsmath,amssymb}
\usepackage[utf8]{inputenc}
\begin{document}

\section*{Problem 9}
If $f(x) = \frac{4}{x-5} + (3x+1)e^{-(x-1)^2} + \sqrt{-2x+3}$, then $f'(1)$ is equal to:

\subsection*{Solution}
We compute the derivative of each term at $x=1$.
1. Term $\frac{4}{x-5}$:
   Derivative is $-4(x-5)^{-2}$.
   At $x=1$: $-4(1-5)^{-2} = -4(-4)^{-2} = -4/16 = -1/4$.

2. Term $(3x+1)e^{-(x-1)^2}$:
   Use product rule: $u = 3x+1, v = e^{-(x-1)^2}$.
   $u' = 3, v' = e^{-(x-1)^2} \cdot (-2(x-1))$.
   At $x=1$: $x-1=0$, so $e^0 = 1$.
   $u(1) = 4, u'(1) = 3, v(1) = 1, v'(1) = 1 \cdot (-2(0)) = 0$.
   $(uv)' = u'v + uv' = 3(1) + 4(0) = 3$.

3. Term $\sqrt{-2x+3}$:
   Derivative is $\frac{1}{2}(-2x+3)^{-1/2} \cdot (-2) = -(-2x+3)^{-1/2}$.
   At $x=1$: $-(-2(1)+3)^{-1/2} = -(1)^{-1/2} = -1$.

Summing the derivatives:
\[ f'(1) = -\frac{1}{4} + 3 - 1 = 2 - \frac{1}{4} = \frac{7}{4} \]

\subsection*{Answer}
$\frac{7}{4}$ (option \textbf{E}).

\end{document}
