\documentclass[12pt]{article}
\usepackage[margin=1in]{geometry}
\usepackage{amsmath,amssymb}
\usepackage[utf8]{inputenc}
\begin{document}

\section*{Problem 5}
A triangle is divided into two parts of equal area by a line parallel to the base. If $a$ is the base of the triangle, then the base of the smaller triangle lying on the given line is equal to:

\subsection*{Solution}
Let the original triangle be $T$ with base $a$ and area $S$.
The line cuts off a smaller triangle $T'$ at the top, which is similar to $T$.
Let the base of $T'$ be $a'$ and its area be $S'$.
We are given that the line divides the triangle into two parts of equal area. This means the area of the smaller triangle $S'$ is half of the total area $S$:
\[ S' = \frac{1}{2}S \]
The ratio of the areas of two similar triangles is equal to the square of the ratio of their corresponding linear dimensions (bases).
\[ \frac{S'}{S} = \left( \frac{a'}{a} \right)^2 \]
Substitute the area ratio:
\[ \frac{1}{2} = \left( \frac{a'}{a} \right)^2 \]
Taking the square root:
\[ \frac{a'}{a} = \frac{1}{\sqrt{2}} = \frac{\sqrt{2}}{2} \]
Solving for $a'$:
\[ a' = a \frac{\sqrt{2}}{2} \]

\subsection*{Answer}
$\frac{a\sqrt{2}}{2}$ (option \textbf{D}).

\end{document}
