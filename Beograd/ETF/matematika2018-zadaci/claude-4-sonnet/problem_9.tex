\documentclass[12pt]{article}
\usepackage[margin=1in]{geometry}
\usepackage{amsmath,amssymb}
\begin{document}

\section*{Problem 9}
If $f(x) = \frac{4}{x - 5} + (3x + 1)e^{-(x-1)^2} + \sqrt{-2x + 3}$, then $f'(1)$ equals:

\subsection*{Solution}
We need to find the derivative of $f(x)$ and evaluate it at $x = 1$.

$f(x) = \frac{4}{x - 5} + (3x + 1)e^{-(x-1)^2} + \sqrt{-2x + 3}$

Let's find the derivative of each term:

First term: $\frac{d}{dx}\left(\frac{4}{x - 5}\right) = \frac{-4}{(x - 5)^2}$

Second term: Using the product rule and chain rule:
\begin{align}
\frac{d}{dx}[(3x + 1)e^{-(x-1)^2}] &= 3e^{-(x-1)^2} + (3x + 1) \cdot e^{-(x-1)^2} \cdot (-2(x-1))\\
&= 3e^{-(x-1)^2} - 2(x-1)(3x + 1)e^{-(x-1)^2}\\
&= e^{-(x-1)^2}[3 - 2(x-1)(3x + 1)]
\end{align}

Third term: $\frac{d}{dx}\sqrt{-2x + 3} = \frac{1}{2\sqrt{-2x + 3}} \cdot (-2) = \frac{-1}{\sqrt{-2x + 3}}$

Therefore:
\[
f'(x) = \frac{-4}{(x - 5)^2} + e^{-(x-1)^2}[3 - 2(x-1)(3x + 1)] + \frac{-1}{\sqrt{-2x + 3}}
\]

Now evaluating at $x = 1$:

First term: $\frac{-4}{(1 - 5)^2} = \frac{-4}{16} = -\frac{1}{4}$

Second term: At $x = 1$:
- $e^{-(1-1)^2} = e^0 = 1$
- $3 - 2(1-1)(3 \cdot 1 + 1) = 3 - 2 \cdot 0 \cdot 4 = 3$
- So this term equals $1 \cdot 3 = 3$

Third term: $\frac{-1}{\sqrt{-2 \cdot 1 + 3}} = \frac{-1}{\sqrt{1}} = -1$

Therefore:
\[
f'(1) = -\frac{1}{4} + 3 + (-1) = -\frac{1}{4} + 2 = \frac{7}{4}
\]

\subsection*{Answer}
$\frac{7}{4}$ (option \textbf{E}).

\end{document}