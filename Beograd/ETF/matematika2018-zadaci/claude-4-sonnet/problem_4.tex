\documentclass[12pt]{article}
\usepackage[margin=1in]{geometry}
\usepackage{amsmath,amssymb}
\begin{document}

\section*{Problem 4}
Calculate the expression:
\[
\left(\frac{x\sqrt{x} - y\sqrt{y}}{\sqrt{x} - \sqrt{y}} + \sqrt{xy}\right) : \left(\frac{\sqrt{x} - \sqrt{y}}{x - y}\right)^{-2}
\]
for all values $x, y \in \mathbb{R}$ for which it is defined, where the identical expression is:

\subsection*{Solution}
First, let's simplify the first part of the expression.

For the numerator $x\sqrt{x} - y\sqrt{y}$:
\[
x\sqrt{x} - y\sqrt{y} = x^{3/2} - y^{3/2} = (\sqrt{x})^3 - (\sqrt{y})^3
\]

Using the identity $a^3 - b^3 = (a-b)(a^2 + ab + b^2)$:
\[
(\sqrt{x})^3 - (\sqrt{y})^3 = (\sqrt{x} - \sqrt{y})(x + \sqrt{xy} + y)
\]

Therefore:
\[
\frac{x\sqrt{x} - y\sqrt{y}}{\sqrt{x} - \sqrt{y}} = \frac{(\sqrt{x} - \sqrt{y})(x + \sqrt{xy} + y)}{\sqrt{x} - \sqrt{y}} = x + \sqrt{xy} + y
\]

So the first part becomes:
\[
x + \sqrt{xy} + y + \sqrt{xy} = x + 2\sqrt{xy} + y = (\sqrt{x} + \sqrt{y})^2
\]

For the second part:
\[
\left(\frac{\sqrt{x} - \sqrt{y}}{x - y}\right)^{-2} = \left(\frac{x - y}{\sqrt{x} - \sqrt{y}}\right)^2
\]

Since $x - y = (\sqrt{x})^2 - (\sqrt{y})^2 = (\sqrt{x} - \sqrt{y})(\sqrt{x} + \sqrt{y})$:
\[
\frac{x - y}{\sqrt{x} - \sqrt{y}} = \frac{(\sqrt{x} - \sqrt{y})(\sqrt{x} + \sqrt{y})}{\sqrt{x} - \sqrt{y}} = \sqrt{x} + \sqrt{y}
\]

Therefore:
\[
\left(\frac{x - y}{\sqrt{x} - \sqrt{y}}\right)^2 = (\sqrt{x} + \sqrt{y})^2
\]

The final result is:
\[
(\sqrt{x} + \sqrt{y})^2 : (\sqrt{x} + \sqrt{y})^2 = 1
\]

\subsection*{Answer}
1 (option \textbf{A}).

\end{document}