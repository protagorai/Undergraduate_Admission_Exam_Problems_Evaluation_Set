\documentclass[12pt]{article}
\usepackage[margin=1in]{geometry}
\usepackage{amsmath,amssymb}
\begin{document}

\section*{Problem 19}
The total number of real solutions of the system $50\left(\frac{2}{5}\right)^{2x+3y-10} + 20\left(\frac{5}{2}\right)^{2x+3y-10} = 133$, $(x + 3)(y - 1) = 6$ is:

\subsection*{Solution}
Let's substitute $u = 2x + 3y - 10$. The first equation becomes:
\[
50\left(\frac{2}{5}\right)^u + 20\left(\frac{5}{2}\right)^u = 133
\]

Since $\left(\frac{5}{2}\right)^u = \left(\frac{2}{5}\right)^{-u}$, let $t = \left(\frac{2}{5}\right)^u$. Then:
\[
50t + 20 \cdot \frac{1}{t} = 133
\]
\[
50t + \frac{20}{t} = 133
\]

Multiplying by $t$:
\[
50t^2 + 20 = 133t
\]
\[
50t^2 - 133t + 20 = 0
\]

Using the quadratic formula:
\[
t = \frac{133 \pm \sqrt{133^2 - 4 \cdot 50 \cdot 20}}{2 \cdot 50} = \frac{133 \pm \sqrt{17689 - 4000}}{100} = \frac{133 \pm \sqrt{13689}}{100} = \frac{133 \pm 117}{100}
\]

So $t = \frac{250}{100} = 2.5$ or $t = \frac{16}{100} = 0.16$.

Case 1: $t = 2.5 = \frac{5}{2}$
$\left(\frac{2}{5}\right)^u = \frac{5}{2} = \left(\frac{2}{5}\right)^{-1}$
Therefore $u = -1$, which gives us $2x + 3y - 10 = -1$, or $2x + 3y = 9$.

Case 2: $t = 0.16 = \frac{2}{5}$
$\left(\frac{2}{5}\right)^u = \frac{2}{5} = \left(\frac{2}{5}\right)^1$
Therefore $u = 1$, which gives us $2x + 3y - 10 = 1$, or $2x + 3y = 11$.

Now we solve each case with the second equation $(x + 3)(y - 1) = 6$.

Case 1: $2x + 3y = 9$ and $(x + 3)(y - 1) = 6$
From the first equation: $y = \frac{9 - 2x}{3}$
Substituting into the second equation:
\[
(x + 3)\left(\frac{9 - 2x}{3} - 1\right) = 6
\]
\[
(x + 3)\left(\frac{9 - 2x - 3}{3}\right) = 6
\]
\[
(x + 3)\left(\frac{6 - 2x}{3}\right) = 6
\]
\[
(x + 3)(6 - 2x) = 18
\]
\[
6x + 18 - 2x^2 - 6x = 18
\]
\[
18 - 2x^2 = 18
\]
\[
-2x^2 = 0
\]
\[
x = 0
\]

When $x = 0$: $y = \frac{9 - 0}{3} = 3$
Check: $(0 + 3)(3 - 1) = 3 \cdot 2 = 6$ ✓

Case 2: $2x + 3y = 11$ and $(x + 3)(y - 1) = 6$
From the first equation: $y = \frac{11 - 2x}{3}$
Substituting into the second equation:
\[
(x + 3)\left(\frac{11 - 2x}{3} - 1\right) = 6
\]
\[
(x + 3)\left(\frac{11 - 2x - 3}{3}\right) = 6
\]
\[
(x + 3)\left(\frac{8 - 2x}{3}\right) = 6
\]
\[
(x + 3)(8 - 2x) = 18
\]
\[
8x + 24 - 2x^2 - 6x = 18
\]
\[
2x + 24 - 2x^2 = 18
\]
\[
-2x^2 + 2x + 6 = 0
\]
\[
x^2 - x - 3 = 0
\]

Using the quadratic formula:
\[
x = \frac{1 \pm \sqrt{1 + 12}}{2} = \frac{1 \pm \sqrt{13}}{2}
\]

So we have two more solutions: $x = \frac{1 + \sqrt{13}}{2}$ and $x = \frac{1 - \sqrt{13}}{2}$.

Total number of real solutions: 3.

\subsection*{Answer}
3 (option \textbf{D}).

\end{document}