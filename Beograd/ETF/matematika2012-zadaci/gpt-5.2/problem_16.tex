\documentclass[12pt]{article}
\usepackage[margin=1in]{geometry}
\usepackage{amsmath,amssymb}
\begin{document}

\section*{Problem 16}
Find the number of real solutions of the equation
\[
\sqrt{3\cdot 2^{\log_{10}(2x)}+1}+\sqrt{2\cdot 2^{\log_{10}(2x)}+9}
=\sqrt{13\cdot 2^{\log_{10}(2x)}-4}.
\]
Options: (A) $0$ \quad (B) $1$ \quad (C) $2$ \quad (D) $3$ \quad (E) none of the above.

\subsection*{Solution}
The expression $\log_{10}(2x)$ requires $x>0$.
Set
\[
u=2^{\log_{10}(2x)}.
\]
As $x>0$ varies, $\log_{10}(2x)$ ranges over all real numbers, hence $u$ ranges over $(0,\infty)$.
The equation becomes
\[
\sqrt{3u+1}+\sqrt{2u+9}=\sqrt{13u-4}.
\]
Square:
\[
3u+1+2u+9+2\sqrt{(3u+1)(2u+9)}=13u-4,
\]
so
\[
2\sqrt{(3u+1)(2u+9)}=8u-14,\qquad \Rightarrow\quad \sqrt{(3u+1)(2u+9)}=4u-7.
\]
Thus we must have $4u-7\ge 0$, i.e. $u\ge \frac74$.
Square again:
\[
(3u+1)(2u+9)=(4u-7)^2.
\]
Compute:
\[
6u^2+29u+9=16u^2-56u+49
\]
\[
\Rightarrow\ 10u^2-85u+40=0 \ \Rightarrow\ 2u^2-17u+8=0.
\]
Discriminant:
\[
\Delta=17^2-4\cdot 2\cdot 8=289-64=225,\quad \sqrt{\Delta}=15,
\]
so
\[
u=\frac{17\pm 15}{4}\in\left\{8,\ \frac12\right\}.
\]
The constraint $u\ge \frac74$ leaves only $u=8$, which indeed satisfies the original equation:
\[
\sqrt{3\cdot 8+1}+\sqrt{2\cdot 8+9}=5+5=10=\sqrt{13\cdot 8-4}.
\]
So there is exactly one $u$, hence exactly one $x$.

\subsection*{Answer}
There is exactly $1$ real solution (option \textbf{B}).

\end{document}

