\documentclass[12pt]{article}
\usepackage[margin=1in]{geometry}
\usepackage{amsmath,amssymb}
\begin{document}

\section*{Problem 15}
Let $i^2=-1$ and let $\varepsilon$ be a complex number satisfying $\varepsilon^2+\varepsilon+1=0$.
Solve for $x$:
\[
\frac{x-1}{x+1}=\varepsilon\cdot \frac{1+i}{1-i}.
\]
Options:
(A) $-2\varepsilon+1-2i$ \quad
(B) $-2\varepsilon-1+2i$ \quad
(C) $-2\varepsilon-1-2i$ \quad
(D) $2\varepsilon+1-2i$ \quad
(E) $2\varepsilon-1-2i$.

\subsection*{Solution}
First simplify
\[
\frac{1+i}{1-i}=\frac{(1+i)^2}{(1-i)(1+i)}=\frac{1+2i+i^2}{1+1}=\frac{2i}{2}=i.
\]
So the equation becomes
\[
\frac{x-1}{x+1}=i\varepsilon.
\]
Solve:
\[
x-1=i\varepsilon(x+1)\ \Rightarrow\ x-i\varepsilon x=1+i\varepsilon
\ \Rightarrow\ x(1-i\varepsilon)=1+i\varepsilon,
\]
thus
\[
x=\frac{1+i\varepsilon}{1-i\varepsilon}.
\]
Rationalize by multiplying numerator and denominator by $(1+i\varepsilon)$:
\[
x=\frac{(1+i\varepsilon)^2}{(1-i\varepsilon)(1+i\varepsilon)}=\frac{1+2i\varepsilon+(i\varepsilon)^2}{1+\varepsilon^2}
=\frac{1+2i\varepsilon-\varepsilon^2}{1+\varepsilon^2}.
\]
Using $\varepsilon^2+\varepsilon+1=0$ we have $\varepsilon^2=-\varepsilon-1$, hence
\[
1+\varepsilon^2=1-\varepsilon-1=-\varepsilon.
\]
Therefore
\[
x=\frac{1+2i\varepsilon-\varepsilon^2}{-\varepsilon}
=-\frac{1}{\varepsilon}-2i+\frac{\varepsilon^2}{\varepsilon}
=-\frac{1}{\varepsilon}-2i+\varepsilon.
\]
Since $\varepsilon^3=1$ for roots of $\varepsilon^2+\varepsilon+1=0$, we have $\frac{1}{\varepsilon}=\varepsilon^2$.
Then
\[
x=\varepsilon-\varepsilon^2-2i.
\]
But $-\varepsilon^2=\varepsilon+1$, so
\[
x=\varepsilon+(\varepsilon+1)-2i=2\varepsilon+1-2i.
\]

\subsection*{Answer}
$2\varepsilon+1-2i$ (option \textbf{D}).

\end{document}

