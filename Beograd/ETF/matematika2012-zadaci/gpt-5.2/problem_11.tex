\documentclass[12pt]{article}
\usepackage[margin=1in]{geometry}
\usepackage{amsmath,amssymb}
\begin{document}

\section*{Problem 11}
Which relation always holds between the roots $x_1,x_2$ of the quadratic equation
\[
x^2-6x+5+m(x^2-5x+6)=0\qquad (m\in\mathbb{R},\ m\neq -1)\ ?
\]
Options:
(A) $x_1+x_2+4x_1x_2=2$ \quad
(B) $x_1+x_2+x_1x_2-11=0$ \quad
(C) $x_1+x_2-x_1x_2+2=0$ \quad
(D) $x_1+x_2-x_1x_2=4$ \quad
(E) $x_1+x_2+3x_1x_2=1$.

\subsection*{Solution}
Expand:
\[
x^2-6x+5+m(x^2-5x+6)=(1+m)x^2+(-6-5m)x+(5+6m).
\]
Since $m\neq -1$, this is a quadratic with
\[
(1+m)x^2+(-6-5m)x+(5+6m)=0.
\]
By Vieta's formulas,
\[
x_1+x_2=\frac{6+5m}{1+m},\qquad x_1x_2=\frac{5+6m}{1+m}.
\]
Add them:
\[
(x_1+x_2)+(x_1x_2)=\frac{(6+5m)+(5+6m)}{1+m}=\frac{11+11m}{1+m}=11,
\]
so
\[
x_1+x_2+x_1x_2-11=0.
\]

\subsection*{Answer}
$x_1+x_2+x_1x_2-11=0$ (option \textbf{B}).

\end{document}

