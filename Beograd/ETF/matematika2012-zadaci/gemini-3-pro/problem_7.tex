\documentclass[12pt]{article}
\usepackage[margin=1in]{geometry}
\usepackage{amsmath,amssymb}
\usepackage[utf8]{inputenc}

\begin{document}

\section*{Problem 7}
Jednačina kruga čiji centar je tačka preseka pravih $x - 2y + 4 = 0$ i $3x + y - 9 = 0$ a koji dodiruje pravu $3x + 4y + 2 = 0$ glasi:

\subsection*{Solution}
First, find the center of the circle $(p, q)$ by solving the system of equations:
1) $x - 2y = -4$
2) $3x + y = 9$

Multiply (2) by 2:
3) $6x + 2y = 18$
Add (1) and (3):
$7x = 14 \implies x = 2$.
Substitute $x=2$ into (2):
$3(2) + y = 9 \implies 6 + y = 9 \implies y = 3$.
So the center is $C(2, 3)$.

The radius $r$ is the distance from the center $C(2, 3)$ to the tangent line $3x + 4y + 2 = 0$.
\[
r = \frac{|Ax_0 + By_0 + C|}{\sqrt{A^2 + B^2}} = \frac{|3(2) + 4(3) + 2|}{\sqrt{3^2 + 4^2}} = \frac{|6 + 12 + 2|}{\sqrt{25}} = \frac{20}{5} = 4
\]
The equation of the circle is $(x-p)^2 + (y-q)^2 = r^2$:
\[
(x-2)^2 + (y-3)^2 = 4^2
\]
\[
x^2 - 4x + 4 + y^2 - 6y + 9 = 16
\]
\[
x^2 + y^2 - 4x - 6y + 13 - 16 = 0
\]
\[
x^2 + y^2 - 4x - 6y - 3 = 0
\]

\subsection*{Answer}
(C) $x^2 + y^2 - 4x - 6y - 3 = 0$

\end{document}
