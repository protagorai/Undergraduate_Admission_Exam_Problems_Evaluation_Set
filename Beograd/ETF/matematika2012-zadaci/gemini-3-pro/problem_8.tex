\documentclass[12pt]{article}
\usepackage[margin=1in]{geometry}
\usepackage{amsmath,amssymb}
\usepackage[utf8]{inputenc}

\begin{document}

\section*{Problem 8}
Pravilna četvorostrana prizma presečena je sa ravni koja sadrži osnovnu ivicu prizme. Ako je površina preseka ravni i prizme dva puta veća od površine baze, tada je ugao između te ravni i baze prizme jednak:

\subsection*{Solution}
Let $B$ be the area of the base (a square). Let $S$ be the area of the cross-section.
The problem states that the plane contains a base edge. The projection of the cross-section onto the base plane is exactly the base itself (assuming the section cuts through the entire prism above the base, intersecting the opposite lateral face or edge).
The relationship between the area of a polygon $S$ in a plane and its projection $S_{proj}$ onto another plane is given by:
\[
S_{proj} = S \cos \alpha
\]
where $\alpha$ is the angle between the planes.
Here $S_{proj} = B$ and $S = 2B$.
\[
B = 2B \cos \alpha
\]
\[
\cos \alpha = \frac{1}{2}
\]
\[
\alpha = 60^\circ
\]

\subsection*{Answer}
(D) $60^\circ$

\end{document}
