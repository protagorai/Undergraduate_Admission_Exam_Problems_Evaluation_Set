\documentclass[12pt]{article}
\usepackage[margin=1in]{geometry}
\usepackage{amsmath,amssymb}
\usepackage[utf8]{inputenc}

\begin{document}

\section*{Problem 5}
Ako se zna da je polinom $x^3 + ax^2 + bx - 4$ ($a, b \in \mathbf{R}$) deljiv polinomom $x^2 - 1$, tada zbir $a^2 + b^2$ iznosi:

\subsection*{Solution}
Let $P(x) = x^3 + ax^2 + bx - 4$.
Since $P(x)$ is divisible by $x^2 - 1 = (x-1)(x+1)$, it must be that $P(1) = 0$ and $P(-1) = 0$.
1) $P(1) = 1 + a + b - 4 = 0 \implies a + b = 3$
2) $P(-1) = -1 + a - b - 4 = 0 \implies a - b = 5$

Adding the two equations:
\[
2a = 8 \implies a = 4
\]
Subtracting the second from the first:
\[
2b = -2 \implies b = -1
\]
We need to find $a^2 + b^2$:
\[
a^2 + b^2 = 4^2 + (-1)^2 = 16 + 1 = 17
\]

\subsection*{Answer}
(B) 17

\end{document}
