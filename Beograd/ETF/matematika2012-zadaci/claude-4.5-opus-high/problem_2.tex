\documentclass[12pt]{article}
\usepackage[margin=1in]{geometry}
\usepackage{amsmath,amssymb}
\begin{document}

\section*{Problem 2}
Ukupan broj dijagonala pravilnog desetougla je:

(A) 15 \quad (B) 20 \quad (C) 25 \quad (D) 30 \quad (E) 35 \quad (N) Ne znam

\subsection*{Solution}
A regular polygon with $n$ vertices has a total number of diagonals given by the formula:
\[
D = \frac{n(n-3)}{2}
\]

This formula comes from the fact that from each vertex we can draw $n-3$ diagonals (we cannot connect to itself or to the two adjacent vertices), giving $n(n-3)$ connections, but we divide by 2 because each diagonal is counted twice.

For a decagon (desetougao), $n = 10$:
\[
D = \frac{10(10-3)}{2} = \frac{10 \cdot 7}{2} = \frac{70}{2} = 35
\]

\subsection*{Answer}
$35$ (option \textbf{E}).

\end{document}
