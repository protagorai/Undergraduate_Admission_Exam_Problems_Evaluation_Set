\documentclass[12pt]{article}
\usepackage[margin=1in]{geometry}
\usepackage{amsmath,amssymb}
\begin{document}

\section*{Problem 19}
Na koliko načina se u red mogu poređati 5 učenika i 2 učenice, tako da učenice ne stoje jedna pored druge?

(A) 240 \quad (B) 3600 \quad (C) 7680 \quad (D) 2400 \quad (E) 250 \quad (N) Ne znam

\subsection*{Solution}
We need to arrange 5 students (boys) and 2 students (girls) such that the 2 girls are not adjacent.

\textbf{Method: Total arrangements minus arrangements where girls are together.}

Total arrangements of 7 people: $7! = 5040$

Arrangements where the 2 girls are together: Treat the 2 girls as one unit. We have 6 units to arrange: $6!$ ways. The 2 girls can be arranged within their unit in $2!$ ways.

So arrangements with girls together: $6! \cdot 2! = 720 \cdot 2 = 1440$

Arrangements where girls are NOT together: $5040 - 1440 = 3600$

\textbf{Alternative method: Direct counting.}

First, arrange the 5 boys: $5! = 120$ ways.

This creates 6 gaps (including the ends): $\_B\_B\_B\_B\_B\_$

We need to place 2 girls in these 6 gaps, choosing 2 different gaps:
\[
P(6,2) = 6 \cdot 5 = 30
\]

Total: $120 \cdot 30 = 3600$

\subsection*{Answer}
$3600$ (option \textbf{B}).

\end{document}
