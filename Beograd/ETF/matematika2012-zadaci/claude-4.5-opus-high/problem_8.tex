\documentclass[12pt]{article}
\usepackage[margin=1in]{geometry}
\usepackage{amsmath,amssymb}
\begin{document}

\section*{Problem 8}
Pravilna četvorostrana prizma presečena je sa ravni koja sadrži osnovnu ivicu prizme. Ako je površina preseka ravni i prizme dva puta veća od površine baze, tada je ugao između te ravni i baze prizme jednak:

(A) $15°$ \quad (B) $30°$ \quad (C) $45°$ \quad (D) $60°$ \quad (E) $75°$ \quad (N) Ne znam

\subsection*{Solution}
Let the base of the regular quadrilateral prism (square prism) have side $a$. The base area is $a^2$.

The cutting plane contains one edge of the base (of length $a$) and cuts through the prism. The cross-section is a rectangle with one side being the base edge $a$ and the other side being the slant height from the base to where the plane exits the prism.

If the angle between the cutting plane and the base is $\alpha$, and the plane contains a base edge, then the cross-section is a rectangle. The width is $a$ (the base edge), and the height in the cutting plane depends on the angle.

For a plane containing one base edge and making angle $\alpha$ with the base, the cross-section is a parallelogram. When the plane passes through the opposite edge at the top, the cross-section becomes a rectangle with dimensions $a$ (base edge) and $\frac{a}{\cos\alpha}$ (the slant across).

Actually, let's reconsider. The cross-section through a base edge at angle $\alpha$ creates a rectangle with base $a$ and the other dimension being $\frac{a}{\sin\alpha}$ where $a$ is the distance to the opposite edge.

If the cross-section area is $2a^2$ and one side is $a$, the other side must be $2a$.

For a plane through one base edge at angle $\alpha$, the cross-section width perpendicular to that edge in the slant plane is $\frac{a}{\sin\alpha}$.

So: $a \cdot \frac{a}{\sin\alpha} = 2a^2$

This gives: $\frac{1}{\sin\alpha} = 2$, so $\sin\alpha = \frac{1}{2}$, meaning $\alpha = 30°$.

\subsection*{Answer}
$30°$ (option \textbf{B}).

\end{document}
