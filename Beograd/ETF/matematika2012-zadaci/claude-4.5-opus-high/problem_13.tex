\documentclass[12pt]{article}
\usepackage[margin=1in]{geometry}
\usepackage{amsmath,amssymb}
\begin{document}

\section*{Problem 13}
Ako je $\log_2 3 = a$ i $\log_5 2 = b$, tada je $\log_{24} 50$ jednako:

(A) $\frac{b+2}{b(a+3)}$ \quad (B) $\frac{b+1}{b(a+4)}$ \quad (C) $\frac{b-2}{b(a-4)}$ \quad (D) $\frac{b+1}{b(a+3)}$ \quad (E) $\frac{b-2}{(b+1)(a+3)}$ \quad (N) Ne znam

\subsection*{Solution}
We have $\log_2 3 = a$, so $2^a = 3$.

We have $\log_5 2 = b$, so $5^b = 2$, which means $\log_2 5 = \frac{1}{b}$.

We need to find $\log_{24} 50$.

Using change of base formula:
\[
\log_{24} 50 = \frac{\log_2 50}{\log_2 24}
\]

Calculate $\log_2 50$:
\[
\log_2 50 = \log_2 (2 \cdot 25) = \log_2 2 + \log_2 25 = 1 + 2\log_2 5 = 1 + \frac{2}{b}
\]

Calculate $\log_2 24$:
\[
\log_2 24 = \log_2 (8 \cdot 3) = \log_2 8 + \log_2 3 = 3 + a
\]

Therefore:
\[
\log_{24} 50 = \frac{1 + \frac{2}{b}}{3 + a} = \frac{\frac{b+2}{b}}{a+3} = \frac{b+2}{b(a+3)}
\]

\subsection*{Answer}
$\frac{b+2}{b(a+3)}$ (option \textbf{A}).

\end{document}
