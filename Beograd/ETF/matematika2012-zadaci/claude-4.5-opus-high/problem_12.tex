\documentclass[12pt]{article}
\usepackage[margin=1in]{geometry}
\usepackage{amsmath,amssymb}
\begin{document}

\section*{Problem 12}
Vrednost izraza $8\sin^2 80° - 2\sqrt{3}\sin 40° - 2\cos 40°$ jednaka je:

(A) 2 \quad (B) $2\sqrt{3}$ \quad (C) $4\sqrt{3}$ \quad (D) 4 \quad (E) 1 \quad (N) Ne znam

\subsection*{Solution}
Let's denote $\alpha = 40°$. Then $80° = 2\alpha$, and we need to find:
\[
8\sin^2 2\alpha - 2\sqrt{3}\sin\alpha - 2\cos\alpha
\]

Using $\sin^2 2\alpha = \frac{1 - \cos 4\alpha}{2}$:
\[
8 \cdot \frac{1 - \cos 4\alpha}{2} = 4(1 - \cos 160°) = 4(1 - \cos 160°)
\]

Since $\cos 160° = -\cos 20°$:
\[
4(1 + \cos 20°)
\]

Also, we can write:
\[
2\sqrt{3}\sin 40° + 2\cos 40° = 4\left(\frac{\sqrt{3}}{2}\sin 40° + \frac{1}{2}\cos 40°\right) = 4\sin(40° + 30°) = 4\sin 70°
\]

Since $\sin 70° = \cos 20°$:
\[
2\sqrt{3}\sin 40° + 2\cos 40° = 4\cos 20°
\]

Now, $8\sin^2 80° = 8 \cdot \sin^2 80° = 4(1 - \cos 160°) = 4(1 + \cos 20°) = 4 + 4\cos 20°$

So the expression becomes:
\[
4 + 4\cos 20° - 4\cos 20° = 4
\]

\subsection*{Answer}
$4$ (option \textbf{D}).

\end{document}
