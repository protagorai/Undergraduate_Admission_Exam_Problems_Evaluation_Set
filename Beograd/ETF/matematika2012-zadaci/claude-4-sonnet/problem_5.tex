\documentclass[12pt]{article}
\usepackage[margin=1in]{geometry}
\usepackage{amsmath,amssymb}
\begin{document}

\section*{Problem 5}
Ako se zna da je polinom $x^3 + ax^2 + bx - 4$ $(a, b \in \mathbb{R})$ deljiv polinomom $x^2 - 1$, tada zbir $a^2 + b^2$ iznosi:

\subsection*{Solution}
Polinom $x^2 - 1 = (x-1)(x+1)$, tako da ako je polinom $P(x) = x^3 + ax^2 + bx - 4$ deljiv sa $x^2 - 1$, tada mora biti deljiv i sa $(x-1)$ i sa $(x+1)$.

To znači da su $x = 1$ i $x = -1$ nule polinoma $P(x)$.

Iz uslova $P(1) = 0$:
\[
1^3 + a \cdot 1^2 + b \cdot 1 - 4 = 0
\]
\[
1 + a + b - 4 = 0
\]
\[
a + b = 3 \quad \text{...(1)}
\]

Iz uslova $P(-1) = 0$:
\[
(-1)^3 + a \cdot (-1)^2 + b \cdot (-1) - 4 = 0
\]
\[
-1 + a - b - 4 = 0
\]
\[
a - b = 5 \quad \text{...(2)}
\]

Rešavamo sistem jednačina (1) i (2):
Sabiranje: $(a + b) + (a - b) = 3 + 5$
\[
2a = 8 \Rightarrow a = 4
\]

Iz jednačine (1): $4 + b = 3 \Rightarrow b = -1$

Proveravamo: $a - b = 4 - (-1) = 5$ ✓

Dakle, $a = 4$ i $b = -1$.

\[
a^2 + b^2 = 4^2 + (-1)^2 = 16 + 1 = 17
\]

\subsection*{Answer}
17 (opcija \textbf{B}).

\end{document}