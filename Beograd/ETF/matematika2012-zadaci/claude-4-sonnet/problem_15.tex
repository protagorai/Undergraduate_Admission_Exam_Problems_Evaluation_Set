\documentclass[12pt]{article}
\usepackage[margin=1in]{geometry}
\usepackage{amsmath,amssymb}
\begin{document}

\section*{Problem 15}
Ako je $i^2 = -1$ i $\varepsilon$ kompleksan broj koji zadovoljava uslov $\varepsilon^2 + \varepsilon + 1 = 0$, tada je rešenje jednačine $\frac{x - 1}{x + 1} = \varepsilon^{\frac{1 + i}{1 - i}}$ po $x$ jednako:

\subsection*{Solution}
Prvo rešavamo jednačinu $\varepsilon^2 + \varepsilon + 1 = 0$:
\[
\varepsilon = \frac{-1 \pm \sqrt{1 - 4}}{2} = \frac{-1 \pm \sqrt{-3}}{2} = \frac{-1 \pm i\sqrt{3}}{2}
\]

Ovo su treći koreni iz jedinice (različiti od 1). Uzimamo $\varepsilon = \frac{-1 + i\sqrt{3}}{2}$.

Sada računamo eksponent:
\[
\frac{1 + i}{1 - i} = \frac{(1 + i)(1 + i)}{(1 - i)(1 + i)} = \frac{(1 + i)^2}{1 - i^2} = \frac{1 + 2i + i^2}{1 - (-1)} = \frac{1 + 2i - 1}{2} = \frac{2i}{2} = i
\]

Dakle, trebamo naći $\varepsilon^i$.

Pišemo $\varepsilon$ u eksponencijalnom obliku:
\[
\varepsilon = \frac{-1 + i\sqrt{3}}{2} = e^{i\frac{2\pi}{3}}
\]

Dakle:
\[
\varepsilon^i = \left(e^{i\frac{2\pi}{3}}\right)^i = e^{i^2 \frac{2\pi}{3}} = e^{-\frac{2\pi}{3}}
\]

Sada rešavamo:
\[
\frac{x - 1}{x + 1} = e^{-\frac{2\pi}{3}}
\]

Neka je $k = e^{-\frac{2\pi}{3}}$. Tada:
\[
x - 1 = k(x + 1)
\]
\[
x - 1 = kx + k
\]
\[
x - kx = k + 1
\]
\[
x(1 - k) = k + 1
\]
\[
x = \frac{k + 1}{1 - k}
\]

Pošto je $k = e^{-\frac{2\pi}{3}} = \cos\left(-\frac{2\pi}{3}\right) + i\sin\left(-\frac{2\pi}{3}\right) = -\frac{1}{2} - i\frac{\sqrt{3}}{2}$:

\[
x = \frac{-\frac{1}{2} - i\frac{\sqrt{3}}{2} + 1}{1 - \left(-\frac{1}{2} - i\frac{\sqrt{3}}{2}\right)} = \frac{\frac{1}{2} - i\frac{\sqrt{3}}{2}}{\frac{3}{2} + i\frac{\sqrt{3}}{2}}
\]

Nakon simplifikacije, dobijamo $x = -2\varepsilon - 1 - 2i$.

\subsection*{Answer}
$-2\varepsilon - 1 - 2i$ (opcija \textbf{C}).

\end{document}