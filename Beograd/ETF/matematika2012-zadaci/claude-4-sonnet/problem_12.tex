\documentclass[12pt]{article}
\usepackage[margin=1in]{geometry}
\usepackage{amsmath,amssymb}
\begin{document}

\section*{Problem 12}
Vrednost izraza $8\sin^2 80° - 2\sqrt{3}\sin 40° - 2\cos 40°$ jednaka je:

\subsection*{Solution}
Koristimo identitete i transformacije.

Prvo, $\sin^2 80° = \sin^2(90° - 10°) = \cos^2 10°$.

Takođe, $\sin 40° = \sin(90° - 50°) = \cos 50°$ i $\cos 40° = \cos(90° - 50°) = \sin 50°$.

Dakle, izraz postaje:
\[
8\cos^2 10° - 2\sqrt{3}\cos 50° - 2\sin 50°
\]

Koristimo identitet $\cos^2 \alpha = \frac{1 + \cos 2\alpha}{2}$:
\[
8 \cdot \frac{1 + \cos 20°}{2} - 2\sqrt{3}\cos 50° - 2\sin 50°
\]
\[
= 4(1 + \cos 20°) - 2\sqrt{3}\cos 50° - 2\sin 50°
\]
\[
= 4 + 4\cos 20° - 2\sqrt{3}\cos 50° - 2\sin 50°
\]

Sada, $\cos 50° = \sin 40°$ i $\sin 50° = \cos 40°$:
\[
= 4 + 4\cos 20° - 2\sqrt{3}\sin 40° - 2\cos 40°
\]

Koristimo identitet za kombinaciju sinusa i kosinusa:
$a\sin \theta + b\cos \theta = \sqrt{a^2 + b^2}\sin(\theta + \phi)$ gde je $\tan \phi = \frac{b}{a}$.

Za $-2\sqrt{3}\sin 40° - 2\cos 40°$:
$a = -2\sqrt{3}$, $b = -2$

$\sqrt{a^2 + b^2} = \sqrt{12 + 4} = 4$

$\tan \phi = \frac{-2}{-2\sqrt{3}} = \frac{1}{\sqrt{3}} \Rightarrow \phi = 30°$

Dakle:
$-2\sqrt{3}\sin 40° - 2\cos 40° = -4\sin(40° + 30°) = -4\sin 70°$

Izraz postaje:
\[
4 + 4\cos 20° - 4\sin 70°
\]

Pošto je $\sin 70° = \cos 20°$:
\[
= 4 + 4\cos 20° - 4\cos 20° = 4
\]

\subsection*{Answer}
4 (opcija \textbf{D}).

\end{document}