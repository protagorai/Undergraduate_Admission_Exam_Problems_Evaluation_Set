\documentclass[12pt]{article}
\usepackage[margin=1in]{geometry}
\usepackage{amsmath,amssymb}
\begin{document}

\section*{Problem 2}
Ukupan broj dijagonala pravilnog desetougla je:

\subsection*{Solution}
Pravilni desetougao ima 10 temena.

Broj dijagonala u n-touglu se računa po formuli:
\[
\text{Broj dijagonala} = \frac{n(n-3)}{2}
\]

Ova formula se izvodi na sledeći način:
- Iz svakog temena možemo povući $n-1$ duži do ostalih temena
- Od tih $n-1$ duži, 2 su stranice (do susednih temena)
- Dakle, iz svakog temena možemo povući $n-3$ dijagonale
- Ukupno imamo $n(n-3)$ dijagonala, ali svaku smo brojali dva puta
- Zato delimo sa 2: $\frac{n(n-3)}{2}$

Za $n = 10$:
\[
\text{Broj dijagonala} = \frac{10(10-3)}{2} = \frac{10 \cdot 7}{2} = \frac{70}{2} = 35
\]

\subsection*{Answer}
35 (opcija \textbf{E}).

\end{document}