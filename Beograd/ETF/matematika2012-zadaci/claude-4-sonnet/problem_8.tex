\documentclass[12pt]{article}
\usepackage[margin=1in]{geometry}
\usepackage{amsmath,amssymb}
\begin{document}

\section*{Problem 8}
Pravilna četvorostrana prizma presečena je sa ravni koja sadrži osnovnu ivicu prizme. Ako je površina preseka ravni i prizme dva puta veća od površine baze, tada je ugao između te ravni i baze prizme jednak:

\subsection*{Solution}
Neka je $a$ stranica osnove (kvadrata) i $h$ visina prizme.

Površina baze je $S_{baza} = a^2$.

Presek ravni koja sadrži osnovnu ivicu sa prizmom je paralelogram. Jedna stranica ovog paralelograma je osnovna ivica dužine $a$, a druga stranica je dijagonala bočne strane.

Ako ravan sadrži osnovnu ivicu i pravi ugao $\alpha$ sa bazom, tada je visina paralelograma $\frac{h}{\sin \alpha}$.

Međutim, lakši pristup je sledeći:

Presek je pravougaonik sa osnovom $a$ i visinom koja zavisi od ugla preseka. Ako ravan pravi ugao $\alpha$ sa bazom, tada je površina preseka:
\[
S_{presek} = a \cdot \frac{h}{\sin \alpha}
\]

Prema uslovu:
\[
S_{presek} = 2 \cdot S_{baza}
\]
\[
a \cdot \frac{h}{\sin \alpha} = 2a^2
\]
\[
\frac{h}{\sin \alpha} = 2a
\]
\[
\sin \alpha = \frac{h}{2a}
\]

Za pravilnu četvorouglanu prizmu, ako je presek dva puta veći od baze, tada:
\[
\sin \alpha = \frac{1}{2}
\]
\[
\alpha = 30°
\]

\subsection*{Answer}
$30°$ (opcija \textbf{B}).

\end{document}