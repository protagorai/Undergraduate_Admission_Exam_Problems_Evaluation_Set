\documentclass[12pt]{article}
\usepackage[margin=1in]{geometry}
\usepackage{amsmath,amssymb}
\begin{document}

\section*{Problem 1}
Vrednost izraza $\sqrt{-89}$ nalazi se između brojeva:

(A) $-10$ i $-9$ \quad (B) $-9$ i $-8$ \quad (C) $-5$ i $-4$ \quad (D) $-4$ i $-3$ \quad (E) $-3$ i $-2$ \quad (N) Ne znam

\subsection*{Solution}
Izraz $\sqrt{-89}$ nije definisan u skupu realnih brojeva jer se ne može naći kvadratni koren iz negativnog broja u realnom domenu.

Međutim, ako posmatramo kompleksne brojeve, imamo:
\[
\sqrt{-89} = i\sqrt{89}
\]

Pošto je $\sqrt{89} \approx 9.43$, imamo $\sqrt{-89} = 9.43i$.

Ako se traži apsolutna vrednost, tada je $|\sqrt{-89}| = \sqrt{89} \approx 9.43$.

Pošto je $9^2 = 81$ i $10^2 = 100$, imamo $9 < \sqrt{89} < 10$.

Međutim, pošto se traži gde se nalazi vrednost izraza, a izraz nije realan broj, odgovor je (N) Ne znam.

\subsection*{Answer}
(N) Ne znam

\end{document}