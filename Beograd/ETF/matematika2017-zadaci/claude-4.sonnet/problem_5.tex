\documentclass[12pt]{article}
\usepackage[margin=1in]{geometry}
\usepackage{amsmath,amssymb}
\begin{document}

\section*{Problem 5}
Neka je $P(x)$ polinom najmaneg stepena čiji su koeficijenti realni brojevi, a koreni $-1$ i $2i$. Ako je $P(0) = -12$, tada je $P(-2)$ jednako:

(A) $-12$ \quad (B) $24$ \quad (C) $-30$ \quad (D) $-36$ \quad (E) $72$ \quad (N) Ne znam

\subsection*{Solution}
Pošto polinom ima realne koeficijente i $2i$ je koren, tada je i $-2i$ koren (kompleksni koreni se javljaju u konjugovanim parovima).

Dakle, koreni su: $-1$, $2i$, $-2i$.

Polinom najmaneg stepena je:
\[
P(x) = a(x + 1)(x - 2i)(x + 2i) = a(x + 1)(x^2 + 4)
\]

Razvijamo:
\[
P(x) = a(x + 1)(x^2 + 4) = a(x^3 + 4x + x^2 + 4) = a(x^3 + x^2 + 4x + 4)
\]

Iz uslova $P(0) = -12$:
\[
P(0) = a(0 + 0 + 0 + 4) = 4a = -12
\]
\[
a = -3
\]

Dakle:
\[
P(x) = -3(x^3 + x^2 + 4x + 4)
\]

Izračunajmo $P(-2)$:
\[
P(-2) = -3((-2)^3 + (-2)^2 + 4(-2) + 4)
\]
\[
P(-2) = -3(-8 + 4 - 8 + 4) = -3(-8) = 24
\]

\subsection*{Answer}
(B) $24$

\end{document}