\documentclass[12pt]{article}
\usepackage[margin=1in]{geometry}
\usepackage{amsmath,amssymb}
\begin{document}

\section*{Problem 2}
Ako je $V$ zapremina lopte, tada je njena površina jednaka:

(A) $\sqrt[3]{36\pi V^2}$ \quad (B) $\sqrt[3]{\pi V^2}$ \quad (C) $\sqrt[3]{\frac{\pi V^2}{2}}$ \quad (D) $\sqrt[3]{\frac{3\pi V^2}{4}}$ \quad (E) $\sqrt[3]{2\pi V^2}$ \quad (N) Ne znam

\subsection*{Solution}
Za loptu poluprečnika $r$:
- Zapremina: $V = \frac{4}{3}\pi r^3$
- Površina: $S = 4\pi r^2$

Iz formule za zapreminu:
\[
V = \frac{4}{3}\pi r^3 \Rightarrow r^3 = \frac{3V}{4\pi} \Rightarrow r = \sqrt[3]{\frac{3V}{4\pi}}
\]

Površina je:
\[
S = 4\pi r^2 = 4\pi \left(\sqrt[3]{\frac{3V}{4\pi}}\right)^2 = 4\pi \left(\frac{3V}{4\pi}\right)^{2/3}
\]

\[
S = 4\pi \cdot \frac{(3V)^{2/3}}{(4\pi)^{2/3}} = 4\pi \cdot \frac{3^{2/3}V^{2/3}}{4^{2/3}\pi^{2/3}} = \frac{4\pi \cdot 3^{2/3}V^{2/3}}{4^{2/3}\pi^{2/3}}
\]

\[
S = \frac{4^{1/3}\pi^{1/3} \cdot 3^{2/3}V^{2/3}}{1} = (4\pi)^{1/3} \cdot 3^{2/3}V^{2/3} = \sqrt[3]{4\pi \cdot 9V^2} = \sqrt[3]{36\pi V^2}
\]

\subsection*{Answer}
(A) $\sqrt[3]{36\pi V^2}$

\end{document}