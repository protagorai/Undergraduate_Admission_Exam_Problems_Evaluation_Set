\documentclass[12pt]{article}
\usepackage[margin=1in]{geometry}
\usepackage{amsmath,amssymb}
\begin{document}

\section*{Problem 13}
Find the acute angle at which the tangent to the curve $2x^2 + 3y^2 = 5$ at point $(1, 1)$ and the tangent to the curve $x^2 - 4x + y^2 + 6y = 3$ at point $(2, 1)$ intersect.

(A) $\arctan\frac{1}{2}$ \quad (B) $\arctan\frac{3}{2}$ \quad (C) $\arctan\frac{2}{3}$ \quad (D) $\arctan\frac{1}{6}$ \quad (E) $\arctan\frac{3}{5}$

\subsection*{Solution}
\textbf{First curve:} $2x^2 + 3y^2 = 5$

Implicit differentiation: $4x + 6y\frac{dy}{dx} = 0$

At $(1, 1)$: $4 + 6\frac{dy}{dx} = 0 \Rightarrow \frac{dy}{dx} = -\frac{2}{3}$

So slope $m_1 = -\frac{2}{3}$.

\textbf{Second curve:} $x^2 - 4x + y^2 + 6y = 3$

Implicit differentiation: $2x - 4 + 2y\frac{dy}{dx} + 6\frac{dy}{dx} = 0$

At $(2, 1)$: $4 - 4 + 2\frac{dy}{dx} + 6\frac{dy}{dx} = 0 \Rightarrow 8\frac{dy}{dx} = 0 \Rightarrow \frac{dy}{dx} = 0$

So slope $m_2 = 0$.

\textbf{Angle between lines:}
The angle $\theta$ between two lines with slopes $m_1$ and $m_2$ is given by:
\[
\tan\theta = \left|\frac{m_1 - m_2}{1 + m_1 m_2}\right|
\]

\[
\tan\theta = \left|\frac{-\frac{2}{3} - 0}{1 + (-\frac{2}{3})(0)}\right| = \left|\frac{-\frac{2}{3}}{1}\right| = \frac{2}{3}
\]

Therefore:
\[
\theta = \arctan\frac{2}{3}
\]

\subsection*{Answer}
(C) $\arctan\frac{2}{3}$

\end{document}
