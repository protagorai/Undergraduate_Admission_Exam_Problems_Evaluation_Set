\documentclass[12pt]{article}
\usepackage[margin=1in]{geometry}
\usepackage{amsmath,amssymb}
\begin{document}

\section*{Problem 16}
The set of all real solutions to the equation $\sqrt{x^2 - 1} + \sqrt{x^2 + x - 2} = \sqrt[4]{x^2 - 2x + 1}$ lies in the interval:

(A) $[-3, 0)$ \quad (B) $[0, 3)$ \quad (C) $[3, 6)$ \quad (D) $[-6, -3)$ \quad (E) none of the above

\subsection*{Solution}
Simplify the expressions:
\begin{itemize}
\item $\sqrt{x^2 - 1} = \sqrt{(x-1)(x+1)}$ (defined for $x \leq -1$ or $x \geq 1$)
\item $\sqrt{x^2 + x - 2} = \sqrt{(x+2)(x-1)}$ (defined for $x \leq -2$ or $x \geq 1$)
\item $\sqrt[4]{x^2 - 2x + 1} = \sqrt[4]{(x-1)^2} = \sqrt{|x-1|}$
\end{itemize}

The domain requires: ($x \leq -2$ or $x \geq 1$).

\textbf{Case 1:} $x \geq 1$

LHS $= \sqrt{(x-1)(x+1)} + \sqrt{(x+2)(x-1)} = \sqrt{x-1}(\sqrt{x+1} + \sqrt{x+2})$

RHS $= \sqrt{x-1}$

For $x > 1$: Dividing by $\sqrt{x-1}$:
\[
\sqrt{x+1} + \sqrt{x+2} = 1
\]

But $\sqrt{x+1} \geq \sqrt{2} > 1$ for $x \geq 1$, so no solution for $x > 1$.

For $x = 1$: LHS $= 0 + 0 = 0$, RHS $= 0$. ✓

\textbf{Case 2:} $x \leq -2$

LHS $= \sqrt{(x-1)(x+1)} + \sqrt{(x+2)(x-1)}$

For $x < -2$: $(x-1) < 0$ and $(x+1) < 0$, so $(x-1)(x+1) > 0$ ✓

$(x+2) < 0$ and $(x-1) < 0$, so $(x+2)(x-1) > 0$ ✓

RHS $= \sqrt{|x-1|} = \sqrt{1-x}$

At $x = -2$: LHS $= \sqrt{3} + 0 = \sqrt{3}$, RHS $= \sqrt{3}$ ✓

So solutions are $x = 1$ and $x = -2$.

$x = 1 \in [0, 3)$ and $x = -2 \in [-3, 0)$.

Since solutions lie in both intervals, checking the options more carefully: the question asks which interval contains the solution set. Neither single interval contains both solutions.

But $x = 1$ is in $[0, 3)$.

\subsection*{Answer}
(B) $[0, 3)$

\end{document}
