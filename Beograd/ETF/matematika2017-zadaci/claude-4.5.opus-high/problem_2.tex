\documentclass[12pt]{article}
\usepackage[margin=1in]{geometry}
\usepackage{amsmath,amssymb}
\begin{document}

\section*{Problem 2}
If $V$ is the volume of a sphere, then its surface area equals:

(A) $\sqrt[3]{36\pi V^2}$ \quad (B) $\sqrt[3]{\pi V^2}$ \quad (C) $\sqrt[3]{\frac{\pi V^2}{2}}$ \quad (D) $\sqrt[3]{\frac{3\pi V^2}{4}}$ \quad (E) $\sqrt[3]{2\pi V^2}$

\subsection*{Solution}
The volume of a sphere with radius $r$ is:
\[
V = \frac{4}{3}\pi r^3
\]

From this, we can express $r^3$:
\[
r^3 = \frac{3V}{4\pi}
\]

The surface area of a sphere is:
\[
S = 4\pi r^2
\]

We need to express $r^2$ in terms of $V$. From $r^3 = \frac{3V}{4\pi}$:
\[
r = \sqrt[3]{\frac{3V}{4\pi}}
\]

Therefore:
\[
r^2 = \left(\frac{3V}{4\pi}\right)^{2/3}
\]

The surface area becomes:
\[
S = 4\pi \cdot \left(\frac{3V}{4\pi}\right)^{2/3} = 4\pi \cdot \frac{(3V)^{2/3}}{(4\pi)^{2/3}} = (4\pi)^{1/3} \cdot (3V)^{2/3}
\]

Let's simplify:
\[
S = \sqrt[3]{4\pi} \cdot \sqrt[3]{9V^2} = \sqrt[3]{36\pi V^2}
\]

\subsection*{Answer}
(A) $\sqrt[3]{36\pi V^2}$

\end{document}
