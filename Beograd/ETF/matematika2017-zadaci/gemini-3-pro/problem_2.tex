\documentclass[12pt]{article}
\usepackage[margin=1in]{geometry}
\usepackage{amsmath,amssymb}
\begin{document}

\section*{Problem 2}
Ako je $V$ zapremina lopte, tada je njena površina jednaka:
(A) $\sqrt[3]{36\pi V^2}$
(B) $\sqrt[3]{\pi V^2}$
(C) $\sqrt[3]{\frac{\pi V^2}{2}}$
(D) $\sqrt[3]{\frac{3\pi V^2}{4}}$
(E) $\sqrt[3]{2\pi V^2}$
(N) Ne znam

\subsection*{Solution}
The volume of a sphere is given by $V = \frac{4}{3}\pi r^3$, and its surface area is $S = 4\pi r^2$.
From the volume formula, we can express $r$ in terms of $V$:
\[
r^3 = \frac{3V}{4\pi} \implies r = \sqrt[3]{\frac{3V}{4\pi}}.
\]
Substitute this into the surface area formula:
\[
S = 4\pi \left(\sqrt[3]{\frac{3V}{4\pi}}\right)^2 = 4\pi \left(\frac{3V}{4\pi}\right)^{2/3}.
\]
To simplify, write $4\pi$ as $(64\pi^3)^{1/3}$ or bring it inside the cube root:
\[
S = \sqrt[3]{(4\pi)^3 \cdot \left(\frac{3V}{4\pi}\right)^2} = \sqrt[3]{64\pi^3 \cdot \frac{9V^2}{16\pi^2}} = \sqrt[3]{4\pi \cdot 9V^2} = \sqrt[3]{36\pi V^2}.
\]

\subsection*{Answer}
$\sqrt[3]{36\pi V^2}$ (option \textbf{A}).

\end{document}
