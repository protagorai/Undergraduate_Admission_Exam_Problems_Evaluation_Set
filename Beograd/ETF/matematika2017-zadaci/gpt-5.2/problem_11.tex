\section*{Zadatak 11}
U aritmeti\v{c}koj progresiji poznati su \v{c}lanovi $a_{54}=\alpha$ i $a_{70}=\beta$ ($\alpha,\beta\in\mathbb{R}$, $\alpha\neq \beta$).
Odrediti zbir prvih $160$ \v{c}lanova te progresije.

\subsection*{Re\v{s}enje}
Za aritmeti\v{c}ku progresiju va\v{z}i
\[
a_n=a_1+(n-1)d.
\]
Iz uslova dobijamo sistem
\[
a_1+53d=\alpha,\qquad a_1+69d=\beta.
\]
Oduzimanjem:
\[
16d=\beta-\alpha \;\Longrightarrow\; d=\frac{\beta-\alpha}{16}.
\]
Zatim
\[
a_1=\alpha-53d=\alpha-53\cdot\frac{\beta-\alpha}{16}.
\]
Zbir prvih $160$ \v{c}lanova je
\[
S_{160}=\frac{160}{2}\bigl(2a_1+159d\bigr)=80\bigl(2a_1+159d\bigr).
\]
Ra\v{c}unamo
\[
2a_1+159d
 =2\left(\alpha-53\frac{\beta-\alpha}{16}\right)+159\frac{\beta-\alpha}{16}
=\frac{-21\alpha+53\beta}{16}.
\]
Otuda
\[
S_{160}=80\cdot \frac{-21\alpha+53\beta}{16}=5(-21\alpha+53\beta).
\]

\subsection*{Odgovor}
\[
\boxed{(D)\;\; 5(-21\alpha+53\beta)}
\]

