\section*{Zadatak 7}
Izracunati granicnu vrednost
\[
\lim_{x\to 5\pi}\frac{e^{-\frac{1}{(5\pi-x)^2}+\sqrt{3\pi}}\cdot \log_2\!\left(\frac{x+3\pi}{2\pi}\right)}
{\left(\sqrt{6x-5\pi}-2\sqrt{\pi}\right)\cdot \tg\!\left(\frac{x}{5}-\frac{4\pi}{3}\right)}.
\]

\subsection*{Resenje}
Pri $x\to 5\pi$ vazi
\[
-\frac{1}{(5\pi-x)^2}\to -\infty \quad\Longrightarrow\quad
e^{-\frac{1}{(5\pi-x)^2}+\sqrt{3\pi}}\to 0,
\]
dok je
\[
\log_2\!\left(\frac{x+3\pi}{2\pi}\right)\to \log_2\!\left(\frac{8\pi}{2\pi}\right)=\log_2 4=2.
\]
Dakle brojilac tezi nuli.
Za imenilac:
\[
\sqrt{6x-5\pi}-2\sqrt{\pi}\to \sqrt{25\pi}-2\sqrt{\pi}=3\sqrt{\pi}\neq 0,
\]
\[
\tg\!\left(\frac{x}{5}-\frac{4\pi}{3}\right)\to \tg\!\left(\pi-\frac{4\pi}{3}\right)=\tg\!\left(-\frac{\pi}{3}\right)=-\sqrt{3}\neq 0.
\]
Zato imenilac tezi nenultoj konstanti, a ceo razlomak tezi $0$.

\subsection*{Odgovor}
\[
\boxed{(B)\;\; 0}
\]

