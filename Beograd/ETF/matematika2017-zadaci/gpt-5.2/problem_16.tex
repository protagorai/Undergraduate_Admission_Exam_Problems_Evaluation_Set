\section*{Zadatak 16}
Na\'ci zbir svih realnih re\v{s}enja jedna\v{c}ine
\[
\sqrt{x^2-1}+\sqrt{x^2+x-2}=\sqrt[4]{x^2-2x+1},
\]
i odrediti u kom se ponu\dj enom intervalu taj zbir nalazi.

\subsection*{Re\v{s}enje}
Desna strana je
\[
\sqrt[4]{x^2-2x+1}=\sqrt[4]{(x-1)^2}=\sqrt{|x-1|}.
\]
Za definisanost leve strane treba
\[
x^2-1\ge 0 \;\Rightarrow\; |x|\ge 1,
\]
\[
x^2+x-2=(x+2)(x-1)\ge 0 \;\Rightarrow\; x\le -2 \ \text{ili}\ x\ge 1.
\]
Dakle domen je $(-\infty,-2]\cup[1,+\infty)$.

\paragraph{1) Slu\v{c}aj $x\ge 1$}
Tada je $\sqrt{|x-1|}=\sqrt{x-1}$ i va\v{z}i
\[
\sqrt{x^2-1}=\sqrt{(x-1)(x+1)}=\sqrt{x-1}\,\sqrt{x+1}.
\]
Za $x>1$ je $\sqrt{x+1}>1$, pa je $\sqrt{x^2-1}>\sqrt{x-1}$, a drugi sabirak na levoj strani je $\ge 0$.
Zato za $x>1$ imamo LHS $>$ RHS i nema re\v{s}enja.
Za $x=1$ dobijamo $0+0=0$, pa je $x=1$ re\v{s}enje.

\paragraph{2) Slu\v{c}aj $x\le -2$}
Tada je $|x-1|=1-x$ i RHS je $\sqrt{1-x}$.
Pi\v{s}emo
\[
\sqrt{x^2-1}=\sqrt{(1-x)(-1-x)}=\sqrt{1-x}\,\sqrt{-1-x}.
\]
Za $x<-2$ je $-1-x>1$, pa je $\sqrt{x^2-1}>\sqrt{1-x}$, a drugi koren na levoj strani je $\ge 0$,
pa nema re\v{s}enja.
Za $x=-2$ dobijamo
\[
\sqrt{3}+\sqrt{0}=\sqrt{3},
\]
pa je $x=-2$ re\v{s}enje.

Dakle re\v{s}enja su $x\in\{-2,1\}$, pa je njihov zbir
\[
(-2)+1=-1.
\]

\subsection*{Odgovor}
Zbir je $-1$, pa pripada intervalu $[-3,0)$.
\[
\boxed{(A)\;\; [-3,0)}
\]

