\section*{Zadatak 13}
O\v{s}tar ugao pod kojim se seku tangenta krive $2x^2+3y^2=5$ u ta\v{c}ki $(1,1)$
i tangenta krive $x^2-4x+y^2+6y=3$ u ta\v{c}ki $(2,1)$ jednak je:

\subsection*{Re\v{s}enje}
\paragraph{1) Prva kriva}
Implicitnim diferenciranjem:
\[
4x+6y\,y'=0 \;\Longrightarrow\; y'=-\frac{4x}{6y}=-\frac{2x}{3y}.
\]
U ta\v{c}ki $(1,1)$ je
\[
m_1=y'(1,1)=-\frac{2}{3}.
\]

\paragraph{2) Druga kriva}
\[
2x-4+(2y+6)y'=0 \;\Longrightarrow\; y'=\frac{4-2x}{2y+6}=\frac{2-x}{y+3}.
\]
U ta\v{c}ki $(2,1)$ je
\[
m_2=y'(2,1)=\frac{2-2}{4}=0.
\]

\paragraph{3) Ugao izme\dj u pravih}
Za ugao $\theta$ izme\dj u pravih sa nagibima $m_1,m_2$ va\v{z}i
\[
\tg\theta=\left|\frac{m_1-m_2}{1+m_1m_2}\right|
=\left|\frac{-\frac23-0}{1+0}\right|=\frac{2}{3}.
\]
Po\v{s}to se tra\v{z}i o\v{s}tar ugao,
\[
\theta=\arctg\frac{2}{3}.
\]

\subsection*{Odgovor}
\[
\boxed{(C)\;\; \arctg\frac{2}{3}}
\]

