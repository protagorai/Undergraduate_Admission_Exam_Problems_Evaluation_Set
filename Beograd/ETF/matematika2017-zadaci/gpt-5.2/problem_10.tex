\section*{Zadatak 10}
U trouglu $ABC$ je $|BC|=3|AB|$ i $\angle ABC=60^\circ$.
Odrediti $\cos\angle BAC+\cos\angle ACB$.

\subsection*{Resenje}
Oznacimo standardno stranice: $a=|BC|$, $b=|CA|$, $c=|AB|$.
Dato je $a=3c$ i $\angle B=60^\circ$.

Po kosinusnoj teoremi:
\[
b^2=a^2+c^2-2ac\cos B=a^2+c^2-ac.
\]
Uz $a=3c$ dobijamo
\[
b^2=9c^2+c^2-3c^2=7c^2 \quad\Rightarrow\quad b=\sqrt{7}\,c.
\]
Zatim
\[
\cos A=\frac{b^2+c^2-a^2}{2bc}
=\frac{7c^2+c^2-9c^2}{2(\sqrt{7}c)c}
=-\frac{1}{2\sqrt{7}},
\]
\[
\cos C=\frac{a^2+b^2-c^2}{2ab}
=\frac{9c^2+7c^2-c^2}{2(3c)(\sqrt{7}c)}
=\frac{5}{2\sqrt{7}}.
\]
Zato je
\[
\cos A+\cos C=\frac{-1+5}{2\sqrt{7}}=\frac{2}{\sqrt{7}}.
\]

\subsection*{Odgovor}
\[
\boxed{(B)\;\; \dfrac{2}{\sqrt{7}}}
\]

