\documentclass[12pt]{article}
\usepackage[margin=1in]{geometry}
\usepackage{amsmath,amssymb}
\begin{document}

\section*{Problem 10}
In an arbitrary triangle with sides $a$, $b$, $c$ and corresponding angles $\alpha$, $\beta$, the coefficient $\displaystyle\frac{\sin(\alpha - \beta)}{\sin(\alpha + \beta)}$ equals:

\subsection*{Solution}
Using the Law of Sines: $\frac{a}{\sin\alpha} = \frac{b}{\sin\beta} = \frac{c}{\sin\gamma} = 2R$

So $a = 2R\sin\alpha$ and $b = 2R\sin\beta$.

Also, in a triangle, $\alpha + \beta + \gamma = \pi$, so $\alpha + \beta = \pi - \gamma$, meaning $\sin(\alpha + \beta) = \sin\gamma$.

Using the sine difference formula:
\[
\sin(\alpha - \beta) = \sin\alpha\cos\beta - \cos\alpha\sin\beta
\]

By the Law of Cosines:
\[
\cos\alpha = \frac{b^2 + c^2 - a^2}{2bc}, \quad \cos\beta = \frac{a^2 + c^2 - b^2}{2ac}
\]

Using Law of Sines: $\sin\alpha = \frac{a}{2R}$, $\sin\beta = \frac{b}{2R}$, $\sin\gamma = \frac{c}{2R}$.

\[
\frac{\sin(\alpha-\beta)}{\sin(\alpha+\beta)} = \frac{\sin\alpha\cos\beta - \cos\alpha\sin\beta}{\sin\gamma}
\]

After substitution and simplification (using $a^2 - b^2$ pattern):
\[
= \frac{a^2 - b^2}{c^2}
\]

This can be verified: The numerator involves the difference of squares of sine values weighted by cosines, which reduces to $\frac{a^2 - b^2}{c^2}$.

\subsection*{Answer}
$\displaystyle\frac{a^2 - b^2}{c^2}$ (option \textbf{C}).

\end{document}
