\documentclass[12pt]{article}
\usepackage[margin=1in]{geometry}
\usepackage{amsmath,amssymb}
\begin{document}

\section*{Problem 12}
Find the difference between the larger and smaller bases of an isosceles trapezoid with perimeter 32 cm and a circumscribed circle of radius 2 cm.

\subsection*{Solution}
Let the isosceles trapezoid have parallel bases $a$ (larger) and $b$ (smaller), and equal legs $c$.

Perimeter: $a + b + 2c = 32$.

For a trapezoid to have a circumscribed circle (inscribed circle), it must satisfy:
\[
a + b = 2c
\]

From these two equations:
\[
a + b + 2c = 32 \quad \text{and} \quad a + b = 2c
\]

Substituting: $2c + 2c = 32 \Rightarrow 4c = 32 \Rightarrow c = 8$.

So $a + b = 16$.

Now, for a circumscribed circle with radius $r = 2$ (the incircle), the area of the trapezoid is:
\[
A = r \cdot s = 2 \cdot \frac{32}{2} = 2 \cdot 16 = 32
\]
where $s$ is the semi-perimeter.

Also, area of trapezoid: $A = \frac{(a+b)}{2} \cdot h = \frac{16}{2} \cdot h = 8h$.

So $8h = 32 \Rightarrow h = 4$.

For an isosceles trapezoid with legs $c = 8$ and height $h = 4$:
The horizontal projection of each leg: $\sqrt{c^2 - h^2} = \sqrt{64 - 16} = \sqrt{48} = 4\sqrt{3}$.

The difference in bases: $a - b = 2 \cdot 4\sqrt{3} = 8\sqrt{3}$.

\subsection*{Answer}
$8\sqrt{3}$ (option \textbf{B}).

\end{document}
