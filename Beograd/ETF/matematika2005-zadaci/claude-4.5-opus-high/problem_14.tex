\documentclass[12pt]{article}
\usepackage[margin=1in]{geometry}
\usepackage{amsmath,amssymb}
\begin{document}

\section*{Problem 14}
Find the total number of solutions to $\sin^2 x + \sin^2 2x = 1$ in the interval $(0, 2\pi)$.

\subsection*{Solution}
We have $\sin^2 x + \sin^2 2x = 1$.

Using $\sin 2x = 2\sin x \cos x$:
\[
\sin^2 x + 4\sin^2 x \cos^2 x = 1
\]

Let $u = \sin^2 x$. Then $\cos^2 x = 1 - u$:
\[
u + 4u(1-u) = 1
\]
\[
u + 4u - 4u^2 = 1
\]
\[
5u - 4u^2 = 1
\]
\[
4u^2 - 5u + 1 = 0
\]

Using quadratic formula:
\[
u = \frac{5 \pm \sqrt{25 - 16}}{8} = \frac{5 \pm 3}{8}
\]

So $u = 1$ or $u = \frac{1}{4}$.

\textbf{Case 1:} $\sin^2 x = 1 \Rightarrow \sin x = \pm 1 \Rightarrow x = \frac{\pi}{2}, \frac{3\pi}{2}$.
Check: $\sin^2(\pi/2) + \sin^2(\pi) = 1 + 0 = 1$ ✓
That's 2 solutions.

\textbf{Case 2:} $\sin^2 x = \frac{1}{4} \Rightarrow \sin x = \pm\frac{1}{2}$.
In $(0, 2\pi)$: $x = \frac{\pi}{6}, \frac{5\pi}{6}, \frac{7\pi}{6}, \frac{11\pi}{6}$.
Check one: $x = \pi/6$: $\sin^2(\pi/6) = 1/4$, $\sin(2 \cdot \pi/6) = \sin(\pi/3) = \sqrt{3}/2$, $\sin^2(\pi/3) = 3/4$.
$1/4 + 3/4 = 1$ ✓
That's 4 solutions.

Total: $2 + 4 = 6$ solutions.

\subsection*{Answer}
$6$ (option \textbf{E}).

\end{document}
