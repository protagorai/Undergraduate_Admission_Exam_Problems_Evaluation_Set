\documentclass[12pt]{article}
\usepackage[margin=1in]{geometry}
\usepackage{amsmath,amssymb}
\begin{document}

\section*{Problem 19}
Broj načina da se poređaju $n$ nula i $k$ jedinica ($k\le n+1$) tako da dve jedinice nikada ne budu susedne.

\subsection*{Rešenje}
Zamislimo da najpre rasporedimo $n$ nula, čime dobijamo $n+1$ "rupa" (pre i posle svake nule, kao i na krajevima). U te rupe sme se odabrati mesta za $k$ jedinica, pod uslovom da u svaku rupu ide najviše jedna jedinica. Broj takvih izbora je upravo binomni koeficijent
\[\binom{n+1}{k}.\]

\subsection*{Odgovor}
$\displaystyle\binom{n+1}{k}$ (опција \textbf{A}).

\end{document}