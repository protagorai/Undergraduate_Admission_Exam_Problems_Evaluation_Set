\documentclass[12pt]{article}
\usepackage[margin=1in]{geometry}
\usepackage{amsmath,amssymb}
\begin{document}

\section*{Problem 20}
Odrediti maksimalnu površinu pravougaonika upisanog u parabolički isečak ograničen parabolom $y=1-x^{2}$ i pravom $y=0$, pri čemu je jedna stranica pravougaonika na $x$-osi.

\subsection*{Rešenje}
Neka temena na paraboli imaju apscise $-x$ i $x$ ($0\le x\le1$). Visina pravougaonika je $y=1-x^{2}$, a dužina osnovice $2x$. Površina je
\[A(x)=2x(1-x^{2}).\]
Maksimum na $(0,1)$ dobijamo iz $A'(x)=2(1-3x^{2})=0\Rightarrow x=\tfrac1{\sqrt3}$.
\[A_{\max}=2\,\frac1{\sqrt3}\Bigl(1-\frac1{3}\Bigr)=\frac{4}{3\sqrt3}=\frac{4\sqrt3}{9}.
\]

\subsection*{Odgovor}
$\dfrac{4\sqrt3}{9}$ (опција \textbf{B}).

\end{document}