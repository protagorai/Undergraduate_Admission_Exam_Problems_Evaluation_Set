\documentclass[12pt]{article}
\usepackage[margin=1in]{geometry}
\usepackage{amsmath,amssymb}
\begin{document}

\section*{Problem 12}
Jednakokraki trapez ima veći osnov $a$, manji osnov $b$ ($a>b$), krak $c$, visinu $h$ i poluprečnik upisanog kruga $r=2\,\mathrm{cm}$. Zadata su još i obim
\[a+b+2c=32\,\mathrm{cm}.\]
Potrebно је наћи разлику $a-b$.

\subsection*{Rešenje}
За једнакокраки трапез постоји уписани круг ако и само ако је
\[a+b=2c.\]
Полупречник уписаног круга дат је односом $r=\dfrac h2\;(\!* )$ па је висина
\[h=2r=4\,\text{cm}.
\]
Из обима имамо
\[a+b+2c=32\;\Longrightarrow\;a+b=16.\]
У троуглу насталом спуштањем висине на већу основу важи
\[c^{2}=h^{2}+\Bigl(\tfrac{a-b}{2}\Bigr)^{2}\;\Longrightarrow\;
8^{2}=4^{2}+\frac{(a-b)^{2}}{4}.
\]
Одатле је $64-16=\dfrac{(a-b)^{2}}{4}\Rightarrow(a-b)^{2}=192$ па
\[a-b=8\sqrt3.\]

\subsection*{Odgovor}
$8\sqrt3\,\mathrm{cm}$ (опција \textbf{B}).

\end{document}