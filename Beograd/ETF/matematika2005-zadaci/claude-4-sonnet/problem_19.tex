\documentclass[12pt]{article}
\usepackage[margin=1in]{geometry}
\usepackage{amsmath,amssymb}
\begin{document}

\section*{Problem 19}
Broj načina na koji se mogu poredati u niz $n$ nula i $k$ jedinica, tako da nikoje dve jedinice nisu susedne, ako je $k \leq n + 1$ je:

\subsection*{Solution}
Imamo $n$ nula i $k$ jedinica, gde je $k \leq n + 1$.

Da bi nikoje dve jedinice nisu susedne, možemo razmisliti ovako:
- Prvo postavimo $n$ nula u niz: $0\ 0\ 0\ \ldots\ 0$
- Ovo stvara $n + 1$ poziciju gde možemo staviti jedinice (pre prve nule, između nula, i posle poslednje nule)
- Trebamo da izaberemo $k$ od ovih $n + 1$ pozicija za jedinice

Broj načina da izaberemo $k$ pozicija od $n + 1$ dostupnih pozicija je $\binom{n+1}{k}$.

\subsection*{Answer}
$\binom{n+1}{k}$ (opcija \textbf{A}).

\end{document}