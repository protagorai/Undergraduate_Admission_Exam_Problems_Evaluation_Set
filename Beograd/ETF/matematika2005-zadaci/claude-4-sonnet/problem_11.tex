\documentclass[12pt]{article}
\usepackage[margin=1in]{geometry}
\usepackage{amsmath,amssymb}
\begin{document}

\section*{Problem 11}
Date su funkcije $f_1(x) = x$, $f_2(x) = \frac{x^2}{x}$, $f_3(x) = \sqrt{x^2}$, $f_4(x) = (\sqrt{x})^2$. Tačan je iskaz:

\subsection*{Solution}
Analizirajmo svaku funkciju:

$f_1(x) = x$ - definisana za sve realne brojeve

$f_2(x) = \frac{x^2}{x} = x$ za $x \neq 0$ - definisana za sve realne brojeve osim $x = 0$

$f_3(x) = \sqrt{x^2} = |x|$ - definisana za sve realne brojeve

$f_4(x) = (\sqrt{x})^2 = x$ za $x \geq 0$ - definisana samo za nenegativne realne brojeve

Dakle:
- $f_1$ i $f_3$ su definisane na celom skupu realnih brojeva
- $f_2$ je definisana za $x \neq 0$
- $f_4$ je definisana za $x \geq 0$

Za pozitivne vrednosti $x$:
- $f_1(x) = x$
- $f_2(x) = x$
- $f_3(x) = |x| = x$
- $f_4(x) = x$

Za negativne vrednosti $x$:
- $f_1(x) = x$
- $f_2(x) = x$
- $f_3(x) = |x| = -x$
- $f_4(x)$ nije definisana

Dakle, $f_1 = f_2 \neq f_3$ na skupu gde su sve definisane.

\subsection*{Answer}
$f_1 = f_3 \neq f_4$ (opcija \textbf{E}).

\end{document}