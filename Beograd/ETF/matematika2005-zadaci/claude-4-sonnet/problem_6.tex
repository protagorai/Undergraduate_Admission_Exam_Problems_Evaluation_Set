\documentclass[12pt]{article}
\usepackage[margin=1in]{geometry}
\usepackage{amsmath,amssymb}
\begin{document}

\section*{Problem 6}
Ako je $\operatorname{tg}\alpha = -7$, $\alpha \in (\pi/2, \pi)$, tada $\frac{3\sin\alpha + \cos\alpha}{\cos\alpha - 3\sin\alpha}$ iznosi:

\subsection*{Solution}
Pošto je $\operatorname{tg}\alpha = -7$, imamo $\frac{\sin\alpha}{\cos\alpha} = -7$, što znači $\sin\alpha = -7\cos\alpha$.

Takođe, iz osnovne trigonometrijske identiteta:
$\sin^2\alpha + \cos^2\alpha = 1$
$(-7\cos\alpha)^2 + \cos^2\alpha = 1$
$49\cos^2\alpha + \cos^2\alpha = 1$
$50\cos^2\alpha = 1$
$\cos^2\alpha = \frac{1}{50}$

Pošto je $\alpha \in (\pi/2, \pi)$ (drugi kvadrant), $\cos\alpha < 0$, pa je:
$\cos\alpha = -\frac{1}{\sqrt{50}} = -\frac{1}{5\sqrt{2}} = -\frac{\sqrt{2}}{10}$

I:
$\sin\alpha = -7\cos\alpha = -7 \cdot \left(-\frac{\sqrt{2}}{10}\right) = \frac{7\sqrt{2}}{10}$

Sada računamo:
$\frac{3\sin\alpha + \cos\alpha}{\cos\alpha - 3\sin\alpha}$

$= \frac{3 \cdot \frac{7\sqrt{2}}{10} + \left(-\frac{\sqrt{2}}{10}\right)}{-\frac{\sqrt{2}}{10} - 3 \cdot \frac{7\sqrt{2}}{10}}$

$= \frac{\frac{21\sqrt{2}}{10} - \frac{\sqrt{2}}{10}}{-\frac{\sqrt{2}}{10} - \frac{21\sqrt{2}}{10}}$

$= \frac{\frac{20\sqrt{2}}{10}}{-\frac{22\sqrt{2}}{10}}$

$= \frac{2\sqrt{2}}{-\frac{22\sqrt{2}}{10}} = \frac{2\sqrt{2} \cdot 10}{-22\sqrt{2}} = \frac{20}{-22} = -\frac{10}{11}$

\subsection*{Answer}
$-\frac{10}{11}$ (opcija \textbf{B}).

\end{document}