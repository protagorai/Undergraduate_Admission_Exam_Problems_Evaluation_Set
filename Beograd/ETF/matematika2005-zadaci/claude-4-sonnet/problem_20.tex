\documentclass[12pt]{article}
\usepackage[margin=1in]{geometry}
\usepackage{amsmath,amssymb}
\begin{document}

\section*{Problem 20}
Maksimalna površina pravougaonika upisanog u parabolički odsetak ograničen parabolom $y = 1 - x^2$ i pravom $y = 0$, tako da mu jedna stranica pripada $x$-osi, jeste:

\subsection*{Solution}
Parabola $y = 1 - x^2$ seče $x$-osu kada je $y = 0$:
$1 - x^2 = 0$
$x^2 = 1$
$x = \pm 1$

Dakle, parabolički odsetak je ograničen tačkama $(-1, 0)$, $(1, 0)$ i vrhom parabole $(0, 1)$.

Neka je pravougaonik sa temenom na $x$-osi u tačkama $(-a, 0)$ i $(a, 0)$ gde je $0 < a \leq 1$.

Visina pravougaonika u tačkama $x = \pm a$ je $y = 1 - a^2$.

Dakle, pravougaonik ima:
- Širinu: $2a$
- Visinu: $1 - a^2$

Površina je:
$P(a) = 2a(1 - a^2) = 2a - 2a^3$

Da bismo našli maksimum, derivišemo po $a$:
$P'(a) = 2 - 6a^2$

Postavimo $P'(a) = 0$:
$2 - 6a^2 = 0$
$6a^2 = 2$
$a^2 = \frac{1}{3}$
$a = \frac{1}{\sqrt{3}} = \frac{\sqrt{3}}{3}$

Proveravamo da je ovo maksimum:
$P''(a) = -12a$
$P''\left(\frac{\sqrt{3}}{3}\right) = -12 \cdot \frac{\sqrt{3}}{3} = -4\sqrt{3} < 0$ ✓

Maksimalna površina je:
$P\left(\frac{\sqrt{3}}{3}\right) = 2 \cdot \frac{\sqrt{3}}{3} \left(1 - \left(\frac{\sqrt{3}}{3}\right)^2\right)$

$= \frac{2\sqrt{3}}{3} \left(1 - \frac{1}{3}\right) = \frac{2\sqrt{3}}{3} \cdot \frac{2}{3} = \frac{4\sqrt{3}}{9}$

\subsection*{Answer}
$\frac{4\sqrt{3}}{9}$ (opcija \textbf{B}).

\end{document}