\documentclass[12pt]{article}
\usepackage[margin=1in]{geometry}
\usepackage{amsmath,amssymb}
\begin{document}

\section*{Problem 1}
Vrednost izraza $\frac{\sqrt{32}}{\sqrt{8}} + \sqrt{3} \cdot \sqrt{12} - \sqrt{50} \cdot \sqrt{2}$ je:

\subsection*{Solution}
Prvo ćemo pojednostaviti svaki deo izraza:

$\frac{\sqrt{32}}{\sqrt{8}} = \sqrt{\frac{32}{8}} = \sqrt{4} = 2$

$\sqrt{3} \cdot \sqrt{12} = \sqrt{3 \cdot 12} = \sqrt{36} = 6$

$\sqrt{50} \cdot \sqrt{2} = \sqrt{50 \cdot 2} = \sqrt{100} = 10$

Dakle, izraz postaje:
$2 + 6 - 10 = -2$

\subsection*{Answer}
$-2$ (opcija \textbf{B}).

\end{document}