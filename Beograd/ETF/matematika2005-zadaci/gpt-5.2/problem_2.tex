\documentclass[12pt]{article}
\usepackage[margin=1in]{geometry}
\usepackage{amsmath,amssymb}
\begin{document}

\section*{Problem 2}
Ako je $f(x)=x^2+x+1$, tada je $f(x+2)-2f(x+1)+f(x)$ za svako $x$ jednako:
\[
\text{(A) }2\quad \text{(B) }0\quad \text{(C) }x+2\quad \text{(D) }x\quad \text{(E) }x+3\quad \text{(N) Ne znam.}
\]

\subsection*{Solution}
Izračunajmo:
\[
f(x+2)=(x+2)^2+(x+2)+1=x^2+5x+7,
\]
\[
f(x+1)=(x+1)^2+(x+1)+1=x^2+3x+3.
\]
Zato je
\begin{align*}
f(x+2)-2f(x+1)+f(x)
&=(x^2+5x+7)-2(x^2+3x+3)+(x^2+x+1)\\
&=2.
\end{align*}

\subsection*{Answer}
$2$ (option \textbf{A}).

\end{document}

