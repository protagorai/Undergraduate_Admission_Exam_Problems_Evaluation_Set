\documentclass[12pt]{article}
\usepackage[margin=1in]{geometry}
\usepackage{amsmath,amssymb}
\begin{document}

\section*{Problem 12}
Razlika veće i manje osnovice jednakokrakog trapeza čiji je obim $32\text{ cm}$ a poluprečnik upisanog kruga $2\text{ cm}$, iznosi (u cm):
\[
\text{(A) }\sqrt3\quad \text{(B) }8\sqrt3\quad \text{(C) }\sqrt6\quad \text{(D) }3\sqrt2\quad \text{(E) }6\sqrt6\quad \text{(N) Ne znam.}
\]

\subsection*{Solution}
Neka su osnovice $a>b$ i kraci (jednaki) $l$. Pošto trapez ima upisan krug, važi osobina tangencijalnog četvorougla:
\[
a+b=l+l=2l.
\]
Obim je $a+b+2l=32$, pa uz $a+b=2l$ dobijamo $4l=32\Rightarrow l=8$ i $a+b=16$.

Poluprečnik upisanog kruga je $r=2$, a poluobim je $s=16$, pa je površina
\[
P=r\cdot s=2\cdot 16=32.
\]
S druge strane, $P=\dfrac{a+b}{2}\cdot h=\dfrac{16}{2}h=8h$, pa je $8h=32\Rightarrow h=4$.

U jednakokrakom trapezu, ako je $d=\dfrac{a-b}{2}$, tada važi Pitagorina teorema na trouglu sa hipotenuzom $l$:
\[
l^2=h^2+d^2 \Rightarrow 8^2=4^2+d^2 \Rightarrow d^2=48 \Rightarrow d=4\sqrt3.
\]
Zato je
\[
a-b=2d=8\sqrt3.
\]

\subsection*{Answer}
$8\sqrt3$ (option \textbf{B}).

\end{document}

