\documentclass[12pt]{article}
\usepackage[margin=1in]{geometry}
\usepackage{amsmath,amssymb}
\begin{document}

\section*{Problem 18}
Ako je
\[
\tan\alpha=\frac{(1+\tan 1^\circ)(1+\tan 2^\circ)-2}{(1-\tan 1^\circ)(1-\tan 2^\circ)-2}
\]
i $\alpha\in(0,90^\circ)$, tada je $\alpha$:
\[
\text{(A) }40^\circ\quad \text{(B) }41^\circ\quad \text{(C) }42^\circ\quad \text{(D) }43^\circ\quad \text{(E) }44^\circ\quad \text{(N) Ne znam.}
\]

\subsection*{Solution}
Označimo $t_1=\tan 1^\circ$, $t_2=\tan 2^\circ$. Tada:
\[
(1+t_1)(1+t_2)-2=t_1+t_2+t_1t_2-1,
\]
\[
(1-t_1)(1-t_2)-2=t_1t_2-t_1-t_2-1.
\]
Dakle,
\[
\tan\alpha=\frac{t_1t_2+(t_1+t_2)-1}{t_1t_2-(t_1+t_2)-1}.
\]
Neka su $S=t_1+t_2$ i $P=t_1t_2$. Iz formule za tangens zbira:
\[
\tan 3^\circ=\tan(1^\circ+2^\circ)=\frac{S}{1-P}\quad\Rightarrow\quad S=\tan 3^\circ\,(1-P).
\]
Ubacivanjem dobijamo:
\begin{align*}
\tan\alpha
&=\frac{P+\tan 3^\circ(1-P)-1}{P-\tan 3^\circ(1-P)-1}\\
&=\frac{(1-\tan 3^\circ)(P-1)}{(1+\tan 3^\circ)(P-1)}
=\frac{1-\tan 3^\circ}{1+\tan 3^\circ}.
\end{align*}
Prepoznajemo formulu
\[
\tan(45^\circ-\theta)=\frac{1-\tan\theta}{1+\tan\theta},
\]
pa je
\[
\tan\alpha=\tan(45^\circ-3^\circ)=\tan 42^\circ.
\]
Kako je $\alpha\in(0,90^\circ)$, sledi $\alpha=42^\circ$.

\subsection*{Answer}
$42^\circ$ (option \textbf{C}).

\end{document}

