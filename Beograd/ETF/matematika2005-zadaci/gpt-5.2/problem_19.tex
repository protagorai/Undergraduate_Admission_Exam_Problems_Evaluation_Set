\documentclass[12pt]{article}
\usepackage[margin=1in]{geometry}
\usepackage{amsmath,amssymb}
\begin{document}

\section*{Problem 19}
Broj načina na koji se mogu poređati u niz $n$ nula i $k$ jedinica, tako da nijedne dve jedinice nisu susedne, ako je $k\le n+1$ je:
\[
\text{(A) }\binom{n+1}{k}\quad
\text{(B) }\binom{n}{k}\quad
\text{(C) }\frac{n!}{k!}\quad
\text{(D) }\frac{(n+1)!}{k!}\quad
\text{(E) }\binom{n-1}{k}\quad
\text{(N) Ne znam.}
\]

\subsection*{Solution}
Najpre poređamo $n$ nula:
\[
0\ \ 0\ \ 0\ \ \cdots\ \ 0
\]
Između i sa obe strane nula postoji ukupno $n+1$ ``mesta'' (razmaka) u koje možemo ubacivati jedinice:
\[
\_\,0\,\_\,0\,\_\,0\,\_ \cdots 0\,\_
\]
Da se dve jedinice ne bi dodirivale, u svaki razmak možemo staviti najviše jednu jedinicu. Zato biramo $k$ razmaka od ukupno $n+1$ i u njih stavljamo jedinice.

Broj načina je
\[
\binom{n+1}{k}.
\]

\subsection*{Answer}
$\binom{n+1}{k}$ (option \textbf{A}).

\end{document}

