\documentclass[12pt]{article}
\usepackage[margin=1in]{geometry}
\usepackage{amsmath,amssymb}
\usepackage[utf8]{inputenc}

\begin{document}

\section*{Problem 10}
U proizvoljnom trouglu čije su stranice $a, b$ i $c$ i odgovarajući uglovi $\alpha$ i $\beta$ količnik $\frac{\sin(\alpha - \beta)}{\sin(\alpha + \beta)}$ jednak je:
(A) $\frac{(a-b)^2}{c^2}$ (B) $\frac{c^2}{a^2-b^2}$ (C) $\frac{a^2-b^2}{c^2}$ (D) $\frac{c^2}{(a-b)^2}$ (E) $\frac{(a-b)^2}{(a+b)^2}$ (N) Ne znam

\subsection*{Rešenje}
Iz sinusne teoreme imamo $a = 2R \sin \alpha$, $b = 2R \sin \beta$, $c = 2R \sin \gamma$.
U trouglu važi $\alpha + \beta + \gamma = 180^\circ$, pa je $\sin(\alpha + \beta) = \sin(180^\circ - \gamma) = \sin \gamma$.
Dakle, imenilac je $\sin \gamma = \frac{c}{2R}$.

Brojilac razvijamo koristeći adicionu formulu:
\[
\sin(\alpha - \beta) = \sin \alpha \cos \beta - \cos \alpha \sin \beta
\]
Koristimo kosinusnu teoremu za $\cos \alpha$ i $\cos \beta$:
\[
\cos \alpha = \frac{b^2 + c^2 - a^2}{2bc}, \quad \cos \beta = \frac{a^2 + c^2 - b^2}{2ac}
\]
Zamenom u izraz za brojilac (uz $\sin \alpha = \frac{a}{2R}, \sin \beta = \frac{b}{2R}$):
\[
\frac{a}{2R} \cdot \frac{a^2 + c^2 - b^2}{2ac} - \frac{b^2 + c^2 - a^2}{2bc} \cdot \frac{b}{2R}
\]
\[
= \frac{1}{4Rc} (a^2 + c^2 - b^2) - \frac{1}{4Rc} (b^2 + c^2 - a^2)
\]
\[
= \frac{1}{4Rc} (a^2 + c^2 - b^2 - b^2 - c^2 + a^2)
\]
\[
= \frac{1}{4Rc} (2a^2 - 2b^2) = \frac{2(a^2 - b^2)}{4Rc} = \frac{a^2 - b^2}{2Rc}
\]
Konačno, traženi količnik je:
\[
\frac{\sin(\alpha - \beta)}{\sin(\alpha + \beta)} = \frac{\frac{a^2 - b^2}{2Rc}}{\frac{c}{2R}} = \frac{a^2 - b^2}{2Rc} \cdot \frac{2R}{c} = \frac{a^2 - b^2}{c^2}
\]

\subsection*{Odgovor}
$\frac{a^2-b^2}{c^2}$ (opcija \textbf{C}).

\end{document}
