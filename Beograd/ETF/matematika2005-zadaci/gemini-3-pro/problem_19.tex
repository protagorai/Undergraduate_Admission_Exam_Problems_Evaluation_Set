\documentclass[12pt]{article}
\usepackage[margin=1in]{geometry}
\usepackage{amsmath,amssymb}
\usepackage[utf8]{inputenc}

\begin{document}

\section*{Problem 19}
Broj načina na koji se mogu poredati u niz $n$ nula i $k$ jedinica, tako da nikoje dve jedinice nisu susedne, ako je $k \leq n+1$ je:
(A) $\binom{n+1}{k}$ (B) $\binom{n}{k}$ (C) $\frac{n!}{k!}$ (D) $\frac{(n+1)!}{k!}$ (E) $\binom{n-1}{k}$ (N) Ne znam

\subsection*{Rešenje}
Prvo poređamo $n$ nula u niz.
\[
0 \quad 0 \quad 0 \quad \dots \quad 0
\]
Ove nule stvaraju $n+1$ praznih mesta (uključujući početak i kraj niza) gde možemo postaviti jedinice:
\[
\_ \, 0 \, \_ \, 0 \, \_ \, \dots \, \_ \, 0 \, \_
\]
Da bismo osigurali da nikoje dve jedinice nisu susedne, moramo postaviti svaku od $k$ jedinica na različito prazno mesto.
Dakle, od $n+1$ mogućih mesta biramo $k$ mesta na koja ćemo staviti jedinice.
Broj načina da se to uradi je broj kombinacija bez ponavljanja od $n+1$ elemenata klase $k$:
\[
\binom{n+1}{k}
\]

\subsection*{Odgovor}
$\binom{n+1}{k}$ (opcija \textbf{A}).

\end{document}
