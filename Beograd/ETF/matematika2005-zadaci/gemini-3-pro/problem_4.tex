\documentclass[12pt]{article}
\usepackage[margin=1in]{geometry}
\usepackage{amsmath,amssymb}
\usepackage[utf8]{inputenc}

\begin{document}

\section*{Problem 4}
Ako kompleksan broj $z$ zadovoljava jednakost $z + 2\bar{z} = 12 + 3i$, ($i^2 = -1$) tada je $|z|$ jednako:
(A) 5 (B) 13 (C) 15 (D) 9 (E) 10 (N) Ne znam

\subsection*{Rešenje}
Neka je $z = x + iy$, gde su $x, y \in \mathbb{R}$. Tada je konjugovano kompleksni broj $\bar{z} = x - iy$.
Zamenimo ovo u datu jednakost:
\[
(x + iy) + 2(x - iy) = 12 + 3i
\]
\[
x + iy + 2x - 2iy = 12 + 3i
\]
\[
3x - iy = 12 + 3i
\]
Izjednačavanjem realnih i imaginarnih delova dobijamo sistem jednačina:
\begin{align*}
3x &= 12 \implies x = 4 \\
-y &= 3 \implies y = -3
\end{align*}
Dakle, $z = 4 - 3i$.
Moduo kompleksnog broja $|z|$ računamo kao:
\[
|z| = \sqrt{x^2 + y^2} = \sqrt{4^2 + (-3)^2} = \sqrt{16 + 9} = \sqrt{25} = 5
\]

\subsection*{Odgovor}
5 (opcija \textbf{A}).

\end{document}
