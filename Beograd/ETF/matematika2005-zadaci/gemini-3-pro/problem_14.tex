\documentclass[12pt]{article}
\usepackage[margin=1in]{geometry}
\usepackage{amsmath,amssymb}
\usepackage[utf8]{inputenc}

\begin{document}

\section*{Problem 14}
Ukupan broj rešenja jednačine $\sin^2 x + \sin^2 2x = 1$ na intervalu $(0, 2\pi)$ jednak je:
(A) 2 (B) 3 (C) 4 (D) 5 (E) 6 (N) Ne znam

\subsection*{Rešenje}
Koristimo identitet $\sin^2 x + \cos^2 x = 1$, pa je $\sin^2 x - 1 = -\cos^2 x$.
Jednačina postaje:
\[
\sin^2 2x = 1 - \sin^2 x = \cos^2 x
\]
Razvijemo $\sin 2x = 2 \sin x \cos x$:
\[
(2 \sin x \cos x)^2 = \cos^2 x
\]
\[
4 \sin^2 x \cos^2 x - \cos^2 x = 0
\]
\[
\cos^2 x (4 \sin^2 x - 1) = 0
\]
Ovo daje dva slučaja:
1. $\cos^2 x = 0 \implies \cos x = 0$.
Na intervalu $(0, 2\pi)$, rešenja su $x = \frac{\pi}{2}$ i $x = \frac{3\pi}{2}$. (2 rešenja)

2. $4 \sin^2 x - 1 = 0 \implies \sin^2 x = \frac{1}{4} \implies \sin x = \pm \frac{1}{2}$.
- $\sin x = \frac{1}{2}$: $x = \frac{\pi}{6}$ i $x = \frac{5\pi}{6}$.
- $\sin x = -\frac{1}{2}$: $x = \frac{7\pi}{6}$ i $x = \frac{11\pi}{6}$.
(4 rešenja)

Ukupan broj rešenja je $2 + 4 = 6$.

\subsection*{Odgovor}
6 (opcija \textbf{E}).

\end{document}
