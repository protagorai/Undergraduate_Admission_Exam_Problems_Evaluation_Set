\documentclass[12pt]{article}
\usepackage[margin=1in]{geometry}
\usepackage{amsmath,amssymb}
\usepackage[utf8]{inputenc}

\begin{document}

\section*{Problem 9}
Zbir koeficijenata pravca tangenti kružnice $x^2 + y^2 = 2$ koje sadrže presečnu tačku pravih $x - y - 1 = 0$ i $x + y - 3 = 0$ je:
(A) 2 (B) $\sqrt{6}$ (C) -2 (D) $-\sqrt{6}$ (E) $2\sqrt{6}$ (N) Ne znam

\subsection*{Rešenje}
Prvo nađimo presečnu tačku pravih rešavanjem sistema:
\begin{align*}
x - y &= 1 \\
x + y &= 3
\end{align*}
Sabiranjem jednačina dobijamo $2x = 4 \implies x = 2$.
Zamenom u prvu jednačinu: $2 - y = 1 \implies y = 1$.
Presečna tačka je $M(2, 1)$.

Kružnica $x^2 + y^2 = 2$ ima centar u $O(0,0)$ i poluprečnik $r = \sqrt{2}$.
Jednačina prave koja prolazi kroz $M(2, 1)$ sa koeficijentom pravca $k$ je:
\[
y - 1 = k(x - 2) \implies kx - y + (1 - 2k) = 0
\]
Da bi ova prava bila tangenta kružnice, rastojanje od centra $(0,0)$ do prave mora biti jednako poluprečniku $r=\sqrt{2}$:
\[
\frac{|k \cdot 0 - 0 + 1 - 2k|}{\sqrt{k^2 + (-1)^2}} = \sqrt{2}
\]
\[
\frac{|1 - 2k|}{\sqrt{k^2 + 1}} = \sqrt{2}
\]
Kvadriranjem:
\[
(1 - 2k)^2 = 2(k^2 + 1)
\]
\[
1 - 4k + 4k^2 = 2k^2 + 2
\]
\[
2k^2 - 4k - 1 = 0
\]
Tražimo zbir koeficijenata pravca $k_1 + k_2$. Na osnovu Vijetovih formula za kvadratnu jednačinu $Ax^2 + Bx + C = 0$, zbir rešenja je $-\frac{B}{A}$.
Ovde je $A=2, B=-4$, pa je:
\[
k_1 + k_2 = -\frac{-4}{2} = 2
\]

\subsection*{Odgovor}
2 (opcija \textbf{A}).

\end{document}
