\documentclass[12pt]{article}
\usepackage[margin=1in]{geometry}
\usepackage{amsmath,amssymb}
\usepackage[utf8]{inputenc}

\begin{document}

\section*{Problem 20}
Maksimalna površina pravougaonika upisanog u parabolički odsečak ograničen parabolom $y = 1 - x^2$ i pravom $y = 0$, tako da mu jedna stranica pripada $x$-osi, jeste:
(A) $\frac{1}{9}\sqrt{3}$ (B) $\frac{4}{9}\sqrt{3}$ (C) $\frac{8}{9}\sqrt{3}$ (D) $\sqrt{3}$ (E) $2\sqrt{3}$ (N) Ne znam

\subsection*{Rešenje}
Neka su temena pravougaonika na $x$-osi tačke $(x, 0)$ i $(-x, 0)$, gde je $x > 0$. Temena na paraboli su $(x, 1-x^2)$ i $(-x, 1-x^2)$.
Širina pravougaonika je $2x$, a visina je $y = 1-x^2$.
Površina pravougaonika kao funkcija od $x$ je:
\[
P(x) = 2x(1-x^2) = 2x - 2x^3
\]
Da bismo našli maksimum, nađimo prvi izvod:
\[
P'(x) = 2 - 6x^2
\]
Izjednačimo izvod sa nulom:
\[
2 - 6x^2 = 0 \implies 6x^2 = 2 \implies x^2 = \frac{1}{3} \implies x = \frac{1}{\sqrt{3}} = \frac{\sqrt{3}}{3}
\]
Proverimo da li je ovo maksimum (drugi izvod $P''(x) = -12x < 0$ za $x>0$, dakle jeste maksimum).
Maksimalna površina je:
\[
P\left(\frac{\sqrt{3}}{3}\right) = 2\left(\frac{\sqrt{3}}{3}\right) \left(1 - \frac{1}{3}\right) = \frac{2\sqrt{3}}{3} \cdot \frac{2}{3} = \frac{4\sqrt{3}}{9}
\]

\subsection*{Odgovor}
$\frac{4}{9}\sqrt{3}$ (opcija \textbf{B}).

\end{document}
