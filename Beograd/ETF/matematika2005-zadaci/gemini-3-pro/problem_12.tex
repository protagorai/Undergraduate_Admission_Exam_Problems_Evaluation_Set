\documentclass[12pt]{article}
\usepackage[margin=1in]{geometry}
\usepackage{amsmath,amssymb}
\usepackage[utf8]{inputenc}

\begin{document}

\section*{Problem 12}
Razlika veće i manje osnovice jednakokrakog trapeza čiji je obim 32cm a poluprečnik upisanog krug 2cm, iznosi (u cm):
(A) $\sqrt{3}$ (B) $8\sqrt{3}$ (C) $\sqrt{6}$ (D) $3\sqrt{2}$ (E) $6\sqrt{6}$ (N) Ne znam

\subsection*{Rešenje}
Neka su osnovice trapeza $a$ i $b$ ($a > b$) i krak $c$.
Pošto je trapez tangentni (može se upisati krug), zbir naspramnih stranica je jednak:
\[
a + b = c + c = 2c
\]
Obim je $P = a + b + 2c = 32$.
Zamenom $a+b=2c$:
\[
2c + 2c = 32 \implies 4c = 32 \implies c = 8 \text{ cm}
\]
Tada je $a+b = 16$.
Poluprečnik upisanog kruga je $r=2$ cm, pa je visina trapeza $h = 2r = 4$ cm.
Kod jednakokrakog trapeza, ako spustimo visinu iz temena manje osnovice na veću, dobijamo pravougli trougao sa hipotenuzom $c$, katetom $h$ i drugom katetom $\frac{a-b}{2}$.
Pitagorina teorema:
\[
c^2 = h^2 + \left(\frac{a-b}{2}\right)^2
\]
\[
8^2 = 4^2 + \left(\frac{a-b}{2}\right)^2
\]
\[
64 = 16 + \left(\frac{a-b}{2}\right)^2
\]
\[
\left(\frac{a-b}{2}\right)^2 = 48
\]
\[
\frac{a-b}{2} = \sqrt{48} = \sqrt{16 \cdot 3} = 4\sqrt{3}
\]
\[
a - b = 8\sqrt{3}
\]

\subsection*{Odgovor}
$8\sqrt{3}$ (opcija \textbf{B}).

\end{document}
