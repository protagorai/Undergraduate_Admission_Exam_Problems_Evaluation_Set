\documentclass[12pt]{article}
\usepackage[margin=1in]{geometry}
\usepackage{amsmath,amssymb}
\begin{document}

\section*{Problem 9}
Broj četvorocifrenih brojeva deljivih sa 5 čije su sve cifre različite jednak je:

(A) 1008 \quad (B) 952 \quad (C) 1200 \quad (D) 896 \quad (E) 840

\subsection*{Solution}
Četvorocifren broj je deljiv sa 5 ako se završava na 0 ili 5.
Cifre moraju biti različite. Skup cifara je $\{0, 1, \dots, 9\}$.

Slučaj 1: Broj se završava na 0.
Poslednja cifra je fiksirana (0).
Prva cifra ne može biti 0 (što je već ispunjeno) i može biti bilo koja od preostalih 9 cifara. Dakle 9 mogućnosti.
Druga cifra može biti bilo koja od preostalih 8.
Treća cifra može biti bilo koja od preostalih 7.
Ukupno za ovaj slučaj: $1 \cdot 9 \cdot 8 \cdot 7 = 504$.

Slučaj 2: Broj se završava na 5.
Poslednja cifra je fiksirana (5).
Prva cifra ne može biti 0 i ne može biti 5. Dakle imamo 8 mogućnosti (sve osim 0 i 5).
Druga cifra ne može biti prva i ne može biti 5, ali može biti 0. Dakle imamo 8 mogućnosti (od 10 cifara oduzmemo 5 i prvu cifru).
Treća cifra ne može biti 5, prva, druga. Dakle 7 mogućnosti.
Ukupno za ovaj slučaj: $1 \cdot 8 \cdot 8 \cdot 7 = 448$.

Ukupan broj takvih brojeva je $504 + 448 = 952$.

\subsection*{Answer}
$952$ (opcija \textbf{(B)}).

\end{document}
