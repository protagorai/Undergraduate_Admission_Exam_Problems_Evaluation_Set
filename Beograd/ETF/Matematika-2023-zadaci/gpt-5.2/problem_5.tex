\documentclass[12pt]{article}
\usepackage[margin=1in]{geometry}
\usepackage{amsmath,amssymb}
\begin{document}

\section*{Problem 5}
Neka je $B$ tačka na kružnici poluprečnika $r$ i $BC$ tangenta duž dužine 8 cm. Ako je $A$ tačka na istoj kružnici takva da je duž $AC$ dužine 9 cm i da sadrži centar kružnice, onda obim kružnice (u cm) iznosi:

(A) $\frac{36}{17}\pi$ \quad (B) $2\pi$ \quad (C) $\frac{11}{9}\pi$ \quad (D) $\frac{17}{9}\pi$ \quad (E) $\frac{289}{324}\pi$

\subsection*{Solution}
Neka je $O$ centar kružnice. Poluprečnik je $r$, dakle $OB = OA = r$.
$BC$ je tangenta na kružnicu u tački $B$, pa je $\triangle OBC$ pravougli trougao sa pravim uglom kod $B$.
Dužina tangente je $BC = 8$.
Duž $AC$ prolazi kroz centar $O$.
Tačke $A$ i $C$ su kolinearne sa $O$. Postoje dve mogućnosti za raspored tačaka na pravoj:
1. $C - A - O$ (A je između C i O). Tada $OC = CA + AO = 9 + r$.
2. $C - O - A$ (O je između C i A). Tada $OC = CA - AO = 9 - r$. Ali ovo bi značilo da sečica prolazi kroz centar, pa je $A$ jedna presečna tačka.
Tekst kaže: "A tačka na istoj kružnici ... duž AC dužine 9 cm i da sadrži centar kružnice". To znači da centar pripada duži $AC$.
Dakle raspored je $A - O - C$? Ne, $A$ je na kružnici. $O$ je centar.
Ako duž $AC$ sadrži centar, onda $O$ leži između $A$ i $C$.
Dakle $C - O - A$.
Onda je $AC = AO + OC$.
Znamo $AO = r$.
Dakle $9 = r + OC \Rightarrow OC = 9 - r$.
U pravouglom trouglu $OBC$:
$OC^2 = OB^2 + BC^2$.
$(9-r)^2 = r^2 + 8^2$.
$81 - 18r + r^2 = r^2 + 64$.
$81 - 64 = 18r$.
$17 = 18r$.
$r = \frac{17}{18}$.

Obim kružnice je $2\pi r = 2\pi \cdot \frac{17}{18} = \frac{17\pi}{9}$.

\subsection*{Answer}
$\frac{17}{9}\pi$ cm (opcija \textbf{(D)}).

\end{document}
