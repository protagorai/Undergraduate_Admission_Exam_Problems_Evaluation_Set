\documentclass[12pt]{article}
\usepackage[margin=1in]{geometry}
\usepackage{amsmath,amssymb}
\begin{document}

\section*{Problem 3}
Trapez je opisan oko kruga poluprečnika $r$. Ako je poznato da je površina trapeza (u cm$^2$) pet puta veća od obima tog trapeza (u cm), tada dužina poluprečnika $r$ (u cm) iznosi:

(A) 5 \quad (B) 30 \quad (C) 10 \quad (D) 20 \quad (E) 40

\subsection*{Solution}
Za trapez opisan oko kruga važi da je visina trapeza jednaka prečniku kruga, $h = 2r$.
Takođe, za tangentni četvorougao (trapez opisan oko kruga) važi da je zbir naspramnih stranica jednak: $a + b = c + d$.
Obim je $O = a + b + c + d = 2(a+b)$.
Površina trapeza je $P = \frac{a+b}{2} \cdot h = \frac{a+b}{2} \cdot 2r = (a+b)r$.

Primećujemo da je poluobim $s = \frac{O}{2} = a+b$.
Dakle $P = s \cdot r$.
Ovo važi za svaki tangentni poligon: $P = s \cdot r$, gde je $s$ poluobim.

Prema uslovu zadatka:
\[
P = 5 \cdot O
\]
Znamo $O = 2s$, pa $P = 5 \cdot 2s = 10s$.
S druge strane $P = s \cdot r$.
Dakle:
\[
s \cdot r = 10s
\]
Skratimo $s$ (jer je obim pozitivan):
\[
r = 10
\]

\subsection*{Answer}
$10$ cm (opcija \textbf{(C)}).

\end{document}
