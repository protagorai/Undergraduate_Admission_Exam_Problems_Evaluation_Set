\documentclass[12pt]{article}
\usepackage[margin=1in]{geometry}
\usepackage{amsmath,amssymb}
\begin{document}

\section*{Problem 15}
Binomni koeficijent četvrtog člana u razvoju binoma $\left(\sqrt{2} + \frac{1}{\sqrt[3]{2}}\right)^n$, $n \in N$, veći je 26 puta od binomnog koeficijenta trećeg člana. Broj racionalnih sabiraka u ovom razvoju iznosi:

(A) 9 \quad (B) 14 \quad (C) 15 \quad (D) 16 \quad (E) 13

\subsection*{Solution}
Binomni koeficijent $k$-tog člana je $\binom{n}{k-1}$.
Četvrti član: $\binom{n}{3}$.
Treći član: $\binom{n}{2}$.
Uslov: $\binom{n}{3} = 26 \binom{n}{2}$.
\[
\frac{n(n-1)(n-2)}{6} = 26 \frac{n(n-1)}{2}
\]
Pošto $n \ge 3$:
\[
\frac{n-2}{6} = 13 \Rightarrow n-2 = 78 \Rightarrow n = 80
\]
Tražimo broj racionalnih članova u razvoju $(\sqrt{2} + 2^{-1/3})^{80}$.
Opšti član:
\[
T_{k+1} = \binom{80}{k} (\sqrt{2})^{80-k} (2^{-1/3})^k = \binom{80}{k} 2^{\frac{80-k}{2}} 2^{-\frac{k}{3}} = \binom{80}{k} 2^{\frac{80-k}{2} - \frac{k}{3}}
\]
Eksponent mora biti ceo broj:
\[
\frac{80-k}{2} - \frac{k}{3} = \frac{240 - 3k - 2k}{6} = \frac{240 - 5k}{6} \in Z
\]
$240 - 5k$ mora biti deljivo sa 6.
Kako je 240 deljivo sa 6, onda $5k$ mora biti deljivo sa 6.
Kako su 5 i 6 uzajamno prosti, $k$ mora biti deljivo sa 6.
$k \in \{0, 6, 12, \dots, 78\}$.
$k$ ide od 0 do 80.
Niz: $0, 6, 12, \dots, 6m, \dots$
$6m \leq 80 \Rightarrow m \leq 13.33$.
Dakle $m$ može biti $0, 1, \dots, 13$.
Ukupno 14 vrednosti.

\subsection*{Answer}
$14$ (opcija \textbf{(B)}).

\end{document}
