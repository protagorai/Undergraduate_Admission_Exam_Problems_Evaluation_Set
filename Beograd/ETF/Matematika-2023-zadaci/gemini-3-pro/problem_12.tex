\documentclass[12pt]{article}
\usepackage[margin=1in]{geometry}
\usepackage{amsmath,amssymb}
\begin{document}

\section*{Problem 12}
Date su funkcije
\[
f_1(x) = \sqrt{x-1} \cdot \log_3 3^{x-1}, \quad f_2(x) = \sqrt{3}^{\log_3(x-1)}, \quad f_3(x) = \sqrt[3]{(x-1)^3}, \quad f_4(x) = 10^{\log_{10}|x-1|^{-3/2}}
\]
Tačan je iskaz:

(A) među datim funkcijama nema jednakih \quad (B) $f_2 \neq f_1 = f_3 \neq f_4$
(C) $f_1 = f_2 = f_3 \neq f_4$ \quad (D) $f_1 \neq f_2 = f_3 \neq f_4$ \quad (E) $f_1 = f_2 = f_3 = f_4$

\subsection*{Solution}
Analizirajmo domene i izraze.
$f_1(x) = \sqrt{x-1} \cdot (x-1) = (x-1)^{3/2}$.
Domen: $x-1 \geq 0 \Rightarrow x \geq 1$.

$f_2(x) = (3^{1/2})^{\log_3(x-1)} = 3^{\frac{1}{2}\log_3(x-1)} = 3^{\log_3((x-1)^{1/2})} = (x-1)^{1/2} = \sqrt{x-1}$.
Domen: $x-1 > 0 \Rightarrow x > 1$. (logaritam mora biti definisan).

$f_3(x) = x-1$.
Domen: $R$.

$f_4(x) = |x-1|^{-3/2}$.
Domen: $x \neq 1$.

Poređenje:
$f_1$ je definisana za $x \geq 1$. Izraz $(x-1)^{3/2}$.
$f_2$ je definisana za $x > 1$. Izraz $(x-1)^{1/2}$.
$f_3$ je definisana za $x \in R$. Izraz $x-1$.
$f_4$ je definisana za $x \neq 1$.

Nijedna funkcija nije jednaka drugoj jer su im ili domeni različiti ili izrazi.
Čak i na preseku domena ($x>1$):
$f_1(x) = (x-1)^{3/2}$
$f_2(x) = (x-1)^{1/2}$
$f_3(x) = x-1$
$f_4(x) = (x-1)^{-3/2}$
Svi izrazi su različiti.

Napomena: Možda sam loše pročitao izraze sa slike.
$f_1$: $\sqrt{x-1} \cdot \log_3 3^{x-1}$. To je $(x-1)\sqrt{x-1}$. Domen $x \geq 1$.
$f_2$: $\sqrt{3}^{3\log_3(x-1)}$. Slika kaže $3\log_3(x-1)$.
Onda $3^{\frac{1}{2} \cdot 3 \log_3(x-1)} = 3^{\log_3 (x-1)^{3/2}} = (x-1)^{3/2}$.
Domen: $x > 1$ (zbog logaritma).
$f_3$: $\sqrt{(x-1)^3}$. Slika kaže kvadratni koren iz $(x-1)^3$.
To je $(x-1)^{3/2}$. Domen $x-1 \geq 0 \Rightarrow x \geq 1$.
$f_4$: $10^{\log \frac{1}{10} |x-1|^{-3/2}}$. Slika je sitna.
Log osnova 1/10?
Ako je $\log_{1/10} |x-1|^{-3/2} = \log_{10^{-1}} |x-1|^{-3/2} = -\log_{10} |x-1|^{-3/2} = \log_{10} (|x-1|^{-3/2})^{-1} = \log_{10} |x-1|^{3/2}$.
Onda $10^{\dots} = |x-1|^{3/2}$.
Domen: $|x-1| \neq 0 \Rightarrow x \neq 1$.

Poređenje sada:
$f_1(x) = (x-1)^{3/2}, x \geq 1$.
$f_2(x) = (x-1)^{3/2}, x > 1$.
$f_3(x) = (x-1)^{3/2}, x \geq 1$. (Ako je koren kvadratni).
Na slici 12. zadatak. $f_3(x) = \sqrt{(x-1)^3}$. Da, kvadratni koren.
$f_4(x) = |x-1|^{3/2}, x \neq 1$. Za $x > 1$ je $(x-1)^{3/2}$.

Zaključak:
$f_1$ i $f_3$ imaju isti domen $[1, \infty)$ i isti izraz. Dakle $f_1 = f_3$.
$f_2$ ima domen $(1, \infty)$, dakle $f_2 \neq f_1$.
$f_4$ ima domen $R \setminus \{1\}$, $f_4 \neq f_1$.

Da li je $f_3 = \sqrt{(x-1)^3}$ definisano samo za $x \geq 1$? Da, ako je realna funkcija.
Dakle $f_1 = f_3$.
Opcije:
(B) $f_2 \neq f_1 = f_3 \neq f_4$.
Ovo izgleda tačno.

\subsection*{Answer}
$f_2 \neq f_1 = f_3 \neq f_4$ (opcija \textbf{(B)}).

\end{document}
