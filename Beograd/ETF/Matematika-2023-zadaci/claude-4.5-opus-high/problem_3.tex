\documentclass[12pt]{article}
\usepackage[margin=1in]{geometry}
\usepackage{amsmath,amssymb}
\begin{document}

\section*{Problem 3}
Trapez je opisan oko kruga poluprečnika $r$. Ako je poznato da je površina trapeza (u cm$^2$) pet puta veća od obima tog trapeza (u cm), tada dužina poluprečnika $r$ (u cm) iznosi:

(A) 5 \quad (B) 30 \quad (C) 10 \quad (D) 20 \quad (E) 40

\subsection*{Solution}
Za trapez opisan oko kruga važi da je površina jednaka:
\[
P = r \cdot s
\]
gde je $s$ poluobim trapeza, tj. $s = \frac{O}{2}$ gde je $O$ obim.

Dakle $P = r \cdot \frac{O}{2} = \frac{rO}{2}$.

Prema uslovu zadatka, površina je 5 puta veća od obima:
\[
P = 5 \cdot O
\]

Zamenjujemo:
\[
\frac{rO}{2} = 5O
\]
\[
\frac{r}{2} = 5
\]
\[
r = 10
\]

\subsection*{Answer}
$10$ cm (opcija \textbf{(C)}).

\end{document}

