\documentclass[12pt]{article}
\usepackage[margin=1in]{geometry}
\usepackage{amsmath,amssymb}
\begin{document}

\section*{Problem 1}
Vrednost izraza 
\[
3 \cdot \frac{\sqrt{8+2\sqrt{7}}}{\sqrt{8-2\sqrt{7}}} - \frac{\sqrt{3+\sqrt{7}}}{\sqrt{3-\sqrt{7}}} \cdot \sqrt{2}
\]
iznosi:

(A) 2 \quad (B) $\sqrt{6}$ \quad (C) $3\sqrt{2}$ \quad (D) $\sqrt{2}$ \quad (E) 1

\subsection*{Solution}
Prvo pojednostavimo izraze pod korenom. Primetimo da:
\[
8 + 2\sqrt{7} = 7 + 2\sqrt{7} + 1 = (\sqrt{7} + 1)^2
\]
\[
8 - 2\sqrt{7} = 7 - 2\sqrt{7} + 1 = (\sqrt{7} - 1)^2
\]
Dakle:
\[
\frac{\sqrt{8+2\sqrt{7}}}{\sqrt{8-2\sqrt{7}}} = \frac{\sqrt{7}+1}{\sqrt{7}-1}
\]
Racionalizujemo:
\[
\frac{\sqrt{7}+1}{\sqrt{7}-1} = \frac{(\sqrt{7}+1)^2}{(\sqrt{7}-1)(\sqrt{7}+1)} = \frac{7 + 2\sqrt{7} + 1}{7-1} = \frac{8 + 2\sqrt{7}}{6} = \frac{4 + \sqrt{7}}{3}
\]

Sada za drugi razlomak:
\[
\frac{\sqrt{3+\sqrt{7}}}{\sqrt{3-\sqrt{7}}} = \sqrt{\frac{3+\sqrt{7}}{3-\sqrt{7}}}
\]
Racionalizujemo unutar korena:
\[
\frac{3+\sqrt{7}}{3-\sqrt{7}} = \frac{(3+\sqrt{7})^2}{(3-\sqrt{7})(3+\sqrt{7})} = \frac{9 + 6\sqrt{7} + 7}{9-7} = \frac{16 + 6\sqrt{7}}{2} = 8 + 3\sqrt{7}
\]
Dakle:
\[
\frac{\sqrt{3+\sqrt{7}}}{\sqrt{3-\sqrt{7}}} = \sqrt{8 + 3\sqrt{7}}
\]
Primetimo da $8 + 3\sqrt{7} = \frac{7}{2} + 3\sqrt{7} + \frac{9}{2} = \left(\frac{\sqrt{14} + 3}{\sqrt{2}}\right)^2$... 

Alternativni pristup: proverimo da li je $8 + 3\sqrt{7} = \left(\frac{a + b\sqrt{7}}{c}\right)^2$ za neke cele brojeve.
$(\sqrt{7/2} + \sqrt{9/2})^2 = 7/2 + 2\sqrt{63/4} + 9/2 = 8 + \sqrt{63} = 8 + 3\sqrt{7}$ ✓

Dakle $\sqrt{8+3\sqrt{7}} = \sqrt{7/2} + \sqrt{9/2} = \frac{\sqrt{14} + 3\sqrt{2}}{2}$

Sada ceo izraz:
\[
3 \cdot \frac{4 + \sqrt{7}}{3} - \frac{\sqrt{14} + 3\sqrt{2}}{2} \cdot \sqrt{2} = (4 + \sqrt{7}) - \frac{\sqrt{28} + 3 \cdot 2}{2}
\]
\[
= 4 + \sqrt{7} - \frac{2\sqrt{7} + 6}{2} = 4 + \sqrt{7} - \sqrt{7} - 3 = 1
\]

\subsection*{Answer}
$1$ (opcija \textbf{(E)}).

\end{document}

