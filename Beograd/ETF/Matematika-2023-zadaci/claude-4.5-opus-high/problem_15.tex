\documentclass[12pt]{article}
\usepackage[margin=1in]{geometry}
\usepackage{amsmath,amssymb}
\begin{document}

\section*{Problem 15}
Binomni koeficijent četvrtog člana u razvoju binoma $\left(\sqrt{2} + \frac{1}{\sqrt[3]{2}}\right)^n$, $n \in N$, veći je 26 puta od binomnog koeficijenta trećeg člana. Broj racionalnih sabiraka u ovom razvoju iznosi:

(A) 9 \quad (B) 14 \quad (C) 15 \quad (D) 16 \quad (E) 13

\subsection*{Solution}
Binomni koeficijent $k$-tog člana (brojimo od 1) je $\binom{n}{k-1}$.

Binomni koeficijent 4. člana: $\binom{n}{3}$

Binomni koeficijent 3. člana: $\binom{n}{2}$

Uslov:
\[
\binom{n}{3} = 26 \cdot \binom{n}{2}
\]
\[
\frac{n!}{3!(n-3)!} = 26 \cdot \frac{n!}{2!(n-2)!}
\]
\[
\frac{n(n-1)(n-2)}{6} = 26 \cdot \frac{n(n-1)}{2}
\]
\[
\frac{n-2}{6} = \frac{26}{2} = 13
\]
\[
n - 2 = 78
\]
\[
n = 80
\]

Sada tražimo broj racionalnih sabiraka u razvoju $\left(\sqrt{2} + \frac{1}{\sqrt[3]{2}}\right)^{80}$.

Opšti član razvoja:
\[
T_{k+1} = \binom{80}{k} \cdot (\sqrt{2})^{80-k} \cdot \left(\frac{1}{\sqrt[3]{2}}\right)^k = \binom{80}{k} \cdot 2^{\frac{80-k}{2}} \cdot 2^{-\frac{k}{3}}
\]
\[
= \binom{80}{k} \cdot 2^{\frac{80-k}{2} - \frac{k}{3}} = \binom{80}{k} \cdot 2^{\frac{3(80-k) - 2k}{6}} = \binom{80}{k} \cdot 2^{\frac{240 - 5k}{6}}
\]

Da bi član bio racionalan, eksponent mora biti ceo broj:
\[
\frac{240 - 5k}{6} \in \mathbb{Z}
\]
\[
240 - 5k \equiv 0 \pmod{6}
\]
\[
5k \equiv 240 \equiv 0 \pmod{6}
\]
\[
5k \equiv 0 \pmod{6}
\]

Pošto je $\gcd(5, 6) = 1$, imamo $k \equiv 0 \pmod{6}$.

Dakle $k \in \{0, 6, 12, 18, 24, 30, 36, 42, 48, 54, 60, 66, 72, 78\}$ (svi umnošci od 6 do 78, uključujući 0 i 78 jer je $k \leq 80$).

Broj takvih vrednosti: $k = 6m$ za $m = 0, 1, 2, \ldots, 13$, što je 14 vrednosti.

\subsection*{Answer}
$14$ (opcija \textbf{(B)}).

\end{document}

