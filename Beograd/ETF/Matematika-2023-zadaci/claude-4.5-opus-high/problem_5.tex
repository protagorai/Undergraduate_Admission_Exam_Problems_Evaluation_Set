\documentclass[12pt]{article}
\usepackage[margin=1in]{geometry}
\usepackage{amsmath,amssymb}
\begin{document}

\section*{Problem 5}
Neka je $B$ tačka na kružnici poluprečnika $r$ i $BC$ tangenta duž dužine 8 cm. Ako je $A$ tačka na istoj kružnici takva da je duž $AC$ dužine 9 cm i da sadrži centar kružnice, onda obim kružnice (u cm) iznosi:

(A) $\frac{36}{17}\pi$ \quad (B) $2\pi$ \quad (C) $\frac{11}{9}\pi$ \quad (D) $\frac{17}{9}\pi$ \quad (E) $\frac{289}{324}\pi$

\subsection*{Solution}
Neka je $O$ centar kružnice poluprečnika $r$. Tačka $B$ je na kružnici, tako da je $OB = r$.

$BC$ je tangenta na kružnicu u tački $B$, pa je $OB \perp BC$, odnosno ugao $OBC = 90°$.

Tačka $A$ je na kružnici i $AC$ prolazi kroz centar $O$, što znači da je $AC$ dijametar kružnice produžen do tačke $C$, ili $A$ i $C$ su na suprotnim stranama od centra.

Ako $C$ nije na kružnici (jer je $BC$ tangenta), a $AC$ sadrži centar $O$, onda:
- $OA = r$ (pošto je $A$ na kružnici)
- $AC = 9$ cm
- $OC = AC - OA = 9 - r$ ili $OC = OA + AC$ zavisno od položaja

Koristimo činjenicu da je $OB \perp BC$ i Pitagorinu teoremu u trouglu $OBC$:
\[
OC^2 = OB^2 + BC^2 = r^2 + 64
\]

Ako je $A$ diametralno suprotna tačka od preseka $AC$ sa kružnicom bliže $C$, tada:
$OC = 9 - r$ (ako je $C$ van kružnice i $O$ između $A$ i $C$)

\[
(9-r)^2 = r^2 + 64
\]
\[
81 - 18r + r^2 = r^2 + 64
\]
\[
81 - 18r = 64
\]
\[
18r = 17
\]
\[
r = \frac{17}{18}
\]

Obim kružnice:
\[
O = 2\pi r = 2\pi \cdot \frac{17}{18} = \frac{17\pi}{9}
\]

\subsection*{Answer}
$\frac{17}{9}\pi$ cm (opcija \textbf{(D)}).

\end{document}

