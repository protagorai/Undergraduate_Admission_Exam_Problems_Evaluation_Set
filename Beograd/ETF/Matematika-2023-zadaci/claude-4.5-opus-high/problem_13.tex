\documentclass[12pt]{article}
\usepackage[margin=1in]{geometry}
\usepackage{amsmath,amssymb}
\begin{document}

\section*{Problem 13}
Pri deljenju polinoma $P_1$ polinomom $x^2 - 1$ dobija se ostatak $x$, a pri deljenju polinoma $P_2$ polinomom $x^2 - 1$ dobija se ostatak $x + 2$. Tada je ostatak pri deljenju polinoma $P_1 \cdot P_2$ polinomom $x^2 - 1$ jednak:

(A) 1 \quad (B) $x + 2$ \quad (C) $2x$ \quad (D) $2x + 1$ \quad (E) $2x - 1$

\subsection*{Solution}
Neka je:
\[
P_1(x) = (x^2 - 1) \cdot Q_1(x) + x
\]
\[
P_2(x) = (x^2 - 1) \cdot Q_2(x) + (x + 2)
\]

Tada:
\[
P_1(x) \cdot P_2(x) = [(x^2-1)Q_1(x) + x] \cdot [(x^2-1)Q_2(x) + (x+2)]
\]

Razvijamo:
\[
= (x^2-1)^2 Q_1(x) Q_2(x) + (x^2-1)Q_1(x)(x+2) + (x^2-1)Q_2(x) \cdot x + x(x+2)
\]

Svi članovi osim poslednjeg su deljivi sa $x^2 - 1$, pa je ostatak jednak ostatku pri deljenju $x(x+2)$ sa $x^2 - 1$:
\[
x(x+2) = x^2 + 2x
\]

Delimo $x^2 + 2x$ sa $x^2 - 1$:
\[
x^2 + 2x = 1 \cdot (x^2 - 1) + (2x + 1)
\]

Provera: $x^2 - 1 + 2x + 1 = x^2 + 2x$ ✓

\subsection*{Answer}
$2x + 1$ (opcija \textbf{(D)}).

\end{document}

