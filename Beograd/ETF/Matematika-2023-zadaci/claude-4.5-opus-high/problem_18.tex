\documentclass[12pt]{article}
\usepackage[margin=1in]{geometry}
\usepackage{amsmath,amssymb}
\begin{document}

\section*{Problem 18}
Broj različitih vrednosti parametra $p \in R$ za koje jednačina $\frac{p^2}{x+1} - \frac{x(p+2)}{x^2-1} = \frac{2p}{1-x^2}$ nema rešenja iznosi:

(A) 1 \quad (B) 2 \quad (C) 0 \quad (D) više od 3 \quad (E) 3

\subsection*{Solution}
Primetimo da je $x^2 - 1 = (x-1)(x+1)$ i $1 - x^2 = -(x^2-1)$.

Jednačina postaje:
\[
\frac{p^2}{x+1} - \frac{x(p+2)}{(x-1)(x+1)} = \frac{-2p}{x^2-1}
\]

Domen: $x \neq 1$ i $x \neq -1$.

Množimo sa $(x-1)(x+1)$:
\[
p^2(x-1) - x(p+2) = -2p
\]
\[
p^2 x - p^2 - px - 2x = -2p
\]
\[
x(p^2 - p - 2) = p^2 - 2p
\]
\[
x(p^2 - p - 2) = p(p - 2)
\]

Faktorizujemo $p^2 - p - 2 = (p-2)(p+1)$:
\[
x(p-2)(p+1) = p(p-2)
\]

\textbf{Slučaj 1:} $p = 2$

Jednačina postaje $0 = 0$, što je tačno za sve $x$ u domenu. Dakle jednačina ima beskonačno rešenja, ne "nema rešenja".

\textbf{Slučaj 2:} $p \neq 2$

Delimo sa $(p-2)$:
\[
x(p+1) = p
\]

\textbf{Podslučaj 2a:} $p = -1$

Jednačina postaje $0 = -1$, što je nemoguće. Dakle za $p = -1$ jednačina nema rešenja.

\textbf{Podslučaj 2b:} $p \neq -1$
\[
x = \frac{p}{p+1}
\]

Ovo rešenje mora biti u domenu, tj. $x \neq 1$ i $x \neq -1$.

$x = 1$: $\frac{p}{p+1} = 1 \Rightarrow p = p + 1$, nemoguće. Dakle $x \neq 1$ za sve $p$.

$x = -1$: $\frac{p}{p+1} = -1 \Rightarrow p = -p - 1 \Rightarrow 2p = -1 \Rightarrow p = -\frac{1}{2}$

Za $p = -\frac{1}{2}$ dobijamo $x = -1$, što nije u domenu. Dakle za $p = -\frac{1}{2}$ jednačina nema rešenja.

Vrednosti $p$ za koje jednačina nema rešenja: $p = -1$ i $p = -\frac{1}{2}$.

\subsection*{Answer}
$2$ (opcija \textbf{(B)}).

\end{document}

