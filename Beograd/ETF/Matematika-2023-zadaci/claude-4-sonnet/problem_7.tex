\documentclass[12pt]{article}
\usepackage[margin=1in]{geometry}
\usepackage{amsmath,amssymb}
\begin{document}

\section*{Problem 7}
U kocku $K_1$ ivice 1 cm upisana je lopta $L_1$, zatim je u loptu $L_1$ upisana kocka $K_2$, zatim u nju lopta $L_2$ i zatim se postupak nastavlja na isti način. Zbir površina (u $\text{cm}^2$) svih kocki $K_n$, $n \in \mathbb{N}$, iznosi:

\subsection*{Solution}
Analizirajmo postupak korak po korak.

\textbf{Korak 1:} Kocka $K_1$ ima ivicu $a_1 = 1$ cm.

Upisana lopta $L_1$ ima poluprečnik $r_1 = \frac{a_1}{2} = \frac{1}{2}$ cm.

\textbf{Korak 2:} U loptu $L_1$ upisujemo kocku $K_2$.

Dijagonala kocke $K_2$ jednaka je prečniku lopte $L_1$:
\[
a_2\sqrt{3} = 2r_1 = 2 \cdot \frac{1}{2} = 1
\]
\[
a_2 = \frac{1}{\sqrt{3}}
\]

\textbf{Korak 3:} U kocku $K_2$ upisujemo loptu $L_2$:
\[
r_2 = \frac{a_2}{2} = \frac{1}{2\sqrt{3}}
\]

\textbf{Opšti obrazac:}
Vidimo da je:
\[
a_{n+1} = \frac{2r_n}{\sqrt{3}} = \frac{2 \cdot \frac{a_n}{2}}{\sqrt{3}} = \frac{a_n}{\sqrt{3}}
\]

Dakle: $a_n = \frac{1}{(\sqrt{3})^{n-1}}$.

Površina kocke $K_n$ je:
\[
P_n = 6a_n^2 = 6 \cdot \frac{1}{(\sqrt{3})^{2(n-1)}} = 6 \cdot \frac{1}{3^{n-1}} = \frac{6}{3^{n-1}} = 2 \cdot 3^{-(n-1)} = 2 \cdot 3^{1-n}
\]

Zbir svih površina:
\[
\sum_{n=1}^{\infty} P_n = \sum_{n=1}^{\infty} 2 \cdot 3^{1-n} = 2 \cdot 3 \sum_{n=1}^{\infty} 3^{-n} = 6 \sum_{n=1}^{\infty} \left(\frac{1}{3}\right)^n
\]

Ovo je geometrijski red sa prvim članom $\frac{1}{3}$ i količnikom $\frac{1}{3}$:
\[
\sum_{n=1}^{\infty} \left(\frac{1}{3}\right)^n = \frac{\frac{1}{3}}{1 - \frac{1}{3}} = \frac{\frac{1}{3}}{\frac{2}{3}} = \frac{1}{2}
\]

Dakle:
\[
\sum_{n=1}^{\infty} P_n = 6 \cdot \frac{1}{2} = 3
\]

\subsection*{Answer}
$3$ (opcija \textbf{E}).

\end{document}
