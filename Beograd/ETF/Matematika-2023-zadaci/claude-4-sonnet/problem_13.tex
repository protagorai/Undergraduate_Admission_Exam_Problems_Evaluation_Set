\documentclass[12pt]{article}
\usepackage[margin=1in]{geometry}
\usepackage{amsmath,amssymb}
\begin{document}

\section*{Problem 13}
Pri deljenju polinoma $P_1$ polinomom $x^2 - 1$ dobija se ostatak $x$, a pri deljenju polinoma $P_2$ polinomom $x^2 - 1$ dobija se ostatak $x + 2$. Tada je ostatak pri deljenju polinoma $P_1 \cdot P_2$ polinomom $x^2 - 1$ jednak:

\subsection*{Solution}
Prema teoremi o deljenju polinoma:

\[
P_1(x) = (x^2 - 1) \cdot Q_1(x) + x
\]
\[
P_2(x) = (x^2 - 1) \cdot Q_2(x) + (x + 2)
\]

gde su $Q_1(x)$ i $Q_2(x)$ količnici.

Treba naći ostatak pri deljenju $P_1(x) \cdot P_2(x)$ sa $x^2 - 1$.

\[
P_1(x) \cdot P_2(x) = [(x^2 - 1) \cdot Q_1(x) + x] \cdot [(x^2 - 1) \cdot Q_2(x) + (x + 2)]
\]

Razvijamo:
\begin{align}
P_1(x) \cdot P_2(x) &= (x^2 - 1)^2 Q_1(x) Q_2(x) + (x^2 - 1) Q_1(x)(x + 2)\\
&\quad + (x^2 - 1) Q_2(x) \cdot x + x(x + 2)\\
&= (x^2 - 1)[Q_1(x) Q_2(x)(x^2 - 1) + Q_1(x)(x + 2) + Q_2(x) \cdot x] + x(x + 2)
\end{align}

Ostatak je $x(x + 2) = x^2 + 2x$.

Međutim, pošto delimo sa $x^2 - 1$, ostatak mora biti polinom stepena manjeg od 2, tj. oblika $ax + b$.

Pošto je $x^2 \equiv 1 \pmod{x^2 - 1}$:
\[
x^2 + 2x \equiv 1 + 2x \pmod{x^2 - 1}
\]

Dakle, ostatak je $2x + 1$.

\subsection*{Answer}
$2x + 1$ (opcija \textbf{D}).

\end{document}
