\documentclass[12pt]{article}
\usepackage[margin=1in]{geometry}
\usepackage{amsmath,amssymb}
\begin{document}

\section*{Problem 3}
Trapez je opisan oko kruga poluprečnika $r$. Ako je poznato da je površina trapeza (u $\text{cm}^2$) pet puta veća od obima tog trapeza (u cm), tada dužina poluprečnika $r$ (u cm) iznosi:

\subsection*{Solution}
Za trapez opisan oko kruga (tangentni trapez) važi da je zbir suprotnih stranica jednak.

Označimo stranice trapeza kao $a$, $b$, $c$, $d$, gde su $a$ i $c$ paralelne stranice (osnove), a $b$ i $d$ kraci.

Za tangentni četvorougao važi: $a + c = b + d$.

Obim trapeza je: $O = a + b + c + d = 2(a + c)$.

Površina trapeza opisanog oko kruga poluprečnika $r$ je:
\[
P = r \cdot s
\]
gde je $s$ poluobim, tj. $s = \frac{O}{2} = a + c$.

Dakle: $P = r(a + c)$.

Prema uslovu zadatka:
\[
P = 5O
\]
\[
r(a + c) = 5 \cdot 2(a + c)
\]
\[
r(a + c) = 10(a + c)
\]

Pošto je $a + c > 0$ (stranice trapeza), možemo podeliti sa $(a + c)$:
\[
r = 10
\]

\subsection*{Answer}
$10$ (opcija \textbf{C}).

\end{document}
