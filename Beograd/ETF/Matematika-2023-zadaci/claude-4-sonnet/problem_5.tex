\documentclass[12pt]{article}
\usepackage[margin=1in]{geometry}
\usepackage{amsmath,amssymb}
\begin{document}

\section*{Problem 5}
Neka je $B$ tačka na kružnici poluprečnika $r$ i $BC$ tangentna duž dužine 8 cm. Ako je $A$ tačka na istoj kružnici takva da je duž $AC$ dužine 9 cm i da sadrži centar kružnice, onda obim kružnice (u cm) iznosi:

\subsection*{Solution}
Označimo centar kružnice sa $O$.

Dato je:
\begin{itemize}
\item $B$ je tačka na kružnici poluprečnika $r$
\item $BC$ je tangenta na kružnicu, $|BC| = 8$ cm
\item $A$ je tačka na kružnici, $|AC| = 9$ cm
\item Duž $AC$ sadrži centar $O$
\end{itemize}

Pošto $A$ leži na kružnici i duž $AC$ sadrži centar $O$, to znači da $A$ i $C$ leže na istoj strani centra ili da je $C$ između $A$ i neke druge tačke na kružnici kroz centar.

Pošto je $BC$ tangenta na kružnicu u tački $B$, ugao $\angle OBC = 90°$.

U pravouglom trouglu $OBC$:
\[
|OC|^2 = |OB|^2 + |BC|^2 = r^2 + 8^2 = r^2 + 64
\]

Dakle: $|OC| = \sqrt{r^2 + 64}$.

Pošto duž $AC$ sadrži centar $O$, imamo dva slučaja:
1. $|AC| = |AO| + |OC| = r + \sqrt{r^2 + 64} = 9$
2. $|AC| = ||AO| - |OC|| = |r - \sqrt{r^2 + 64}| = 9$

Pošto je $\sqrt{r^2 + 64} > r$ za $r > 0$, drugi slučaj daje:
$\sqrt{r^2 + 64} - r = 9$

Rešavamo:
\[
\sqrt{r^2 + 64} = r + 9
\]

Kvadriramo obe strane:
\[
r^2 + 64 = (r + 9)^2 = r^2 + 18r + 81
\]
\[
64 = 18r + 81
\]
\[
18r = 64 - 81 = -17
\]
\[
r = -\frac{17}{18}
\]

Ovo nije moguće jer je $r > 0$.

Probajmo prvi slučaj:
\[
r + \sqrt{r^2 + 64} = 9
\]
\[
\sqrt{r^2 + 64} = 9 - r
\]

Kvadriramo:
\[
r^2 + 64 = (9 - r)^2 = 81 - 18r + r^2
\]
\[
64 = 81 - 18r
\]
\[
18r = 81 - 64 = 17
\]
\[
r = \frac{17}{18}
\]

Proveravamo: $\sqrt{(\frac{17}{18})^2 + 64} = \sqrt{\frac{289}{324} + 64} = \sqrt{\frac{289 + 20736}{324}} = \sqrt{\frac{21025}{324}} = \frac{145}{18}$

$r + \sqrt{r^2 + 64} = \frac{17}{18} + \frac{145}{18} = \frac{162}{18} = 9$ ✓

Obim kružnice je:
\[
O = 2\pi r = 2\pi \cdot \frac{17}{18} = \frac{17\pi}{9}
\]

\subsection*{Answer}
$\frac{17\pi}{9}$ (opcija \textbf{D}).

\end{document}
