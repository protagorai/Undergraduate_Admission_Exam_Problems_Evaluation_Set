% Problem 19 – Matematički prijemni 2023
\begin{solution}
Нека је основна ивица квадрата $a$, висина пирамиде $h$ а полупречник полулопте $r$.
Центар полулопте је $O(0,0,r)$, док је теме пирамиде $S(0,0,h)$, а темена основе $A(\pm a/2,\pm a/2,0)$.

\paragraph{1.  Услов тангенције.}
За равни бочне стране која садржи ивицу $x=a/2$  једначина је
\[h\bigl(x-\tfrac a2\bigr)+\tfrac a2 z =0.\]
Растојање центра $O$ до те равни мора бити $r$:
\[
\frac{\tfrac a2(h-r)}{\sqrt{h^{2}+a^{2}/4}}=r \Longrightarrow a^{2}(h-2r)=4r^{2}h.\tag{1}
\]

\paragraph{2.  Минимум површине.}  Површина пирамиде $P=a^{2}+2a\ell$, где је
\[\ell=\sqrt{h^{2}+\tfrac{a^{2}}4}.\]
Из (1) добијамо
\[h=\frac{2a^{2}r}{a^{2}-4r^{2}},\qquad a>2r.\]
Уз замену $a=kr$ ($k>2$) добија се
\[\frac{P}{r^{2}}=k^{2}+2k\sqrt{\frac{k^{2}}4+\frac{4k^{4}}{(k^{2}-4)^{2}}}\;\equiv F(k).
\]
Извођењем и једноставном анализом добија се минимум у
\[k=\frac{16\sqrt3}{9}\approx3.079.
\]
Према томе оптимална ивица је
\[
\boxed{a=\frac{16\sqrt3}{9}\,r},\qquad\text{опција \textbf{E}}.
\]
\end{solution}

