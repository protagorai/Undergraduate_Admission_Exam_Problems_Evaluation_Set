% Problem 13 – Matematički prijemni 2023
\begin{solution}
Нека су
\[
P_{1}(x)=Q_{1}(x)(x^{2}-1)+x, \qquad
P_{2}(x)=Q_{2}(x)(x^{2}-1)+(x+2)
\]
остатци при дељењу полиномима $x^{2}-1$.  Жељени остатак добијамо рачунањем у 
конгруенцији $\bmod(x^{2}-1)$, где важи $x^{2}\equiv1$:
\[
P_{1}P_{2}\equiv x(x+2)=x^{2}+2x\equiv1+2x.\tag{\*}
\]
Пошто степен остатка мора бити мањи од $2$, вредност (\*) већ 
представља остатак.  Дакле
\[
\boxed{\;1+2x\;}\qquad (\text{опција\;D}).
\]
\end{solution}

