% Problem 5 – Matematički prijemni 2023
\begin{solution}
Neka je \(O\) centar kružnice poluprečnika \(r\).  Tačka \(B\) je na kružnici, a \(C\) je spoljašnja tačka tako da je \(BC=8\,\text{cm}\) tangentna duž.  Tangenta iz spoljašnje tačke je normalna na radijus: \(OB\perp BC\).

Data je i tačka \(A\) na kružnici takva da pravac \(AC\) prolazi kroz centar \(O\) i \(|AC|=9\,\text{cm}.\)  Тиме је
\[
OC = AC - AO = 9-r.  \tag{1}
\]
Пошто је \(BC\) тангента, важи Питагора у троуглу \(OBC\):
\[
OC^{2}=OB^{2}+BC^{2}=r^{2}+8^{2}. \tag{2}
\]
Убацујући (1) у (2):
\[
(9-r)^{2}=r^{2}+64 \Longrightarrow 81-18r+r^{2}=r^{2}+64 \Longrightarrow 81-18r=64.
\]
Одатле је
\[r=\frac{17}{18}\;\text{cm}.\]

Обим круга је
\[
O=2\pi r = 2\pi\cdot\frac{17}{18}=\boxed{\tfrac{17}{9}\,\pi\,\text{cm}}.
\]
То је понуђена опција \textbf{D}. \qed
\end{solution}

