% Problem 7 – Matematički prijemni 2023
\begin{solution}
У кockу $K_{1}$ ивице 1 уписујемо лопту, затим у лопту кockу $K_{2}$ итд.
Свакa лоптa је описана око наредне кockе, па се низ ивица формира правилом:
\[
a_{1}=1,\qquad a_{n+1}=\frac{\sqrt2}{2}\,a_{n}\quad(n\ge1).
\]
Разлог: дијагонала квадрата основе кockе $a_{n}\sqrt2$ једнака је пречнику лопте, а ивица наредне кockе је дијаметар лопте пута $\tfrac{1}{\sqrt2}$.

Дакле $$a_{n}=\Bigl(\tfrac{\sqrt2}{2}\Bigr)^{n-1}. $$
Површина кockе је $6a_{n}^{2}$, па је збир површина
\[
S=6\sum_{n=1}^{\infty} a_{n}^{2}=6\sum_{n=1}^{\infty}\Bigl(\tfrac{1}{2}\Bigr)^{n-1}=6\cdot\frac{1}{1-\tfrac12}=\boxed{12},
\]
који није међу понуђеним одговорима, па бирамо последњу опцију.
\[
\boxed{\text{N (Не знам)}}
\]
\end{solution}

