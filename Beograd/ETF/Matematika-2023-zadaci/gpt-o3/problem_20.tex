% Problem 20 – Matematički prijemni 2023
\begin{solution}
Разматрамо пресеци праве $y=x+n$ са кривом $f(x)=x^{3}-2x^{2}+x-2$.  То су реални корени кубне
\[
g_{n}(x)=x^{3}-2x^{2}+x-2-(x+n)=x^{3}-2x^{2}-n-2.
\]
Изведена функција $g_{n}'(x)=3x^{2}-4x$ има корене $x=0$ и $x=\tfrac43$.  Вредности
\[
g_{n}(0)=-n-2,\qquad g_{n}\bigl(\tfrac43\bigr)=\Bigl(\tfrac43\Bigr)^{3}-2\Bigl(\tfrac43\Bigr)^{2}-n-2=-\tfrac{16}{27}-n.
\]
Граф кубне има два екстрема: $P(0,-n-2)$ (макс) и $Q(\tfrac43,-\tfrac{16}{27}-n)$ (мин).  Максимални број пресека (3) добија се ако су $P$ изнад $x$-осе, а $Q$ испод:
\[
 -n-2>0\;\Longrightarrow\;n<-2,\qquad -\tfrac{16}{27}-n<0\;\Longrightarrow\;n>-\tfrac{16}{27}.
\]
Ниједно $n$ не може истовремено испунити оба услова, па је максималан број 2.  До двоструког корена долази кад $g_{n}(\tfrac43)=0\;\Rightarrow\;n=-\tfrac{16}{27}$.  За $n<-\tfrac{16}{27}$ добијамо 3 корена (један негативан, два позитивна), за $n>-\tfrac{16}{27}$ само 1.
Стога скуп $n$ за који број пресека прелази на максимум је $(-\infty,-\tfrac{16}{27}]$.

То одговара облику $( -\infty,a]\cup\{b\}$ са $a=b=-\tfrac{16}{27}$, па је опција \textbf{A}.  
\end{solution}

