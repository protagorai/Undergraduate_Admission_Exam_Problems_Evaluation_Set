% Problem 1 – Matematički prijemni 2023
\begin{solution}
Tražimo vrednost izraza
\[
E=\frac{\sqrt 8+2\sqrt 7}{\sqrt 8-2\sqrt 7}\cdot\frac{\sqrt 3+\sqrt 7}{\sqrt 3-\sqrt 7}\cdot\sqrt 2.
\]
Prvi razlomak racionališemo pomoću konjugata:
\[
\frac{\sqrt 8+2\sqrt 7}{\sqrt 8-2\sqrt 7}=\frac{(\sqrt 8+2\sqrt 7)(\sqrt 8+2\sqrt 7)}{8-(2\sqrt 7)^2}=\frac{(\sqrt 8+2\sqrt 7)^2}{8-28}
  =\frac{9+2\sqrt{14}}{-20}= -\,\frac{9+2\sqrt{14}}{5}.
\]
Za drugi razlomak dobijamo
\[
\frac{\sqrt 3+\sqrt 7}{\sqrt 3-\sqrt 7}=\frac{(\sqrt 3+\sqrt 7)^2}{3-7}=\frac{10+2\sqrt{21}}{-4}= -\,\frac{5+\sqrt{21}}{2}.
\]
Dakle
\[
E=\Bigl(-\frac{9+2\sqrt{14}}{5}\Bigr)\Bigl(-\frac{5+\sqrt{21}}{2}\Bigr)\sqrt 2
  =\frac{(9+2\sqrt{14})(5+\sqrt{21})}{10}\,\sqrt 2.
\]
Iz množenja brojnika sledi
\[
(9+2\sqrt{14})(5+\sqrt{21})=45+9\sqrt{21}+10\sqrt{14}+14\sqrt 6,
\]
pa je
\[E=\frac{45+9\sqrt{21}+10\sqrt{14}+14\sqrt 6}{10}\,\sqrt 2\approx22{,}36.\]
Ta vrednost se ne poklapa ni sa jednom od ponuđenih (2,\,\sqrt 6,\,3\sqrt2,\,\sqrt2,\,1).
Prema uputstvu biramo poslednju opciju.
\[
\boxed{\text{N (Не знам)}}
\]
\end{solution}

