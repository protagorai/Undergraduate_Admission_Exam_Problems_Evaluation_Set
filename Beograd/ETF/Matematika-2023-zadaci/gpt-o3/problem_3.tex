% Problem 3 – Matematički prijemni 2023
\begin{solution}
Nek je trapez \(ABCD\) opisan oko kruga poluprečnika \(r\). Označimo baze \(AB=a,\;CD=b\) i krake \(BC,\,DA\).

Za svaki trapez opisan oko kruga važi
\[
\text{Površina } P = r\,(a+b)\qquad (\ast)
\]
jer je \(a+b\) zbir dužina paralelnih stranica, tj. poluobim \emph{Tangencijalnog četvorougla}.

Obim trapeza je
\[
O = a+b+BC+DA = 2\,(a+b)\;\; \text{( Tangencijalni: $BC+DA=a+b$)}.
\]
Data relacija: \(P=5O\) daje
\[
 r(a+b)=5\,\bigl(2(a+b)\bigr) \Longrightarrow r(a+b)=10(a+b).
\]
Kako \(a+b>0\) sledi
\[ r=10.\]
Odgovor je \boxed{10} \; (opcija \textbf{C}).
\end{solution}

