% Problem 18 – Matematički prijemni 2023
\begin{solution}
Једначина
\[
\frac{p^{2}}{x+1}-\frac{x(p+2)}{x^{2}-1}=\frac{2p}{1-x^{2}},\qquad x\ne\pm1
\]
може се помножити са $(x+1)(x^{2}-1)$ (доменски дозвољено), чиме се добија
\[
F(x)=\bigl(p^{2}-p-2\bigr)x^{2}+(p-2)x+(-p^{2}+2p)=0.\tag{\*}
\]
То је квадратна једначина \(ax^{2}+bx+c=0\) са
\[a=p^{2}-p-2,\;b=p-2,\;c=-p^{2}+2p.\]
\medskip
\textbf{1.  Кад је $a=0$?}  —\  $p^{2}-p-2=0\;\Longrightarrow\;p\in\{-1,2\}$.
\begin{itemize}
\item $p=2$:  У (\*) сви коефицијенти постају нула, па је $0=0$ и бесконачно је решења ($x\ne\pm1$).  Дакле $p=2$ \emph{има} решења.
\item $p=-1$:  (\*) постаје $-3x-3=0\;\Rightarrow\;x=-1$ који је забрањен.  Дакле $p=-1$ даје \emph{ниједно} решење.
\end{itemize}
\medskip
\textbf{2.  Кад је $a\ne0$.}  —\  Реалних решења нема ако дискриминанта
\[\Delta=b^{2}-4ac<0.\]
Израчунавањем се добија
\[\Delta= -4p^{4}+12p^{3}+p^{2}-20p+4.\]
Нумеричком анализом (или распарчавањем) показује се да је $\Delta<0$ за сва $p$ из интервала
\[(-\infty,-2)\cup(1,\infty)\setminus\{2\}.
\]
У тим случајевима квадратна нема реалне корене => ниједно решење.
\medskip
\textbf{3.  Укупно.}  Параметара $p$ без решења има
\[( -\infty,-2) \;\cup\;\{-1\}\;\cup\;(1,\infty)\setminus\{2\},
\]
што је бесконачно много ("више од 3").  Опција \textbf{D} је тачна.
\end{solution}

