% Problem 15 – Matematički prijemni 2023
\begin{solution}
Посматрамо биномен $(\sqrt2+1/\sqrt2)^{n}$, $n\in\mathbb N$.  Трећи ("3."), односно четврти ("4.") члан развоја (нумерисано од 1) имају бинoмне коефицијенте $\binom n2$ и $\binom n3$.  Дато је
\[
\binom n3 = 26\,\binom n2.
\]
Пошто је $\binom n3/\binom n2 = (n-2)/3$, тачно је кад
\[
\frac{n-2}{3}=26\;\Longrightarrow\;n=80.
\]
\medskip
Сваком $k$-том члану одговара степен
\[
(\sqrt2)^{n-k}\Bigl(\tfrac1{\sqrt2}\Bigr)^{k}=2^{(n-2k)/2}=2^{40-k},\qquad k=0,1,\dots,80.
\]
За парно $n$ сваки израз $2^{40-k}$ је рационалан, па је 
\emph{свих $n+1=81$ сабирака рационално}.  Тај број није ни у једном понуђеном одговору (12,14,15,16,17), па према задатом правилу бирамо последњу опцију.
\[
\boxed{\text{N (Не знам)}}
\]
\end{solution}

