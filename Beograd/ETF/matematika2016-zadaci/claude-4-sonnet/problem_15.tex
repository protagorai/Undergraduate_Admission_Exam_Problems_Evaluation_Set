\documentclass[12pt]{article}
\usepackage[margin=1in]{geometry}
\usepackage{amsmath,amssymb}
\begin{document}

\section*{Problem 15}
Find the total number of real solutions to the system of equations:
\[
\begin{cases}
\sqrt{x - 1} + \sqrt[3]{y} = 1\\
x - y = 2
\end{cases}
\]

\subsection*{Solution}
From the second equation: $x = y + 2$.

Substituting into the first equation:
\[
\sqrt{(y + 2) - 1} + \sqrt[3]{y} = 1
\]
\[
\sqrt{y + 1} + \sqrt[3]{y} = 1
\]

For $\sqrt{y + 1}$ to be defined, we need $y \geq -1$.

Let $u = \sqrt[3]{y}$, so $y = u^3$. The equation becomes:
\[
\sqrt{u^3 + 1} + u = 1
\]
\[
\sqrt{u^3 + 1} = 1 - u
\]

For the right side to be non-negative, we need $1 - u \geq 0$, so $u \leq 1$.

Also, since $y = u^3 \geq -1$, we need $u^3 \geq -1$, which gives $u \geq -1$.

So we have $-1 \leq u \leq 1$.

Squaring both sides (valid since both sides are non-negative):
\[
u^3 + 1 = (1 - u)^2 = 1 - 2u + u^2
\]
\[
u^3 + 1 = 1 - 2u + u^2
\]
\[
u^3 - u^2 + 2u = 0
\]
\[
u(u^2 - u + 2) = 0
\]

This gives us $u = 0$ or $u^2 - u + 2 = 0$.

For $u^2 - u + 2 = 0$:
The discriminant is $\Delta = 1 - 8 = -7 < 0$, so there are no real solutions.

Therefore, $u = 0$ is the only solution.

When $u = 0$: $y = u^3 = 0$ and $x = y + 2 = 2$.

Let's verify: $\sqrt{2 - 1} + \sqrt[3]{0} = 1 + 0 = 1$ ✓
And: $2 - 0 = 2$ ✓

We should also check that our squaring didn't introduce extraneous solutions:
When $u = 0$: $\sqrt{0^3 + 1} = 1$ and $1 - u = 1 - 0 = 1$, so $1 = 1$ ✓

Therefore, there is exactly 1 real solution: $(x, y) = (2, 0)$.

\subsection*{Answer}
1 (option \textbf{B}).

\end{document}