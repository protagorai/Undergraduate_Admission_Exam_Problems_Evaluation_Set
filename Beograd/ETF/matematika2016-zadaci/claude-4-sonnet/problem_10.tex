\documentclass[12pt]{article}
\usepackage[margin=1in]{geometry}
\usepackage{amsmath,amssymb}
\begin{document}

\section*{Problem 10}
In isosceles triangle ABC, $AB = AC = b$ and $\angle BAC = 30°$. Find the height of the triangle.

\subsection*{Solution}
Let's denote the height from vertex A to base BC as h, and let it meet BC at point D.

Since triangle ABC is isosceles with $AB = AC = b$, the height from A bisects both the angle BAC and the base BC.

Therefore, $\angle BAD = \angle CAD = \frac{30°}{2} = 15°$.

In right triangle ABD:
- $AB = b$ (hypotenuse)
- $AD = h$ (height we want to find)
- $\angle BAD = 15°$

Using trigonometry:
\[
\cos(15°) = \frac{AD}{AB} = \frac{h}{b}
\]

Therefore:
\[
h = b \cos(15°)
\]

We need to find $\cos(15°)$. Using the half-angle formula:
\[
\cos(15°) = \cos\left(\frac{30°}{2}\right) = \sqrt{\frac{1 + \cos(30°)}{2}}
\]

Since $\cos(30°) = \frac{\sqrt{3}}{2}$:
\[
\cos(15°) = \sqrt{\frac{1 + \frac{\sqrt{3}}{2}}{2}} = \sqrt{\frac{2 + \sqrt{3}}{4}} = \frac{\sqrt{2 + \sqrt{3}}}{2}
\]

We can simplify $\sqrt{2 + \sqrt{3}}$. Note that:
\[
\left(\frac{\sqrt{6} + \sqrt{2}}{2}\right)^2 = \frac{6 + 2\sqrt{12} + 2}{4} = \frac{8 + 4\sqrt{3}}{4} = 2 + \sqrt{3}
\]

Therefore:
\[
\sqrt{2 + \sqrt{3}} = \frac{\sqrt{6} + \sqrt{2}}{2}
\]

So:
\[
\cos(15°) = \frac{\sqrt{6} + \sqrt{2}}{4}
\]

Therefore, the height is:
\[
h = b \cdot \frac{\sqrt{6} + \sqrt{2}}{4} = \frac{b(\sqrt{6} + \sqrt{2})}{4}
\]

This can also be written as $\frac{b}{4}(\sqrt{6} + \sqrt{2})$.

\subsection*{Answer}
$\frac{b}{4}(\sqrt{6} + \sqrt{2})$ (option \textbf{C}).

\end{document}