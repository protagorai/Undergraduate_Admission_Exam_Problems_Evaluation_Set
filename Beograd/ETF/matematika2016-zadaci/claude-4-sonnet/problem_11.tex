\documentclass[12pt]{article}
\usepackage[margin=1in]{geometry}
\usepackage{amsmath,amssymb}
\begin{document}

\section*{Problem 11}
Given vertices of triangle $A(-8,4)$, $B(-2,1)$ and $C(1,-3)$, and orthocenter $H(x_0,y_0)$, find the value of difference $y_0 - x_0$.

\subsection*{Solution}
The orthocenter is the intersection point of the altitudes of the triangle.

First, let's find the slopes of the sides:
\begin{align}
m_{AB} &= \frac{1-4}{-2-(-8)} = \frac{-3}{6} = -\frac{1}{2}\\
m_{BC} &= \frac{-3-1}{1-(-2)} = \frac{-4}{3}\\
m_{AC} &= \frac{-3-4}{1-(-8)} = \frac{-7}{9}
\end{align}

The altitude from C to AB has slope perpendicular to AB:
\[
m_{\perp AB} = -\frac{1}{m_{AB}} = -\frac{1}{-\frac{1}{2}} = 2
\]

The equation of altitude from C(1,-3) with slope 2:
\[
y - (-3) = 2(x - 1) \Rightarrow y + 3 = 2x - 2 \Rightarrow y = 2x - 5
\]

The altitude from A to BC has slope perpendicular to BC:
\[
m_{\perp BC} = -\frac{1}{m_{BC}} = -\frac{1}{-\frac{4}{3}} = \frac{3}{4}
\]

The equation of altitude from A(-8,4) with slope $\frac{3}{4}$:
\[
y - 4 = \frac{3}{4}(x - (-8)) \Rightarrow y - 4 = \frac{3}{4}(x + 8) \Rightarrow y = \frac{3}{4}x + 6 + 4 = \frac{3}{4}x + 10
\]

To find the orthocenter, we solve:
\[
2x - 5 = \frac{3}{4}x + 10
\]

Multiplying by 4:
\[
8x - 20 = 3x + 40 \Rightarrow 5x = 60 \Rightarrow x = 12
\]

Substituting back:
\[
y = 2(12) - 5 = 24 - 5 = 19
\]

Therefore, $H(12, 19)$, so $x_0 = 12$ and $y_0 = 19$.

The difference is:
\[
y_0 - x_0 = 19 - 12 = 7
\]

\subsection*{Answer}
7 (option \textbf{A}).

\end{document}