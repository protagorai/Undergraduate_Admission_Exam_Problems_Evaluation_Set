\documentclass[12pt]{article}
\usepackage[margin=1in]{geometry}
\usepackage{amsmath,amssymb}
\begin{document}

\section*{Problem 2}
Find the number of real solutions to the equation:
\[
||1 - |x|| - 1| - 2 = 0
\]

\subsection*{Solution}
We solve this step by step, working from the inside out.

First, rewrite the equation as:
\[
||1 - |x|| - 1| = 2
\]

This means:
\[
|1 - |x|| - 1 = 2 \quad \text{or} \quad |1 - |x|| - 1 = -2
\]

From the first case: $|1 - |x|| = 3$
From the second case: $|1 - |x|| = -1$ (impossible since absolute value is non-negative)

So we only need to solve $|1 - |x|| = 3$, which gives us:
\[
1 - |x| = 3 \quad \text{or} \quad 1 - |x| = -3
\]

From $1 - |x| = 3$: $|x| = -2$ (impossible since $|x| \geq 0$)

From $1 - |x| = -3$: $|x| = 4$

This gives us $x = 4$ or $x = -4$.

Let's verify:
- For $x = 4$: $|x| = 4$, $|1-4| = 3$, $|3-1| = 2$, $|2| - 2 = 0$ ✓
- For $x = -4$: $|x| = 4$, $|1-4| = 3$, $|3-1| = 2$, $|2| - 2 = 0$ ✓

\subsection*{Answer}
There are 2 real solutions (option \textbf{C}).

\end{document}