\documentclass[12pt]{article}
\usepackage[margin=1in]{geometry}
\usepackage{amsmath,amssymb}
\begin{document}

\section*{Problem 15}
Find the total number of real solutions of the system
\[
\sqrt{x-1}+\sqrt[3]{y}=1,\qquad x-y=2.
\]

\subsection*{Solution}
From $x-y=2$ we have $x=y+2$. Substitute into the first equation:
\[
\sqrt{(y+2)-1}+\sqrt[3]{y}=1
\quad\Rightarrow\quad
\sqrt{y+1}+\sqrt[3]{y}=1.
\]
Domain requires $y\ge -1$.
Let $t=\sqrt[3]{y}$, so $y=t^3$ and $t\ge -1$. Then
\[
\sqrt{t^3+1}+t=1 \quad\Rightarrow\quad \sqrt{t^3+1}=1-t.
\]
Hence we also need $1-t\ge 0$, i.e. $t\le 1$.
Squaring:
\[
t^3+1=(1-t)^2=1-2t+t^2
\quad\Rightarrow\quad
t^3-t^2+2t=0
\quad\Rightarrow\quad
t(t^2-t+2)=0.
\]
The quadratic $t^2-t+2$ has discriminant $1-8=-7<0$, so the only real solution is $t=0$.
Then $y=t^3=0$ and $x=y+2=2$.
Thus there is exactly one real solution.

\subsection*{Answer}
$1$ (option \textbf{B}).

\end{document}

