\documentclass[12pt]{article}
\usepackage[margin=1in]{geometry}
\usepackage{amsmath,amssymb}
\begin{document}

\section*{Problem 6}
If $\log_2(\sqrt{3}+1) + \log_2(\sqrt{6}-2) = A$, find the value of $\log_{\frac{1}{4}}(\sqrt{3}-1) + \log_{\frac{1}{4}}(\sqrt{6}+2)$.

\subsection*{Solution}
First, let's simplify the given expression $A$:
\[A = \log_2(\sqrt{3}+1) + \log_2(\sqrt{6}-2) = \log_2[(\sqrt{3}+1)(\sqrt{6}-2)]\]

Now for the expression we need to find:
\[\log_{\frac{1}{4}}(\sqrt{3}-1) + \log_{\frac{1}{4}}(\sqrt{6}+2)\]

Note that $\frac{1}{4} = 2^{-2}$, so $\log_{\frac{1}{4}} x = \frac{\log_2 x}{\log_2 2^{-2}} = \frac{\log_2 x}{-2} = -\frac{1}{2}\log_2 x$.

Therefore:
\begin{align*}
\log_{\frac{1}{4}}(\sqrt{3}-1) + \log_{\frac{1}{4}}(\sqrt{6}+2) &= -\frac{1}{2}\log_2(\sqrt{3}-1) - \frac{1}{2}\log_2(\sqrt{6}+2)\\
&= -\frac{1}{2}\log_2[(\sqrt{3}-1)(\sqrt{6}+2)]
\end{align*}

Now let's compute the products:
\[(\sqrt{3}+1)(\sqrt{3}-1) = 3 - 1 = 2\]
\[(\sqrt{6}-2)(\sqrt{6}+2) = 6 - 4 = 2\]

So:
\[(\sqrt{3}-1) = \frac{2}{\sqrt{3}+1} \quad \text{and} \quad (\sqrt{6}+2) = \frac{2}{\sqrt{6}-2}\]

Therefore:
\[(\sqrt{3}-1)(\sqrt{6}+2) = \frac{2}{\sqrt{3}+1} \cdot \frac{2}{\sqrt{6}-2} = \frac{4}{(\sqrt{3}+1)(\sqrt{6}-2)}\]

Let $P = (\sqrt{3}+1)(\sqrt{6}-2)$. Then:
\[A = \log_2 P\]

And:
\begin{align*}
-\frac{1}{2}\log_2[(\sqrt{3}-1)(\sqrt{6}+2)] &= -\frac{1}{2}\log_2\frac{4}{P}\\
&= -\frac{1}{2}(\log_2 4 - \log_2 P)\\
&= -\frac{1}{2}(2 - A)\\
&= -1 + \frac{A}{2}\\
&= \frac{A}{2} - 1
\end{align*}

\subsection*{Answer}
$\dfrac{A}{2} - 1$ (option \textbf{D}).

\end{document}
