\documentclass[12pt]{article}
\usepackage[margin=1in]{geometry}
\usepackage{amsmath,amssymb}
\begin{document}

\section*{Problem 15}
Find the total number of real solutions of the system of equations:
\[\sqrt{x-1} + \sqrt[3]{y} = 1, \quad x - y = 2\]

\subsection*{Solution}
From the second equation: $x = y + 2$.

Substituting into the first equation:
\[\sqrt{y + 2 - 1} + \sqrt[3]{y} = 1\]
\[\sqrt{y + 1} + \sqrt[3]{y} = 1\]

For this equation to be valid, we need $y + 1 \geq 0$, i.e., $y \geq -1$.

Let $f(y) = \sqrt{y+1} + \sqrt[3]{y}$.

At $y = -1$: $f(-1) = 0 + \sqrt[3]{-1} = -1 < 1$

At $y = 0$: $f(0) = 1 + 0 = 1$ ✓

So $y = 0$ is a solution, giving $x = 2$.

Let's check if there are other solutions by analyzing $f(y)$:

For $y > 0$: $f(y) = \sqrt{y+1} + \sqrt[3]{y} > 1 + 0 = 1$, so $f(y) > 1$.

For $-1 \leq y < 0$: 
- $\sqrt{y+1}$ ranges from $0$ to $1$
- $\sqrt[3]{y}$ is negative

At $y = -1$: $f(-1) = 0 + (-1) = -1$
At $y = 0$: $f(0) = 1 + 0 = 1$

$f(y)$ is continuous and strictly increasing on $[-1, 0]$ (both terms are increasing).

$f'(y) = \frac{1}{2\sqrt{y+1}} + \frac{1}{3y^{2/3}} > 0$ for $y \in (-1, 0)$

Since $f$ is strictly increasing from $-1$ to $1$ on $[-1, 0]$, there is exactly one value where $f(y) = 1$, which is $y = 0$.

Therefore, there is exactly \textbf{1} solution: $(x, y) = (2, 0)$.

\subsection*{Verification}
For $(x, y) = (2, 0)$:
\[\sqrt{2-1} + \sqrt[3]{0} = 1 + 0 = 1 \checkmark\]
\[2 - 0 = 2 \checkmark\]

\subsection*{Answer}
$1$ (option \textbf{B}).

\end{document}
