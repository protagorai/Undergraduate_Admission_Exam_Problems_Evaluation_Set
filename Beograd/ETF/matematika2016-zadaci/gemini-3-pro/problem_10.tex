\documentclass[12pt]{article}
\usepackage[margin=1in]{geometry}
\usepackage{amsmath,amssymb}
\begin{document}

\section*{Problem 10}
U jednakokrakom trouglu $ABC$ je $AB=AC=b$ i $\angle BAC = 30^\circ$. Tada je zbir visina tog trougla jednak:

\subsection*{Solution}
Altitudes $h_b$ and $h_c$ correspond to sides $AC$ and $AB$.
$h_b = b \sin 30^\circ = b/2$.
$h_c = b \sin 30^\circ = b/2$.
Altitude $h_a$ to base $BC$ bisects angle $A$.
$h_a = b \cos 15^\circ$.
Calculate $\cos 15^\circ$:
$\cos 15^\circ = \sqrt{\frac{1+\cos 30^\circ}{2}} = \sqrt{\frac{1+\sqrt{3}/2}{2}} = \frac{\sqrt{2+\sqrt{3}}}{2} = \frac{\sqrt{3}+1}{2\sqrt{2}} = \frac{\sqrt{6}+\sqrt{2}}{4}$.
Sum of altitudes $H = h_a + h_b + h_c = b \frac{\sqrt{6}+\sqrt{2}}{4} + b$.
$H = \frac{b}{4}(\sqrt{6}+\sqrt{2} + 4)$.


\subsection*{Answer}
$\frac{b}{4}(4+\sqrt{2}+\sqrt{6})$ (option \textbf{C})

\end{document}
