\documentclass[12pt]{article}
\usepackage[margin=1in]{geometry}
\usepackage{amsmath,amssymb}
\begin{document}

\section*{Problem 11}
Ako su temena trougla tačke $A(-8,4)$, $B(-2,1)$ i $C(1,-3)$, a ortocentar $H(x_0, y_0)$, tada je vrednost razlike $y_0-x_0$ jednaka:

\subsection*{Solution}
Orthocenter is intersection of altitudes.
Altitude $h_a \perp BC$. Slope $BC = \frac{-3-1}{1-(-2)} = \frac{-4}{3}$.
Slope $h_a = 3/4$. Through $A(-8,4)$:
$y - 4 = \frac{3}{4}(x+8) \implies 4y - 16 = 3x + 24 \implies 3x - 4y = -40$.
Altitude $h_b \perp AC$. Slope $AC = \frac{-3-4}{1-(-8)} = \frac{-7}{9}$.
Slope $h_b = 9/7$. Through $B(-2,1)$:
$y - 1 = \frac{9}{7}(x+2) \implies 7y - 7 = 9x + 18 \implies 9x - 7y = -25$.
Solving system:
Multiply first eq by 3: $9x - 12y = -120$.
Subtract from second: $(9x-7y) - (9x-12y) = -25 - (-120) \implies 5y = 95 \implies y=19$.
$3x = 4y - 40 = 76 - 40 = 36 \implies x=12$.
$H(12, 19)$.
Difference $y_0 - x_0 = 19 - 12 = 7$.


\subsection*{Answer}
7 (option \textbf{A})

\end{document}
