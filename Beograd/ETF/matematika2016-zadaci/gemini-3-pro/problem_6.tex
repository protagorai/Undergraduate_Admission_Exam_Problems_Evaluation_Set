\documentclass[12pt]{article}
\usepackage[margin=1in]{geometry}
\usepackage{amsmath,amssymb}
\begin{document}

\section*{Problem 6}
Ako je $\log_2(\sqrt{3}+1) + \log_2(\sqrt{6}-2) = A$, tada je izraz $\log_{\frac{1}{4}}(\sqrt{3}-1) + \log_{\frac{1}{4}}(\sqrt{6}+2)$ jednak:

\subsection*{Solution}
Let $X = \log_{\frac{1}{4}}(\sqrt{3}-1) + \log_{\frac{1}{4}}(\sqrt{6}+2)$.
Using $\log_{1/4} y = \log_{2^{-2}} y = -\frac{1}{2} \log_2 y$:
\[ X = -\frac{1}{2} [\log_2(\sqrt{3}-1) + \log_2(\sqrt{6}+2)] \]
Note relations between conjugate pairs:
$\log_2(\sqrt{3}-1) = \log_2 \frac{2}{\sqrt{3}+1} = 1 - \log_2(\sqrt{3}+1)$
$\log_2(\sqrt{6}+2) = \log_2 \frac{2}{\sqrt{6}-2} = 1 - \log_2(\sqrt{6}-2)$
Substitute into X:
\[ X = -\frac{1}{2} [ (1 - \log_2(\sqrt{3}+1)) + (1 - \log_2(\sqrt{6}-2)) ] \]
\[ X = -\frac{1}{2} [ 2 - (\log_2(\sqrt{3}+1) + \log_2(\sqrt{6}-2)) ] \]
Using the given $A$:
\[ X = -\frac{1}{2} (2 - A) = \frac{A}{2} - 1 \]

\subsection*{Answer}
$\frac{A}{2}-1$ (option \textbf{D})

\end{document}
