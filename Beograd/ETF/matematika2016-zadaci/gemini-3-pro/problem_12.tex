\documentclass[12pt]{article}
\usepackage[margin=1in]{geometry}
\usepackage{amsmath,amssymb}
\begin{document}

\section*{Problem 12}
U razvoju binoma $\left(\sqrt[3]{a} + \frac{1}{\sqrt[6]{a}}\right)^n$ ($a>0, n \in \mathbb{N}$) zbir prva tri binomna koeficijenta je 121. Član koji sadrži $\frac{1}{a}$ jednak je:

\subsection*{Solution}
Sum of first three coefficients: $\binom{n}{0} + \binom{n}{1} + \binom{n}{2} = 121$.
$1 + n + \frac{n(n-1)}{2} = 121$.
$n^2 + n - 240 = 0 \implies (n+16)(n-15)=0 \implies n=15$.
General term $T_{k+1} = \binom{15}{k} (a^{1/3})^{15-k} (a^{-1/6})^k = \binom{15}{k} a^{\frac{15-k}{3} - \frac{k}{6}} = \binom{15}{k} a^{\frac{30-3k}{6}} = \binom{15}{k} a^{\frac{10-k}{2}}$.
For $1/a = a^{-1}$:
$\frac{10-k}{2} = -1 \implies 10-k = -2 \implies k=12$.
Coefficient: $\binom{15}{12} = \binom{15}{3} = \frac{15 \cdot 14 \cdot 13}{6} = 455$.
Term is $455 a^{-1} = \frac{455}{a}$.


\subsection*{Answer}
$\frac{455}{a}$ (option \textbf{C})

\end{document}
