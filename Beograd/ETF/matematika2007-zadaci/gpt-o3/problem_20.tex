\documentclass[12pt]{article}
\usepackage[margin=1in]{geometry}
\usepackage{amsmath}
\begin{document}

\section*{Problem 20}
У групи је 2 математичара и 8 физичара. Колико начина постоји да се формира петочлана комисија у којој је бар један математичар?

\subsection*{Решење}
Укупно 10 кандидата (2 М + 8 Ф). Број свих петочланих комисија:
\[
\binom{10}{5}=252.
\]
Комисије без математичара: бира се 5 од 8 физичара
\[
\binom{8}{5}=56.
\]
Комисије са бар једним математичаром:
\[
252-56=196.
\]
\subsection*{Одговор}
$196$ (опција \textbf{B}).

\end{document}