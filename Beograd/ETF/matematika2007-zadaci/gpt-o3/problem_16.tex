\documentclass[12pt]{article}
\usepackage[margin=1in]{geometry}
\usepackage{amsmath,amssymb}
\begin{document}

\section*{Problem 16}
U geometrijskoj progresiji dati su
\[
a_{1}+a_{5}=51,\qquad a_{2}+a_{6}=102.
\]
Odredити $n$ za koji je zbir prvih $n$ članova $S_{n}=3069$.

\subsection*{Rešenje}
Neka je prvi član $a$ a količnik $q$.
\[
a_{1}=a,\quad a_{5}=aq^{4}\;\Rightarrow\;a(1+q^{4})=51\tag{1}
\]
\[
a_{2}=aq,\quad a_{6}=aq^{5}\;\Rightarrow\;aq(1+q^{4})=102.\tag{2}
\]
Podelimo (2) sa (1):
\[q=\frac{102}{51}=2.\]
Iz (1) potom sledi $a(1+16)=51\Rightarrow a=\frac{51}{17}=3$.

Zbir $n$ prvih članova geometrijskog niza sa količnikom $q=2$ je
\[
S_{n}=a\frac{q^{n}-1}{q-1}=3(2^{n}-1).
\]
Postavljamo $3(2^{n}-1)=3069\;\Rightarrow\;2^{n}-1=1023\;\Rightarrow\;2^{n}=1024=2^{10}$,
pa je $n=10$.

\subsection*{Odgovor}
$n=10$ (opcija \textbf{D}).

\end{document}