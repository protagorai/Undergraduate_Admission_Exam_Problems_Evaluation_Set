\documentclass[12pt]{article}
\usepackage[margin=1in]{geometry}
\usepackage{amsmath,amssymb}
\begin{document}

\section*{Problem 8}
Odrediti skup svih $x$ koji zadovoljavaju nejednačinu
\[
\frac{(x+1)(x+2)}{(x-1)(x-2)}\ge1.
\]

\subsection*{Rešenje}
Prebacivanjem na zajednički imenilac:
\[
\frac{(x+1)(x+2)-(x-1)(x-2)}{(x-1)(x-2)}\ge0.
\]
Brojilac:
\begin{align*}
(x+1)(x+2)&=x^{2}+3x+2,\\
(x-1)(x-2)&=x^{2}-3x+2,\\
\Rightarrow&(x+1)(x+2)-(x-1)(x-2)=6x.
\end{align*}
Stoga nejednačina postaje
\[
\frac{6x}{(x-1)(x-2)}\ge0.
\]
Faktori su $6>0$, pa znak zavisi od $x$ i imenilaca.
Kritične tačke: $x=-\infty,0,1,2$. Postavimo tabelu znakova:
\[
\begin{array}{c|ccccc}
 x & -\infty & 0 & 1 & 2 & +\infty\\\hline
6x & - & - & 0 & + & +\\
(x-1) & - & - & 0 & + & +\\
(x-2) & - & - & - & 0 & +\\\hline
\text{Ceо izraz} & - & - & \times & 0+ & +
\end{array}
\]
Izraz je nenegativan na segmentima $[0,1)\cup(2,\infty)$.
\subsection*{Odgovor}
$[0,1]\cup(2,\infty)$ (opcija \textbf{D}).

\end{document}