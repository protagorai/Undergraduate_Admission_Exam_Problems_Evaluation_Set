\section*{Zadatak 16}

U geometrijskoj progresiji je \(a_1+a_5=51\), \(a_2+a_6=102\).
Za koju vrednost \(n\) je zbir prvih \(n\) članova \(S_n=3069\)?

\subsection*{Rešenje}
Neka je \(a_1=A\), a količnik \(r\). Tada su
\[
a_5=Ar^4,\qquad a_6=Ar^5,\qquad a_2=Ar.
\]
Uslovi daju:
\[
A+Ar^4=A(1+r^4)=51,
\]
\[
Ar+Ar^5=Ar(1+r^4)=102.
\]
Deljenjem druge jednačine prvom dobijamo
\[
r=\frac{102}{51}=2.
\]
Zatim iz \(A(1+2^4)=A\cdot 17=51\) sledi \(A=3\).

Zbir prvih \(n\) članova je
\[
S_n=A\frac{r^n-1}{r-1}=3\frac{2^n-1}{1}=3(2^n-1).
\]
Postavljamo \(3(2^n-1)=3069\):
\[
2^n-1=1023 \Rightarrow 2^n=1024=2^{10}\Rightarrow n=10.
\]

\subsection*{Odgovor}
\[
\boxed{(D)\;n=10}
\]

