\section*{Zadatak 13}

Izračunati vrednost izraza
\[
\frac{\sin 86^\circ+\sin 76^\circ-\sin 26^\circ-\sin 16^\circ}{\cos 86^\circ+\cos 76^\circ+\cos 26^\circ+\cos 16^\circ}.
\]

\subsection*{Rešenje}
Grupišemo:
\[
\sin 86^\circ-\sin 16^\circ + \sin 76^\circ-\sin 26^\circ.
\]
Koristimo formulu \(\sin A-\sin B=2\cos\frac{A+B}{2}\sin\frac{A-B}{2}\):
\[
\sin 86^\circ-\sin 16^\circ
=2\cos 51^\circ \sin 35^\circ,
\]
\[
\sin 76^\circ-\sin 26^\circ
=2\cos 51^\circ \sin 25^\circ.
\]
Zato je brojilac
\[
2\cos 51^\circ(\sin 35^\circ+\sin 25^\circ).
\]
A \(\sin u+\sin v=2\sin\frac{u+v}{2}\cos\frac{u-v}{2}\), pa
\[
\sin 35^\circ+\sin 25^\circ
=2\sin 30^\circ\cos 5^\circ
=\cos 5^\circ.
\]
Dakle, brojilac je \(2\cos 51^\circ\cos 5^\circ\).

Za imenilac grupišemo
\[
(\cos 86^\circ+\cos 16^\circ)+(\cos 76^\circ+\cos 26^\circ),
\]
i koristimo \(\cos A+\cos B=2\cos\frac{A+B}{2}\cos\frac{A-B}{2}\):
\[
\cos 86^\circ+\cos 16^\circ=2\cos 51^\circ\cos 35^\circ,
\]
\[
\cos 76^\circ+\cos 26^\circ=2\cos 51^\circ\cos 25^\circ.
\]
Zato je imenilac
\[
2\cos 51^\circ(\cos 35^\circ+\cos 25^\circ).
\]
A \(\cos u+\cos v=2\cos\frac{u+v}{2}\cos\frac{u-v}{2}\), pa
\[
\cos 35^\circ+\cos 25^\circ
=2\cos 30^\circ\cos 5^\circ
=\sqrt{3}\,\cos 5^\circ.
\]
Dakle, imenilac je \(2\sqrt{3}\cos 51^\circ\cos 5^\circ\).

Konačno,
\[
\frac{2\cos 51^\circ\cos 5^\circ}{2\sqrt{3}\cos 51^\circ\cos 5^\circ}=\frac{1}{\sqrt{3}}.
\]

\subsection*{Odgovor}
\[
\boxed{(D)\;\frac{1}{\sqrt{3}}}
\]

