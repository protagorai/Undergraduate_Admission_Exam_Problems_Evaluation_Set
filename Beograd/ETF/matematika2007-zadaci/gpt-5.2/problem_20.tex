\section*{Zadatak 20}

Broj načina na koji se može formirati petočlana komisija od 2 matematičara i 8 fizičara,
tako da u njoj bude bar jedan matematičar, iznosi:

\subsection*{Rešenje}
Ukupan broj petočlanih komisija od ukupno \(10\) ljudi je
\[
\binom{10}{5}=252.
\]
Oduzimamo komisije bez matematičara (dakle svih 5 su fizičari):
\[
\binom{8}{5}=56.
\]
Zato je broj komisija sa bar jednim matematičarem
\[
\binom{10}{5}-\binom{8}{5}=252-56=196.
\]

\subsection*{Odgovor}
\[
\boxed{(B)\;196}
\]

