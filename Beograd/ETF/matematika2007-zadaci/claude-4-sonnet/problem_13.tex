\documentclass[12pt]{article}
\usepackage[margin=1in]{geometry}
\usepackage{amsmath,amssymb}
\begin{document}

\section*{Problem 13}
Vrednost izraza $\frac{\sin 86° + \sin 76° - \sin 26° - \sin 16°}{\cos 86° + \cos 76° + \cos 26° + \cos 16°}$ iznosi:
\begin{enumerate}
\item[(A)] $\sqrt{3}$
\item[(B)] $\frac{\sqrt{3}}{2}$
\item[(C)] $\frac{2}{\sqrt{3}}$
\item[(D)] $\frac{1}{\sqrt{3}}$
\item[(E)] 0
\item[(N)] Ne znam
\end{enumerate}

\subsection*{Solution}
Koristimo identitete za zbir i razliku sinusa i kosinusa.

Primetimo da:
\begin{align}
86° &= 90° - 4° \\
76° &= 90° - 14° \\
26° &= 90° - 64° \\
16° &= 90° - 74°
\end{align}

Takođe:
\begin{align}
\sin 86° &= \sin(90° - 4°) = \cos 4° \\
\sin 76° &= \sin(90° - 14°) = \cos 14° \\
\cos 86° &= \cos(90° - 4°) = \sin 4° \\
\cos 76° &= \cos(90° - 14°) = \sin 14°
\end{align}

Dakle:
\begin{align}
\text{Brojilac} &= \cos 4° + \cos 14° - \sin 26° - \sin 16° \\
\text{Imenilac} &= \sin 4° + \sin 14° + \cos 26° + \cos 16°
\end{align}

Koristimo formule za zbir:
\[
\sin A + \sin B = 2\sin\left(\frac{A+B}{2}\right)\cos\left(\frac{A-B}{2}\right)
\]
\[
\cos A + \cos B = 2\cos\left(\frac{A+B}{2}\right)\cos\left(\frac{A-B}{2}\right)
\]

Za brojilac:
\begin{align}
\cos 4° + \cos 14° &= 2\cos 9° \cos 5° \\
\sin 26° + \sin 16° &= 2\sin 21° \cos 5°
\end{align}

Dakle:
\[
\text{Brojilac} = 2\cos 9° \cos 5° - 2\sin 21° \cos 5° = 2\cos 5°(\cos 9° - \sin 21°)
\]

Pošto je $\sin 21° = \sin(90° - 69°) = \cos 69°$ i $\cos 9° = \cos 9°$, imamo:
\[
\cos 9° - \sin 21° = \cos 9° - \cos 69°
\]

Koristimo $\cos A - \cos B = -2\sin\left(\frac{A+B}{2}\right)\sin\left(\frac{A-B}{2}\right)$:
\[
\cos 9° - \cos 69° = -2\sin 39° \sin(-30°) = 2\sin 39° \sin 30° = 2\sin 39° \cdot \frac{1}{2} = \sin 39°
\]

Dakle:
\[
\text{Brojilac} = 2\cos 5° \sin 39°
\]

Slično za imenilac:
\begin{align}
\sin 4° + \sin 14° &= 2\sin 9° \cos 5° \\
\cos 26° + \cos 16° &= 2\cos 21° \cos 5°
\end{align}

\[
\text{Imenilac} = 2\sin 9° \cos 5° + 2\cos 21° \cos 5° = 2\cos 5°(\sin 9° + \cos 21°)
\]

Pošto je $\cos 21° = \sin 69°$:
\[
\sin 9° + \cos 21° = \sin 9° + \sin 69°
\]

Koristimo formulu za zbir sinusa:
\[
\sin 9° + \sin 69° = 2\sin 39° \cos 30° = 2\sin 39° \cdot \frac{\sqrt{3}}{2} = \sqrt{3}\sin 39°
\]

Dakle:
\[
\text{Imenilac} = 2\cos 5° \cdot \sqrt{3}\sin 39° = 2\sqrt{3}\cos 5° \sin 39°
\]

Konačno:
\[
\frac{\text{Brojilac}}{\text{Imenilac}} = \frac{2\cos 5° \sin 39°}{2\sqrt{3}\cos 5° \sin 39°} = \frac{1}{\sqrt{3}}
\]

\subsection*{Answer}
Vrednost izraza je $\frac{1}{\sqrt{3}}$, što odgovara opciji \textbf{(D)}.

\end{document}