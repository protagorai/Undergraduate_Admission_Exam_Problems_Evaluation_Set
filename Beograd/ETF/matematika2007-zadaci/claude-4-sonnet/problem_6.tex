\documentclass[12pt]{article}
\usepackage[margin=1in]{geometry}
\usepackage{amsmath,amssymb}
\begin{document}

\section*{Problem 6}
Jednačina $\log_2(1 - x) = \log_2(x - 3)$:
\begin{enumerate}
\item[(A)] Nema rešenja
\item[(B)] Ima beskonačno mnogo rešenja
\item[(C)] $x = 3$ je jedinstveno rešenje
\item[(D)] $x = 1$ je jedinstveno rešenje
\item[(E)] Zadovoljena je za $x = 2$
\item[(N)] Ne znam
\end{enumerate}

\subsection*{Solution}
Da bi logaritmi bili definisani, moraju biti ispunjeni uslovi:
\begin{itemize}
\item $1 - x > 0 \Rightarrow x < 1$
\item $x - 3 > 0 \Rightarrow x > 3$
\end{itemize}

Ovi uslovi su međusobno kontradiktni jer ne postoji $x$ takvo da je istovremeno $x < 1$ i $x > 3$.

Dakle, ne postoji vrednost $x$ za koju su oba logaritma definisana.

Alternativno, ako pretpostavimo da su logaritmi definisani, iz jednačine:
\[
\log_2(1 - x) = \log_2(x - 3)
\]

sledi:
\[
1 - x = x - 3
\]
\[
1 + 3 = x + x
\]
\[
4 = 2x
\]
\[
x = 2
\]

Međutim, proveravamo da li je $x = 2$ u domenu:
\begin{itemize}
\item Za $x = 2$: $1 - x = 1 - 2 = -1 < 0$ (logaritam nije definisan)
\item Za $x = 2$: $x - 3 = 2 - 3 = -1 < 0$ (logaritam nije definisan)
\end{itemize}

Dakle, $x = 2$ ne pripada domenu funkcije.

\subsection*{Answer}
Jednačina \textbf{(A) nema rešenja} jer ne postoji vrednost $x$ za koju su oba logaritma definisana.

\end{document}