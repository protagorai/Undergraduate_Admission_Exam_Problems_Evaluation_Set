\documentclass[12pt]{article}
\usepackage[margin=1in]{geometry}
\usepackage{amsmath,amssymb}
\begin{document}

\section*{Problem 20}
Broj načina na koji se može formirati petočlana komisija od 2 matematičara i 8 fizičara, tako da u njoj bude bar jedan matematičar iznosi:
\begin{enumerate}
\item[(A)] 132
\item[(B)] 196
\item[(C)] 212
\item[(D)] 314
\item[(E)] 422
\item[(N)] Ne znam
\end{enumerate}

\subsection*{Solution}
Imamo ukupno 2 matematičara i 8 fizičara, što je ukupno 10 ljudi.

Treba da formiramo petočlanu komisiju sa bar jednim matematičarom.

Koristićemo komplementarni pristup: ukupan broj načina minus broj načina bez matematičara.

\textbf{Ukupan broj načina da se izabere 5 ljudi od 10:}
\[
\binom{10}{5} = \frac{10!}{5! \cdot 5!} = \frac{10 \cdot 9 \cdot 8 \cdot 7 \cdot 6}{5 \cdot 4 \cdot 3 \cdot 2 \cdot 1} = \frac{30240}{120} = 252
\]

\textbf{Broj načina da se izabere 5 ljudi samo od 8 fizičara (bez matematičara):}
\[
\binom{8}{5} = \binom{8}{3} = \frac{8!}{3! \cdot 5!} = \frac{8 \cdot 7 \cdot 6}{3 \cdot 2 \cdot 1} = \frac{336}{6} = 56
\]

\textbf{Broj načina sa bar jednim matematičarom:}
\[
252 - 56 = 196
\]

Alternativno, možemo direktno računati:

\textbf{Slučaj 1: Tačno 1 matematičar i 4 fizičara}
\[
\binom{2}{1} \cdot \binom{8}{4} = 2 \cdot \frac{8!}{4! \cdot 4!} = 2 \cdot \frac{8 \cdot 7 \cdot 6 \cdot 5}{4 \cdot 3 \cdot 2 \cdot 1} = 2 \cdot 70 = 140
\]

\textbf{Slučaj 2: Tačno 2 matematičara i 3 fizičara}
\[
\binom{2}{2} \cdot \binom{8}{3} = 1 \cdot \frac{8!}{3! \cdot 5!} = 1 \cdot \frac{8 \cdot 7 \cdot 6}{3 \cdot 2 \cdot 1} = 56
\]

\textbf{Ukupno:}
\[
140 + 56 = 196
\]

\subsection*{Answer}
Broj načina je 196, što odgovara opciji \textbf{(B)}.

\end{document}