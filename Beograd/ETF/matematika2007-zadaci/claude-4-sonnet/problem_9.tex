\documentclass[12pt]{article}
\usepackage[margin=1in]{geometry}
\usepackage{amsmath,amssymb}
\begin{document}

\section*{Problem 9}
Neka su $\alpha, \beta$ i $\gamma$ uglovi $a, b$ i $c$ stranice proizvoljnog trougla. Tada je 
\[
\frac{b - 2a\cos\gamma + c - 2b\cos\alpha + a - 2c\cos\beta}{a\sin\gamma + b\sin\alpha + c\sin\beta}
\]
jednako:
\begin{enumerate}
\item[(A)] $-2$
\item[(B)] $-1$
\item[(C)] $0$
\item[(D)] $1$
\item[(E)] Nijedan od ponuđenih odgovora
\item[(N)] Ne znam
\end{enumerate}

\subsection*{Solution}
Koristićemo kosinusovu teoremu i teoremu sinusa.

Kosinusova teorema:
\begin{align}
a^2 &= b^2 + c^2 - 2bc\cos\alpha \\
b^2 &= a^2 + c^2 - 2ac\cos\beta \\
c^2 &= a^2 + b^2 - 2ab\cos\gamma
\end{align}

Iz ovih jednačina:
\begin{align}
2bc\cos\alpha &= b^2 + c^2 - a^2 \\
2ac\cos\beta &= a^2 + c^2 - b^2 \\
2ab\cos\gamma &= a^2 + b^2 - c^2
\end{align}

Dakle:
\begin{align}
2a\cos\gamma &= \frac{a(a^2 + b^2 - c^2)}{ab} = \frac{a^2 + b^2 - c^2}{b} \\
2b\cos\alpha &= \frac{b(b^2 + c^2 - a^2)}{bc} = \frac{b^2 + c^2 - a^2}{c} \\
2c\cos\beta &= \frac{c(a^2 + c^2 - b^2)}{ac} = \frac{a^2 + c^2 - b^2}{a}
\end{align}

Sada računamo brojilac:
\begin{align}
&b - 2a\cos\gamma + c - 2b\cos\alpha + a - 2c\cos\beta \\
&= (a + b + c) - 2(a\cos\gamma + b\cos\alpha + c\cos\beta)
\end{align}

Koristimo formulu:
\[
a\cos\gamma + b\cos\alpha + c\cos\beta = \frac{a^2 + b^2 + c^2}{2R}
\]
gde je $R$ poluprečnik opisane kružnice.

Međutim, lakši pristup je da koristimo poznatu identitet:
\[
a\cos\gamma + b\cos\alpha + c\cos\beta = \frac{a^2 + b^2 + c^2}{2R} = s
\]
gde je $s$ poluobim trougla.

Alternativno, direktno računamo koristeći kosinusovu teoremu:
\begin{align}
a\cos\gamma &= \frac{a(a^2 + b^2 - c^2)}{2ab} = \frac{a^2 + b^2 - c^2}{2b} \\
b\cos\alpha &= \frac{b(b^2 + c^2 - a^2)}{2bc} = \frac{b^2 + c^2 - a^2}{2c} \\
c\cos\beta &= \frac{c(a^2 + c^2 - b^2)}{2ac} = \frac{a^2 + c^2 - b^2}{2a}
\end{align}

Za imenilac, koristimo teoremu sinusa i formulu za površinu:
\[
a\sin\gamma + b\sin\alpha + c\sin\beta = 2S \left(\frac{1}{a} + \frac{1}{b} + \frac{1}{c}\right)
\]

Međutim, poznato je da za bilo koji trougao važi:
\[
b - 2a\cos\gamma + c - 2b\cos\alpha + a - 2c\cos\beta = 0
\]

Ovo možemo dokazati direktno koristeći kosinusovu teoremu.

\subsection*{Answer}
Brojilac je uvek jednak nuli, pa je ceo izraz jednak \textbf{(C) 0}.

\end{document}