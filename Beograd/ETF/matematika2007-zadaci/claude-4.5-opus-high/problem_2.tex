\documentclass[12pt]{article}
\usepackage[margin=1in]{geometry}
\usepackage{amsmath,amssymb}
\begin{document}

\section*{Problem 2}
Ako je $x > 0$, onda je $\sqrt{x\sqrt{x\sqrt{x}}}$ jednako:

(A) $x\sqrt{x}$ \quad (B) $x\sqrt[4]{x}$ \quad (C) $\sqrt[8]{x}$ \quad (D) $\sqrt[8]{x^3}$ \quad (E) $\sqrt[8]{x^7}$ \quad (N) Ne znam

\subsection*{Solution}
We work from the inside out. Let's express everything in terms of powers of $x$.

Start with the innermost expression:
\[
\sqrt{x} = x^{1/2}
\]

Then:
\[
x\sqrt{x} = x \cdot x^{1/2} = x^{3/2}
\]

Next:
\[
\sqrt{x\sqrt{x}} = \sqrt{x^{3/2}} = x^{3/4}
\]

Then:
\[
x\sqrt{x\sqrt{x}} = x \cdot x^{3/4} = x^{7/4}
\]

Finally:
\[
\sqrt{x\sqrt{x\sqrt{x}}} = \sqrt{x^{7/4}} = x^{7/8} = \sqrt[8]{x^7}
\]

\subsection*{Answer}
$\sqrt[8]{x^7}$ (option \textbf{E}).

\end{document}
