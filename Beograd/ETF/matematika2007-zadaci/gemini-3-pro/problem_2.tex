\documentclass[12pt]{article}
\usepackage[margin=1in]{geometry}
\usepackage{amsmath,amssymb}
\begin{document}

\section*{Problem 2}
Ako je $x > 0$, onda je $\sqrt{x\sqrt{x\sqrt{x}}}$ jednako :
\[
(A)\ x\sqrt{x} \quad (B)\ x\sqrt[4]{x} \quad (C)\ \sqrt[8]{x} \quad (D)\ \sqrt[8]{x^3} \quad (E)\ \sqrt[8]{x^7} \quad (N)\ \text{Ne znam}
\]

\subsection*{Solution}
Zapišimo korene kao stepene sa racionalnim izložiocem:
\[
\sqrt{x} = x^{1/2}
\]
Izraz ispod srednjeg korena je:
\[
x \sqrt{x} = x \cdot x^{1/2} = x^{3/2}
\]
Zatim primenimo srednji koren:
\[
\sqrt{x \sqrt{x}} = (x^{3/2})^{1/2} = x^{3/4}
\]
Izraz ispod spoljašnjeg korena je:
\[
x \sqrt{x \sqrt{x}} = x \cdot x^{3/4} = x^{7/4}
\]
Konačno, primenimo spoljašnji koren:
\[
\sqrt{x \sqrt{x \sqrt{x}}} = (x^{7/4})^{1/2} = x^{7/8}
\]
Ovo se može zapisati kao $\sqrt[8]{x^7}$.

\subsection*{Answer}
$\sqrt[8]{x^7}$ (option \textbf{E}).

\end{document}
