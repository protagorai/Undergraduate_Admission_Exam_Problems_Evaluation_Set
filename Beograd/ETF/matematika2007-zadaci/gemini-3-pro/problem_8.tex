\documentclass[12pt]{article}
\usepackage[margin=1in]{geometry}
\usepackage{amsmath,amssymb}
\begin{document}

\section*{Problem 8}
Skup svih rešenja nejednačine $\frac{(x+1)(x+2)}{(x-1)(x-2)} \ge 1$ je :
\[
(A)\ (-\infty, +\infty) \quad (B)\ [0, +\infty) \quad (C)\ [0, 1] \quad (D)\ [0, 1) \cup (2, +\infty) \quad (E)\ (-\infty, 0] \cup (2, +\infty) \quad (N)\ \text{Ne znam}
\]

\subsection*{Solution}
Prebacimo 1 na levu stranu:
\[
\frac{(x+1)(x+2)}{(x-1)(x-2)} - 1 \ge 0
\]
Svedemo na zajednički imenilac:
\[
\frac{(x+1)(x+2) - (x-1)(x-2)}{(x-1)(x-2)} \ge 0
\]
Izmnožimo brojilac:
\[
(x^2 + 3x + 2) - (x^2 - 3x + 2) = 6x
\]
Nejednačina postaje:
\[
\frac{6x}{(x-1)(x-2)} \ge 0
\]
Analiziramo znak izraza pomoću tabele ili intervala. Nule su $x=0$, a prekidne tačke $x=1$ i $x=2$.
\begin{itemize}
    \item Za $x < 0$: brojilac $(-)$, imenilac $(-)(-) = (+)$, količnik $(-)$.
    \item Za $0 \le x < 1$: brojilac $(+)$, imenilac $(-)(-) = (+)$, količnik $(+)$.
    \item Za $1 < x < 2$: brojilac $(+)$, imenilac $(+)(-) = (-)$, količnik $(-)$.
    \item Za $x > 2$: brojilac $(+)$, imenilac $(+)(+) = (+)$, količnik $(+)$.
\end{itemize}
Rešenje je unija intervala gde je izraz nenegativan:
\[
x \in [0, 1) \cup (2, +\infty)
\]
Napomena: $x=1$ i $x=2$ su isključeni jer funkcija nije definisana u tim tačkama.

\subsection*{Answer}
$[0, 1) \cup (2, +\infty)$ (option \textbf{D}).

\end{document}
