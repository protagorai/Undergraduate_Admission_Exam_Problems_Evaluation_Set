\documentclass[12pt]{article}
\usepackage[margin=1in]{geometry}
\usepackage{amsmath,amssymb}
\begin{document}

\section*{Problem 18}
Date su funkcije : $f_1(x) = 2^{\log_2 x}, f_2(x) = \log_2 2^x, f_3(x) = x, f_4(x) = (x \cdot 2^{-\log_2 \sqrt{x}})^2$. Tačan je iskaz :
\[
(A)\ \text{Među funkcijama nema međusobno jednakih} \quad (B)\ f_1 = f_2 = f_3 \neq f_4
\]
\[
(C)\ f_1 \neq f_2 = f_3 \neq f_4 \neq f_1 \quad (D)\ f_3 \neq f_1 = f_2 \neq f_4 \quad (E)\ f_2 = f_3 \neq f_1 = f_4 \quad (N)\ \text{Ne znam}
\]

\subsection*{Solution}
Analizirajmo svaku funkciju i njen domen:
1) $f_1(x) = 2^{\log_2 x}$.
Domen: $x > 0$ (zbog logaritma).
Na domenu, $f_1(x) = x$.

2) $f_2(x) = \log_2 2^x$.
Domen: $x \in \mathbb{R}$ (eksponencijalna funkcija je uvek pozitivna).
Na domenu, $f_2(x) = x$.

3) $f_3(x) = x$.
Domen: $x \in \mathbb{R}$.

4) $f_4(x) = (x \cdot 2^{-\log_2 \sqrt{x}})^2$.
Domen: $x > 0$ (zbog $\sqrt{x}$ i logaritma).
Uprostimo izraz:
\[
2^{-\log_2 \sqrt{x}} = \frac{1}{2^{\log_2 \sqrt{x}}} = \frac{1}{\sqrt{x}}
\]
\[
f_4(x) = \left(x \cdot \frac{1}{\sqrt{x}}\right)^2 = (\sqrt{x})^2 = x
\]
Dakle, $f_4(x) = x$ za $x > 0$.

Poređenje:
- $f_2$ i $f_3$ imaju isti domen ($\mathbb{R}$) i isti zakon dodele ($x$), pa je $f_2 = f_3$.
- $f_1$ i $f_4$ imaju isti domen ($(0, \infty)$) i isti zakon dodele ($x$), pa je $f_1 = f_4$.
- $f_1$ nije jednako $f_2$ (različiti domeni).

Dakle: $f_2 = f_3 \neq f_1 = f_4$.

\subsection*{Answer}
$f_2 = f_3 \neq f_1 = f_4$ (option \textbf{E}).

\end{document}
