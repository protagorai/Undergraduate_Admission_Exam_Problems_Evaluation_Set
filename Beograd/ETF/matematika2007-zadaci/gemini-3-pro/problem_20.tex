\documentclass[12pt]{article}
\usepackage[margin=1in]{geometry}
\usepackage{amsmath,amssymb}
\begin{document}

\section*{Problem 20}
Broj načina na koji se može formirati petočlana komisija od 2 matematičara i 8 fizičara, tako da u njoj bude bar jedan matematičar iznosi :
\[
(A)\ 132 \quad (B)\ 196 \quad (C)\ 212 \quad (D)\ 314 \quad (E)\ 422 \quad (N)\ \text{Ne znam}
\]

\subsection*{Solution}
Ukupan broj osoba je $2 + 8 = 10$.
Ukupan broj načina da se izabere 5 osoba od 10 je:
\[
\binom{10}{5} = \frac{10 \cdot 9 \cdot 8 \cdot 7 \cdot 6}{5 \cdot 4 \cdot 3 \cdot 2 \cdot 1} = 2 \cdot 9 \cdot 2 \cdot 7 = 252
\]
Broj komisija u kojima nema nijednog matematičara (svi su fizičari) je broj načina da izaberemo 5 osoba od 8 fizičara:
\[
\binom{8}{5} = \binom{8}{3} = \frac{8 \cdot 7 \cdot 6}{3 \cdot 2 \cdot 1} = 56
\]
Broj komisija sa bar jednim matematičarom je razlika ukupnog broja i broja komisija bez matematičara:
\[
252 - 56 = 196
\]

\subsection*{Answer}
196 (option \textbf{B}).

\end{document}
