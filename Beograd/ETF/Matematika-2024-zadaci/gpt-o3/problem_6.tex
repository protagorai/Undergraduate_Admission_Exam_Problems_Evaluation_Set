\documentclass[12pt]{article}
\usepackage[margin=1in]{geometry}
\usepackage{amsmath,amssymb}
\begin{document}

\section*{Problem 6}
Koliko ima različitih polinoma oblika
\[
P(x)=x^{2}+px+q, \qquad p,q\in\mathbb R,
\]
koji \\emph{deli} (kao činilac) polinom
\[
F(x)=x^{4}-3x^{2}+2?
\]

\subsection*{Rešenje}
\textbf{1.}  Faktorizujemo četvrtostepeni polinom.
Stavimo $t=x^{2}$, тада
\[
F(x)=t^{2}-3t+2=(t-1)(t-2).
\]
Vraćajući $t=x^{2}$ dobijamo
\[
F(x)=(x^{2}-1)(x^{2}-2).
\]
\textbf{2.}  Njegove nule su
\[
\pm1,\;\pm\sqrt2.
\]
Kvadratni delilac sa realnim koeficijentima nastaje izborom произвольне двојке ovih četири корена (budući da se faktor polinoma formira kao $(x-r_{1})(x-r_{2})$).

Broj takvih izbora jednak je броју комбинација $\binom{4}{2}=6$.
Свака изабрана двојка даје различит полином степена 2 са водећим коефицијентом 1 (на пример $x^{2}-1$, $x^{2}-2$, $x^{2}-(1+\sqrt2)x+\sqrt2$ итд.).  Сви они заиста деле $F(x)$ јер су њихови корени подскуп скупа корена четворостепеног полинома.

\subsection*{Odgovor}
\[6\qquad (\text{opcija B})\]

\end{document}
