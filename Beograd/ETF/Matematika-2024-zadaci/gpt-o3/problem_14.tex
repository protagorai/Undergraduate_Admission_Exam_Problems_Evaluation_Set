\documentclass[12pt]{article}
\usepackage[margin=1in]{geometry}
\usepackage{amsmath,amssymb}
\begin{document}

\section*{Problem 14}
Tri tačke dele obim kružnice u odnosu $6:7:11$.
Koliki je najmanji ugao trougla koji te tačke određuju?

\subsection*{Rešenje}
Obim je podeljen u tri luka čiji centralni uglovi (u stepеnima) су сразмерни 6:7:11 i zbir im je $360^{\circ}$.  Dakle
\[
\alpha: \beta : \gamma =6:7:11,\qquad \alpha+\beta+\gamma=360^{\circ}.
\]
Zbir koeficijenata je 24, pa
\[
\alpha=\frac{6}{24}\cdot360^{\circ}=90^{\circ},\qquad \beta=105^{\circ},\qquad \gamma=165^{\circ}.
\]

\textbf{Ugao trougla} kraće luče naspram chordi, pa je unutrašnji ugao trougla na periferiji polovina odgovarajućeg centralnog ugla.
Minimalni takav ugao je
\[
\frac{\alpha}{2}=45^{\circ}.
\]

\subsection*{Odgovor}
\[45^{\circ}\qquad(\text{opcija D})\]

\end{document}
