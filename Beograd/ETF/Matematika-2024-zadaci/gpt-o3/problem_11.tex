\documentclass[12pt]{article}
\usepackage[margin=1in]{geometry}
\usepackage{amsmath,amssymb}
\begin{document}

\section*{Problem 11}
Date su funkcije
\[
\begin{aligned}
&f_{1}(x)=|x-1|,\\
&f_{2}(x)=\frac{x^{2}-1}{x+1},\\
&f_{3}(x)=\frac{x^{2}-x^{3}+1}{x+1},\\
&f_{4}(x)=2\log_{2}(x-1).
\end{aligned}
\]
Odabrati tačan odnos među navedenim funkcijama.

\subsection*{Rešenje}
\textbf{1.}  Poređenje \(f_{1}\) i \(f_{2}\).
Za \(x\neq-1\):
\[
\frac{x^{2}-1}{x+1}=\frac{(x-1)(x+1)}{x+1}=x-1.
\]
Dakle \(f_{2}(x)=x-1\) za \(x\neq-1\), a to je isto što i \(|x-1|\) samo kada je \(x\ge1\).  Za \(x<1\) vrednosti se razlikuju.
Zaključak: \(f_{1}\ne f_{2}.\)

\textbf{2.}  Poređenje \(f_{1}\) i \(f_{4}\).
Za \(x>1\) imamo
\[f_{1}(x)=x-1,\qquad f_{4}(x)=2\log_{2}(x-1).\]
Jednakost bi zahtevala
\[x-1=2\log_{2}(x-1).\]
Leva strana raste linearno, desna logaritamski; poklapaju se jedino u \(x=\tfrac32\).  Dakle nisu identične.

\textbf{3.}  Poređenje \(f_{2}\) i \(f_{3}\).
Za \(x\neq-1\)
\[
f_{3}(x)=\frac{x^{2}-x^{3}+1}{x+1}=\frac{(x^{2}+1)(1-x)}{x+1}=-(x^{2}+1)\frac{x-1}{x+1}.
\]
To očigledno nije jednako \(x-1\) sem možda u izolovanim tačkama.

\textbf{4.}  Zaključak.
Ni dve funkcije nisu identične, pa su sve četiri različite.

\subsection*{Odgovor}
Sve funkcije su različite – \textbf{opcija E}.

\end{document}
