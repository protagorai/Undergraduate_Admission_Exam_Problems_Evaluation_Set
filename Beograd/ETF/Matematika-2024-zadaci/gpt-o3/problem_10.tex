\documentclass[12pt]{article}
\usepackage[margin=1in]{geometry}
\usepackage{amsmath,amssymb}
\begin{document}

\section*{Problem 10}
Odrediti zbir svih realnih rešenja jednačine
\[
\sqrt{3-x}+\sqrt{x-1}=\sqrt{2x-4}.
\]

\subsection*{Rešenje}
\textbf{1.}  Domen.
Svi izrazi pod korenom moraju biti nenegativni:
\[
\begin{aligned}
3-x &\ge0 &\Rightarrow&\;x\le3,\\
x-1 &\ge0 &\Rightarrow&\;x\ge1,\\
2x-4 &\ge0 &\Rightarrow&\;x\ge2.
\end{aligned}
\]
Zajednički domen je \(2\le x\le3\).

\textbf{2.}  Kvadriranje.
Stavimo
\[
a=\sqrt{3-x},\qquad b=\sqrt{x-1},\qquad c=\sqrt{2x-4}=\sqrt{2}\,\sqrt{x-2}.
\]
Jednačina je \(a+b=c\).  Kvadriranjem dobijamo
\[
a^{2}+b^{2}+2ab=c^{2}.
\]
Ali \(a^{2}=3-x,\;b^{2}=x-1\Rightarrow a^{2}+b^{2}=2\).  Tako
\[
2+2ab=2x-4\quad\Longrightarrow\quad ab=x-3.
\]
S leve strane je \(ab=\sqrt{(3-x)(x-1)}\ge0\), dok je desna strana \(x-3\le0\) (jer je \(x\le3\)).  Jedini način da budu jednake jeste da su 
obе нуле, што подразумева
\[
\begin{cases}
\sqrt{(3-x)(x-1)}=0\\
x-3=0.
\end{cases}\qquad\Longrightarrow\qquad x=3.
\]
Provera pokazuje da \(x=3\) zaista zadovoljava originalnu jednakost.

\textbf{3.}  Zbir rešenja.
Pošto postoji samo jedno rešenje, zbir je \(3\).

\subsection*{Odgovor}
\[3\qquad(\text{opcija C})\]

\end{document}
