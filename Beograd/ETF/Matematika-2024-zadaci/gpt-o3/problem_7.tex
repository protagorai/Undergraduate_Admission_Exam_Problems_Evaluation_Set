\documentclass[12pt]{article}
\usepackage[margin=1in]{geometry}
\usepackage{amsmath,amssymb}
\begin{document}

\section*{Problem 7}
Baza piramide je pravougli trougao $ABC$ sa hipotenuzom $AB=c$ i uglom $\angle BAC = 30^{\circ}$.  Neka je $O$ središte hipotenuze $AB$ i ortogonalna projekcija vrha piramide $S$ na ravan baze, a neka je visina piramide $h=SO = AO$ (što se iščitava iz uslova zadatka).
Izračunati zapreminu piramide.

\subsection*{Rešenje}
\textbf{1.}  Dimenzije osnove.
U pravouglom trouglu sa hipotenuzom $c$ i uglom od $30^{\circ}$ kod temena $A$ važi poznati odnos $1:\sqrt3:2$.  Stoga
\[
BC = c\sin30^{\circ}=\frac{c}{2},\qquad AC = c\cos30^{\circ}=\frac{c\sqrt3}{2}.
\]

\textbf{2.}  Površina osnove.
\[
P_{\triangle ABC}=\frac12\,BC\,AC = \frac12\cdot\frac{c}{2}\cdot\frac{c\sqrt3}{2}=\frac{c^{2}\sqrt3}{8}.
\]

\textbf{3.}  Visina piramide.
Tačka $O$ je polovina hipotenuze, pa je $AO = \tfrac{c}{2}$.  Uslov zadatka kaže $SO = AO$, dakle
\[
h = SO = \frac{c}{2}.
\]

\textbf{4.}  Zapremina.
\[
V = \frac13\,P_{\text{osnove}}\cdot h = \frac13\,\frac{c^{2}\sqrt3}{8}\cdot\frac{c}{2}=\frac{c^{3}\sqrt3}{48}=\frac{c^{3}}{16\sqrt3}.
\]

\subsection*{Odgovor}
\[\displaystyle \frac{c^{3}}{16\sqrt3}\qquad(\text{opcija D})\]

\end{document}
