\documentclass[12pt]{article}
\usepackage[margin=1in]{geometry}
\usepackage{amsmath,amssymb}
\begin{document}

\section*{Problem 19}
Skup svih re\u{s}enja nejedna\u{c}ine
\[
\frac{(x^2-4)\left(5^{2x}-6\cdot 5^{x+1}+5^3\right)}{\sqrt{\,9-3^{x^2-3x+2}\,}}\le 0
\]
je oblika (za neke realne brojeve $a<b<c<d$):

\subsection*{Solution}
\noindent\textbf{1) Domena (imenilac).}
Mora va\u{z}iti
\[
9-3^{x^2-3x+2}>0
\quad\Longleftrightarrow\quad
3^{x^2-3x+2}<3^2
\quad\Longleftrightarrow\quad
x^2-3x+2<2.
\]
Dakle,
\[
x^2-3x<0 \;\Longleftrightarrow\; x(x-3)<0 \;\Longleftrightarrow\; x\in(0,3).
\]
Na tom intervalu je imenilac pozitivan, pa znak razlomka zavisi samo od brojica.

\medskip
\noindent\textbf{2) Brojilac.}
\[
x^2-4=(x-2)(x+2).
\]
Za drugi faktor uvedimo $t=5^x>0$:
\[
5^{2x}-6\cdot 5^{x+1}+5^3=t^2-30t+125=(t-5)(t-25).
\]
Vra\u{c}anjem $t=5^x$:
\[
5^{2x}-6\cdot 5^{x+1}+5^3=(5^x-5)(5^x-25).
\]
Na domeni $(0,3)$ kriti\u{c}ne ta\u{c}ke su $x=1$ i $x=2$ (gde $5^x=5$ i $5^x=25$) i $x=2$ (gde je $x^2-4=0$).

Na intervalima:
\begin{itemize}
\item Ako je $x\in(0,1)$, onda je $5^x<5<25$, pa je $(5^x-5)(5^x-25)>0$, a $(x-2)<0$, pa je brojilac $<0$.
\item Ako je $x\in(1,2)$, tada je $(5^x-5)>0$, $(5^x-25)<0$, pa je proizvod $<0$, a $(x-2)<0$, pa je brojilac $>0$.
\item Ako je $x\in(2,3)$, oba faktora su $>0$, pa je brojilac $>0$.
\end{itemize}
Nule brojica su u $x=1$ i $x=2$ (obe pripadaju domeni), pa su te ta\u{c}ke uklju\u{c}ene.

Zato je nejedna\u{c}ina $\le 0$ ispunjena za
\[
x\in(0,1]\cup\{2\}.
\]

\subsection*{Answer}
$(0,1]\cup\{2\}$ (option \textbf{E}).

\end{document}




