\documentclass[12pt]{article}
\usepackage[margin=1in]{geometry}
\usepackage{amsmath,amssymb}
\begin{document}

\section*{Problem 2}
Vrednost izraza
\[
\frac{(1+i)^{2024}+(1-i)^{2025}i}{(1-i)^{2024}+(1+i)^{2025}i}
\]
iznosi:

\subsection*{Solution}
Pi\u{s}emo $1\pm i=\sqrt2\,e^{\pm i\pi/4}$.
\[
(1+i)^{2024}=(\sqrt2)^{2024}e^{i2024\pi/4}=2^{1012}e^{i506\pi}=2^{1012}.
\]
Dalje,
\[
(1-i)^{2025}=(\sqrt2)^{2025}e^{-i2025\pi/4}
 =2^{1012}\sqrt2\cdot e^{-i(506\pi+\pi/4)}=2^{1012}\sqrt2\cdot e^{-i\pi/4}.
\]
Kako je $e^{-i\pi/4}=\frac{1-i}{\sqrt2}$, dobijamo $(1-i)^{2025}=2^{1012}(1-i)$, pa
\[
(1-i)^{2025}i=2^{1012}(1-i)i=2^{1012}(1+i).
\]
Zato je brojilac
\[
N=2^{1012}+2^{1012}(1+i)=2^{1012}(2+i).
\]
Sli\u{c}no,
\[
(1-i)^{2024}=2^{1012},\qquad (1+i)^{2025}i=2^{1012}(i-1),
\]
pa je imenilac
\[
D=2^{1012}+2^{1012}(i-1)=2^{1012}i.
\]
Zato
\[
\frac{N}{D}=\frac{2+i}{i}=(2+i)\cdot\frac{-i}{-i\,i}=(2+i)(-i)=1-2i.
\]

\subsection*{Answer}
$1-2i$ (option \textbf{D}).

\end{document}


