\documentclass[12pt]{article}
\usepackage[margin=1in]{geometry}
\usepackage{amsmath,amssymb}
\begin{document}

\section*{Problem 18}
Maksimalna zapremina $V$ pravilne \v{s}estostrane piramide upisane u loptu polupre\u{c}nika $R$ iznosi:

\subsection*{Solution}
Neka je lopta polupre\u{c}nika $R$ sa centrom u koordinatnom po\u{c}etku.
Rotacijom mo\u{z}emo uzeti da je vrh piramide $S$ na severnom polu:
\[
S=(0,0,R).
\]
Baza je pravilni \v{s}estougao u ravni $z=z_0$ ($z_0<R$), sa centrom na $z$-osi.
Temena baze su na lopti, pa za polupre\u{c}nik opisane kru\u{z}nice baze va\u{z}i
\[
\rho^2+z_0^2=R^2 \;\Longrightarrow\; \rho^2=R^2-z_0^2.
\]
Povr\u{s}ina pravilnog \v{s}estougla sa opisanim polupre\u{c}nikom $\rho$ je
\[
P_{\text{b}}=\frac{3\sqrt3}{2}\rho^2=\frac{3\sqrt3}{2}(R^2-z_0^2).
\]
Visina piramide je
\[
h=R-z_0.
\]
Zapremina:
\[
V(z_0)=\frac13 P_{\text{b}}h
=\frac13\cdot \frac{3\sqrt3}{2}(R^2-z_0^2)(R-z_0)
=\frac{\sqrt3}{2}(R^2-z_0^2)(R-z_0).
\]
Primetimo da je
\[
(R^2-z_0^2)(R-z_0)=(R-z_0)(R+z_0)(R-z_0)=(R+z_0)(R-z_0)^2.
\]
Neka je $u=R-z_0>0$. Tada je $R+z_0=2R-u$, pa
\[
V=\frac{\sqrt3}{2}(2R-u)u^2.
\]
Maksimum: derivacijom $g(u)=(2R-u)u^2=2Ru^2-u^3$ dobijamo
\[
g'(u)=4Ru-3u^2=u(4R-3u)=0 \;\Rightarrow\; u=\frac{4R}{3}.
\]
Tada je
\[
g_{\max}=\left(2R-\frac{4R}{3}\right)\left(\frac{4R}{3}\right)^2
=\frac{2R}{3}\cdot \frac{16R^2}{9}=\frac{32R^3}{27}.
\]
Zato
\[
V_{\max}=\frac{\sqrt3}{2}\cdot \frac{32R^3}{27}=\frac{16\sqrt3}{27}R^3.
\]

\subsection*{Answer}
$\dfrac{16\sqrt3}{27}R^3$ (option \textbf{A}).

\end{document}




