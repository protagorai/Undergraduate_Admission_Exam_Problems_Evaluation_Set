\documentclass[12pt]{article}
\usepackage[margin=1in]{geometry}
\usepackage{amsmath,amssymb}
\begin{document}

\section*{Problem 4}
Neka su $x_1$ i $x_2$ re\u{s}enja jedna\u{c}ine
\[
x^2+(m-2)x+m=0,\qquad m\in\mathbb{R}.
\]
Razlika najve\u{c}e i najmanje vrednosti parametra $m$ za koje va\u{z}i uslov
\[
x_1^2+x_2^2\le 8-9x_1x_2
\]
iznosi:

\subsection*{Solution}
Po Vijetovim formulama:
\[
x_1+x_2=2-m,\qquad x_1x_2=m.
\]
Zato je
\[
x_1^2+x_2^2=(x_1+x_2)^2-2x_1x_2=(2-m)^2-2m=m^2-6m+4.
\]
Uslov postaje
\[
m^2-6m+4\le 8-9m\quad\Longleftrightarrow\quad m^2+3m-4\le 0.
\]
Faktori\u{s}emo:
\[
m^2+3m-4=(m+4)(m-1)\le 0 \;\Longleftrightarrow\; -4\le m\le 1.
\]
Najmanje je $m_{\min}=-4$, najve\u{c}e $m_{\max}=1$, pa je razlika
\[
m_{\max}-m_{\min}=1-(-4)=5.
\]

\subsection*{Answer}
$5$ (option \textbf{A}).

\end{document}


