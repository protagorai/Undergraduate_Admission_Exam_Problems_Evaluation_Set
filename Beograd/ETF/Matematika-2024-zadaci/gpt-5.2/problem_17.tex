\documentclass[12pt]{article}
\usepackage[margin=1in]{geometry}
\usepackage{amsmath,amssymb}
\begin{document}

\section*{Problem 17}
Neka je oko trougla sa stranicama $a,b,c$ i povr\u{s}inom $P=15\sqrt3\ \text{cm}^2$
opisana kru\u{z}nica polupre\u{c}nika $R=\dfrac{14\sqrt3}{3}\ \text{cm}$.
Ako je $a=10\ \text{cm}$ (naspram jednog o\u{s}trog ugla), zbir du\u{z}ina stranica $b$ i $c$ iznosi (u cm):

\subsection*{Solution}
Va\u{z}i formula za povr\u{s}inu preko opisanog polupre\u{c}nika:
\[
P=\frac{abc}{4R}\quad\Longrightarrow\quad bc=\frac{4RP}{a}.
\]
Ra\u{c}unamo:
\[
bc=\frac{4\cdot \frac{14\sqrt3}{3}\cdot 15\sqrt3}{10}
=\frac{4\cdot 14\cdot 15\cdot 3}{3\cdot 10}
=\frac{4\cdot 14\cdot 15}{10}=84.
\]
Po sinusnom teoremu $a=2R\sin A$, pa
\[
\sin A=\frac{a}{2R}=\frac{10}{2\cdot \frac{14\sqrt3}{3}}=\frac{5\sqrt3}{14}.
\]
Kako je ugao $A$ o\u{s}tar, $\cos A>0$, pa
\[
\cos A=\sqrt{1-\sin^2 A}=\sqrt{1-\frac{75}{196}}=\sqrt{\frac{121}{196}}=\frac{11}{14}.
\]
Po kosinusnom teoremu:
\[
a^2=b^2+c^2-2bc\cos A
\ \Longrightarrow\ 
b^2+c^2=a^2+2bc\cos A
=100+2\cdot 84\cdot \frac{11}{14}=100+132=232.
\]
Zato
\[
(b+c)^2=b^2+c^2+2bc=232+168=400\ \Longrightarrow\ b+c=20.
\]

\subsection*{Answer}
$20$ (option \textbf{D}).

\end{document}




