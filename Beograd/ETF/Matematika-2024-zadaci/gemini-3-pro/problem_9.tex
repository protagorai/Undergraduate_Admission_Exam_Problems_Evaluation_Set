\documentclass[12pt]{article}
\usepackage[margin=1in]{geometry}
\usepackage{amsmath,amssymb}
\usepackage[utf8]{inputenc}

\begin{document}

\section*{Problem 9}
Skup rešenja nejednačine $-\frac{3}{5} \le x + x^2 + x^3 + \dots + x^n + \dots < 1$, $x \in \mathbb{R}$, je oblika (za neke realne brojeve $a, b$ takve da je $-\infty < a < b < +\infty$):

\subsection*{Solution}
Izraz $x + x^2 + x^3 + \dots$ predstavlja zbir geometrijskog reda sa prvim članom $x$ i količnikom $q=x$.
Da bi red konvergirao (i imao konačnu sumu o kojoj možemo da govorimo u kontekstu nejednačine), mora važiti $|x| < 1$, tj. $-1 < x < 1$.
Suma reda je $S = \frac{x}{1-x}$.

Nejednačina postaje:
$-\frac{3}{5} \le \frac{x}{1-x} < 1$.

Rešavamo levi deo:
$\frac{x}{1-x} \ge -\frac{3}{5}$
$\frac{x}{1-x} + \frac{3}{5} \ge 0$
$\frac{5x + 3(1-x)}{5(1-x)} \ge 0$
$\frac{2x+3}{5(1-x)} \ge 0$.
Kako je $|x| < 1$, imamo $1-x > 0$, pa imenilac je pozitivan.
Dakle, brojilac mora biti nenegativan:
$2x + 3 \ge 0 \implies x \ge -1.5$.
Uz uslov $-1 < x < 1$, ovaj uslov ($x \ge -1.5$) je uvek zadovoljen za sve $x$ iz domena konvergencije.

Rešavamo desni deo:
$\frac{x}{1-x} < 1$
$\frac{x}{1-x} - 1 < 0$
$\frac{x - (1-x)}{1-x} < 0$
$\frac{2x - 1}{1-x} < 0$.
Opet, $1-x > 0$, pa mora biti $2x - 1 < 0$.
$2x < 1 \implies x < 1/2$.

Kombinujemo sa uslovom konvergencije $|x| < 1$:
$x \in (-1, 1/2)$.

Ovaj skup je oblika $(a, b)$ gde je $a=-1, b=1/2$.

\subsection*{Answer}
$(a, b)$ (option \textbf{E}).

\end{document}

