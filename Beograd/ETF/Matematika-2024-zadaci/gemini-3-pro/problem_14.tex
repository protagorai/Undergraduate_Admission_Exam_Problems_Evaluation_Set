\documentclass[12pt]{article}
\usepackage[margin=1in]{geometry}
\usepackage{amsmath,amssymb}
\usepackage[utf8]{inputenc}

\begin{document}

\section*{Problem 14}
Tri tačke dele kružnicu u odnosu 6:7:11. Vrednost najmanjeg ugla trougla određenog tim tačkama iznosi:

\subsection*{Solution}
Neka su tačke $A, B, C$ na kružnici. One dele kružnicu na tri luka čije su dužine u odnosu 6:7:11.
Ukupan broj delova je $6+7+11 = 24$.
Pun krug je $360^\circ$.
Mere centralnih uglova koji odgovaraju ovim lukovima su:
$\alpha_1 = \frac{6}{24} \cdot 360^\circ = \frac{1}{4} \cdot 360^\circ = 90^\circ$.
$\alpha_2 = \frac{7}{24} \cdot 360^\circ = 7 \cdot 15^\circ = 105^\circ$.
$\alpha_3 = \frac{11}{24} \cdot 360^\circ = 11 \cdot 15^\circ = 165^\circ$.

Unutrašnji uglovi trougla $ABC$ su periferijski uglovi nad ovim lukovima.
Ugao kod temena $A$ nad lukom $BC$ (koji odgovara npr. $\alpha_1$) bi bio $\alpha_1/2$? Ne tačno.
Temena trougla su tačke podele.
Stranice trougla su tetive koje odgovaraju lukovima.
Ugao naspram luka mere $\alpha$ je periferijski ugao nad tim lukom i iznosi $\alpha/2$.
Dakle, uglovi trougla su polovine centralnih uglova nad odgovarajućim stranicama.
Tri luka su $90^\circ, 105^\circ, 165^\circ$.
Uglovi trougla su:
$\beta_1 = 90^\circ / 2 = 45^\circ$.
$\beta_2 = 105^\circ / 2 = 52.5^\circ$.
$\beta_3 = 165^\circ / 2 = 82.5^\circ$.

Provera zbira: $45 + 52.5 + 82.5 = 180$.
Najmanji ugao je $45^\circ$.

\subsection*{Answer}
$45^\circ$ (option \textbf{D}).

\end{document}

