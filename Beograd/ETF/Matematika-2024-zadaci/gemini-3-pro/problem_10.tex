\documentclass[12pt]{article}
\usepackage[margin=1in]{geometry}
\usepackage{amsmath,amssymb}
\usepackage[utf8]{inputenc}

\begin{document}

\section*{Problem 10}
Zbir svih realnih rešenja jednačine $\sqrt[3]{x-3} + \sqrt[3]{x-1} = \sqrt[3]{2x-4}$ iznosi:

\subsection*{Solution}
Uočimo da je $2x-4 = (x-3) + (x-1)$.
Uvedimo smenu $u = \sqrt[3]{x-3}$ i $v = \sqrt[3]{x-1}$.
Tada je desna strana $\sqrt[3]{u^3 + v^3}$.
Jednačina postaje:
$u + v = \sqrt[3]{u^3 + v^3}$.
Kubiranjem obe strane:
$(u+v)^3 = u^3 + v^3$
$u^3 + v^3 + 3uv(u+v) = u^3 + v^3$
$3uv(u+v) = 0$.

Ovo je ekvivalentno sa:
$u = 0$ ili $v = 0$ ili $u+v = 0$.

Slučaj 1: $u = 0$
$\sqrt[3]{x-3} = 0 \implies x = 3$.

Slučaj 2: $v = 0$
$\sqrt[3]{x-1} = 0 \implies x = 1$.

Slučaj 3: $u = -v$
$\sqrt[3]{x-3} = -\sqrt[3]{x-1} = \sqrt[3]{-(x-1)} = \sqrt[3]{1-x}$
$x-3 = 1-x$
$2x = 4 \implies x = 2$.

Sva tri rešenja su realna.
Zbir rešenja: $1 + 2 + 3 = 6$.

\subsection*{Answer}
$6$ (option \textbf{B}).

\end{document}

