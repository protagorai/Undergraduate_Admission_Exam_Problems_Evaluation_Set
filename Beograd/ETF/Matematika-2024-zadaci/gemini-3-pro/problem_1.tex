\documentclass[12pt]{article}
\usepackage[margin=1in]{geometry}
\usepackage{amsmath,amssymb}
\usepackage[utf8]{inputenc}

\begin{document}

\section*{Problem 1}
Za $a \in \mathbb{R}, a \neq 3$ izraz $\frac{a^3-27}{(a-3)^2} \cdot \frac{a^2-2a-3}{(a+3)^2-3a}$ je jednak:

\subsection*{Solution}
Počnimo sa uprošćavanjem pojedinačnih delova izraza.

Prvi razlomak:
Brojilac: $a^3 - 27$ je razlika kubova. Formula: $x^3 - y^3 = (x-y)(x^2+xy+y^2)$.
\[ a^3 - 3^3 = (a-3)(a^2 + 3a + 9) \]
Imenilac: $(a-3)^2$.

Drugi razlomak:
Brojilac: $a^2 - 2a - 3$. Ovo je kvadratni trinom. Nule su $a = \frac{2 \pm \sqrt{4+12}}{2} = \frac{2 \pm 4}{2}$, tj. $a=3$ i $a=-1$.
Dakle, $a^2 - 2a - 3 = (a-3)(a+1)$.
Imenilac: $(a+3)^2 - 3a = a^2 + 6a + 9 - 3a = a^2 + 3a + 9$.

Sada zamenimo sve u polazni izraz:
\[
\frac{(a-3)(a^2+3a+9)}{(a-3)^2} \cdot \frac{(a-3)(a+1)}{a^2+3a+9}
\]
Skratimo $(a^2+3a+9)$ u brojiocu prvog i imeniocu drugog dela (pošto $a^2+3a+9 = (a+1.5)^2 + 6.75 > 0$, nije nula).
\[
\frac{a-3}{(a-3)^2} \cdot \frac{(a-3)(a+1)}{1}
\]
\[
\frac{a-3}{a-3} \cdot \frac{a+1}{1} \cdot \frac{a-3}{a-3}
\]
Pažljivije:
\[
\frac{a^3-27}{(a-3)^2} = \frac{a^2+3a+9}{a-3}
\]
Množimo sa drugim delom:
\[
\frac{a^2+3a+9}{a-3} \cdot \frac{(a-3)(a+1)}{a^2+3a+9}
\]
Skratimo $a^2+3a+9$ i $a-3$:
\[
1 \cdot (a+1) = a+1
\]
Uslov $a \neq 3$ je dat.

\subsection*{Answer}
$a+1$ (option \textbf{C}).

\end{document}

