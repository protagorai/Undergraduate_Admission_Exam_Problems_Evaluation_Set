\documentclass[12pt]{article}
\usepackage[margin=1in]{geometry}
\usepackage{amsmath,amssymb}
\usepackage[utf8]{inputenc}

\begin{document}

\section*{Problem 12}
Ugao pod kojim se vidi kružnica $x^2 + (y-25)^2 = 225$ iz koordinatnog početka iznosi:

\subsection*{Solution}
Jednačina kružnice je $x^2 + (y-25)^2 = 15^2$.
Centar kružnice je $C(0, 25)$, a poluprečnik $r=15$.
Posmatramo ugao pod kojim se kružnica vidi iz tačke $O(0,0)$.
Tangente povučene iz $O$ na kružnicu formiraju taj ugao.
Neka je $\phi$ traženi ugao. Neka je $\alpha = \phi/2$.
Uočimo pravougli trougao koji čine koordinatni početak $O$, centar kružnice $C$ i tačka dodira $T$ jedne tangente.
Ugao kod temena $O$ u tom trouglu je $\alpha$.
Hipotenuza je $d = OC = 25$ (rastojanje od koordinatnog početka do centra $(0, 25)$).
Kateta naspram ugla $\alpha$ je poluprečnik $r=15$.
Iz definicije sinusa:
$\sin \alpha = \frac{r}{d} = \frac{15}{25} = \frac{3}{5}$.

Treba nam ugao $\phi = 2\alpha$.
Možemo izračunati $\tan \phi$ ili izraziti preko $\arctan$.
Iz $\sin \alpha = 3/5$, sledi $\cos \alpha = 4/5$ i $\tan \alpha = 3/4$.
Koristimo formulu za tangens dvostrukog ugla:
$\tan \phi = \frac{2 \tan \alpha}{1 - \tan^2 \alpha} = \frac{2 \cdot \frac{3}{4}}{1 - (\frac{3}{4})^2} = \frac{\frac{3}{2}}{1 - \frac{9}{16}} = \frac{\frac{3}{2}}{\frac{7}{16}} = \frac{3}{2} \cdot \frac{16}{7} = \frac{24}{7}$.
Dakle, $\phi = \arctan \frac{24}{7}$.

\subsection*{Answer}
$\mathrm{arctg}\, \frac{24}{7}$ (option \textbf{A}).

\end{document}

