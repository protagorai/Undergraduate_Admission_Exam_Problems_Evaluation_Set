\documentclass[12pt]{article}
\usepackage[margin=1in]{geometry}
\usepackage{amsmath,amssymb}
\begin{document}

\section*{Problem 14}
Tri tačke dele kružnicu u odnosu $6:7:11$. Vrednost najmanjeg ugla trougla određenog tim tačkama iznosi:

\subsection*{Solution}
Neka su tačke $A$, $B$, i $C$ na kružnici koje dele kružnicu u odnosu $6:7:11$.

Ukupan obim kružnice odgovara uglu od $360°$, pa su centralni uglovi:
\[\alpha = \frac{6}{6+7+11} \cdot 360° = \frac{6}{24} \cdot 360° = 90°\]
\[\beta = \frac{7}{24} \cdot 360° = 105°\]
\[\gamma = \frac{11}{24} \cdot 360° = 165°\]

Uglovi trougla $ABC$ su upisani uglovi koji se oslanjaju na lukove:
- Ugao $A$ se oslanja na luk $BC$ (centralni ugao $105°$), pa je $\angle A = \frac{105°}{2} = 52.5°$
- Ugao $B$ se oslanja na luk $AC$ (centralni ugao $165°$), pa je $\angle B = \frac{165°}{2} = 82.5°$  
- Ugao $C$ se oslanja na luk $AB$ (centralni ugao $90°$), pa je $\angle C = \frac{90°}{2} = 45°$

Proveravamo: $52.5° + 82.5° + 45° = 180°$ ✓

Najmanji ugao je $45°$.

\subsection*{Answer}
$45°$ (opcija \textbf{D}).

\end{document}
