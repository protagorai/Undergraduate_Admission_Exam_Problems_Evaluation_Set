\documentclass[12pt]{article}
\usepackage[margin=1in]{geometry}
\usepackage{amsmath,amssymb}
\begin{document}

\section*{Problem 17}
Neka je oko trougla čije su stranice $a$, $b$ i $c$ i površina $P = 15\sqrt{3}\text{cm}^2$, opisana kružnica poluprečnika $R = \frac{14\sqrt{3}}{3}\text{cm}$. Ako je $a$ stranica dužine 10cm, koja se nalazi naspram jednog oštrog ugla, zbir dužina stranica $b$ i $c$ iznosi (u cm):

\subsection*{Solution}
Koristimo formulu za površinu trougla preko poluprečnika opisane kružnice:
\[P = \frac{abc}{4R}\]

Odatle:
\[abc = 4PR = 4 \cdot 15\sqrt{3} \cdot \frac{14\sqrt{3}}{3} = 4 \cdot 15\sqrt{3} \cdot \frac{14\sqrt{3}}{3} = 4 \cdot 15 \cdot 14 \cdot \frac{3}{3} = 840\]

Pošto je $a = 10$:
\[10bc = 840 \Rightarrow bc = 84\]

Takođe, koristimo formulu za površinu preko Heronovog obrasca. Neka je $s = \frac{a+b+c}{2}$ poluobim.

\[P = \sqrt{s(s-a)(s-b)(s-c)}\]
\[15\sqrt{3} = \sqrt{s(s-10)(s-b)(s-c)}\]

Takođe, iz sinusne teoreme:
\[\frac{a}{\sin A} = 2R \Rightarrow \sin A = \frac{a}{2R} = \frac{10}{2 \cdot \frac{14\sqrt{3}}{3}} = \frac{10 \cdot 3}{28\sqrt{3}} = \frac{30}{28\sqrt{3}} = \frac{15}{14\sqrt{3}} = \frac{15\sqrt{3}}{42} = \frac{5\sqrt{3}}{14}\]

Pošto je ugao $A$ oštar, $\cos A = \sqrt{1 - \sin^2 A} = \sqrt{1 - \frac{75}{196}} = \sqrt{\frac{121}{196}} = \frac{11}{14}$.

Iz kosinusne teoreme:
\[a^2 = b^2 + c^2 - 2bc\cos A\]
\[100 = b^2 + c^2 - 2 \cdot 84 \cdot \frac{11}{14}\]
\[100 = b^2 + c^2 - 132\]
\[b^2 + c^2 = 232\]

Imamo sistem:
\[bc = 84\]
\[b^2 + c^2 = 232\]

Iz $(b+c)^2 = b^2 + 2bc + c^2 = 232 + 2 \cdot 84 = 232 + 168 = 400$.

Dakle $b + c = 20$.

\subsection*{Answer}
$20$ (opcija \textbf{D}).

\end{document}
