\documentclass[12pt]{article}
\usepackage[margin=1in]{geometry}
\usepackage{amsmath,amssymb}
\begin{document}

\section*{Problem 18}
Maksimalna zapremina $V$ pravilne šestostrane piramide upisane u loptu poluprečnika $R$ iznosi:

\subsection*{Solution}
Neka je $a$ stranica osnove (pravilnog šestougla) i $h$ visina piramide.

Za pravilnu šestostanu piramidu upisanu u loptu poluprečnika $R$:
- Centar lopte se nalazi na visini piramide
- Rastojanje od centra lopte do bilo kog temena je $R$

Neka je centar lopte na rastojanju $x$ od osnove piramide. Tada je visina piramide $h = R + x$.

Poluprečnik opisane kružnice oko pravilnog šestougla sa stranicom $a$ je takođe $a$.

Iz uslova da su sva temena na lopti:
\[a^2 + x^2 = R^2\]

Dakle $a^2 = R^2 - x^2$.

Površina pravilnog šestougla sa stranicom $a$:
\[P = \frac{3\sqrt{3}}{2}a^2 = \frac{3\sqrt{3}}{2}(R^2 - x^2)\]

Zapremina piramide:
\[V = \frac{1}{3}Ph = \frac{1}{3} \cdot \frac{3\sqrt{3}}{2}(R^2 - x^2)(R + x)\]
\[= \frac{\sqrt{3}}{2}(R^2 - x^2)(R + x)\]
\[= \frac{\sqrt{3}}{2}(R^3 + R^2x - Rx^2 - x^3)\]

Da bismo našli maksimum, derivišemo po $x$:
\[\frac{dV}{dx} = \frac{\sqrt{3}}{2}(R^2 - 2Rx - 3x^2) = 0\]
\[R^2 - 2Rx - 3x^2 = 0\]
\[3x^2 + 2Rx - R^2 = 0\]

Koristeći kvadratnu formulu:
\[x = \frac{-2R \pm \sqrt{4R^2 + 12R^2}}{6} = \frac{-2R \pm 4R}{6}\]

Uzimamo pozitivno rešenje: $x = \frac{R}{3}$.

Maksimalna zapremina:
\[V_{max} = \frac{\sqrt{3}}{2}(R^2 - \frac{R^2}{9})(R + \frac{R}{3})\]
\[= \frac{\sqrt{3}}{2} \cdot \frac{8R^2}{9} \cdot \frac{4R}{3}\]
\[= \frac{\sqrt{3}}{2} \cdot \frac{32R^3}{27} = \frac{16\sqrt{3}R^3}{27}\]

\subsection*{Answer}
$\frac{16\sqrt{3}R^3}{27}$ (opcija \textbf{A}).

\end{document}
