\documentclass[12pt]{article}
\usepackage[margin=1in]{geometry}
\usepackage{amsmath,amssymb}
\begin{document}

\section*{Problem 6}
Broj različitih polinoma oblika $x^2 + px + q$, za $p, q \in \mathbb{R}$, koji dele polinom $x^4 - 3x^2 + 2$ iznosi:

\subsection*{Solution}
Prvo ćemo faktorizovati polinom $x^4 - 3x^2 + 2$.

Neka je $y = x^2$, tada imamo:
\[y^2 - 3y + 2 = 0\]

Faktorizujemo:
\[y^2 - 3y + 2 = (y-1)(y-2) = (x^2-1)(x^2-2)\]

Dakle:
\[x^4 - 3x^2 + 2 = (x^2-1)(x^2-2) = (x-1)(x+1)(x-\sqrt{2})(x+\sqrt{2})\]

Sada tražimo polinome drugog stepena koji dele ovaj polinom. Polinom drugog stepena može imati:
1. Dva realna korena
2. Dva konjugovano kompleksna korena

Mogući polinomi drugog stepena koji dele $x^4 - 3x^2 + 2$ su:
1. $(x-1)(x+1) = x^2 - 1$
2. $(x-\sqrt{2})(x+\sqrt{2}) = x^2 - 2$
3. $(x-1)(x-\sqrt{2}) = x^2 - (1+\sqrt{2})x + \sqrt{2}$
4. $(x-1)(x+\sqrt{2}) = x^2 + (\sqrt{2}-1)x - \sqrt{2}$
5. $(x+1)(x-\sqrt{2}) = x^2 + (1-\sqrt{2})x - \sqrt{2}$
6. $(x+1)(x+\sqrt{2}) = x^2 + (1+\sqrt{2})x + \sqrt{2}$

Dakle, imamo 6 različitih polinoma drugog stepena.

\subsection*{Answer}
$6$ (opcija \textbf{B}).

\end{document}
