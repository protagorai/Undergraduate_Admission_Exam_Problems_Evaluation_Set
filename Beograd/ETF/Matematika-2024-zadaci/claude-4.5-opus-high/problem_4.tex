\documentclass[12pt]{article}
\usepackage[margin=1in]{geometry}
\usepackage{amsmath,amssymb}
\begin{document}

\section*{Problem 4}
Let $x_1$ and $x_2$ be solutions of equation $x^2 + (m-2)x + m = 0$, $m \in \mathbb{R}$. 
Find the difference between the maximum and minimum values of parameter $m$ for which 
$x_1^2 + x_2^2 \leq 8 - 9x_1x_2$ holds.

\subsection*{Solution}
By Vieta's formulas:
\[
x_1 + x_2 = -(m-2) = 2-m, \quad x_1 x_2 = m
\]

We know that:
\[
x_1^2 + x_2^2 = (x_1 + x_2)^2 - 2x_1x_2 = (2-m)^2 - 2m
\]

The condition becomes:
\[
(2-m)^2 - 2m \leq 8 - 9m
\]

Expanding:
\[
4 - 4m + m^2 - 2m \leq 8 - 9m
\]
\[
m^2 - 6m + 4 \leq 8 - 9m
\]
\[
m^2 - 6m + 4 - 8 + 9m \leq 0
\]
\[
m^2 + 3m - 4 \leq 0
\]

Factoring:
\[
(m + 4)(m - 1) \leq 0
\]

This holds when $-4 \leq m \leq 1$.

The difference between maximum and minimum:
\[
1 - (-4) = 5
\]

\subsection*{Answer}
$5$ (option \textbf{A}).

\end{document}

