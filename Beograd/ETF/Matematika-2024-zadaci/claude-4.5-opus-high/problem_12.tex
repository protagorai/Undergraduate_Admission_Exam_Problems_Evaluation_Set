\documentclass[12pt]{article}
\usepackage[margin=1in]{geometry}
\usepackage{amsmath,amssymb}
\begin{document}

\section*{Problem 12}
Find the angle from which the circle $x^2 + (y-25)^2 = 225$ is seen from the coordinate origin.

\subsection*{Solution}
The circle has:
\begin{itemize}
\item Center: $(0, 25)$
\item Radius: $r = \sqrt{225} = 15$
\end{itemize}

Distance from origin to center:
\[
d = 25
\]

The angle $\theta$ from which the circle is seen from the origin is the angle between the two tangent lines from the origin to the circle.

For a tangent line from an external point, if $\theta/2$ is the half-angle:
\[
\sin\left(\frac{\theta}{2}\right) = \frac{r}{d} = \frac{15}{25} = \frac{3}{5}
\]

From $\sin(\theta/2) = \frac{3}{5}$, we get:
\[
\cos\left(\frac{\theta}{2}\right) = \sqrt{1 - \frac{9}{25}} = \sqrt{\frac{16}{25}} = \frac{4}{5}
\]

Therefore:
\[
\tan\left(\frac{\theta}{2}\right) = \frac{\sin(\theta/2)}{\cos(\theta/2)} = \frac{3/5}{4/5} = \frac{3}{4}
\]

Using the double angle formula:
\[
\tan\theta = \frac{2\tan(\theta/2)}{1 - \tan^2(\theta/2)} = \frac{2 \cdot \frac{3}{4}}{1 - \frac{9}{16}} = \frac{\frac{3}{2}}{\frac{7}{16}} = \frac{3}{2} \cdot \frac{16}{7} = \frac{24}{7}
\]

So:
\[
\theta = \arctan\frac{24}{7}
\]

\subsection*{Answer}
$\arctan\frac{24}{7}$ (option \textbf{A}).

\end{document}

