\documentclass[12pt]{article}
\usepackage[margin=1in]{geometry}
\usepackage{amsmath,amssymb}
\begin{document}

\section*{Problem 14}
U jednakokraki trougao osnovice dužine 12cm i odgovarajuće visine dužine 8cm, upisan je pravougaonik maksimalne površine tako da mu jedna stranica pripada osnovici trougla. Obim pravougaonika (u cm) je:

\subsection*{Solution}
Postavimo koordinatni sistem tako da je osnova trougla na $x$-osi od $(-6, 0)$ do $(6, 0)$, a vrh trougla u $(0, 8)$.

Jednačina stranice trougla od $(6, 0)$ do $(0, 8)$ je:
\[
\frac{x}{6} + \frac{y}{8} = 1 \Rightarrow y = 8 - \frac{4x}{3}
\]

Neka je pravougaonik sa temenima u $(-a, 0)$, $(a, 0)$, $(a, h)$, $(-a, h)$, gde je $0 < a \leq 6$.

Visina pravougaonika je $h = 8 - \frac{4a}{3}$.

Površina pravougaonika je:
\[
P = 2a \cdot h = 2a\left(8 - \frac{4a}{3}\right) = 16a - \frac{8a^2}{3}
\]

Da nađemo maksimum, derivišemo po $a$:
\[
\frac{dP}{da} = 16 - \frac{16a}{3} = 0
\]
\[
16 = \frac{16a}{3} \Rightarrow a = 3
\]

Za $a = 3$: $h = 8 - \frac{4 \cdot 3}{3} = 8 - 4 = 4$.

Obim pravougaonika je:
\[
O = 2(2a + h) = 2(6 + 4) = 20
\]

\subsection*{Answer}
$20$ (option \textbf{A}).

\end{document}