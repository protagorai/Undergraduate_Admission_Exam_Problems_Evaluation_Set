\documentclass[12pt]{article}
\usepackage[margin=1in]{geometry}
\usepackage{amsmath,amssymb}
\usepackage[utf8]{inputenc}
\begin{document}

\section*{Problem 4}
Ako je $a > b > 0$ i $a^2 + b^2 = 6ab$, tada je $\frac{a+b}{a-b}$ jednako:
\begin{itemize}
    \item[A)] $-\sqrt{2}$
    \item[B)] $\sqrt{2}$
    \item[C)] $\sqrt{6}$
    \item[D)] $1$
    \item[E)] $\frac{1}{\sqrt{2}}$
    \item[N)] Ne znam
\end{itemize}

\subsection*{Solution}
Let $X = \frac{a+b}{a-b}$. We can compute $X^2$:
\[ X^2 = \left(\frac{a+b}{a-b}\right)^2 = \frac{a^2 + 2ab + b^2}{a^2 - 2ab + b^2} = \frac{(a^2 + b^2) + 2ab}{(a^2 + b^2) - 2ab}. \]
Substitute the given condition $a^2 + b^2 = 6ab$:
\[ X^2 = \frac{6ab + 2ab}{6ab - 2ab} = \frac{8ab}{4ab} = 2. \]
Thus $X = \pm \sqrt{2}$.
Since $a > b > 0$, we have $a+b > 0$ and $a-b > 0$. Therefore, the ratio $X$ must be positive.
\[ X = \sqrt{2}. \]

\subsection*{Answer}
The value is $\sqrt{2}$, which corresponds to option \textbf{B}.

\end{document}