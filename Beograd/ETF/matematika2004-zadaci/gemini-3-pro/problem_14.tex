\documentclass[12pt]{article}
\usepackage[margin=1in]{geometry}
\usepackage{amsmath,amssymb}
\usepackage[utf8]{inputenc}
\begin{document}

\section*{Problem 14}
U jednakokraki trougao osnovice dužine 12cm i odgovarajuće visine dužine 8cm, upisan je pravougaonik maksimalne površine tako da mu jedna stranica pripada osnovici trougla. Obim pravougaonika (u cm) je:
\begin{itemize}
    \item[A)] $20$
    \item[B)] $16$
    \item[C)] $14$
    \item[D)] $24$
    \item[E)] $10$
    \item[N)] Ne znam
\end{itemize}

\subsection*{Solution}
Let the base of the triangle be $b=12$ and the height be $h=8$.
Let the rectangle have width $x$ (along the base of the triangle) and height $y$.
Using similar triangles, the ratio of the height of the smaller triangle above the rectangle to the base of that triangle ($x$) is equal to the ratio of the height of the large triangle ($h$) to its base ($b$):
Actually, it's better to relate the height and base directly.
The triangle above the rectangle has height $h-y$ and base $x$.
\[ \frac{h-y}{h} = \frac{x}{b}. \]
Substitute values:
\[ \frac{8-y}{8} = \frac{x}{12}. \]
\[ 1 - \frac{y}{8} = \frac{x}{12} \implies \frac{y}{8} = 1 - \frac{x}{12} = \frac{12-x}{12}. \]
\[ y = \frac{8}{12}(12-x) = \frac{2}{3}(12-x). \]
The area of the rectangle is $A = xy$:
\[ A(x) = x \cdot \frac{2}{3}(12-x) = \frac{2}{3}(12x - x^2). \]
This is a downward opening parabola. The maximum occurs at the vertex $x = -\frac{12}{2(-1)} = 6$.
When $x=6$, the height is:
\[ y = \frac{2}{3}(12-6) = \frac{2}{3}(6) = 4. \]
The dimensions of the rectangle are $6$ cm and $4$ cm.
The perimeter is:
\[ P = 2(x+y) = 2(6+4) = 2(10) = 20 \text{ cm}. \]

\subsection*{Answer}
The perimeter is $20$, which corresponds to option \textbf{A}.

\end{document}