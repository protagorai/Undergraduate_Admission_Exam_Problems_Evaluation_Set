\documentclass[12pt]{article}
\usepackage[margin=1in]{geometry}
\usepackage{amsmath,amssymb}
\usepackage[utf8]{inputenc}
\begin{document}

\section*{Problem 20}
Ako su $a, b, i c$ istovremeno peti, sedamnaesti i trideset sedmi član i aritmetičke i geometrijske progresije, tada je $a^{b-c} \cdot b^{c-a} \cdot c^{a-b}$ jednako:
\begin{itemize}
    \item[A)] $1/3$
    \item[B)] $1$
    \item[C)] $1/2$
    \item[D)] $1/4$
    \item[E)] $2$
    \item[N)] Ne znam
\end{itemize}

\subsection*{Solution}
Let the numbers be $a, b, c$.
They are the 5th, 17th, and 37th terms of an arithmetic progression (AP).
So $a = k_1 + 4d$, $b = k_1 + 16d$, $c = k_1 + 36d$.
This implies:
\[ b - a = 12d, \quad c - b = 20d. \]
So $\frac{b-a}{12} = \frac{c-b}{20} \implies \frac{b-a}{3} = \frac{c-b}{5}$.
They are also the 5th, 17th, and 37th terms of a geometric progression (GP).
So $a = g_1 q^4$, $b = g_1 q^{16}$, $c = g_1 q^{36}$.
This implies:
\[ b = a q^{12}, \quad c = b q^{20} = a q^{32}. \]
Substituting into the AP relation:
\[ \frac{a q^{12} - a}{3} = \frac{a q^{32} - a q^{12}}{5}. \]
Assuming $a \ne 0$ (otherwise the product is 0 or undefined, but options suggest non-zero), we can divide by $a$:
\[ \frac{q^{12} - 1}{3} = \frac{q^{12}(q^{20} - 1)}{5}. \]
If $q=1$, then $a=b=c$.
If $q \ne 1$, this equation generally leads to no other positive real solutions for $q$ that fit the indices (it can be shown using the function derivative that $q=1$ is the unique solution for positive bases).
So we must have $q=1$, which implies $a=b=c$.
Substitute $a=b=c$ into the target expression:
\[ a^{b-c} \cdot b^{c-a} \cdot c^{a-b} = a^0 \cdot a^0 \cdot a^0 = 1 \cdot 1 \cdot 1 = 1. \]

\subsection*{Answer}
The value is $1$, which corresponds to option \textbf{B}.

\end{document}