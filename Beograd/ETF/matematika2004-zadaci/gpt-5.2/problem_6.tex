\documentclass[12pt]{article}
\usepackage[margin=1in]{geometry}
\usepackage{amsmath,amssymb}
\begin{document}

\section*{Problem 6}
Neka su $\alpha,\beta,\gamma$ uglovi, a $a,b,c$ stranice trougla. Odrediti vrednost izraza
\[
a\sin(\beta-\gamma)+b\sin(\gamma-\alpha)+c\sin(\alpha-\beta).
\]

\subsection*{Solution}
Po sinusnom teoremu važi $a=2R\sin\alpha$, $b=2R\sin\beta$, $c=2R\sin\gamma$ za opisani poluprečnik $R$.
Zato je traženi izraz jednak
\[
2R\Big(\sin\alpha\sin(\beta-\gamma)+\sin\beta\sin(\gamma-\alpha)+\sin\gamma\sin(\alpha-\beta)\Big).
\]
Razvijemo razlike:
\begin{align*}
\sin\alpha\sin(\beta-\gamma)
&=\sin\alpha(\sin\beta\cos\gamma-\cos\beta\sin\gamma)
=\sin\alpha\sin\beta\cos\gamma-\sin\alpha\cos\beta\sin\gamma,\\
\sin\beta\sin(\gamma-\alpha)
&=\sin\beta(\sin\gamma\cos\alpha-\cos\gamma\sin\alpha)
=\sin\beta\sin\gamma\cos\alpha-\sin\beta\cos\gamma\sin\alpha,\\
\sin\gamma\sin(\alpha-\beta)
&=\sin\gamma(\sin\alpha\cos\beta-\cos\alpha\sin\beta)
=\sin\gamma\sin\alpha\cos\beta-\sin\gamma\cos\alpha\sin\beta.
\end{align*}
Sabiranjem, članovi se poništavaju parovima:
\[
\sin\alpha\sin\beta\cos\gamma-\sin\beta\cos\gamma\sin\alpha=0,\quad
-\sin\alpha\cos\beta\sin\gamma+\sin\gamma\sin\alpha\cos\beta=0,\quad
\sin\beta\sin\gamma\cos\alpha-\sin\gamma\cos\alpha\sin\beta=0.
\]
Dakle, ceo izraz je $2R\cdot 0=0$.

\subsection*{Answer}
$0$ (option \textbf{D}).

\end{document}

