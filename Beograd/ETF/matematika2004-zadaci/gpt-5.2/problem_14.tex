\documentclass[12pt]{article}
\usepackage[margin=1in]{geometry}
\usepackage{amsmath,amssymb}
\begin{document}

\section*{Problem 14}
U jednakokraki trougao osnovice dužine $12\text{ cm}$ i odgovarajuće visine dužine $8\text{ cm}$ upisan je pravougaonik maksimalne površine tako da mu jedna stranica pripada osnovici trougla.
Odrediti obim pravougaonika (u cm).

\subsection*{Solution}
Neka trougao ima osnovicu $B=12$ i visinu $H=8$.
Pravougaonik ima visinu $h$ i osnovicu $w$ na osnovici trougla.
Na visini $h$ presečna duž trougla je, po sličnosti, proporcionalno manja:
\[
w=B\left(1-\frac{h}{H}\right).
\]
Površina pravougaonika je
\[
A(h)=w\cdot h
=B h\left(1-\frac{h}{H}\right)
=B\left(h-\frac{h^2}{H}\right),
\qquad 0<h<H.
\]
To je kvadratna funkcija koja maksimum postiže kada je
\[
A'(h)=B\left(1-\frac{2h}{H}\right)=0 \quad\Rightarrow\quad h=\frac{H}{2}=4.
\]
Tada je
\[
w=B\left(1-\frac{1}{2}\right)=\frac{B}{2}=6.
\]
Obim pravougaonika je
\[
O=2(w+h)=2(6+4)=20.
\]

\subsection*{Answer}
$20$ (option \textbf{A}).

\end{document}

