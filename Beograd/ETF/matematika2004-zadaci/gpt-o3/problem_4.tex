\documentclass[12pt]{article}
\usepackage[margin=1in]{geometry}
\usepackage{amsmath,amssymb}
\begin{document}

\section*{Problem 4}
Ako je $a>b>0$ i $a^2+b^2=6ab$, tada je $\dfrac{a+b}{a-b}$ jednako:
A) $-\sqrt2$\; B) $\sqrt2$\; C) $\sqrt6$\; D) $1$\; E) $\dfrac1{\sqrt2}$\; N) Ne znam.

\subsection*{Solution}
Posmatrajmo
\[X=\frac{a+b}{a-b},\qquad X^2=\frac{(a+b)^2}{(a-b)^2}=\frac{a^2+2ab+b^2}{a^2-2ab+b^2}.\]
Uslov $a^2+b^2=6ab$ daje
\[X^2=\frac{6ab+2ab}{6ab-2ab}=\frac{8ab}{4ab}=2\implies X=\pm\sqrt2.\]
Pošto je $a>b$, brojilac i imenilac su pozitivni, te je količnik pozitivan:
\[X=\sqrt2.\]

\subsection*{Answer}
$\sqrt2$ (opcija \textbf{B}).

\end{document}