\documentclass[12pt]{article}
\usepackage[margin=1in]{geometry}
\usepackage{amsmath}
\begin{document}

\section*{Problem 14}
U jednakokraki trougao sa osnovicom $12$ cm i visinom na osnovicu $8$ cm upisan je pravougaonik maksimalne površine tako da mu jedna stranica leži na osnovici. Odrediti njegov obim.

Opcije: A) 20\; B) 16\; C) 14\; D) 24\; E) 10\; N) Ne znam.

\subsection*{Solution}
Koordinate trougla: neka je osnovica na osi $x$ od $(-6,0)$ do $(6,0)$, a vrh $T$ na $y=8$. Jednačina leve stranice je $y=\tfrac43(x+6)$ (nagib $\frac{8}{6}=\tfrac43$). Za dati $x$ ($-6<x<6$) visina do stranice je $y=\tfrac43(x+6)$. Ako pravougaonik ima temena na $(-x,0),(x,0)$, njegova visina je $h=\tfrac43(6-x)$. Površina $P=2x\cdot h=2x\cdot\tfrac43(6-x)=\tfrac83x(6-x)$. Maksimum parabole postiže se u sredini segmenta $[0,6]$, tj. za $x=3$. Tada su dimenzije $2x=6$ i $h=\tfrac43\cdot3=4$. Obim je $2(6+4)=20$.

\subsection*{Answer}
20 (opcija \textbf{A}).

\end{document}