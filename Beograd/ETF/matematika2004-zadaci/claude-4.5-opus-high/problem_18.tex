\documentclass[12pt]{article}
\usepackage[margin=1in]{geometry}
\usepackage{amsmath,amssymb}
\begin{document}

\section*{Problem 18}
Date su dve paralelne prave. Na jednoj od njih je 10 a na drugoj 12 različitih tačaka. Broj trouglova koje određuju ove tačke je:

A) $\binom{10}{2}\binom{12}{1} + \binom{10}{1}\binom{12}{2}$; \quad B) $\binom{22}{3} - \binom{22}{2}$; \quad C) $\binom{10}{1}\binom{12}{2}$; 

D) $\binom{10}{2}\binom{12}{1}$; \quad E) $10 \cdot 9 \cdot 12 \cdot 11$; \quad N) Ne znam.

\subsection*{Rešenje}
Da bismo formirali trougao, potrebne su nam 3 nekolinearne tačke. Pošto su sve tačke na dve paralelne prave, ne možemo uzeti sve 3 tačke sa iste prave (bile bi kolinearne).

Imamo dva slučaja:

\textbf{Slučaj 1:} 2 tačke sa prve prave (10 tačaka) i 1 tačka sa druge prave (12 tačaka)

Broj načina: $\binom{10}{2} \cdot \binom{12}{1} = 45 \cdot 12 = 540$

\textbf{Slučaj 2:} 1 tačka sa prve prave i 2 tačke sa druge prave

Broj načina: $\binom{10}{1} \cdot \binom{12}{2} = 10 \cdot 66 = 660$

\textbf{Ukupan broj trouglova:}
\[
\binom{10}{2}\binom{12}{1} + \binom{10}{1}\binom{12}{2} = 540 + 660 = 1200
\]

\subsection*{Odgovor}
$\binom{10}{2}\binom{12}{1} + \binom{10}{1}\binom{12}{2}$ (opcija \textbf{A}).

\end{document}
