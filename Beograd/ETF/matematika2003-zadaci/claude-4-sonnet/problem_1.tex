\documentclass[12pt]{article}
\usepackage[margin=1in]{geometry}
\usepackage{amsmath,amssymb}
\begin{document}

\section*{Problem 1}
If $a = \frac{\sqrt{5}+1}{2}$ and $b = \frac{\sqrt{5}-1}{2}$, then $a^2 - b^2$ equals:

A) $\sqrt{5}$; B) $10$; C) $5\sqrt{5}$; D) $5$; E) $\frac{1}{\sqrt{5}}$; N) Ne znam.

\subsection*{Solution}
We use the identity $a^2 - b^2 = (a+b)(a-b)$.

First, let's find $a + b$:
\[
a + b = \frac{\sqrt{5}+1}{2} + \frac{\sqrt{5}-1}{2} = \frac{\sqrt{5}+1+\sqrt{5}-1}{2} = \frac{2\sqrt{5}}{2} = \sqrt{5}
\]

Next, let's find $a - b$:
\[
a - b = \frac{\sqrt{5}+1}{2} - \frac{\sqrt{5}-1}{2} = \frac{\sqrt{5}+1-\sqrt{5}+1}{2} = \frac{2}{2} = 1
\]

Therefore:
\[
a^2 - b^2 = (a+b)(a-b) = \sqrt{5} \cdot 1 = \sqrt{5}
\]

\subsection*{Answer}
$\sqrt{5}$ (option \textbf{A}).

\end{document}