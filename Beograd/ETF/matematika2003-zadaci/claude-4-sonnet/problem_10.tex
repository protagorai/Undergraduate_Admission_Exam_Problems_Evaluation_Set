\documentclass[12pt]{article}
\usepackage[margin=1in]{geometry}
\usepackage{amsmath,amssymb}
\begin{document}

\section*{Problem 10}
The equation $x^3 + x^2 + ax + b = 0$ $(a, b \in \mathbb{R})$ has solutions $1 - \sqrt{2}i$ and $1 + \sqrt{2}$. Find all solutions:

A) $-2$; B) $-3$; C) $2$; D) $0$; E) $3$; N) Ne znam.

\subsection*{Solution}
Since the polynomial has real coefficients and $1 - \sqrt{2}i$ is a solution, its complex conjugate $1 + \sqrt{2}i$ must also be a solution.

So the three solutions are:
- $x_1 = 1 - \sqrt{2}i$
- $x_2 = 1 + \sqrt{2}i$  
- $x_3 = 1 + \sqrt{2}$ (given)

Using Vieta's formulas for $x^3 + x^2 + ax + b = 0$:
- Sum of roots: $x_1 + x_2 + x_3 = -1$
- Sum of products taken two at a time: $x_1x_2 + x_1x_3 + x_2x_3 = a$
- Product of all roots: $x_1x_2x_3 = -b$

Let's calculate:
$x_1 + x_2 + x_3 = (1 - \sqrt{2}i) + (1 + \sqrt{2}i) + (1 + \sqrt{2}) = 3 + \sqrt{2}$

But this should equal $-1$, so:
$3 + \sqrt{2} = -1$

This is impossible since $3 + \sqrt{2} > 0$.

Let me reconsider. The problem asks to "find all solutions" and gives options that are single numbers. This suggests we need to find the third root.

We know two roots: $1 - \sqrt{2}i$ and $1 + \sqrt{2}$.
Since complex roots come in conjugate pairs for real polynomials, if $1 - \sqrt{2}i$ is a root, then $1 + \sqrt{2}i$ must also be a root.

Wait, let me re-read the problem. It says the solutions are $1 - \sqrt{2}i$ and $1 + \sqrt{2}$. 

Actually, let me assume there's a typo and the given roots are $1 - \sqrt{2}i$ and $1 + \sqrt{2}i$ (complex conjugates), and we need to find the third root.

Sum of roots = $-1$
$(1 - \sqrt{2}i) + (1 + \sqrt{2}i) + x_3 = -1$
$2 + x_3 = -1$
$x_3 = -3$

\subsection*{Answer}
$-3$ (option \textbf{B}).

\end{document}