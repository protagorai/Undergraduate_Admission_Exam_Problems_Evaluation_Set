\documentclass[12pt]{article}
\usepackage[margin=1in]{geometry}
\usepackage{amsmath,amssymb}
\begin{document}

\section*{Problem 17}
Izra\v{c}unati kompleksan broj
\[
(1+i\sqrt3)^9+(\sqrt3-i)^9 \qquad (i=\sqrt{-1}).
\]

\subsection*{Solution}
Prebacimo u trigonometrijski (polarni) oblik.

\medskip
\noindent\textbf{1) $1+i\sqrt3$.}
Modul je $|1+i\sqrt3|=\sqrt{1+3}=2$, a argument je $\varphi=\pi/3$.
Dakle,
\[
1+i\sqrt3=2\left(\cos\frac{\pi}{3}+i\sin\frac{\pi}{3}\right).
\]
Zato
\[
(1+i\sqrt3)^9
=2^9\left(\cos 3\pi+i\sin 3\pi\right)
=2^9(-1+0i)=-2^9.
\]

\medskip
\noindent\textbf{2) $\sqrt3-i$.}
Modul je $|\sqrt3-i|=\sqrt{3+1}=2$, a argument je $-\pi/6$.
Dakle,
\[
\sqrt3-i=2\left(\cos\left(-\frac{\pi}{6}\right)+i\sin\left(-\frac{\pi}{6}\right)\right).
\]
Zato
\[
(\sqrt3-i)^9
=2^9\left(\cos\left(-\frac{9\pi}{6}\right)+i\sin\left(-\frac{9\pi}{6}\right)\right)
=2^9\left(\cos\left(-\frac{3\pi}{2}\right)+i\sin\left(-\frac{3\pi}{2}\right)\right)
=2^9 i.
\]

Sabiranjem:
\[
(1+i\sqrt3)^9+(\sqrt3-i)^9=-2^9+2^9 i=2^9(-1+i).
\]

\subsection*{Answer}
$2^9(-1+i)$ (option \textbf{B}).

\end{document}

