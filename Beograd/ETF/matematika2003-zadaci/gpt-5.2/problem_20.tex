\documentclass[12pt]{article}
\usepackage[margin=1in]{geometry}
\usepackage{amsmath,amssymb}
\begin{document}

\section*{Problem 20}
Brojevi $2$, $\sqrt6$, $\dfrac{9}{2}$ su \v{c}lanovi:

\subsection*{Solution}
Treba proveriti da li postoji geometrijska progresija koja sadr\v{z}i ova tri broja kao neke svoje \v{c}lanove (ne moraju biti uzastopni).

Uzmimo geometrijsku progresiju sa
\[
a_1=2,\qquad r=\frac{\sqrt6}{2}.
\]
Tada je
\[
a_2=a_1r=2\cdot\frac{\sqrt6}{2}=\sqrt6,
\]
pa se drugi broj pojavljuje kao drugi \v{c}lan.

Proverimo da li se $\dfrac{9}{2}$ pojavljuje kao neki \v{c}lan:
\[
a_n=a_1r^{n-1}=2\left(\frac{\sqrt6}{2}\right)^{n-1}.
\]
Tra\v{z}imo $n$ takvo da je $a_n=\dfrac{9}{2}$, tj.
\[
2\left(\frac{\sqrt6}{2}\right)^{n-1}=\frac{9}{2}
\quad\Longleftrightarrow\quad
\left(\frac{\sqrt6}{2}\right)^{n-1}=\frac{9}{4}.
\]
Prime\v{c}ujemo da
\[
\left(\frac{\sqrt6}{2}\right)^4=\frac{(\sqrt6)^4}{2^4}=\frac{6^2}{16}=\frac{36}{16}=\frac{9}{4}.
\]
Dakle va\v{z}i $n-1=4$, tj.\ $n=5$, i stvarno je
\[
a_5=2\left(\frac{\sqrt6}{2}\right)^4=2\cdot\frac{9}{4}=\frac{9}{2}.
\]
Sva tri broja su \v{c}lanovi jedne geometrijske progresije.

\subsection*{Answer}
Geometrijske progresije (option \textbf{C}).

\end{document}

