\documentclass[12pt]{article}
\usepackage[margin=1in]{geometry}
\usepackage{amsmath,amssymb}
\begin{document}

\section*{Problem 9}
Ako je prava $y=kx+n$ zajedni\v{c}ka tangenta kruga $x^2+y^2=4$ i elipse $2x^2+5y^2=10$, na\'ci $k^2+n^2$.

\subsection*{Solution}
\textbf{Tangenta na krug.}
Za krug $x^2+y^2=4$ polupre\v{c}nik je $2$, pa je rastojanje od ishodi\v{s}ta do prave $y=kx+n$ jednako $2$:
\[
\frac{|n|}{\sqrt{k^2+1}}=2 \quad\Longrightarrow\quad n^2=4(k^2+1).
\]

\textbf{Tangenta na elipsu.}
Elipsu prepi\v{s}imo kao
\[
2x^2+5y^2=10 \quad\Longleftrightarrow\quad \frac{x^2}{5}+\frac{y^2}{2}=1,
\]
pa je $a^2=5$, $b^2=2$. Za elipsu $\frac{x^2}{a^2}+\frac{y^2}{b^2}=1$ va\v{z}i uslov tangencije za pravu $y=kx+n$:
\[
n^2=a^2k^2+b^2.
\]
Ovde je zato
\[
n^2=5k^2+2.
\]

Izjedna\v{c}avanjem dve formule za $n^2$:
\[
4(k^2+1)=5k^2+2 \;\Longrightarrow\; k^2=2.
\]
Tada je
\[
n^2=4(k^2+1)=4(3)=12,
\]
pa je
\[
k^2+n^2=2+12=14.
\]

\subsection*{Answer}
$14$ (option \textbf{E}).

\end{document}

