\documentclass[12pt]{article}
\usepackage[margin=1in]{geometry}
\usepackage{amsmath,amssymb}
\begin{document}

\section*{Problem 8}
Date su funkcije
\[
f_1(x)=1,\qquad
f_2(x)=\frac{|\sin x|}{\sqrt{1-\cos^2 x}},\qquad
f_3(x)=\frac{|\cos x|}{\sqrt{1-\sin^2 x}},\qquad
f_4(x)=\tan x\cdot \cot x.
\]
Odrediti ta\v{c}an iskaz o njihovoj jednakosti.

\subsection*{Solution}
Sre\dj ivanjem dobijamo:
\[
\sqrt{1-\cos^2 x}=\sqrt{\sin^2 x}=|\sin x|\quad\Rightarrow\quad
f_2(x)=\frac{|\sin x|}{|\sin x|}=1,
\]
ali samo tamo gde je definisana, tj.\ za $\sin x\neq 0$ (ina\v{c}e se deli nulom).

Sli\v{c}no,
\[
\sqrt{1-\sin^2 x}=\sqrt{\cos^2 x}=|\cos x|\quad\Rightarrow\quad
f_3(x)=\frac{|\cos x|}{|\cos x|}=1,
\]
ali samo za $\cos x\neq 0$.

Za $f_4$ va\v{z}i
\[
f_4(x)=\tan x\cdot \cot x=\frac{\sin x}{\cos x}\cdot\frac{\cos x}{\sin x}=1,
\]
ali samo kada su i $\sin x\neq 0$ i $\cos x\neq 0$.

Dakle, sve tri funkcije $f_2,f_3,f_4$ imaju vrednost $1$ tamo gde su definisane, ali imaju razli\v{c}ite prirodne domene:
\[
D_{f_1}=\mathbb{R},\quad
D_{f_2}=\mathbb{R}\setminus\{k\pi\},\quad
D_{f_3}=\mathbb{R}\setminus\left\{\frac{\pi}{2}+k\pi\right\},\quad
D_{f_4}=\mathbb{R}\setminus\left(\{k\pi\}\cup\left\{\frac{\pi}{2}+k\pi\right\}\right).
\]
Po\v{s}to jednake funkcije moraju imati isti domen, me\dj u datim funkcijama nema me\dj usobno jednakih.

\subsection*{Answer}
Me\dj u datim funkcijama nema me\dj usobno jednakih (option \textbf{A}).

\end{document}

