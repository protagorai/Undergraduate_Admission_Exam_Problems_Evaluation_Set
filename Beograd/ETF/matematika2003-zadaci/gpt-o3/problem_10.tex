\documentclass[12pt]{article}
\usepackage[margin=1in]{geometry}
\usepackage{amsmath,amssymb}
\begin{document}

\section*{Problem 10}
Полином
\[
 p(x)=x^{3}+x^{2}+ax+b
\]
има два реална корена $x_{1}=1-\sqrt{2}$ и $x_{2}=1+\sqrt{2}$.  Тражи се производ сва три корена.

\subsection*{Solution}
Ако је $x_{3}$ трећи корен, важи
\[
 (x- x_{1})(x-x_{2})(x-x_{3})=x^{3}+x^{2}+ax+b.
\]
По Вијетовим формулама
\[
 x_{1}+x_{2}+x_{3}=-1,\qquad x_{1}x_{2}+x_{1}x_{3}+x_{2}x_{3}=a,\qquad x_{1}x_{2}x_{3}=-b.
\]
Прво помножимо дата два корена:
\[
 x_{1}x_{2}=(1-\sqrt2)(1+\sqrt2)=1-2=-1.
\]
Суму $x_{1}+x_{2}$ налазимо:
\[
 x_{1}+x_{2}=2.
\]
Из прве Вијетове формуле добијамо
\[
 x_{3}=-1-(x_{1}+x_{2})=-1-2=-3.
\]
Производ сва три корена је
\[
 x_{1}x_{2}x_{3}=(-1)\cdot(-3)=3.
\]

\subsection*{Answer}
$3$ (опција \textbf{E}).

\end{document}