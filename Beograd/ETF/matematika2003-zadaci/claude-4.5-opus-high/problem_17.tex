\documentclass[12pt]{article}
\usepackage[margin=1in]{geometry}
\usepackage{amsmath,amssymb}
\begin{document}

\section*{Problem 17}
Kompleksan broj $(1 + i\sqrt{3})^9 + (\sqrt{3} - i)^9$ $(i = \sqrt{-1})$ jednak je:

\textbf{A)} $2^9(1+i)$; \quad \textbf{B)} $2^9(-1+i)$; \quad \textbf{C)} $2^9(1-i)$; \quad \textbf{D)} $2^9$; \quad \textbf{E)} $-2^9 i$

\subsection*{Solution}
Pretvorimo kompleksne brojeve u trigonometrijski oblik.

\textbf{Za $1 + i\sqrt{3}$:}
\[
|1 + i\sqrt{3}| = \sqrt{1 + 3} = 2
\]
\[
\arg(1 + i\sqrt{3}) = \arctan\frac{\sqrt{3}}{1} = \frac{\pi}{3}
\]
\[
1 + i\sqrt{3} = 2\left(\cos\frac{\pi}{3} + i\sin\frac{\pi}{3}\right)
\]

\textbf{Za $\sqrt{3} - i$:}
\[
|\sqrt{3} - i| = \sqrt{3 + 1} = 2
\]
\[
\arg(\sqrt{3} - i) = -\arctan\frac{1}{\sqrt{3}} = -\frac{\pi}{6}
\]
\[
\sqrt{3} - i = 2\left(\cos\left(-\frac{\pi}{6}\right) + i\sin\left(-\frac{\pi}{6}\right)\right)
\]

Koristimo De Moivreovu teoremu:
\[
(1 + i\sqrt{3})^9 = 2^9\left(\cos\frac{9\pi}{3} + i\sin\frac{9\pi}{3}\right) = 2^9(\cos 3\pi + i\sin 3\pi) = 2^9(-1 + 0) = -2^9
\]

\[
(\sqrt{3} - i)^9 = 2^9\left(\cos\left(-\frac{9\pi}{6}\right) + i\sin\left(-\frac{9\pi}{6}\right)\right) = 2^9\left(\cos\left(-\frac{3\pi}{2}\right) + i\sin\left(-\frac{3\pi}{2}\right)\right)
\]
\[
= 2^9(0 + i \cdot 1) = 2^9 i
\]

Zbir:
\[
(1 + i\sqrt{3})^9 + (\sqrt{3} - i)^9 = -2^9 + 2^9 i = 2^9(-1 + i)
\]

\subsection*{Answer}
$2^9(-1+i)$ (option \textbf{B}).

\end{document}
