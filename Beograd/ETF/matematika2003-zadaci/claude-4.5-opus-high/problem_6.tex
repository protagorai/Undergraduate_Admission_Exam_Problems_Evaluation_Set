\documentclass[12pt]{article}
\usepackage[margin=1in]{geometry}
\usepackage{amsmath,amssymb}
\begin{document}

\section*{Problem 6}
Površina trougla, čiji su uglovi $\alpha$, $\beta$ i $\gamma$ a $R$ poluprečnik opisanog kruga, jednaka je:

\textbf{A)} $2R^2\sin\alpha\sin\beta\sin\gamma$; \quad \textbf{B)} $\frac{1}{2}R^2\sin\alpha\sin\beta\sin\gamma$; \quad \textbf{C)} $\frac{1}{2}R^2\cos\alpha\cos\beta\cos\gamma$; \quad \textbf{D)} $R^2\sin\alpha\cos\beta\cos\gamma$; \quad \textbf{E)} $2R^2\cos\alpha\sin(\beta+\gamma)$

\subsection*{Solution}
Za trougao sa stranicama $a$, $b$, $c$ i poluprečnikom opisanog kruga $R$, važi sinusna teorema:
\[
\frac{a}{\sin\alpha} = \frac{b}{\sin\beta} = \frac{c}{\sin\gamma} = 2R
\]

Odatle:
\[
a = 2R\sin\alpha, \quad b = 2R\sin\beta, \quad c = 2R\sin\gamma
\]

Površina trougla je:
\[
P = \frac{1}{2}ab\sin\gamma
\]

Zamenom:
\[
P = \frac{1}{2}(2R\sin\alpha)(2R\sin\beta)\sin\gamma
\]
\[
P = \frac{1}{2} \cdot 4R^2\sin\alpha\sin\beta\sin\gamma
\]
\[
P = 2R^2\sin\alpha\sin\beta\sin\gamma
\]

\subsection*{Answer}
$2R^2\sin\alpha\sin\beta\sin\gamma$ (option \textbf{A}).

\end{document}
