\documentclass[12pt]{article}
\usepackage[margin=1in]{geometry}
\usepackage{amsmath,amssymb}
\begin{document}

\section*{Problem 9}
Ako je prava $y = kx + n$ zajednička tangenta kruga $x^2 + y^2 = 4$ i elipse $2x^2 + 5y^2 = 10$, tada je $k^2 + n^2$ jednako:

\textbf{A)} 4; \quad \textbf{B)} 7; \quad \textbf{C)} 6; \quad \textbf{D)} 5; \quad \textbf{E)} 14

\subsection*{Solution}
\textbf{Uslov tangentnosti za krug $x^2 + y^2 = 4$:}

Rastojanje centra $(0,0)$ od prave $kx - y + n = 0$ mora biti jednako poluprečniku $r = 2$:
\[
\frac{|n|}{\sqrt{k^2 + 1}} = 2 \Rightarrow n^2 = 4(k^2 + 1) = 4k^2 + 4
\]

\textbf{Uslov tangentnosti za elipsu $\frac{x^2}{5} + \frac{y^2}{2} = 1$:}

Za elipsu $\frac{x^2}{a^2} + \frac{y^2}{b^2} = 1$, tangenta oblika $y = kx + n$ mora zadovoljavati:
\[
n^2 = a^2k^2 + b^2
\]

Ovde je $a^2 = 5$ i $b^2 = 2$, pa:
\[
n^2 = 5k^2 + 2
\]

\textbf{Rešavanje sistema:}

Iz dva uslova:
\[
4k^2 + 4 = 5k^2 + 2
\]
\[
4 - 2 = 5k^2 - 4k^2
\]
\[
k^2 = 2
\]

Zamenom u $n^2 = 5k^2 + 2$:
\[
n^2 = 5 \cdot 2 + 2 = 12
\]

Dakle:
\[
k^2 + n^2 = 2 + 12 = 14
\]

\subsection*{Answer}
$14$ (option \textbf{E}).

\end{document}
