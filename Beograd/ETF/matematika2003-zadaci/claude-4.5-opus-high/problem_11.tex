\documentclass[12pt]{article}
\usepackage[margin=1in]{geometry}
\usepackage{amsmath,amssymb}
\begin{document}

\section*{Problem 11}
Visina valjka maksimalne zapremine upisanog u loptu poluprečnika $\sqrt{3}$ jednaka je:

\textbf{A)} 1; \quad \textbf{B)} 2; \quad \textbf{C)} $2\sqrt{2}$; \quad \textbf{D)} $\sqrt{3}$; \quad \textbf{E)} $\frac{3}{2}$

\subsection*{Solution}
Neka je $R = \sqrt{3}$ poluprečnik lopte, $r$ poluprečnik osnove valjka i $h$ visina valjka.

Iz geometrije upisanog valjka u loptu (centar lopte je u centru valjka):
\[
r^2 + \left(\frac{h}{2}\right)^2 = R^2 = 3
\]
\[
r^2 = 3 - \frac{h^2}{4}
\]

Zapremina valjka:
\[
V = \pi r^2 h = \pi \left(3 - \frac{h^2}{4}\right) h = \pi \left(3h - \frac{h^3}{4}\right)
\]

Da bismo našli maksimum, izjednačavamo izvod sa nulom:
\[
\frac{dV}{dh} = \pi \left(3 - \frac{3h^2}{4}\right) = 0
\]
\[
3 - \frac{3h^2}{4} = 0
\]
\[
3 = \frac{3h^2}{4}
\]
\[
h^2 = 4
\]
\[
h = 2 \quad (h > 0)
\]

Proverimo da li je ovo maksimum:
\[
\frac{d^2V}{dh^2} = -\frac{6\pi h}{4} = -\frac{3\pi h}{2} < 0 \text{ za } h > 0
\]

Dakle, $h = 2$ daje maksimalnu zapreminu.

\subsection*{Answer}
$2$ (option \textbf{B}).

\end{document}
