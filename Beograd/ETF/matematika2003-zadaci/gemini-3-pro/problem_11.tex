\documentclass[12pt]{article}
\usepackage[margin=1in]{geometry}
\usepackage{amsmath,amssymb}
\usepackage[utf8]{inputenc}
\begin{document}
\section*{Problem 11}
Visina valjka maksimalne zapremine upisanog u loptu poluprečnika $\sqrt{3}$ jednaka je:

A) 1; B) 2; C) $2\sqrt{2}$; D) $\sqrt{3}$; E) $\frac{3}{2}$; N) Ne znam.

\subsection*{Solution}
Neka je $R=\sqrt{3}$ poluprečnik lopte, $r$ poluprečnik osnove valjka i $H$ visina valjka.
Poprečni presek valjka upisanog u loptu je pravougaonik stranica $2r$ i $H$ upisan u krug poluprečnika $R$.
Važi veza:
\[ (2r)^2 + H^2 = (2R)^2 \]
\[ 4r^2 + H^2 = 4R^2 \quad \Rightarrow \quad r^2 = R^2 - \frac{H^2}{4} \]
Zapremina valjka je $V = r^2 \pi H$.
Zamenimo $r^2$:
\[ V(H) = \pi \left( R^2 - \frac{H^2}{4} \right) H = \pi \left( R^2 H - \frac{H^3}{4} \right) \]
Da bismo našli maksimum, tražimo izvod po $H$:
\[ V'(H) = \pi \left( R^2 - \frac{3H^2}{4} \right) \]
Izjednačimo izvod sa nulom:
\[ R^2 - \frac{3H^2}{4} = 0 \quad \Rightarrow \quad H^2 = \frac{4R^2}{3} \]
\[ H = \frac{2R}{\sqrt{3}} \]
Zamenom $R=\sqrt{3}$:
\[ H = \frac{2\sqrt{3}}{\sqrt{3}} = 2 \]

\subsection*{Answer}
2 (option \textbf{B}).
\end{document}