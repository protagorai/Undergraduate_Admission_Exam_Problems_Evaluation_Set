\documentclass[12pt]{article}
\usepackage[margin=1in]{geometry}
\usepackage{amsmath,amssymb}
\usepackage[utf8]{inputenc}
\begin{document}
\section*{Problem 6}
Površina trougla, čiji su uglovi $\alpha, \beta$ i $\gamma$ a $R$ poluprečnik opisanog kruga, jednaka je:

A) $2R^2 \sin \alpha \sin \beta \sin \gamma$; B) $\frac{1}{2}R^2 \sin \alpha \sin \beta \sin \gamma$; C) $\frac{1}{2}R^2 \cos \alpha \cos \beta \cos \gamma$; D) $R^2 \sin \alpha \cos \beta \cos \gamma$; E) $2R^2 \cos \alpha \sin (\beta+\gamma)$; N) Ne znam.

\subsection*{Solution}
Površina trougla se može izračunati formulom:
\[ P = \frac{abc}{4R} \]
Koristeći sinusnu teoremu, stranice možemo izraziti preko poluprečnika opisanog kruga $R$ i uglova:
\[ a = 2R \sin \alpha, \quad b = 2R \sin \beta, \quad c = 2R \sin \gamma \]
Zamenom u formulu za površinu:
\[ P = \frac{(2R \sin \alpha)(2R \sin \beta)(2R \sin \gamma)}{4R} \]
\[ P = \frac{8R^3 \sin \alpha \sin \beta \sin \gamma}{4R} \]
\[ P = 2R^2 \sin \alpha \sin \beta \sin \gamma \]

\subsection*{Answer}
$2R^2 \sin \alpha \sin \beta \sin \gamma$ (option \textbf{A}).
\end{document}