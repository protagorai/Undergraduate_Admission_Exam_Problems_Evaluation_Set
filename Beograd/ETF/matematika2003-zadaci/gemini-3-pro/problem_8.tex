\documentclass[12pt]{article}
\usepackage[margin=1in]{geometry}
\usepackage{amsmath,amssymb}
\usepackage[utf8]{inputenc}
\begin{document}
\section*{Problem 8}
Date su funkcije $f_1(x) = 1$, $f_2(x) = \frac{|\sin x|}{\sqrt{1-\cos^2 x}}$, $f_3(x) = \frac{|\cos x|}{\sqrt{1-\sin^2 x}}$, $f_4(x) = \tan x \cdot \cot x$. Tačan je iskaz:

A) Među datim funkcijama nema međusobno jednakih; B) Sve funkcije su međusobno jednake; C) $f_1 \neq f_2 = f_3$; D) $f_1 = f_4 \neq f_3$; E) $f_2 \neq f_3 = f_4 \neq f_1$; N) Ne znam.

\subsection*{Solution}
Analizirajmo svaku funkciju i njen domen.
$f_1(x) = 1$. Domen $D_1 = \mathbb{R}$.

$f_2(x) = \frac{|\sin x|}{\sqrt{1-\cos^2 x}} = \frac{|\sin x|}{\sqrt{\sin^2 x}} = \frac{|\sin x|}{|\sin x|} = 1$.
Definisana je kada $\sin x \neq 0$, tj. $x \neq k\pi, k \in \mathbb{Z}$.
Domen $D_2 = \mathbb{R} \setminus \{k\pi\}$.

$f_3(x) = \frac{|\cos x|}{\sqrt{1-\sin^2 x}} = \frac{|\cos x|}{\sqrt{\cos^2 x}} = \frac{|\cos x|}{|\cos x|} = 1$.
Definisana je kada $\cos x \neq 0$, tj. $x \neq \frac{\pi}{2} + k\pi, k \in \mathbb{Z}$.
Domen $D_3 = \mathbb{R} \setminus \{\frac{\pi}{2} + k\pi\}$.

$f_4(x) = \tan x \cdot \cot x = \tan x \cdot \frac{1}{\tan x} = 1$.
Definisana je kada su definisani i tangens i kotangens, tj. $\sin x \neq 0$ i $\cos x \neq 0$.
To znači $x \neq \frac{k\pi}{2}, k \in \mathbb{Z}$.
Domen $D_4 = \mathbb{R} \setminus \{\frac{k\pi}{2}\}$.

Dve funkcije su jednake ako imaju isti domen i iste vrednosti na tom domenu.
Vrednosti su svuda 1, ali su domeni različiti ($D_1 \neq D_2 \neq D_3 \neq D_4$).
Dakle, nema međusobno jednakih funkcija.

\subsection*{Answer}
Među datim funkcijama nema međusobno jednakih (option \textbf{A}).
\end{document}