\begin{solution}
\textbf{Zadatak 8.}

Имамо израз
\[
E = 3-\cos^{2}\alpha-\cos^{2}\beta-\cos^{2}\gamma
\]
за углове троугла ($\alpha+\beta+\gamma=\pi$).  Користећи идентитет
\[
\cos^{2}\theta = 1-\sin^{2}\theta
\]
dобијамо
\[
E = \sin^{2}\alpha+\sin^{2}\beta+\sin^{2}\gamma.
\]
Са формулом синуса преко описаног круга $a=2R\sin\alpha,\;b=2R\sin\beta,\;c=2R\sin\gamma$ добија се  
\[
E = \frac{a^{2}+b^{2}+c^{2}}{4R^{2}},
\]
што уопштено \\није константно крат R^{2}. Пошто ниједан од понуђених израза $(A)$–$(E)$ не одговара за све троуглове, изабрали бисмо \boxed{N\;(Ne\;znam)}.
\end{solution}
