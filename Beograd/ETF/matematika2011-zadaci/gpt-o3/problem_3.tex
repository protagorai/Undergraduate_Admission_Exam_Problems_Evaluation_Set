\begin{solution}
\textbf{Zadatak 3.}

Neka je $|x|>2$. Označimo
\[
 t = \sqrt{x^{2}-4}\quad (t>0).
\]
Posmatrani izraz je
\[
S = \frac{x+2+t}{x+2-t}+\frac{x+2-t}{x+2+t}=a+\frac1a,\qquad a:=\frac{x+2+t}{x+2-t}.
\]
Пошто је $a>0$, важи стандардна формула
\[
a+\frac1a = \frac{a^{2}+1}{a}= \frac{(x+2+t)^{2}+(x+2-t)^{2}}{(x+2+t)(x+2-t)}.
\]
Израчунајмо:
\begin{align*}
(x+2+t)^{2}+(x+2-t)^{2} &= (x+2)^{2}+2(x+2)t+t^{2}+(x+2)^{2}-2(x+2)t+t^{2}\\
&=2\bigl[(x+2)^{2}+t^{2}\bigr].
\\
(x+2+t)(x+2-t) &= (x+2)^{2}-t^{2}.
\end{align*}
Дакле
\[
S = 2\,\frac{(x+2)^{2}+t^{2}}{(x+2)^{2}-t^{2}}.
\]
Сада заменимо $t^{2}=x^{2}-4$ и $(x+2)^{2}=x^{2}+4x+4$:
\begin{align*}
S &=2\frac{(x^{2}+4x+4)+(x^{2}-4)}{(x^{2}+4x+4)-(x^{2}-4)}
=2\frac{2x^{2}+4x}{4x+8}=\frac{2x(x+2)}{2(x+2)}=x.
\end{align*}
\[
\boxed{S=x}
\]
Што одговара понуди (C).
\end{solution}
