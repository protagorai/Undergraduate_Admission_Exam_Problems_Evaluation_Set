\documentclass[12pt]{article}
\usepackage[margin=1in]{geometry}
\usepackage{amsmath,amssymb}
\begin{document}

\section*{Problem 3}
Ako je $|x| > 2, x \in \mathbb{R}$ tada je izraz $\frac{x + 2 + \sqrt{x^2 - 4}}{x + 2 - \sqrt{x^2 - 4}} + \frac{x + 2 - \sqrt{x^2 - 4}}{x + 2 + \sqrt{x^2 - 4}}$ identički jednak:

(A) $4$ \quad (B) $-4$ \quad (C) $x$ \quad (D) $2x$ \quad (E) $4x$ \quad (N) Ne znam

\subsection*{Solution}
Let $a = x + 2$ and $b = \sqrt{x^2 - 4}$. Then our expression becomes:
$$\frac{a + b}{a - b} + \frac{a - b}{a + b}$$

To add these fractions, we find a common denominator:
\begin{align}
\frac{a + b}{a - b} + \frac{a - b}{a + b} &= \frac{(a + b)^2 + (a - b)^2}{(a - b)(a + b)}\\
&= \frac{(a + b)^2 + (a - b)^2}{a^2 - b^2}
\end{align}

Expanding the numerator:
\begin{align}
(a + b)^2 + (a - b)^2 &= a^2 + 2ab + b^2 + a^2 - 2ab + b^2\\
&= 2a^2 + 2b^2\\
&= 2(a^2 + b^2)
\end{align}

So our expression becomes:
$$\frac{2(a^2 + b^2)}{a^2 - b^2}$$

Substituting back $a = x + 2$ and $b = \sqrt{x^2 - 4}$:
\begin{align}
a^2 + b^2 &= (x + 2)^2 + (\sqrt{x^2 - 4})^2\\
&= x^2 + 4x + 4 + x^2 - 4\\
&= 2x^2 + 4x
\end{align}

\begin{align}
a^2 - b^2 &= (x + 2)^2 - (\sqrt{x^2 - 4})^2\\
&= x^2 + 4x + 4 - (x^2 - 4)\\
&= x^2 + 4x + 4 - x^2 + 4\\
&= 4x + 8\\
&= 4(x + 2)
\end{align}

Therefore:
\begin{align}
\frac{2(a^2 + b^2)}{a^2 - b^2} &= \frac{2(2x^2 + 4x)}{4(x + 2)}\\
&= \frac{4x(x + 2)}{4(x + 2)}\\
&= \frac{4x(x + 2)}{4(x + 2)}\\
&= x
\end{align}

\subsection*{Answer}
(C) $x$

\end{document}