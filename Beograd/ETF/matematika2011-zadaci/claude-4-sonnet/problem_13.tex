\documentclass[12pt]{article}
\usepackage[margin=1in]{geometry}
\usepackage{amsmath,amssymb}
\begin{document}

\section*{Problem 13}
Ostatak pri deljenju polinoma $x^{243} + x^{81} + x^{27} + x^9 + x^3 + x$ polinomom $x^2 - 1$ iznosi:

(A) $0$ \quad (B) $1$ \quad (C) $2x$ \quad (D) $4x$ \quad (E) $6x$ \quad (N) Ne znam

\subsection*{Solution}
When dividing a polynomial $P(x)$ by $x^2 - 1$, the remainder is a polynomial of degree at most 1, so it has the form $ax + b$.

We can write: $P(x) = Q(x)(x^2 - 1) + ax + b$

Since $x^2 - 1 = (x-1)(x+1)$, we have $x^2 \equiv 1 \pmod{x^2 - 1}$.

This means that for any even power $x^{2k} \equiv 1 \pmod{x^2 - 1}$ and for any odd power $x^{2k+1} \equiv x \pmod{x^2 - 1}$.

Let's analyze each term:
- $x^{243} = x^{2 \cdot 121 + 1} \equiv x \pmod{x^2 - 1}$
- $x^{81} = x^{2 \cdot 40 + 1} \equiv x \pmod{x^2 - 1}$
- $x^{27} = x^{2 \cdot 13 + 1} \equiv x \pmod{x^2 - 1}$
- $x^9 = x^{2 \cdot 4 + 1} \equiv x \pmod{x^2 - 1}$
- $x^3 = x^{2 \cdot 1 + 1} \equiv x \pmod{x^2 - 1}$
- $x^1 \equiv x \pmod{x^2 - 1}$

Therefore:
$$P(x) \equiv x + x + x + x + x + x = 6x \pmod{x^2 - 1}$$

So the remainder is $6x$.

We can verify this by substituting $x = 1$ and $x = -1$:

For $x = 1$:
$P(1) = 1 + 1 + 1 + 1 + 1 + 1 = 6$
Remainder at $x = 1$: $6(1) = 6$ ✓

For $x = -1$:
$P(-1) = (-1)^{243} + (-1)^{81} + (-1)^{27} + (-1)^9 + (-1)^3 + (-1) = -1 - 1 - 1 - 1 - 1 - 1 = -6$
Remainder at $x = -1$: $6(-1) = -6$ ✓

\subsection*{Answer}
(E) $6x$

\end{document}