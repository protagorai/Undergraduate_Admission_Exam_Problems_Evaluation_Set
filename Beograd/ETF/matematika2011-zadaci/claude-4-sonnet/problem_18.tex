\documentclass[12pt]{article}
\usepackage[margin=1in]{geometry}
\usepackage{amsmath,amssymb}
\begin{document}

\section*{Problem 18}
Osnova prave četvorostrane piramide je pravougaonik dijagonale $d$ i ugla $\alpha$ među dijagonalama. Ako bočne ivice obrazuju sa osnovom piramide ugao $\beta$, tada je zapremina ove piramide jednaka:

(A) $\frac{d^3}{12} \sin \alpha \operatorname{ctg} \beta$ \quad (B) $\frac{d^3}{12} \sin \alpha \operatorname{tg} \beta$ \quad (C) $\frac{d^3}{4} \sin \alpha \operatorname{tg} \beta$ \quad (D) $\frac{d^3}{12} \sin \frac{\alpha}{2} \operatorname{tg} \beta$ \quad (E) $\frac{d^3}{12} \cos \alpha \operatorname{tg} \beta$ \quad (N) Ne znam

\subsection*{Solution}
Let's set up the problem systematically.

The base is a rectangle with diagonals of length $d$ and angle $\alpha$ between them.

For a rectangle with sides $a$ and $b$:
- Diagonal length: $d = \sqrt{a^2 + b^2}$
- Area of rectangle: $S = ab$

The angle $\alpha$ between diagonals can be related to the sides. If we place the rectangle with center at origin, the diagonals make angles with the coordinate axes. The angle between diagonals is related to the rectangle dimensions.

For a rectangle with sides $a$ and $b$, if $\alpha$ is the acute angle between diagonals, then:
$$\sin \alpha = \frac{2ab}{a^2 + b^2} = \frac{2ab}{d^2}$$

Therefore: $ab = \frac{d^2 \sin \alpha}{2}$

So the area of the base is: $S = ab = \frac{d^2 \sin \alpha}{2}$

For a right pyramid where all lateral edges make the same angle $\beta$ with the base, the apex is directly above the center of the base rectangle.

The distance from the center to any vertex of the rectangle is $\frac{d}{2}$ (half the diagonal).

If the lateral edges make angle $\beta$ with the base, and the distance from center to vertex is $\frac{d}{2}$, then the height $h$ of the pyramid satisfies:
$$\tan \beta = \frac{h}{\frac{d}{2}} = \frac{2h}{d}$$

Therefore: $h = \frac{d \tan \beta}{2}$

The volume of the pyramid is:
$$V = \frac{1}{3} \cdot S \cdot h = \frac{1}{3} \cdot \frac{d^2 \sin \alpha}{2} \cdot \frac{d \tan \beta}{2} = \frac{d^3 \sin \alpha \tan \beta}{12}$$

\subsection*{Answer}
(B) $\frac{d^3}{12} \sin \alpha \operatorname{tg} \beta$

\end{document}