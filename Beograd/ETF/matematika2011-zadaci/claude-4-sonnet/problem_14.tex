\documentclass[12pt]{article}
\usepackage[margin=1in]{geometry}
\usepackage{amsmath,amssymb}
\begin{document}

\section*{Problem 14}
Na koliko načina od 2 matematičara i 8 inženjera možemo formirati petočlanu komisiju u kojoj će biti bar jedan matematičar?

(A) $196$ \quad (B) $248$ \quad (C) $70$ \quad (D) $56$ \quad (E) $140$ \quad (N) Ne znam

\subsection*{Solution}
We have 2 mathematicians and 8 engineers, and we want to form a 5-member committee with at least one mathematician.

It's easier to use the complement: total ways minus ways with no mathematicians.

Total ways to choose 5 people from 10:
$$\binom{10}{5} = \frac{10!}{5! \cdot 5!} = \frac{10 \cdot 9 \cdot 8 \cdot 7 \cdot 6}{5 \cdot 4 \cdot 3 \cdot 2 \cdot 1} = \frac{30240}{120} = 252$$

Ways to choose 5 people with no mathematicians (all from 8 engineers):
$$\binom{8}{5} = \binom{8}{3} = \frac{8!}{3! \cdot 5!} = \frac{8 \cdot 7 \cdot 6}{3 \cdot 2 \cdot 1} = \frac{336}{6} = 56$$

Therefore, ways to choose 5 people with at least one mathematician:
$$252 - 56 = 196$$

Alternatively, we can count directly:
- 1 mathematician, 4 engineers: $\binom{2}{1} \cdot \binom{8}{4} = 2 \cdot 70 = 140$
- 2 mathematicians, 3 engineers: $\binom{2}{2} \cdot \binom{8}{3} = 1 \cdot 56 = 56$

Total: $140 + 56 = 196$

\subsection*{Answer}
(A) $196$

\end{document}