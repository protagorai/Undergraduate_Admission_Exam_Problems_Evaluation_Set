\documentclass[12pt]{article}
\usepackage[margin=1in]{geometry}
\usepackage{amsmath,amssymb}
\begin{document}

\section*{Problem 18}
Osnova prave četvorostrane piramide je pravougaonik dijagonale $d$ i ugla $\alpha$ među dijagonalama. Ako bočne ivice obrazuju sa osnovom piramide ugao $\beta$, tada je zapremina ove piramide jednaka:
\[
\text{(A)}~\frac{d^3}{12}\sin\alpha\ctg\beta\quad
\text{(B)}~\frac{d^3}{12}\sin\alpha\tg\beta\quad
\text{(C)}~\frac{d^3}{4}\sin\alpha\ctg\beta\quad
\text{(D)}~\frac{d^3}{12}\sin\frac{\alpha}{2}\tg\beta\quad
\text{(E)}~\frac{d^3}{12}\cos\alpha\tg\beta.
\]

\subsection*{Solution}
Zapremina piramide je
\[
V=\frac13\,S\,h,
\]
gde je $S$ površina osnove, a $h$ visina.

\textbf{Površina osnove.} Diagonale pravougaonika imaju jednake dužine $d$ i seku se pod uglom $\alpha$. Za paralelogram važi formula
\[
S=\frac12\,d_1d_2\sin\alpha,
\]
pa ovde ($d_1=d_2=d$):
\[
S=\frac12\,d^2\sin\alpha.
\]

\textbf{Visina.} Kod prave piramide temena je iznad preseka dijagonala osnove. Projekcija bočne ivice na ravan osnove je duž od centra do temena osnove, tj.\ polovina dijagonale: $\frac d2$.

Ako bočna ivica dužine $l$ obrazuje sa osnovom ugao $\beta$, tada je njen horizontalni (ravanski) projekat $l\cos\beta$, pa
\[
l\cos\beta=\frac d2 \;\Rightarrow\; l=\frac{d}{2\cos\beta}.
\]
Visina je vertikalna komponenta: $h=l\sin\beta$, dakle
\[
h=\frac{d}{2\cos\beta}\sin\beta=\frac d2\tg\beta.
\]

Zato je
\[
V=\frac13\cdot\frac12 d^2\sin\alpha\cdot\frac d2\tg\beta
=\frac{d^3}{12}\sin\alpha\,\tg\beta.
\]

\subsection*{Answer}
$\dfrac{d^3}{12}\sin\alpha\,\tg\beta$ (option \textbf{B}).

\end{document}

