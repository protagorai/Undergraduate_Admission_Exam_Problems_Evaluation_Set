\documentclass[12pt]{article}
\usepackage[margin=1in]{geometry}
\usepackage{amsmath,amssymb}
\begin{document}

\section*{Problem 3}
Ako je $|x|>2$, $x\in\mathbb{R}$, tada je izraz
\[
\frac{x+2+\sqrt{x^2-4}}{x+2-\sqrt{x^2-4}}
+
\frac{x+2-\sqrt{x^2-4}}{x+2+\sqrt{x^2-4}}
\]
identički jednak:
\[
\text{(A)}~4\quad
\text{(B)}~-4\quad
\text{(C)}~x\quad
\text{(D)}~2x\quad
\text{(E)}~4x.
\]

\subsection*{Solution}
Označimo $a=x+2$ i $b=\sqrt{x^2-4}$ (za $|x|>2$ je $b\in\mathbb{R}$). Tada je
\[
\frac{a+b}{a-b}+\frac{a-b}{a+b}
 =\frac{(a+b)^2+(a-b)^2}{a^2-b^2}
 =\frac{2a^2+2b^2}{a^2-b^2}
 =2\cdot\frac{a^2+b^2}{a^2-b^2}.
\]
Sada:
\[
a^2=(x+2)^2=x^2+4x+4,\qquad b^2=x^2-4.
\]
Zato
\[
a^2+b^2=(x^2+4x+4)+(x^2-4)=2x^2+4x=2x(x+2),
\]
\[
a^2-b^2=(x^2+4x+4)-(x^2-4)=4x+8=4(x+2).
\]
Ubacivanjem dobijamo
\[
2\cdot\frac{2x(x+2)}{4(x+2)}=x.
\]

\subsection*{Answer}
$x$ (option \textbf{C}).

\end{document}

