\documentclass[12pt]{article}
\usepackage[margin=1in]{geometry}
\usepackage{amsmath,amssymb}
\begin{document}

\section*{Problem 19}
Sva realna rešenja jednačine
\[
\log_{2011}(2010x)=\log_{2010}(2011x)
\]
pripadaju intervalu:
\[
\text{(A)}~\left(0,\frac{1}{2011}\right]\quad
\text{(B)}~\left(\frac{1}{2011},\frac{1}{2010}\right]\quad
\text{(C)}~\left(\frac{1}{2010},1\right]\quad
\text{(D)}~\left(1,\frac{2011}{2010}\right]\quad
\text{(E)}~\left(\frac{2011}{2010},+\infty\right).
\]

\subsection*{Solution}
Uslov definisanosti: $2010x>0$ i $2011x>0$, pa je $x>0$.

Koristimo promenu osnove:
\[
\log_{2011}(2010x)=\frac{\ln(2010x)}{\ln 2011},\qquad
\log_{2010}(2011x)=\frac{\ln(2011x)}{\ln 2010}.
\]
Jednačina postaje
\[
\frac{\ln(2010x)}{\ln 2011}=\frac{\ln(2011x)}{\ln 2010}
\;\Rightarrow\;
\ln(2010x)\,\ln 2010=\ln(2011x)\,\ln 2011.
\]
Neka su $a=\ln 2010$ i $b=\ln 2011$. Tada:
\[
a(a+\ln x)=b(b+\ln x)
\;\Rightarrow\;
a^2+a\ln x=b^2+b\ln x
\;\Rightarrow\;
(a-b)\ln x=b^2-a^2=-(a-b)(a+b).
\]
Pošto je $a\neq b$, delimo sa $(a-b)$ i dobijamo
\[
\ln x=-(a+b)=-\ln 2010-\ln 2011=-\ln(2010\cdot 2011).
\]
Dakle
\[
x=\frac{1}{2010\cdot 2011}.
\]
Očigledno je $0<x<\frac{1}{2011}$, pa rešenje pripada intervalu iz opcije (A).

\subsection*{Answer}
$x=\dfrac{1}{2010\cdot 2011}\in\left(0,\dfrac{1}{2011}\right]$ (option \textbf{A}).

\end{document}

