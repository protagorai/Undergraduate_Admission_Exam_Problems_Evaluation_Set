\documentclass[12pt]{article}
\usepackage[margin=1in]{geometry}
\usepackage{amsmath,amssymb}
\begin{document}

\section*{Problem 16}
U razvoju binoma $(x+y)^n$, $(x,y\in\mathbb{R},\, n\in\mathbb{N})$ drugi član je jednak $240$, treći član $720$ a četvrti $1080$. Tada je zbir $x+y+n$ jednak:
\[
\text{(A)}~11\quad
\text{(B)}~9\quad
\text{(C)}~10\quad
\text{(D)}~25\quad
\text{(E)}~280.
\]

\subsection*{Solution}
U razvoju $(x+y)^n$ važi:
\[
T_2=\binom{n}{1}x^{n-1}y=nx^{n-1}y=240,
\]
\[
T_3=\binom{n}{2}x^{n-2}y^2=720,\qquad
T_4=\binom{n}{3}x^{n-3}y^3=1080.
\]
Iz odnosa:
\[
\frac{T_3}{T_2}=\frac{\binom{n}{2}x^{n-2}y^2}{nx^{n-1}y}
=\frac{n(n-1)}{2n}\cdot\frac{y}{x}
=\frac{n-1}{2}\cdot\frac{y}{x}
=\frac{720}{240}=3,
\]
pa je
\[
\frac{y}{x}=\frac{6}{n-1}.
\]
Takođe,
\[
\frac{T_4}{T_3}
=\frac{\binom{n}{3}x^{n-3}y^3}{\binom{n}{2}x^{n-2}y^2}
=\frac{n-2}{3}\cdot\frac{y}{x}
=\frac{1080}{720}=\frac32,
\]
odakle
\[
\frac{y}{x}=\frac{9}{2(n-2)}.
\]
Izjednačimo:
\[
\frac{6}{n-1}=\frac{9}{2(n-2)}
\;\Rightarrow\;
12(n-2)=9(n-1)
\;\Rightarrow\;
3n=15
\;\Rightarrow\;
n=5.
\]
Tada je $\dfrac{y}{x}=\dfrac{6}{4}=\dfrac32$, tj.\ $y=\frac32x$.
Iz $T_2=240$:
\[
5x^4y=240
\;\Rightarrow\;
5x^4\cdot \frac32x=240
\;\Rightarrow\;
\frac{15}{2}x^5=240
\;\Rightarrow\;
x^5=32
\;\Rightarrow\;
x=2,
\]
pa je $y=3$. Zato je
\[
x+y+n=2+3+5=10.
\]

\subsection*{Answer}
$10$ (option \textbf{C}).

\end{document}

