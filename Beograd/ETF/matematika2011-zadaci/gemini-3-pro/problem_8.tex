\documentclass[12pt]{article}
\usepackage[margin=1in]{geometry}
\usepackage{amsmath,amssymb}
\usepackage[utf8]{inputenc}
\usepackage[T1]{fontenc}

\begin{document}

\section*{Problem 8}
Neka su u proizvoljnom trouglu $\alpha, \beta$ i $\gamma$ uglovi, $a, b$ i $c$ dužine stranica naspram datih uglova i $R$ poluprečnik opisanog kruga, tada je $\frac{a^2+b^2+c^2}{3-\cos^2 \alpha - \cos^2 \beta - \cos^2 \gamma}$ jednako:
\[
(A)\; R^2 \quad (B)\; 2R^2 \quad (C)\; 3R^2 \quad (D)\; 4R^2 \quad (E)\; 5R^2
\]

\subsection*{Solution}
Na osnovu sinusne teoreme, stranice trougla se mogu izraziti kao:
\[
a = 2R \sin \alpha, \quad b = 2R \sin \beta, \quad c = 2R \sin \gamma
\]
Brojilac izraza postaje:
\[
a^2+b^2+c^2 = 4R^2 (\sin^2 \alpha + \sin^2 \beta + \sin^2 \gamma)
\]
Imenilac izraza transformišemo koristeći osnovni trigonometrijski identitet $\sin^2 x + \cos^2 x = 1 \Rightarrow 1 - \cos^2 x = \sin^2 x$:
\[
3 - \cos^2 \alpha - \cos^2 \beta - \cos^2 \gamma = (1 - \cos^2 \alpha) + (1 - \cos^2 \beta) + (1 - \cos^2 \gamma)
\]
\[
= \sin^2 \alpha + \sin^2 \beta + \sin^2 \gamma
\]
Tada je ceo izraz:
\[
\frac{4R^2 (\sin^2 \alpha + \sin^2 \beta + \sin^2 \gamma)}{\sin^2 \alpha + \sin^2 \beta + \sin^2 \gamma} = 4R^2
\]

\subsection*{Answer}
$4R^2$ (option \textbf{D}).

\end{document}
