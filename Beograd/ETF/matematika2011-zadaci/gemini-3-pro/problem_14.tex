\documentclass[12pt]{article}
\usepackage[margin=1in]{geometry}
\usepackage{amsmath,amssymb}
\usepackage[utf8]{inputenc}
\usepackage[T1]{fontenc}

\begin{document}

\section*{Problem 14}
Na koliko načina od 2 matematičara i 8 inženjera možemo formirati petočlanu komisiju u kojoj će biti bar jedan matematičar?
\[
(A)\; 196 \quad (B)\; 248 \quad (C)\; 70 \quad (D)\; 56 \quad (E)\; 140
\]

\subsection*{Solution}
Ukupan broj osoba je $2+8=10$.
Ukupan broj načina da se izabere 5 osoba od 10 je:
\[
\binom{10}{5} = \frac{10 \cdot 9 \cdot 8 \cdot 7 \cdot 6}{5 \cdot 4 \cdot 3 \cdot 2 \cdot 1} = 2 \cdot 9 \cdot 2 \cdot 7 = 252
\]
Broj komisija u kojima nema matematičara (svi su inženjeri) je izbor 5 inženjera od 8:
\[
\binom{8}{5} = \binom{8}{3} = \frac{8 \cdot 7 \cdot 6}{3 \cdot 2 \cdot 1} = 56
\]
Broj komisija sa bar jednim matematičarom je ukupan broj komisija minus broj komisija bez matematičara:
\[
252 - 56 = 196
\]

\subsection*{Answer}
196 (option \textbf{A}).

\end{document}
