\documentclass[12pt]{article}
\usepackage[margin=1in]{geometry}
\usepackage{amsmath,amssymb}
\usepackage[utf8]{inputenc}
\usepackage[T1]{fontenc}

\begin{document}

\section*{Problem 16}
U razvoju binoma $(x+y)^n$ ($x,y \in \mathbf{R}, n \in \mathbf{N}$) drugi član je jednak 240, treći član 720 a četvrti 1080. Tada je zbir $x+y+n$ jednak:
\[
(A)\; 11 \quad (B)\; 9 \quad (C)\; 10 \quad (D)\; 25 \quad (E)\; 280
\]

\subsection*{Solution}
Opšti član u razvoju $(x+y)^n$ je $T_{k+1} = \binom{n}{k} x^{n-k} y^k$.
Dati su članovi:
\[
T_2 = \binom{n}{1} x^{n-1} y = n x^{n-1} y = 240
\]
\[
T_3 = \binom{n}{2} x^{n-2} y^2 = \frac{n(n-1)}{2} x^{n-2} y^2 = 720
\]
\[
T_4 = \binom{n}{3} x^{n-3} y^3 = \frac{n(n-1)(n-2)}{6} x^{n-3} y^3 = 1080
\]
Posmatrajmo odnos uzastopnih članova:
\[
\frac{T_3}{T_2} = \frac{\frac{n(n-1)}{2} x^{n-2} y^2}{n x^{n-1} y} = \frac{n-1}{2} \cdot \frac{y}{x} = \frac{720}{240} = 3 \Rightarrow (n-1)\frac{y}{x} = 6
\]
\[
\frac{T_4}{T_3} = \frac{\frac{n(n-1)(n-2)}{6} x^{n-3} y^3}{\frac{n(n-1)}{2} x^{n-2} y^2} = \frac{n-2}{3} \cdot \frac{y}{x} = \frac{1080}{720} = \frac{3}{2} \Rightarrow (n-2)\frac{y}{x} = \frac{9}{2}
\]
Deljenjem ove dve jednačine eliminišemo $y/x$:
\[
\frac{n-1}{n-2} = \frac{6}{9/2} = \frac{12}{9} = \frac{4}{3}
\]
\[
3(n-1) = 4(n-2) \Rightarrow 3n - 3 = 4n - 8 \Rightarrow n = 5
\]
Sada nađemo $y/x$:
\[
(5-1)\frac{y}{x} = 6 \Rightarrow 4\frac{y}{x} = 6 \Rightarrow \frac{y}{x} = \frac{3}{2} \Rightarrow y = 1.5x
\]
Zamenimo u prvu jednačinu $n x^{n-1} y = 240$:
\[
5 x^4 (1.5x) = 240
\]
\[
7.5 x^5 = 240 \Rightarrow x^5 = \frac{240}{7.5} = 32 \Rightarrow x = 2
\]
Tada je $y = 1.5 \cdot 2 = 3$.
Traženi zbir je:
\[
x + y + n = 2 + 3 + 5 = 10
\]

\subsection*{Answer}
10 (option \textbf{C}).

\end{document}
