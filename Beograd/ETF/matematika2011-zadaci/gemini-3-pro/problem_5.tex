\documentclass[12pt]{article}
\usepackage[margin=1in]{geometry}
\usepackage{amsmath,amssymb}
\usepackage[utf8]{inputenc}
\usepackage[T1]{fontenc}

\begin{document}

\section*{Problem 5}
Zbir $\sin \frac{3\pi}{7} + \sin \frac{4\pi}{7}$ jednak je:
\[
(A)\; -2\sin \frac{\pi}{14} \quad (B)\; -2\cos \frac{\pi}{14} \quad (C)\; 2\sin \frac{\pi}{14} \quad (D)\; 2\cos \frac{\pi}{7} \quad (E)\; 2\cos \frac{\pi}{14}
\]

\subsection*{Solution}
Koristimo osobinu $\sin(\pi - \alpha) = \sin \alpha$.
Primetimo da je $\frac{3\pi}{7} + \frac{4\pi}{7} = \frac{7\pi}{7} = \pi$, pa je $\frac{4\pi}{7} = \pi - \frac{3\pi}{7}$.
Tada je:
\[
\sin \frac{4\pi}{7} = \sin \left(\pi - \frac{3\pi}{7}\right) = \sin \frac{3\pi}{7}
\]
Naš izraz postaje:
\[
\sin \frac{3\pi}{7} + \sin \frac{3\pi}{7} = 2 \sin \frac{3\pi}{7}
\]
Da bismo dobili rešenje u ponuđenom obliku (preko kosinusa), koristimo vezu $\sin \alpha = \cos(\frac{\pi}{2} - \alpha)$:
\[
\frac{\pi}{2} - \frac{3\pi}{7} = \frac{7\pi - 6\pi}{14} = \frac{\pi}{14}
\]
Dakle:
\[
2 \sin \frac{3\pi}{7} = 2 \cos \frac{\pi}{14}
\]

\subsection*{Answer}
$2\cos \frac{\pi}{14}$ (option \textbf{E}).

\end{document}
