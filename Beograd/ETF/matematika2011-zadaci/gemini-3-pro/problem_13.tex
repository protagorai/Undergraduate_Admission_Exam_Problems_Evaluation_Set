\documentclass[12pt]{article}
\usepackage[margin=1in]{geometry}
\usepackage{amsmath,amssymb}
\usepackage[utf8]{inputenc}
\usepackage[T1]{fontenc}

\begin{document}

\section*{Problem 13}
Ostatak pri deljenju polinoma $x^{243} + x^{81} + x^{27} + x^9 + x^3 + x$ polinomom $x^2-1$ iznosi:
\[
(A)\; 0 \quad (B)\; 1 \quad (C)\; 2x \quad (D)\; 4x \quad (E)\; 6x
\]

\subsection*{Solution}
Neka je $P(x) = x^{243} + x^{81} + x^{27} + x^9 + x^3 + x$.
Delimo polinomom $Q(x) = x^2-1 = (x-1)(x+1)$.
Ostatak $R(x)$ je polinom stepena manjeg od 2, dakle oblika $R(x) = ax+b$.
Važi $P(x) = Q(x)S(x) + R(x)$.
Za $x=1$:
\[
P(1) = 1^{243} + 1^{81} + 1^{27} + 1^9 + 1^3 + 1 = 6
\]
\[
R(1) = a(1) + b = 6
\]
Za $x=-1$:
\[
P(-1) = (-1)^{243} + (-1)^{81} + (-1)^{27} + (-1)^9 + (-1)^3 + (-1) = -1-1-1-1-1-1 = -6
\]
(svi stepeni su neparni).
\[
R(-1) = a(-1) + b = -6 \Rightarrow -a+b = -6
\]
Rešavamo sistem:
\[
a+b=6
\]
\[
-a+b=-6
\]
Sabiranjem jednačina: $2b=0 \Rightarrow b=0$.
Oduzimanjem: $2a=12 \Rightarrow a=6$.
Ostatak je $R(x) = 6x$.

\subsection*{Answer}
$6x$ (option \textbf{E}).

\end{document}
