\documentclass[12pt]{article}
\usepackage[margin=1in]{geometry}
\usepackage{amsmath,amssymb}
\usepackage[utf8]{inputenc}
\usepackage[T1]{fontenc}

\begin{document}

\section*{Problem 18}
Osnova prave četvorostrane piramide je pravougaonik dijagonale $d$ i ugla $\alpha$ među dijagonalama. Ako bočne ivice obrazuju sa osnovom piramide ugao $\beta$, tada je zapremina ove piramide jednaka:
\[
(A)\; \frac{d^3}{12} \sin \alpha \text{ctg} \beta \quad (B)\; \frac{d^3}{12} \sin \alpha \text{tg} \beta \quad (C)\; \frac{d^3}{4} \sin \alpha \text{tg} \beta \quad (D)\; \frac{d^3}{12} \sin \frac{\alpha}{2} \text{tg} \beta \quad (E)\; \frac{d^3}{12} \cos \alpha \text{tg} \beta
\]

\subsection*{Solution}
Površina osnove (pravougaonika) se može izraziti preko dijagonala i ugla između njih:
\[
B = \frac{1}{2} d_1 d_2 \sin \alpha = \frac{1}{2} d^2 \sin \alpha
\]
Visina piramide $H$ se može naći iz pravouglog trougla koji čine visina piramide, bočna ivica i polovina dijagonale osnove (jer je piramida prava, vrh se projektuje u presek dijagonala).
Kateta na osnovi je $d/2$. Ugao naspram visine je $\beta$.
\[
\tan \beta = \frac{H}{d/2} \Rightarrow H = \frac{d}{2} \tan \beta
\]
Zapremina piramide je:
\[
V = \frac{1}{3} B H = \frac{1}{3} \left( \frac{1}{2} d^2 \sin \alpha \right) \left( \frac{d}{2} \tan \beta \right)
\]
\[
V = \frac{1}{3 \cdot 2 \cdot 2} d^3 \sin \alpha \tan \beta = \frac{d^3}{12} \sin \alpha \text{tg} \beta
\]

\subsection*{Answer}
$\frac{d^3}{12} \sin \alpha \text{tg} \beta$ (option \textbf{B}).

\end{document}
