\documentclass[12pt]{article}
\usepackage[margin=1in]{geometry}
\usepackage{amsmath,amssymb}
\usepackage[utf8]{inputenc}
\usepackage[T1]{fontenc}

\begin{document}

\section*{Problem 3}
Ako je $|x| > 2$, $x \in \mathbf{R}$ tada je izraz $\frac{x+2+\sqrt{x^2-4}}{x+2-\sqrt{x^2-4}} + \frac{x+2-\sqrt{x^2-4}}{x+2+\sqrt{x^2-4}}$ identički jednak:
\[
(A)\; 4 \quad (B)\; -4 \quad (C)\; x \quad (D)\; 2x \quad (E)\; 4x
\]

\subsection*{Solution}
Neka je $A$ dati izraz. Primetimo da je zajednički imenilac razlika kvadrata.
\[
A = \frac{(x+2+\sqrt{x^2-4})^2 + (x+2-\sqrt{x^2-4})^2}{(x+2)^2 - (\sqrt{x^2-4})^2}
\]
Imenilac je:
\[
(x+2)^2 - (x^2-4) = (x^2 + 4x + 4) - x^2 + 4 = 4x + 8 = 4(x+2)
\]
Brojilac je oblika $(a+b)^2 + (a-b)^2 = 2(a^2+b^2)$, gde je $a=x+2$ i $b=\sqrt{x^2-4}$:
\[
2((x+2)^2 + (\sqrt{x^2-4})^2) = 2(x^2 + 4x + 4 + x^2 - 4) = 2(2x^2 + 4x) = 4x(x+2)
\]
Dakle, izraz postaje:
\[
A = \frac{4x(x+2)}{4(x+2)} = x
\]
Uslov $|x|>2$ obezbeđuje da je koren definisan i da imenilac nije nula (jer $x \ne -2$).

\subsection*{Answer}
$x$ (option \textbf{C}).

\end{document}
