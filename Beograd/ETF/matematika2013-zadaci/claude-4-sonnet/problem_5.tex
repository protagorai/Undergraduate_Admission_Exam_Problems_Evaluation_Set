\documentclass[12pt]{article}
\usepackage[margin=1in]{geometry}
\usepackage{amsmath,amssymb}
\begin{document}

\section*{Problem 5}
U jednakostraničnom trouglu čija je osnovica $a = 10$ cm i krak $b = 13$ cm upisan je kvadrat tako da mu dva temena leže na osnovici trougla, a druga dva na kracima. Dužina stranice kvadrata (u cm) jednaka je:

\subsection*{Solution}
Postavimo koordinatni sistem tako da je osnovica trougla na x-osi od $(0,0)$ do $(10,0)$, a vrh trougla u tački $(5,h)$ gde je $h$ visina trougla.

Visina trougla se računa iz Pitagorine teoreme:
\[
h = \sqrt{13^2 - 5^2} = \sqrt{169 - 25} = \sqrt{144} = 12 \text{ cm}
\]

Neka je stranica kvadrata $s$. Kvadrat ima temena u tačkama $(x, 0)$, $(x+s, 0)$, $(x, s)$, $(x+s, s)$.

Zbog simetrije, kvadrat je centriran, pa je $x = \frac{10-s}{2}$.

Gornja temena kvadrata moraju da leže na stranama trougla. Jednačina leve strane trougla je:
\[
y = \frac{12}{5}x
\]

Jednačina desne strane trougla je:
\[
y = \frac{12}{5}(10-x) = 24 - \frac{12}{5}x
\]

Levo gornje teme $(x, s)$ mora da leži na levoj strani:
\[
s = \frac{12}{5} \cdot \frac{10-s}{2} = \frac{12(10-s)}{10} = \frac{6(10-s)}{5}
\]
\[
5s = 6(10-s) = 60 - 6s
\]
\[
11s = 60
\]
\[
s = \frac{60}{11}
\]

\subsection*{Answer}
$\frac{60}{11}$ (option \textbf{A}).

\end{document}