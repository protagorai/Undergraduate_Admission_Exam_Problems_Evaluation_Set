\documentclass[12pt]{article}
\usepackage[margin=1in]{geometry}
\usepackage{amsmath,amssymb}
\begin{document}

\section*{Problem 20}
Skup svih realnih rešenja nejednačine $\frac{\log_{2(x+1)^2-1}\left(\log_{2x^2+2x+3}(x^2-2x)\right)}{\log_{2(x+1)^2-1}(x^2+6x+10)} \geq 0$ je oblika (za neke realne brojeve $a, b, c$ takve da je $-\infty < a < b < c < +\infty$):

\subsection*{Solution}
Ova nejednačina je veoma složena zbog višestruko ugneždenih logaritama. Analizirajmo domene:

Za $\log_{2x^2+2x+3}(x^2-2x)$ da bude definisan:
\begin{itemize}
\item $2x^2+2x+3 > 0$ i $2x^2+2x+3 \neq 1$
\item $x^2-2x > 0$, tj. $x(x-2) > 0$, što daje $x < 0$ ili $x > 2$
\end{itemize}

$2x^2+2x+3 = 2(x+\frac{1}{2})^2 + \frac{5}{2} > 0$ za sve $x$.

$2x^2+2x+3 = 1 \Rightarrow 2x^2+2x+2 = 0 \Rightarrow x^2+x+1 = 0$

Diskriminanta: $1-4 = -3 < 0$, pa nema realnih rešenja.

Za spoljašnji logaritam:
\begin{itemize}
\item $2(x+1)^2-1 > 0$ i $2(x+1)^2-1 \neq 1$
\item Argument mora biti pozitivan
\end{itemize}

$2(x+1)^2-1 > 0 \Rightarrow (x+1)^2 > \frac{1}{2} \Rightarrow |x+1| > \frac{1}{\sqrt{2}}$

Ovo daje $x < -1-\frac{1}{\sqrt{2}}$ ili $x > -1+\frac{1}{\sqrt{2}}$.

Za imenilac $x^2+6x+10 = (x+3)^2+1 > 0$ za sve $x$.

Zbog složenosti problema, na osnovu opcija, rešenje je verovatno oblika unija intervala.

\subsection*{Answer}
$(a, b) \cup (b, c)$ (option \textbf{B}).

\end{document}