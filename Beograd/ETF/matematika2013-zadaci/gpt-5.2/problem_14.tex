\documentclass[12pt]{article}
\usepackage[utf8]{inputenc}
\usepackage[T1]{fontenc}
\usepackage[serbian]{babel}
\usepackage{amsmath,amssymb}
\usepackage{geometry}
\geometry{margin=2.2cm}

\begin{document}
\section*{Zadatak 14}
U valjak prečnika osnove \(14\sqrt{3}\,\mathrm{cm}\) i visine \(20\,\mathrm{cm}\) upisana je prava trostrana prizma čija je osnova trougao \(ABC\) sa \(BC=9\,\mathrm{cm}\) i uglom naspram stranice \(AC\) jednakim \(120^\circ\). Naći zapreminu prizme.

\subsection*{Rešenje}
Poluprečnik opisane kružnice trougla (kružnice osnove valjka) je
\[
R=\frac{14\sqrt{3}}{2}=7\sqrt{3}.
\]
Ugao naspram stranice \(AC\) je \(\angle B=120^\circ\), pa po proširenoj sinovoj teoremi:
\[
\frac{AC}{\sin 120^\circ}=2R \Rightarrow AC=2R\sin 120^\circ
=2\cdot 7\sqrt{3}\cdot \frac{\sqrt{3}}{2}=21.
\]
Označimo \(a=BC=9\), \(b=AC=21\), \(c=AB\). Po kosinusnoj teoremi za stranicu \(b\):
\[
b^2=a^2+c^2-2ac\cos B.
\]
Kako je \(\cos 120^\circ=-\frac12\), dobijamo
\[
21^2=9^2+c^2-2\cdot 9\cdot c\left(-\frac12\right)
 \Rightarrow 441=81+c^2+9c.
\]
\[
c^2+9c-360=0 \Rightarrow c=\frac{-9+39}{2}=15.
\]
Površina osnove:
\[
P_{\triangle}=\frac12\cdot a\cdot c\cdot \sin B
=\frac12\cdot 9\cdot 15\cdot \frac{\sqrt{3}}{2}
=\frac{135\sqrt{3}}{4}.
\]
Visina prizme je \(20\), pa je zapremina
\[
V=P_{\triangle}\cdot 20=\frac{135\sqrt{3}}{4}\cdot 20=675\sqrt{3}.
\]

\subsection*{Odgovor}
\[
\boxed{(D)\ 675\sqrt{3}}
\]
\end{document}

