\documentclass[12pt]{article}
\usepackage[utf8]{inputenc}
\usepackage[T1]{fontenc}
\usepackage[serbian]{babel}
\usepackage{amsmath,amssymb}
\usepackage{geometry}
\geometry{margin=2.2cm}

\begin{document}
\section*{Zadatak 15}
Odrediti interval u kome se nalazi skup realnih rešenja jednačine
\[
4^x-7\cdot 2^{\frac{x-3}{2}}=2^{-x}.
\]

\subsection*{Rešenje}
Neka je \(t=2^{x/2}>0\). Tada je \(4^x=2^{2x}=t^4\), a \(2^{-x}=t^{-2}\), i
\[
2^{\frac{x-3}{2}}=2^{x/2}\cdot 2^{-3/2}=\frac{t}{2\sqrt{2}}.
\]
Jednačina postaje
\[
t^4-7\cdot \frac{t}{2\sqrt{2}}=t^{-2}.
\]
Množenjem sa \(t^2\) dobijamo
\[
t^6-\frac{7}{2\sqrt{2}}t^3=1.
\]
Neka je \(u=t^3>0\). Tada
\[
u^2-\frac{7}{2\sqrt{2}}u-1=0.
\]
\[
u=\frac{\frac{7}{2\sqrt{2}}\pm\sqrt{\left(\frac{7}{2\sqrt{2}}\right)^2+4}}{2}
=\frac{\frac{7\sqrt{2}}{4}\pm \frac{9\sqrt{2}}{4}}{2}.
\]
Dobijamo \(u_1=2\sqrt{2}>0\), \(u_2=-\frac{\sqrt{2}}{4}<0\) (odbacujemo). Dakle
\[
t^3=2\sqrt{2}=2^{3/2}\Rightarrow t=2^{1/2}=\sqrt{2}.
\]
Pošto je \(t=2^{x/2}\), sledi \(2^{x/2}=2^{1/2}\Rightarrow x=1\).

Dakle rešenje je \(x=1\), što pripada intervalu \((0,3]\).

\subsection*{Odgovor}
\[
\boxed{(B)\ (0,3]}
\]
\end{document}

