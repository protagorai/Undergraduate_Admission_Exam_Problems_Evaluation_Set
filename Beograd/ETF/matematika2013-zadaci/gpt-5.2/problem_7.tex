\documentclass[12pt]{article}
\usepackage[utf8]{inputenc}
\usepackage[T1]{fontenc}
\usepackage[serbian]{babel}
\usepackage{amsmath,amssymb}
\usepackage{geometry}
\geometry{margin=2.2cm}

\begin{document}
\section*{Zadatak 7}
Neka je \(S_1\) skup rešenja nejednačine \(\lvert \sqrt{x+1}\rvert>1\), a \(S_2\) skup rešenja nejednačine \(\sqrt{\lvert x+1\rvert}>1\). Odrediti odnos skupova.

\subsection*{Rešenje}
Pošto je \(\sqrt{x+1}\ge 0\), važi \(\lvert \sqrt{x+1}\rvert=\sqrt{x+1}\). Uz uslov \(x\ge -1\):
\[
\sqrt{x+1}>1 \;\Longleftrightarrow\; x+1>1 \;\Longleftrightarrow\; x>0.
\]
Dakle \(S_1=(0,+\infty)\).

Za drugi skup:
\[
\sqrt{\lvert x+1\rvert}>1 \;\Longleftrightarrow\; \lvert x+1\rvert>1
\;\Longleftrightarrow\; (x>0)\ \text{ili}\ (x<-2).
\]
Zato je \(S_2=(-\infty,-2)\cup(0,+\infty)\).

Jasno je \(S_1\subset S_2\).

\subsection*{Odgovor}
\[
\boxed{(C)\ S_1\subset S_2}
\]
\end{document}

