\documentclass[12pt]{article}
\usepackage[margin=1in]{geometry}
\usepackage{amsmath,amssymb}
\usepackage[utf8]{inputenc}
\usepackage[T1]{fontenc}

\begin{document}

\section*{Problem 10}
Ako je $f(x) = 2x + |x|$ i $g(x) = \frac{2}{3}x - \frac{1}{3}|x|$, tada je $f(g(x))$ jednako:
\[ (A)\ \frac{2}{3}x \quad (B)\ |x| \quad (C)\ -x \quad (D)\ x \quad (E)\ 3x \quad (N)\ \text{Ne znam} \]

\subsection*{Solution}
Let's analyze the functions based on the sign of $x$.

Case 1: $x \ge 0$.
Then $|x| = x$.
$f(x) = 2x + x = 3x$.
$g(x) = \frac{2}{3}x - \frac{1}{3}x = \frac{1}{3}x$.
Since $x \ge 0$, $g(x) = \frac{1}{3}x \ge 0$.
So $f(g(x)) = f(\frac{1}{3}x) = 3(\frac{1}{3}x) = x$.

Case 2: $x < 0$.
Then $|x| = -x$.
$f(x) = 2x + (-x) = x$.
$g(x) = \frac{2}{3}x - \frac{1}{3}(-x) = \frac{2}{3}x + \frac{1}{3}x = x$.
Since $x < 0$, $g(x) = x < 0$.
So $f(g(x)) = f(x) = x$.

In both cases, $f(g(x)) = x$.

\subsection*{Answer}
$x$ (option \textbf{D}).

\end{document}
