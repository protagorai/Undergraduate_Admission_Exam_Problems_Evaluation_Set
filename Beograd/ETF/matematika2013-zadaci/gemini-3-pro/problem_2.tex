\documentclass[12pt]{article}
\usepackage[margin=1in]{geometry}
\usepackage{amsmath,amssymb}
\usepackage[utf8]{inputenc}
\usepackage[T1]{fontenc}

\begin{document}

\section*{Problem 2}
Jednačina prave koja prolazi kroz tačke $M_1(-1,1)$ i $M_2(2,4)$ glasi:
\[ (A)\ x-y+2=0 \quad (B)\ x+y=0 \quad (C)\ -2x+y=0 \quad (D)\ -3x-y-2=0 \quad (E)\ x-y-2=0 \quad (N)\ \text{Ne znam} \]

\subsection*{Solution}
The equation of a line passing through two points $(x_1, y_1)$ and $(x_2, y_2)$ is given by:
\[ y - y_1 = \frac{y_2 - y_1}{x_2 - x_1} (x - x_1) \]
Given $M_1(-1, 1)$ and $M_2(2, 4)$, we have:
\[ k = \frac{4 - 1}{2 - (-1)} = \frac{3}{3} = 1 \]
Using point-slope form with $M_1$:
\[ y - 1 = 1(x - (-1)) \]
\[ y - 1 = x + 1 \]
\[ y = x + 2 \]
Rearranging to the general form $Ax + By + C = 0$:
\[ x - y + 2 = 0 \]

\subsection*{Answer}
$x - y + 2 = 0$ (option \textbf{A}).

\end{document}
