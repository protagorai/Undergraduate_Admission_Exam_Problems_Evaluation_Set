\documentclass[12pt]{article}
\usepackage[margin=1in]{geometry}
\usepackage{amsmath,amssymb}
\usepackage[utf8]{inputenc}
\usepackage[T1]{fontenc}

\begin{document}

\section*{Problem 7}
Neka je $S_1$ skup rešenja nejednačine $|\sqrt{x+1}| > 1$ i $S_2$ skup rešenja nejednačine $\sqrt{|x+1|} > 1$. Tada je:
\[ (A)\ S_1=S_2 \quad (B)\ S_1 \supset S_2 \quad (C)\ S_1 \subset S_2 \quad (D)\ S_1 = \mathbf{R}, S_2 \ne \varnothing \quad (E)\ \text{nijedan od ponuđenih odgovora} \quad (N)\ \text{Ne znam} \]

\subsection*{Solution}
For $S_1$: $|\sqrt{x+1}| > 1$.
The square root $\sqrt{x+1}$ is defined for $x+1 \ge 0 \implies x \ge -1$.
Since $\sqrt{x+1}$ is non-negative, $|\sqrt{x+1}| = \sqrt{x+1}$.
So $\sqrt{x+1} > 1$.
Square both sides: $x+1 > 1 \implies x > 0$.
So $S_1 = (0, +\infty)$.

For $S_2$: $\sqrt{|x+1|} > 1$.
Defined for all $x \in \mathbf{R}$ (since $|x+1| \ge 0$).
Square both sides: $|x+1| > 1$.
This means $x+1 > 1$ or $x+1 < -1$.
$x > 0$ or $x < -2$.
So $S_2 = (-\infty, -2) \cup (0, +\infty)$.

Comparing $S_1$ and $S_2$, we see that $S_1$ is a proper subset of $S_2$.
$S_1 \subset S_2$.

\subsection*{Answer}
$S_1 \subset S_2$ (option \textbf{C}).

\end{document}
