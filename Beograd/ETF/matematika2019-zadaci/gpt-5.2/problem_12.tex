\documentclass[12pt]{article}
\usepackage[margin=1in]{geometry}
\usepackage{amsmath,amssymb}
\begin{document}

\section*{Problem 12}
Rotacijom pravouglog trougla koji nije jednakokraki oko hipotenuze formirano je obrtno telo $T_1$,
a rotacijom oko duže katete obrtno telo $T_2$.
Ako je $\alpha$ najmanji ugao datog trougla, odrediti odnos zapremina $T_1$ i $T_2$.

\subsection*{Solution}
Neka su katete $a<b$ (kraća i duža), a hipotenuza $c$.
Pošto je $\alpha$ najmanji ugao, on leži naspram kraće katete $a$, pa je
\[
\cos\alpha=\frac{b}{c}.
\]

\medskip
\noindent\textbf{Zapremina $T_2$:}
Rotacijom oko duže katete $b$ dobijamo pravu kružnu kupu sa visinom $b$ i poluprečnikom osnove $a$, pa je
\[
V_2=\frac{1}{3}\pi a^2 b.
\]

\medskip
\noindent\textbf{Zapremina $T_1$:}
Rotacijom oko hipotenuze dobija se dvostruka kupa (dve kupe sa zajedničkom osnovom).
Poluprečnik zajedničke osnove je visina na hipotenuzu $h=\frac{ab}{c}$, a zbir visina te dve kupe jednak je $c$.
Zato je
\[
V_1=\frac{1}{3}\pi h^2\cdot c=\frac{1}{3}\pi\left(\frac{ab}{c}\right)^2 c=\frac{1}{3}\pi\frac{a^2b^2}{c}.
\]

Odnos zapremina je
\[
\frac{V_1}{V_2}=\frac{\frac{1}{3}\pi\frac{a^2b^2}{c}}{\frac{1}{3}\pi a^2 b}=\frac{b}{c}=\cos\alpha.
\]

\subsection*{Answer}
$\cos\alpha$ (option \textbf{B}).

\end{document}

