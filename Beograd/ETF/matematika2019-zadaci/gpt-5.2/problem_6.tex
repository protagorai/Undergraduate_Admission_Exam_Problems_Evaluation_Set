\documentclass[12pt]{article}
\usepackage[margin=1in]{geometry}
\usepackage{amsmath,amssymb}
\begin{document}

\section*{Problem 6}
Neka su $x_1,x_2,x_3$ rešenja jednačine $x^3+px+q=0$, gde su $p,q\in\mathbb{R}$, $p\neq 0$, $q\neq 0$.
Odrediti vrednost izraza
\[
\frac{x_1+x_2+x_3}{x_1^2+x_2^2+x_3^2}-x_1x_2(1+x_3)-(x_1+x_2)x_3.
\]

\subsection*{Solution}
Po Vijetovim formulama za $x^3+px+q=0$ važi
\[
x_1+x_2+x_3=0,\qquad x_1x_2+x_1x_3+x_2x_3=p,\qquad x_1x_2x_3=-q.
\]
Prvi član je zato nula:
\[
\frac{x_1+x_2+x_3}{x_1^2+x_2^2+x_3^2}=0.
\]
Dalje, iz $x_1+x_2=-x_3$ sledi
\[
-(x_1+x_2)x_3=-(-x_3)x_3=x_3^2.
\]
Takođe,
\[
-x_1x_2(1+x_3)=-x_1x_2-x_1x_2x_3=-x_1x_2+q.
\]
Zato je ceo izraz jednak
\[
(-x_1x_2+x_3^2)+q.
\]
Ali iz formule za $p$ dobijamo
\[
p=x_1x_2+x_3(x_1+x_2)=x_1x_2-x_3^2 \quad\Rightarrow\quad x_3^2-x_1x_2=-p.
\]
Otuda je vrednost izraza
\[
-p+q=q-p.
\]

\subsection*{Answer}
$q-p$ (option \textbf{D}).

\end{document}

