\documentclass[12pt]{article}
\usepackage[margin=1in]{geometry}
\usepackage{amsmath,amssymb}
\begin{document}

\section*{Problem 4}
Ako je
\[
f(x)=x(x+1)(x+2)(x+3)(x+4),
\]
odrediti $f'(-2)$.

\subsection*{Solution}
Neka su nule polinoma $0,-1,-2,-3,-4$. Važi
\[
f(x)=\prod_{k\in\{0,1,2,3,4\}}(x+k).
\]
Za polinom oblika $f(x)=\prod (x-r_i)$ važi $f'(r_j)=\prod_{i\neq j}(r_j-r_i)$.
Ovde je $r_j=-2$, pa
\[
f'(-2)=(-2-0)(-2-(-1))(-2-(-3))(-2-(-4)).
\]
Odnosno
\[
f'(-2)=(-2)(-1)(1)(2)=4.
\]

\subsection*{Answer}
$4$ (option \textbf{D}).

\end{document}

