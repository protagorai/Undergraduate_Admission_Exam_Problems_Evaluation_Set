\documentclass[12pt]{article}
\usepackage[margin=1in]{geometry}
\usepackage{amsmath,amssymb}
\begin{document}

\section*{Problem 18}
Ako su uglovi trougla $\alpha,\beta,\gamma$ i važi $\alpha\ge \beta\ge \gamma$, a
\[
\sin\alpha-\sin\beta+\sin\gamma=1,
\]
onda je ugao $\gamma$ jednak:

\subsection*{Solution}
Kako je $\alpha+\beta+\gamma=\pi$, važi $\alpha=\pi-\beta-\gamma$ i
\[
\sin\alpha=\sin(\pi-\beta-\gamma)=\sin(\beta+\gamma).
\]
Uslov postaje
\[
\sin(\beta+\gamma)-\sin\beta+\sin\gamma=1.
\]
Prebacimo:
\[
\sin(\beta+\gamma)-\sin\beta=1-\sin\gamma.
\]
Primena identiteta $\sin u-\sin v=2\cos\frac{u+v}{2}\sin\frac{u-v}{2}$ daje (za $u=\beta+\gamma$, $v=\beta$)
\[
2\cos\left(\beta+\frac{\gamma}{2}\right)\sin\frac{\gamma}{2}=1-\sin\gamma.
\]
Kako je $\sin\gamma=2\sin\frac{\gamma}{2}\cos\frac{\gamma}{2}$, dobijamo
\[
2\sin\frac{\gamma}{2}\left(\cos\left(\beta+\frac{\gamma}{2}\right)+\cos\frac{\gamma}{2}\right)=1.
\]
Sada proverimo kandidate iz ponuđenih odgovora koristeći i uslov poretka $\alpha\ge\beta\ge\gamma$.

Ako je $\gamma=45^\circ$, možemo uzeti $\beta=45^\circ$ (dozvoljeno jer $\beta\ge\gamma$), tada je
\[
\alpha=180^\circ-45^\circ-45^\circ=90^\circ,
\]
i zaista
\[
\sin 90^\circ-\sin 45^\circ+\sin 45^\circ=1.
\]
Poredak $\alpha\ge\beta\ge\gamma$ je ispunjen: $90^\circ\ge 45^\circ\ge 45^\circ$.

Zato je $\gamma=45^\circ$.

\subsection*{Answer}
$45^\circ$ (option \textbf{E}).

\end{document}

