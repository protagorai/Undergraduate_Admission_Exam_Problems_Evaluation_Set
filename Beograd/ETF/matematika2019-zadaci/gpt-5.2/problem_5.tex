\documentclass[12pt]{article}
\usepackage[margin=1in]{geometry}
\usepackage{amsmath,amssymb}
\begin{document}

\section*{Problem 5}
Ako je $\sin\alpha=-\frac{5}{13}$ i $\alpha\in\left(\frac{3\pi}{2},2\pi\right)$, odrediti
\[
\tan\!\left(\frac{\pi}{4}+\alpha\right).
\]

\subsection*{Solution}
Kako je $\alpha$ u IV kvadrantu, $\cos\alpha>0$. Zato
\[
\cos\alpha=\sqrt{1-\sin^2\alpha}=\sqrt{1-\frac{25}{169}}=\sqrt{\frac{144}{169}}=\frac{12}{13}.
\]
Tada je
\[
\tan\alpha=\frac{\sin\alpha}{\cos\alpha}=\frac{-5/13}{12/13}=-\frac{5}{12}.
\]
Koristimo formulu za tangens zbira:
\[
\tan\left(\frac{\pi}{4}+\alpha\right)=\frac{\tan\frac{\pi}{4}+\tan\alpha}{1-\tan\frac{\pi}{4}\tan\alpha}
=\frac{1+\tan\alpha}{1-\tan\alpha}.
\]
Za $\tan\alpha=-\frac{5}{12}$ dobijamo
\[
\frac{1-\frac{5}{12}}{1+\frac{5}{12}}=\frac{\frac{7}{12}}{\frac{17}{12}}=\frac{7}{17}.
\]

\subsection*{Answer}
$\dfrac{7}{17}$ (option \textbf{D}).

\end{document}

