\documentclass[12pt]{article}
\usepackage[margin=1in]{geometry}
\usepackage{amsmath,amssymb}
\begin{document}

\section*{Problem 11}
Neka je $x\in\mathbb{R}$ pozitivan broj i neka je $n\in\mathbb{N}$ paran broj.
Zbir svih binomnih koeficijenata u razvoju binoma
\[
\left(x^{2019}+\frac{1}{x^{2019}}\right)^n
\]
četiri puta je veći od zbira svih binomnih koeficijenata u razvoju binoma
\[
\left(\sqrt{x}+\frac{1}{\sqrt{x}}\right)^{\frac{n}{2}}.
\]
Zbir onih članova ova dva razvoja binoma koji ne sadrže $x$ jednak je:

\subsection*{Solution}
Zbir svih binomnih koeficijenata u razvoju $(a+b)^m$ jednak je $2^m$.
Zato je uslov
\[
2^n=4\cdot 2^{n/2}.
\]
Odavde
\[
2^{n-n/2}=4 \;\Rightarrow\; 2^{n/2}=4 \;\Rightarrow\; \frac{n}{2}=2 \;\Rightarrow\; n=4.
\]

Sada tražimo zbir članova koji ne sadrže $x$ (konstantni članovi).

\medskip
\noindent\textbf{1) }$\left(x^{2019}+x^{-2019}\right)^4$:
opšti član je $\binom{4}{k}x^{2019(4-2k)}$. Konstantni član nastaje za $4-2k=0$, tj.\ $k=2$,
pa je konstanta $\binom{4}{2}=6$.

\medskip
\noindent\textbf{2) }$\left(x^{1/2}+x^{-1/2}\right)^{2}$:
\[
\left(\sqrt{x}+\frac{1}{\sqrt{x}}\right)^2=x+2+\frac1x,
\]
pa je konstantni član $2$.

Zbir konstantnih članova je $6+2=8$.

\subsection*{Answer}
$8$ (option \textbf{C}).

\end{document}

