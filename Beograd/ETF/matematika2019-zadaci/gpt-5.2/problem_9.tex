\documentclass[12pt]{article}
\usepackage[margin=1in]{geometry}
\usepackage{amsmath,amssymb}
\begin{document}

\section*{Problem 9}
Obim trougla čije stranice obrazuju aritmetičku progresiju sa razlikom $4\,\text{cm}$ i koji ima jedan ugao $120^\circ$ jednak je:

\subsection*{Solution}
Neka su stranice u aritmetičkoj progresiji
\[
a-4,\quad a,\quad a+4,
\]
pa je najduža strana $a+4$. Kako je ugao $120^\circ$ najveći (obtuzan), on leži naspram najduže stranice, dakle ugao između stranica $a$ i $a-4$ je $120^\circ$.

Primena kosinusne teoreme daje:
\[
(a+4)^2=a^2+(a-4)^2-2\cdot a(a-4)\cos 120^\circ.
\]
Kako je $\cos120^\circ=-\frac12$, dobijamo
\[
(a+4)^2=a^2+(a-4)^2+a(a-4).
\]
Računamo:
\[
a^2+(a-4)^2+a(a-4)=a^2+(a^2-8a+16)+(a^2-4a)=3a^2-12a+16.
\]
Leva strana je $(a+4)^2=a^2+8a+16$, pa
\[
a^2+8a+16=3a^2-12a+16 \;\Rightarrow\; 0=2a^2-20a=2a(a-10).
\]
Pošto je $a>0$, sledi $a=10$, pa su stranice $6,10,14$ i obim je
\[
6+10+14=30\ \text{cm}.
\]

\subsection*{Answer}
$30\,\text{cm}$ (option \textbf{D}).

\end{document}

