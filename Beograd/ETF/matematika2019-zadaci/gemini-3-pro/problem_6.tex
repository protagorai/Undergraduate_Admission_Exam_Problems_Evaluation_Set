\documentclass[12pt]{article}
\usepackage[margin=1in]{geometry}
\usepackage{amsmath,amssymb}
\usepackage[utf8]{inputenc}
\begin{document}

\section*{Problem 6}
Neka su $x_1, x_2$ i $x_3$ rešenja jednačine $x^3+px+q=0$, $p,q\in\mathbb{R}$, $p\neq 0, q\neq 0$.
Tada je vrednost izraza $\frac{x_1+x_2+x_3}{x_1^2+x_2^2+x_3^2}-x_1x_2(1+x_3)-(x_1+x_2)x_3$ jednaka:

\subsection*{Solution}
Koristeći Vijetove formule za jednačinu $x^3+0x^2+px+q=0$, imamo:
\[
x_1+x_2+x_3 = 0
\]
\[
x_1x_2+x_1x_3+x_2x_3 = p
\]
\[
x_1x_2x_3 = -q
\]
Posmatrajmo dati izraz:
\[
\frac{x_1+x_2+x_3}{x_1^2+x_2^2+x_3^2}-x_1x_2(1+x_3)-(x_1+x_2)x_3
\]
Prvi član je jednak 0 jer je brojilac $x_1+x_2+x_3=0$.
Ostaje:
\[
-x_1x_2(1+x_3) - (x_1+x_2)x_3 = -x_1x_2 - x_1x_2x_3 - x_1x_3 - x_2x_3
\]
Grupisanjem članova:
\[
-(x_1x_2 + x_1x_3 + x_2x_3) - x_1x_2x_3
\]
Zamenom vrednosti iz Vijetovih formula:
\[
-(p) - (-q) = -p + q = q-p
\]

\subsection*{Answer}
$q-p$ (Opcija \textbf{D})

\end{document}