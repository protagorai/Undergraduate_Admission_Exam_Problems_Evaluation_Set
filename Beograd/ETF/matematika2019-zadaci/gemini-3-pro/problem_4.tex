\documentclass[12pt]{article}
\usepackage[margin=1in]{geometry}
\usepackage{amsmath,amssymb}
\usepackage[utf8]{inputenc}
\begin{document}

\section*{Problem 4}
Ako je $f(x)=x(x+1)(x+2)(x+3)(x+4)$, onda je $f'(-2)$ jednako:

\subsection*{Solution}
Funkciju možemo zapisati kao $f(x) = (x+2)g(x)$, gde je $g(x) = x(x+1)(x+3)(x+4)$.
Primenom pravila za izvod proizvoda:
\[
f'(x) = (x+2)'g(x) + (x+2)g'(x) = 1 \cdot g(x) + (x+2)g'(x)
\]
Za $x=-2$:
\[
f'(-2) = g(-2) + (-2+2)g'(-2) = g(-2) + 0 = g(-2)
\]
Izračunajmo $g(-2)$:
\[
g(-2) = (-2)(-2+1)(-2+3)(-2+4) = (-2)(-1)(1)(2) = 4
\]
Dakle, $f'(-2) = 4$.

\subsection*{Answer}
$4$ (Opcija \textbf{D})

\end{document}