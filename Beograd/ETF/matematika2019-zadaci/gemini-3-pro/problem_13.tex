\documentclass[12pt]{article}
\usepackage[margin=1in]{geometry}
\usepackage{amsmath,amssymb}
\usepackage[utf8]{inputenc}
\begin{document}

\section*{Problem 13}
Jednačina kružnice čiji je centar na $x$-osi i koja sa parabolom $y^2=12x$ u tački $A(3,6)$ ima zajedničku tangentu jeste:

\subsection*{Solution}
Jednačina parabole je $y^2 = 2px \Rightarrow 2p=12 \Rightarrow p=6$.
Jednačina tangente parabole u tački $A(x_0, y_0)$ je $yy_0 = p(x+x_0)$.
Za $A(3,6)$: $6y = 6(x+3) \Rightarrow y = x+3$. Koeficijent pravca tangente je $k_t=1$.
Normala u tački $A$ je normalna na tangentu, pa je njen koeficijent pravca $k_n = -1/k_t = -1$.
Jednačina normale kroz $A(3,6)$: $y-6 = -1(x-3) \Rightarrow y = -x+9$.
Centar kružnice $C(x_c, 0)$ leži na normali.
$0 = -x_c + 9 \Rightarrow x_c = 9$. Centar je $C(9,0)$.
Poluprečnik $R$ je rastojanje $CA$:
$R^2 = (9-3)^2 + (0-6)^2 = 36 + 36 = 72$.
Jednačina kružnice:
$(x-9)^2 + y^2 = 72$.

\subsection*{Answer}
$(x-9)^2+y^2=72$ (Opcija \textbf{D})

\end{document}