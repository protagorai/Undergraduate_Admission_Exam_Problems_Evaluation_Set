\documentclass[12pt]{article}
\usepackage[margin=1in]{geometry}
\usepackage{amsmath,amssymb}
\usepackage[utf8]{inputenc}
\begin{document}

\section*{Problem 5}
Ako je $\sin\alpha=-\frac{5}{13}$ i $\alpha\in\left(\frac{3\pi}{2}, 2\pi\right)$, onda je $\text{tg}\left(\frac{\pi}{4}+\alpha\right)$ jednak:

\subsection*{Solution}
Kako je $\alpha$ u IV kvadrantu, kosinus je pozitivan:
\[
\cos\alpha = \sqrt{1-\sin^2\alpha} = \sqrt{1-\left(-\frac{5}{13}\right)^2} = \sqrt{1-\frac{25}{169}} = \sqrt{\frac{144}{169}} = \frac{12}{13}
\]
Tada je $\text{tg}\alpha = \frac{\sin\alpha}{\cos\alpha} = \frac{-5/13}{12/13} = -\frac{5}{12}$.
Koristimo adicionu formulu za tangens:
\[
\text{tg}\left(\frac{\pi}{4}+\alpha\right) = \frac{\text{tg}\frac{\pi}{4} + \text{tg}\alpha}{1 - \text{tg}\frac{\pi}{4}\text{tg}\alpha}
\]
Kako je $\text{tg}\frac{\pi}{4} = 1$:
\[
\text{tg}\left(\frac{\pi}{4}+\alpha\right) = \frac{1 + (-\frac{5}{12})}{1 - 1 \cdot (-\frac{5}{12})} = \frac{1 - \frac{5}{12}}{1 + \frac{5}{12}} = \frac{\frac{7}{12}}{\frac{17}{12}} = \frac{7}{17}
\]

\subsection*{Answer}
$\frac{7}{17}$ (Opcija \textbf{D})

\end{document}