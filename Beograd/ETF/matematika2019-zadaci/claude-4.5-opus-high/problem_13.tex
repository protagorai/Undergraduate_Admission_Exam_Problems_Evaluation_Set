\documentclass[12pt]{article}
\usepackage[margin=1in]{geometry}
\usepackage{amsmath,amssymb}
\begin{document}

\section*{Problem 13}
A circle whose center is on the $x$-axis and which has the parabola $y^2 = 12x$ as a tangent at point $A(3,6)$ has the equation:

\subsection*{Solution}
Let the center of the circle be at $(h, 0)$ since it's on the $x$-axis.

First, let's verify that $A(3,6)$ is on the parabola $y^2 = 12x$:
$6^2 = 36$ and $12 \cdot 3 = 36$ ✓

For the parabola $y^2 = 12x$, we can find the slope of the tangent by implicit differentiation:
$2y \frac{dy}{dx} = 12$
$\frac{dy}{dx} = \frac{6}{y}$

At point $A(3,6)$: $\frac{dy}{dx} = \frac{6}{6} = 1$

So the tangent line at $A(3,6)$ has slope $m = 1$.

For the circle to be tangent to the parabola at point $A$, the line from the center $(h,0)$ to point $A(3,6)$ must be perpendicular to the tangent line.

The slope of the line from $(h,0)$ to $(3,6)$ is $\frac{6-0}{3-h} = \frac{6}{3-h}$.

Since this line is perpendicular to the tangent (which has slope 1), we have:
$\frac{6}{3-h} \cdot 1 = -1$

$\frac{6}{3-h} = -1$

$6 = -(3-h) = -3 + h$

$h = 9$

So the center is at $(9,0)$.

The radius is the distance from $(9,0)$ to $(3,6)$:
$r = \sqrt{(9-3)^2 + (0-6)^2} = \sqrt{36 + 36} = \sqrt{72} = 6\sqrt{2}$

The equation of the circle is:
$(x-9)^2 + y^2 = (6\sqrt{2})^2 = 72$

$(x-9)^2 + y^2 = 72$

Expanding: $x^2 - 18x + 81 + y^2 = 72$

$x^2 + y^2 - 18x + 9 = 0$

Wait, let me double-check: $81 - 72 = 9$ ✓

Looking at the options:
(A) $(x-3)^2 + y^2 = 9$
(B) $(x-6)^2 + y^2 = 36$  
(C) $(x-9)^2 + y^2 = 81$
(D) $(x-9)^2 + y^2 = 72$
(E) $(x-6)^2 + y^2 = 72$

My calculation gives $(x-9)^2 + y^2 = 72$, which is option (D).

\subsection*{Answer}
$(x-9)^2 + y^2 = 72$ (option \textbf{D}).

\end{document}