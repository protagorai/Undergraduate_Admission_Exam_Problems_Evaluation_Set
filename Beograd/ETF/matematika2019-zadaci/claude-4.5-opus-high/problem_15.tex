\documentclass[12pt]{article}
\usepackage[margin=1in]{geometry}
\usepackage{amsmath,amssymb}
\begin{document}

\section*{Problem 15}
The number of real solutions of the equation $3^x + 4^x = 5^x$ is:

\subsection*{Solution}
We need to solve $3^x + 4^x = 5^x$.

Dividing both sides by $5^x$ (which is always positive):
$\frac{3^x}{5^x} + \frac{4^x}{5^x} = 1$

$\left(\frac{3}{5}\right)^x + \left(\frac{4}{5}\right)^x = 1$

Let $f(x) = \left(\frac{3}{5}\right)^x + \left(\frac{4}{5}\right)^x$.

Since $\frac{3}{5} < 1$ and $\frac{4}{5} < 1$, both terms are decreasing functions.

Let's analyze the behavior of $f(x)$:

- As $x \to +\infty$: $f(x) \to 0 + 0 = 0$
- As $x \to -\infty$: $f(x) \to +\infty + +\infty = +\infty$
- At $x = 0$: $f(0) = 1 + 1 = 2$

Since $f(x)$ is strictly decreasing (as the sum of two strictly decreasing functions) and continuous, and since:
- $f(0) = 2 > 1$
- $\lim_{x \to +\infty} f(x) = 0 < 1$

By the Intermediate Value Theorem, there exists exactly one value of $x > 0$ such that $f(x) = 1$.

Let's check a few values:
- $f(1) = \frac{3}{5} + \frac{4}{5} = \frac{7}{5} = 1.4 > 1$
- $f(2) = \left(\frac{3}{5}\right)^2 + \left(\frac{4}{5}\right)^2 = \frac{9}{25} + \frac{16}{25} = \frac{25}{25} = 1$

So $x = 2$ is the solution!

Let's verify: $3^2 + 4^2 = 9 + 16 = 25 = 5^2$ ✓

Since $f(x)$ is strictly decreasing and we found that $f(2) = 1$, this is the unique solution.

\subsection*{Answer}
1 (option \textbf{B}).

\end{document}