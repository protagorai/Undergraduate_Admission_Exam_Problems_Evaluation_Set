\documentclass[12pt]{article}
\usepackage[margin=1in]{geometry}
\usepackage{amsmath,amssymb}
\begin{document}

\section*{Problem 12}
By rotating a right triangle that is not isosceles around the hypotenuse, a solid body $T_1$ is formed, and by rotating around the longer leg, a solid body $T_2$ is formed. If $\alpha$ is the smallest angle of the triangle, then the ratio of volumes $T_1$ and $T_2$ is:

\subsection*{Solution}
Let the right triangle have legs of length $a$ and $b$ (with $a > b$) and hypotenuse of length $c = \sqrt{a^2 + b^2}$.

The smallest angle $\alpha$ is opposite to the shorter leg $b$, so $\sin \alpha = \frac{b}{c}$ and $\cos \alpha = \frac{a}{c}$.

\textbf{Volume $T_1$ (rotation around hypotenuse):}
When we rotate around the hypotenuse, we get a double cone. The height of the triangle from the right angle to the hypotenuse is $h = \frac{ab}{c}$.

The volume is: $V_1 = \frac{1}{3}\pi h^2 c = \frac{1}{3}\pi \left(\frac{ab}{c}\right)^2 c = \frac{\pi a^2 b^2}{3c}$

\textbf{Volume $T_2$ (rotation around longer leg):}
When we rotate around the leg of length $a$, we get a cone with base radius $b$ and height $a$.

The volume is: $V_2 = \frac{1}{3}\pi b^2 a$

\textbf{Ratio of volumes:}
\[
\frac{V_1}{V_2} = \frac{\frac{\pi a^2 b^2}{3c}}{\frac{1}{3}\pi b^2 a} = \frac{a^2 b^2}{c} \cdot \frac{1}{ab^2} = \frac{a}{c} = \cos \alpha
\]

Therefore, the ratio is $\cos \alpha$.

\subsection*{Answer}
$\cos \alpha$ (option \textbf{B}).

\end{document}