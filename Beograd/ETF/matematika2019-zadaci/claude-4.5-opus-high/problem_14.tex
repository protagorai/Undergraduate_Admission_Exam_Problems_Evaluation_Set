\documentclass[12pt]{article}
\usepackage[margin=1in]{geometry}
\usepackage{amsmath,amssymb}
\begin{document}

\section*{Problem 14}
The sum of all values of parameter $a \in \mathbb{R}$ for which the graphs of functions $y = (a + 2)x^2 - ax - 3$ and $y = ax - 4$ have exactly one common point is:

\subsection*{Solution}
For the graphs to have exactly one common point, the system of equations must have exactly one solution:
\begin{align}
y &= (a + 2)x^2 - ax - 3\\
y &= ax - 4
\end{align}

Setting them equal:
$(a + 2)x^2 - ax - 3 = ax - 4$

$(a + 2)x^2 - ax - 3 - ax + 4 = 0$

$(a + 2)x^2 - 2ax + 1 = 0$

For this quadratic equation in $x$ to have exactly one solution, we need to consider two cases:

\textbf{Case 1:} The equation is linear (coefficient of $x^2$ is zero)
$a + 2 = 0 \Rightarrow a = -2$

When $a = -2$: $0 \cdot x^2 - 2(-2)x + 1 = 4x + 1 = 0$
This gives $x = -\frac{1}{4}$, which is exactly one solution.

\textbf{Case 2:} The equation is quadratic with discriminant equal to zero
$a + 2 \neq 0$ and $\Delta = 0$

$\Delta = (-2a)^2 - 4(a + 2)(1) = 4a^2 - 4(a + 2) = 4a^2 - 4a - 8$

Setting $\Delta = 0$:
$4a^2 - 4a - 8 = 0$
$a^2 - a - 2 = 0$
$(a - 2)(a + 1) = 0$

So $a = 2$ or $a = -1$.

We need to verify that $a + 2 \neq 0$ for these values:
- For $a = 2$: $a + 2 = 4 \neq 0$ ✓
- For $a = -1$: $a + 2 = 1 \neq 0$ ✓

Therefore, the values of $a$ are: $a = -2, -1, 2$.

The sum is: $(-2) + (-1) + 2 = -1$.

\subsection*{Answer}
$-1$ (option \textbf{C}).

\end{document}