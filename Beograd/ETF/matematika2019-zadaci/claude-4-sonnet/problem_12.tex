\documentclass[12pt]{article}
\usepackage[margin=1in]{geometry}
\usepackage{amsmath,amssymb}
\begin{document}

\section*{Problem 12}
By rotating a right triangle that is not isosceles around the hypotenuse, a solid body $T_1$ is formed, and by rotating around the longer leg, a solid body $T_2$ is formed. If $\alpha$ is the smallest angle of the given triangle, then the ratio of volumes $T_1$ and $T_2$ is:

\subsection*{Solution}
Let the right triangle have legs $a$ and $b$ with $b > a$, and hypotenuse $c$. The smallest angle $\alpha$ is opposite to the shorter leg $a$.

We have: $\sin \alpha = \frac{a}{c}$, $\cos \alpha = \frac{b}{c}$, and $c^2 = a^2 + b^2$.

\textbf{Volume $T_1$ (rotation around hypotenuse):}
When rotating around the hypotenuse, we get a double cone. The height from the right angle to the hypotenuse is $h = \frac{ab}{c}$.

The two cones have radii equal to this height $h$ and bases with lengths that sum to $c$.
The volume is: $V_1 = \frac{1}{3}\pi h^2 c = \frac{1}{3}\pi \left(\frac{ab}{c}\right)^2 c = \frac{\pi a^2 b^2}{3c}$

\textbf{Volume $T_2$ (rotation around longer leg):}
When rotating around the longer leg $b$, we get a cone with radius $a$ and height $b$.
The volume is: $V_2 = \frac{1}{3}\pi a^2 b$

\textbf{Ratio of volumes:}
$\frac{V_1}{V_2} = \frac{\frac{\pi a^2 b^2}{3c}}{\frac{1}{3}\pi a^2 b} = \frac{a^2 b^2}{c} \cdot \frac{3c}{3a^2 b} = \frac{b}{c}$

Since $\cos \alpha = \frac{b}{c}$, we have:
$\frac{V_1}{V_2} = \cos \alpha$

\subsection*{Answer}
$\cos \alpha$ (option \textbf{B}).

\end{document}