\documentclass[12pt]{article}
\usepackage[margin=1in]{geometry}
\usepackage{amsmath,amssymb}
\begin{document}

\section*{Problem 14}
The sum of all values of parameter $a \in \mathbb{R}$ for which the graphs of functions $y = (a+2)x^2 - ax - 3$ and $y = ax - 4$ have exactly one common point is:

\subsection*{Solution}
For the graphs to intersect, we set the functions equal:
$(a+2)x^2 - ax - 3 = ax - 4$

Rearranging:
$(a+2)x^2 - ax - 3 - ax + 4 = 0$
$(a+2)x^2 - 2ax + 1 = 0$

For the graphs to have exactly one common point, this quadratic equation in $x$ must have exactly one solution.

\textbf{Case 1:} $a + 2 = 0$, i.e., $a = -2$

The equation becomes: $0 \cdot x^2 - 2(-2)x + 1 = 0$
$4x + 1 = 0$
$x = -\frac{1}{4}$

This gives exactly one solution, so $a = -2$ is valid.

\textbf{Case 2:} $a + 2 \neq 0$, i.e., $a \neq -2$

For a quadratic equation $Ax^2 + Bx + C = 0$ to have exactly one solution, the discriminant must be zero:
$\Delta = B^2 - 4AC = 0$

Here: $A = a + 2$, $B = -2a$, $C = 1$

$\Delta = (-2a)^2 - 4(a+2)(1) = 4a^2 - 4(a+2) = 4a^2 - 4a - 8$

Setting $\Delta = 0$:
$4a^2 - 4a - 8 = 0$
$a^2 - a - 2 = 0$
$(a - 2)(a + 1) = 0$

So $a = 2$ or $a = -1$.

Let's verify these solutions:

For $a = 2$: $(2+2)x^2 - 2(2)x + 1 = 0$ → $4x^2 - 4x + 1 = 0$ → $(2x-1)^2 = 0$ → $x = \frac{1}{2}$ (one solution) ✓

For $a = -1$: $(-1+2)x^2 - 2(-1)x + 1 = 0$ → $x^2 + 2x + 1 = 0$ → $(x+1)^2 = 0$ → $x = -1$ (one solution) ✓

Therefore, the values of $a$ are: $a = -2, -1, 2$

The sum is: $(-2) + (-1) + 2 = -1$

\subsection*{Answer}
$-1$ (option \textbf{C}).

\end{document}