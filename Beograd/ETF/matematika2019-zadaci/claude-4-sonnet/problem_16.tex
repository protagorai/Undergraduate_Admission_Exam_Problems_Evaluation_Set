\documentclass[12pt]{article}
\usepackage[margin=1in]{geometry}
\usepackage{amsmath,amssymb}
\begin{document}

\section*{Problem 16}
Given the set $S = \left\{\left(\frac{-1+i\sqrt{3}}{2}\right)^{2019}, \text{Im}\left(\left(\frac{1+i}{\sqrt{2}}\right)^{2019}\right), 0.3333333, \frac{\pi}{3}, \sin\frac{\pi}{3}, \frac{22}{7}\right\}$ and the set of rational numbers $\mathbb{Q} = \left\{\frac{p}{q} \mid p \in \mathbb{Z}, q \in \mathbb{N}\right\}, i^2 = -1$. The number of elements in the set $S \cap \mathbb{Q}$ is:

\subsection*{Solution}
Let's analyze each element of $S$ to determine which are rational.

\textbf{Element 1:} $\left(\frac{-1+i\sqrt{3}}{2}\right)^{2019}$

Let $\omega = \frac{-1+i\sqrt{3}}{2}$. This is a primitive cube root of unity.
We have $\omega^3 = 1$ and $1 + \omega + \omega^2 = 0$.

Since $2019 = 3 \times 673$, we have:
$\omega^{2019} = \omega^{3 \times 673} = (\omega^3)^{673} = 1^{673} = 1$

So this element equals 1, which is rational.

\textbf{Element 2:} $\text{Im}\left(\left(\frac{1+i}{\sqrt{2}}\right)^{2019}\right)$

Let $z = \frac{1+i}{\sqrt{2}} = \frac{1}{\sqrt{2}} + \frac{i}{\sqrt{2}}$.

In polar form: $|z| = \sqrt{\frac{1}{2} + \frac{1}{2}} = 1$ and $\arg(z) = \arctan(1) = \frac{\pi}{4}$.

So $z = e^{i\pi/4}$ and $z^{2019} = e^{i \cdot 2019\pi/4}$.

$2019\pi/4 = 504.75\pi = 504\pi + \frac{3\pi}{4}$

Since $e^{i \cdot 504\pi} = 1$, we have:
$z^{2019} = e^{i \cdot 3\pi/4} = \cos\frac{3\pi}{4} + i\sin\frac{3\pi}{4} = -\frac{\sqrt{2}}{2} + i\frac{\sqrt{2}}{2}$

Therefore: $\text{Im}(z^{2019}) = \frac{\sqrt{2}}{2}$, which is irrational.

\textbf{Element 3:} $0.3333333$

This appears to be $\frac{1}{3}$ (assuming the 3's continue infinitely), which is rational.

\textbf{Element 4:} $\frac{\pi}{3}$

Since $\pi$ is irrational, $\frac{\pi}{3}$ is irrational.

\textbf{Element 5:} $\sin\frac{\pi}{3}$

$\sin\frac{\pi}{3} = \frac{\sqrt{3}}{2}$, which is irrational.

\textbf{Element 6:} $\frac{22}{7}$

This is clearly rational (it's a fraction of two integers).

\textbf{Summary:}
Rational elements: $1$, $\frac{1}{3}$, $\frac{22}{7}$
Irrational elements: $\frac{\sqrt{2}}{2}$, $\frac{\pi}{3}$, $\frac{\sqrt{3}}{2}$

Therefore, $|S \cap \mathbb{Q}| = 3$.

\subsection*{Answer}
3 (option \textbf{B}).

\end{document}