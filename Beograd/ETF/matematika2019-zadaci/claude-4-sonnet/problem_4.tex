\documentclass[12pt]{article}
\usepackage[margin=1in]{geometry}
\usepackage{amsmath,amssymb}
\begin{document}

\section*{Problem 4}
If $f(x) = x(x+1)(x+2)(x+3)(x+4)$, then $f'(-2)$ equals:

\subsection*{Solution}
We need to find the derivative of $f(x) = x(x+1)(x+2)(x+3)(x+4)$ and evaluate it at $x = -2$.

Notice that $f(-2) = (-2)(-2+1)(-2+2)(-2+3)(-2+4) = (-2)(-1)(0)(1)(2) = 0$.

Since one of the factors is zero at $x = -2$, we can use the fact that $f(x)$ has a zero at $x = -2$.

Let's rewrite $f(x)$ by factoring out $(x+2)$:
\[
f(x) = x(x+1)(x+2)(x+3)(x+4) = (x+2) \cdot g(x)
\]
where $g(x) = x(x+1)(x+3)(x+4)$.

Using the product rule:
\[
f'(x) = g(x) + (x+2)g'(x)
\]

At $x = -2$:
\[
f'(-2) = g(-2) + (-2+2)g'(-2) = g(-2) + 0 \cdot g'(-2) = g(-2)
\]

Now we calculate $g(-2)$:
\[
g(-2) = (-2)(-2+1)(-2+3)(-2+4) = (-2)(-1)(1)(2) = 4
\]

Therefore, $f'(-2) = 4$.

\subsection*{Answer}
4 (option \textbf{D}).

\end{document}