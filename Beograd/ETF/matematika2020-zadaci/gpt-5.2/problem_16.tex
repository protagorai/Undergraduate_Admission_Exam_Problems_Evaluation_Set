\documentclass[12pt]{article}
\usepackage[utf8]{inputenc}
\usepackage[T1]{fontenc}
\usepackage{amsmath,amssymb}
\usepackage[margin=2.2cm]{geometry}

\begin{document}
\section*{Zadatak 16}

U pravilnom tetraedru \(ABCD\) ivice su du\v zine \(a\).
Ta\v cke \(M,N,P,Q\) su sredi\v sta ivica \(DA,AB,BC,CD\).
Tra\v zi se povr\v sina \v cetvorougla \(MNPQ\).

\subsection*{Koordinate}
Postavimo pravilni tetraedar u prostoru standardno:
\[
A(0,0,0),\quad B(a,0,0),\quad
C\!\left(\frac a2,\frac{\sqrt3}{2}a,0\right),\quad
D\!\left(\frac a2,\frac{\sqrt3}{6}a,h\right),
\]
gde je visina tetraedra
\[
h=a\sqrt{\frac{2}{3}}.
\]

Sredi\v sta ivica su:
\[
N=\frac{A+B}{2}=\left(\frac a2,0,0\right),
\]
\[
P=\frac{B+C}{2}=\left(\frac{3a}{4},\frac{\sqrt3}{4}a,0\right),
\]
\[
M=\frac{D+A}{2}=\left(\frac a4,\frac{\sqrt3}{12}a,\frac h2\right),
\]
\[
Q=\frac{C+D}{2}=\left(\frac a2,\frac{\sqrt3}{3}a,\frac h2\right).
\]

\subsection*{Paralelogram i povr\v sina}
Vektori susednih stranica:
\[
\overrightarrow{MN}=N-M=\left(\frac a4,-\frac{\sqrt3}{12}a,-\frac h2\right),
\qquad
\overrightarrow{NP}=P-N=\left(\frac a4,\frac{\sqrt3}{4}a,0\right).
\]
Uo\v cava se i \(\overrightarrow{PQ}=-\overrightarrow{MN}\) i \(\overrightarrow{QM}=-\overrightarrow{NP}\),
pa je \(MNPQ\) paralelogram. Njegova povr\v sina je
\[
S=\bigl\lVert \overrightarrow{MN}\times \overrightarrow{NP}\bigr\rVert.
\]
Ra\v cunamo vektorski proizvod:
\[
\overrightarrow{MN}\times \overrightarrow{NP}
=\left(\frac{ha\sqrt3}{8},-\frac{ha}{8},\frac{a^2\sqrt3}{12}\right).
\]
Zato je
\[
S^2=\left(\frac{ha\sqrt3}{8}\right)^2+\left(\frac{ha}{8}\right)^2+\left(\frac{a^2\sqrt3}{12}\right)^2
=\frac{h^2a^2}{16}+\frac{a^4}{48}.
\]
Po\v sto je \(h^2=a^2\cdot\frac{2}{3}\), sledi
\[
S^2=\frac{a^4}{16}\cdot\frac{2}{3}+\frac{a^4}{48}=\frac{a^4}{24}+\frac{a^4}{48}=\frac{a^4}{16},
\]
pa je
\[
S=\frac{a^2}{4}.
\]

\subsection*{Odgovor}
\[
\boxed{\frac{a^2}{4}}\qquad\text{(A)}
\]

\end{document}

