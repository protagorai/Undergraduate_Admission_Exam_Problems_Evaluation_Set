\documentclass[12pt]{article}
\usepackage[utf8]{inputenc}
\usepackage[T1]{fontenc}
\usepackage{amsmath,amssymb}
\usepackage[margin=2.2cm]{geometry}

\begin{document}
\section*{Zadatak 12}

Kvadrat \(ABCD\) ima stranicu \(8\text{ cm}\).
Tra\v zi se povr\v sina kru\v znice koja prolazi kroz temena \(A\) i \(D\) i dodiruje stranicu \(BC\).

Postavimo koordinate:
\[
A(0,0),\quad B(8,0),\quad C(8,8),\quad D(0,8).
\]
Kru\v znica prolazi kroz \(A\) i \(D\), pa je sredi\v ste kru\v znice na simetrali du\v zi \(AD\).
Du\v z \(AD\) je vertikalna, sredina joj je \((0,4)\), pa je simetrala prava \(y=4\).
Neka je sredi\v ste \(O(h,4)\), a polupre\v cnik \(r\).

Po\v sto \(A\) le\v zi na kru\v znici,
\[
r^2=OA^2=h^2+4^2=h^2+16.
\]
Kru\v znica dodiruje pravu \(BC\), tj. liniju \(x=8\). Rastojanje centra do te prave je \(8-h\), i to mora biti jednako \(r\):
\[
8-h=r.
\]
Kvadriranjem:
\[
(8-h)^2=r^2=h^2+16.
\]
\[
64-16h+h^2=h^2+16 \quad\Rightarrow\quad 48=16h \quad\Rightarrow\quad h=3.
\]
Tada je
\[
r=8-h=5.
\]
Povr\v sina kruga je
\[
P=\pi r^2=25\pi\ \text{cm}^2.
\]

\subsection*{Odgovor}
\[
\boxed{25\pi\ \text{cm}^2}\qquad\text{(B)}
\]

\end{document}

