\documentclass[12pt]{article}
\usepackage[utf8]{inputenc}
\usepackage[T1]{fontenc}
\usepackage{amsmath,amssymb}
\usepackage[margin=2.2cm]{geometry}

\begin{document}
\section*{Zadatak 5}

Sa slike se vidi da parabola se\v ce \(x\)-osu u ta\v ckama \(x=-2\) i \(x=3\), a vrednost u temenu je \(\dfrac{25}{4}\).

Zato je
\[
f(x)=a(x+2)(x-3).
\]
Osa simetrije je sredina korena:
\[
x_v=\frac{-2+3}{2}=\frac12.
\]
Tada je
\[
f\!\left(\frac12\right)=a\left(\frac12+2\right)\left(\frac12-3\right)
=a\cdot\frac52\cdot\left(-\frac52\right)
=-a\cdot\frac{25}{4}.
\]
Po slici je \(f(x_v)=\dfrac{25}{4}\), pa sledi
\[
-a\cdot\frac{25}{4}=\frac{25}{4}\quad\Rightarrow\quad a=-1.
\]
Dalje,
\[
f(x)=-(x+2)(x-3)=-(x^2-x-6)=-x^2+x+6,
\]
pa su \(b=1\) i \(c=6\). Tra\v zi se \(a(b+c)\):
\[
a(b+c)=-1(1+6)=-7.
\]

\subsection*{Odgovor}
\[
\boxed{-7}\qquad\text{(D)}
\]

\end{document}

