\documentclass[12pt]{article}
\usepackage[utf8]{inputenc}
\usepackage[T1]{fontenc}
\usepackage{amsmath,amssymb}
\usepackage[margin=2.2cm]{geometry}

\begin{document}
\section*{Zadatak 20}

Tra\v zi se vrednost izraza
\[
\tan 10^\circ-\tan 50^\circ+\tan 70^\circ.
\]
Ozna\v cimo \(t=\tan 10^\circ\). Koristimo formule za tangens zbira i razlike sa \(60^\circ\):
\[
\tan 70^\circ=\tan(60^\circ+10^\circ)=\frac{\sqrt3+t}{1-\sqrt3\,t},
\]
\[
\tan 50^\circ=\tan(60^\circ-10^\circ)=\frac{\sqrt3-t}{1+\sqrt3\,t}.
\]
Ra\v cunamo razliku:
\[
\tan 70^\circ-\tan 50^\circ=
\frac{\sqrt3+t}{1-\sqrt3 t}-\frac{\sqrt3-t}{1+\sqrt3 t}
=\frac{(\sqrt3+t)(1+\sqrt3 t)-(\sqrt3-t)(1-\sqrt3 t)}{1-3t^2}.
\]
Brojilac je
\[
(\sqrt3+t)(1+\sqrt3 t)=\sqrt3+4t+\sqrt3 t^2,
\]
\[
(\sqrt3-t)(1-\sqrt3 t)=\sqrt3-4t+\sqrt3 t^2,
\]
pa je razlika \(8t\). Dakle
\[
\tan 70^\circ-\tan 50^\circ=\frac{8t}{1-3t^2}.
\]
Zato je ceo izraz
\[
t+\frac{8t}{1-3t^2}=t\left(1+\frac{8}{1-3t^2}\right)
=t\cdot\frac{9-3t^2}{1-3t^2}
=3\,t\cdot\frac{3-t^2}{1-3t^2}.
\]
Sada koristimo formulu za \(\tan 3\theta\):
\[
\tan 30^\circ=\tan(3\cdot 10^\circ)=\frac{3t-t^3}{1-3t^2}
=t\cdot\frac{3-t^2}{1-3t^2}.
\]
Po\v sto je \(\tan 30^\circ=\frac{1}{\sqrt3}\), dobijamo
\[
t\cdot\frac{3-t^2}{1-3t^2}=\frac{1}{\sqrt3}.
\]
Zato je
\[
3\,t\cdot\frac{3-t^2}{1-3t^2}=3\cdot\frac{1}{\sqrt3}=\sqrt3.
\]

\subsection*{Odgovor}
\[
\boxed{\sqrt3}\qquad\text{(E)}
\]

\end{document}

