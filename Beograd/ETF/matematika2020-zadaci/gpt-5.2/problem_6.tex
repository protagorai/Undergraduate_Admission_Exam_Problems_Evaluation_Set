\documentclass[12pt]{article}
\usepackage[utf8]{inputenc}
\usepackage[T1]{fontenc}
\usepackage{amsmath,amssymb}
\usepackage[margin=2.2cm]{geometry}

\begin{document}
\section*{Zadatak 6}

Dat je polinom
\[
P(x)=x^4-9x^2+18.
\]
Uvedimo smenu \(t=x^2\). Tada
\[
P(x)=t^2-9t+18=(t-3)(t-6).
\]
Zato su nule dobijene iz
\[
x^2=3 \quad\text{ili}\quad x^2=6,
\]
pa je skup korena \(\{-\sqrt6,-\sqrt3,\sqrt3,\sqrt6\}\).

Najmanji koren je \(-\sqrt6\), a najve\'ci \(\sqrt6\), pa je njihov proizvod
\[
(-\sqrt6)(\sqrt6)=-6.
\]

\subsection*{Odgovor}
\[
\boxed{-6}\qquad\text{(D)}
\]

\end{document}

