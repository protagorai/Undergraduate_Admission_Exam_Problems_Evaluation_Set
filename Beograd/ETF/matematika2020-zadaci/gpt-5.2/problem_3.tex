\documentclass[12pt]{article}
\usepackage[utf8]{inputenc}
\usepackage[T1]{fontenc}
\usepackage{amsmath,amssymb}
\usepackage[margin=2.2cm]{geometry}

\begin{document}
\section*{Zadatak 3}

Izraz je
\[
(4+\sqrt{15})(\sqrt{10}-\sqrt{6})\sqrt{\,4-\sqrt{15}\,}.
\]
Ozna\v cimo \(A=4+\sqrt{15}\) i \(B=4-\sqrt{15}\). Va\v zi
\[
AB=(4+\sqrt{15})(4-\sqrt{15})=16-15=1 \quad\Rightarrow\quad B=\frac1A.
\]
Zato je \(\sqrt{B}=\dfrac1{\sqrt{A}}\), pa se izraz svodi na
\[
A(\sqrt{10}-\sqrt{6})\cdot\frac1{\sqrt{A}}=\sqrt{A}\,(\sqrt{10}-\sqrt{6}).
\]
Dalje, uo\v cimo da je
\[
4+\sqrt{15}=\left(\sqrt{\frac52}+\sqrt{\frac32}\right)^2,
\]
jer
\[
\left(\sqrt{\frac52}+\sqrt{\frac32}\right)^2=\frac52+\frac32+2\sqrt{\frac{15}{4}}=4+\sqrt{15}.
\]
Dakle
\[
\sqrt{A}=\sqrt{\frac52}+\sqrt{\frac32}=\frac{\sqrt5+\sqrt3}{\sqrt2}.
\]
Tako dobijamo
\[
\sqrt{A}(\sqrt{10}-\sqrt{6})
\;=\;\frac{\sqrt5+\sqrt3}{\sqrt2}\cdot \sqrt2(\sqrt5-\sqrt3)
\;=\;(\sqrt5+\sqrt3)(\sqrt5-\sqrt3)
\;=\;5-3
\;=\;2.
\]

\subsection*{Odgovor}
\[
\boxed{2}\qquad\text{(C)}
\]

\end{document}

