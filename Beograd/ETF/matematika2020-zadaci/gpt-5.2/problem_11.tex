\documentclass[12pt]{article}
\usepackage[utf8]{inputenc}
\usepackage[T1]{fontenc}
\usepackage{amsmath,amssymb}
\usepackage[margin=2.2cm]{geometry}

\begin{document}
\section*{Zadatak 11}

Re\v savamo jedna\v cinu
\[
x+\log_{21}(3^x+1)=x\log_{21}7+\log_{21}756.
\]
Prebacimo \v clanove:
\[
x-x\log_{21}7=\log_{21}756-\log_{21}(3^x+1).
\]
Leva strana je
\[
x\left(1-\log_{21}7\right)=x\left(\log_{21}21-\log_{21}7\right)=x\log_{21}3.
\]
Desna strana je
\[
\log_{21}\frac{756}{3^x+1}.
\]
Dakle,
\[
x\log_{21}3=\log_{21}\frac{756}{3^x+1}.
\]
Po osobini \(x\log_a b=\log_a(b^x)\) dobijamo
\[
\log_{21}(3^x)=\log_{21}\frac{756}{3^x+1}
\quad\Rightarrow\quad
3^x=\frac{756}{3^x+1}.
\]
Ozna\v cimo \(y=3^x>0\). Tada
\[
y=\frac{756}{y+1}\quad\Rightarrow\quad y(y+1)=756
\quad\Rightarrow\quad y^2+y-756=0.
\]
\[
\Delta=1+4\cdot 756=3025=55^2,
\]
pa je
\[
y=\frac{-1\pm 55}{2}\in\{27,-28\}.
\]
Po\v sto je \(y>0\), sledi \(y=27\), tj.
\[
3^x=27=3^3\quad\Rightarrow\quad x=3.
\]
Broj \(3\) pripada intervalu \((0,21)\).

\subsection*{Odgovor}
\[
\boxed{(0,21)}\qquad\text{(C)}
\]

\end{document}

