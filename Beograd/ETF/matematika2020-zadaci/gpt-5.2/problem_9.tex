\documentclass[12pt]{article}
\usepackage[utf8]{inputenc}
\usepackage[T1]{fontenc}
\usepackage{amsmath,amssymb}
\usepackage[margin=2.2cm]{geometry}

\begin{document}
\section*{Zadatak 9}

Kre\'cemo od jednakostrani\v cnog trougla povr\v sine \(P\). Iz njega se izbacuje trougao \v cija su temena sredi\v sta stranica (medijalni trougao), pa se dobija figura \(F_1\).

\subsection*{Korak 1: povr\v sina \(F_1\)}
Medijalni trougao ima stranice duplo manje, pa je njegova povr\v sina \(\frac14 P\).
Zato je
\[
\operatorname{Pov}(F_1)=P-\frac14P=\frac34P.
\]

\subsection*{Op\v sti korak}
Svaki put se u svakom preostalom trouglu izbacuje njegov medijalni trougao, tj. u svakom trouglu se zadr\v zi \(\frac34\) povr\v sine.
Zato va\v zi rekurzija
\[
\operatorname{Pov}(F_{n})=\left(\frac34\right)^n P,\qquad n\in\mathbb{N}.
\]

\subsection*{Zbir povr\v sina}
Tra\v zi se
\[
\sum_{n=1}^{\infty}\operatorname{Pov}(F_n)=\sum_{n=1}^{\infty}P\left(\frac34\right)^n
=P\cdot\frac{\frac34}{1-\frac34}
=P\cdot 3
=3P.
\]

\subsection*{Odgovor}
\[
\boxed{3P}\qquad\text{(D)}
\]

\end{document}

