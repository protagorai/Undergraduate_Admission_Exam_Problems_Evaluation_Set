\documentclass[12pt]{article}
\usepackage[utf8]{inputenc}
\usepackage[T1]{fontenc}
\usepackage{amsmath,amssymb}
\usepackage[margin=2.2cm]{geometry}

\begin{document}
\section*{Zadatak 17}

Jedna\v cina je
\[
\bigl(\log_{\sin x} 2\bigr)\bigl(\log_{\sin^2 x}\tfrac{4}{3}\bigr)=2\log_{\frac43}2.
\]
\subsection*{Domen}
Da bi \(\log_{\sin x}2\) bio definisan, mora biti \(\sin x>0\) i \(\sin x\ne 1\).
Za \(\log_{\sin^2 x}\frac43\) mora biti \(\sin x\ne 0\) i \(\sin^2 x\ne 1\).
Dakle:
\[
\sin x\in(0,1).
\]

\subsection*{Svo\dj enje jedna\v cine}
Koristimo promenu osnove:
\[
\log_{\sin x}2=\frac{\ln 2}{\ln(\sin x)},\qquad
\log_{\sin^2 x}\frac43=\frac{\ln\frac43}{\ln(\sin^2 x)}=\frac{\ln\frac43}{2\ln(\sin x)}
\]
(jer je \(\sin x>0\)).
Zato je leva strana
\[
\frac{\ln2}{\ln(\sin x)}\cdot \frac{\ln\frac43}{2\ln(\sin x)}
=\frac{\ln2\cdot \ln\frac43}{2(\ln(\sin x))^2}.
\]
Desna strana je
\[
2\log_{\frac43}2=2\cdot\frac{\ln2}{\ln\frac43}.
\]
Izjedna\v cimo i skratimo \(\ln2\ne 0\):
\[
\frac{\ln\frac43}{2(\ln(\sin x))^2}= \frac{2}{\ln\frac43}
\quad\Rightarrow\quad
\left(\ln\frac43\right)^2=4(\ln(\sin x))^2.
\]
Zato je
\[
|\ln(\sin x)|=\frac12\ln\frac43=\ln\sqrt{\frac43}=\ln\frac{2}{\sqrt3}.
\]
Dakle,
\[
\ln(\sin x)=\pm \ln\frac{2}{\sqrt3}
\quad\Rightarrow\quad
\sin x\in\left\{\frac{2}{\sqrt3},\frac{\sqrt3}{2}\right\}.
\]
Po\v sto je \(\sin x\in(0,1)\), mogu\'ce je samo
\[
\sin x=\frac{\sqrt3}{2}.
\]
Re\v senja su
\[
x=\frac{\pi}{3}+2k\pi\quad\text{ili}\quad x=\frac{2\pi}{3}+2k\pi,\qquad k\in\mathbb{Z}.
\]

\subsection*{Uslov iz teksta}
Tra\v zi se skup \(S\) koji ima ta\v cno tri zajedni\v cka elementa sa skupom re\v senja.
U ponu\dj enim skupovima jedino (E) sadr\v zi ta\v cno tri ugla koja su kongruentna sa \(\pi/3\) ili \(2\pi/3\) po modulu \(2\pi\):
\[
\frac{2\pi}{3},\quad -\frac{4\pi}{3}=\frac{2\pi}{3}-2\pi,\quad -\frac{5\pi}{3}=\frac{\pi}{3}-2\pi.
\]

\subsection*{Odgovor}
\[
\boxed{S\ \text{je dat u (E)}}.
\]

\end{document}

