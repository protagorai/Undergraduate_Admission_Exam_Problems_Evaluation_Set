\documentclass[12pt]{article}
\usepackage[utf8]{inputenc}
\usepackage[T1]{fontenc}
\usepackage{amsmath,amssymb}
\usepackage[margin=2.2cm]{geometry}

\begin{document}
\section*{Zadatak 10}

Posmatramo jedna\v cinu
\[
2x^2-6x+\sqrt{x^2-3x+6}+2=0.
\]
Po\v sto je
\[
x^2-3x+6=\left(x-\frac32\right)^2+\frac{15}{4}>0,
\]
koren je definisan za svaki realan \(x\).

Prebacimo koren:
\[
\sqrt{x^2-3x+6}=-2x^2+6x-2=-2(x^2-3x+1).
\]
Leva strana je \(\ge 0\), pa mora va\v ziti
\[
-2(x^2-3x+1)\ge 0 \quad\Rightarrow\quad x^2-3x+1\le 0.
\]
Ozna\v cimo
\[
t=x^2-3x+6.
\]
Tada je \(x^2-3x+1=t-5\), pa jedna\v cina postaje
\[
\sqrt{t}= -2(t-5)=10-2t.
\]
Odavde je neophodno \(10-2t\ge 0\Rightarrow t\le 5\).

Sada kvadriramo:
\[
t=(10-2t)^2=4(5-t)^2=4(t^2-10t+25)=4t^2-40t+100,
\]
pa
\[
4t^2-41t+100=0.
\]
Diskriminanta je
\[
\Delta=41^2-4\cdot 4\cdot 100=1681-1600=81,
\]
pa su re\v senja
\[
t=\frac{41\pm 9}{8}\in\left\{\frac{25}{4},\,4\right\}.
\]
Uslov \(t\le 5\) daje jedino \(t=4\).

Dakle
\[
x^2-3x+6=4\quad\Rightarrow\quad x^2-3x+2=0
\quad\Rightarrow\quad (x-1)(x-2)=0,
\]
pa su kandidati \(x=1\) i \(x=2\). Provera u po\v cetnoj jedna\v cini daje da oba zadovoljavaju.

Zato broj realnih re\v senja iznosi \(2\).

\subsection*{Odgovor}
\[
\boxed{2}\qquad\text{(A)}
\]

\end{document}

