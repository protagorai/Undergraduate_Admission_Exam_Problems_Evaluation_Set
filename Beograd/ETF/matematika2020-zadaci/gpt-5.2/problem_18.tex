\documentclass[12pt]{article}
\usepackage[utf8]{inputenc}
\usepackage[T1]{fontenc}
\usepackage{amsmath,amssymb}
\usepackage[margin=2.2cm]{geometry}

\begin{document}
\section*{Zadatak 18}

U tetivnom (cikli\v cnom) \v cetvorouglu \(ABCD\) va\v zi:
\[
BD\perp BC,\qquad \angle ABC=120^\circ,\qquad \angle BAD=120^\circ,\qquad AD=1\text{ cm}.
\]
Tra\v zi se proizvod \(BD\cdot CD\).

\subsection*{1) Trougao \(ABD\)}
Po\v sto je \(BD\perp BC\), ugao izme\dj u \(BA\) i \(BD\) je
\[
\angle ABD = |\angle ABC-90^\circ|=|120^\circ-90^\circ|=30^\circ.
\]
U trouglu \(ABD\) va\v zi \(\angle BAD=120^\circ\), pa su uglovi:
\[
\angle A=120^\circ,\quad \angle B=30^\circ,\quad \angle D=30^\circ.
\]
Dakle, trougao \(ABD\) je jednakokrak sa \(AB=AD\). Po\v sto je \(AD=1\), sledi \(AB=1\).

Sada po kosinusnom teoremu:
\[
BD^2=AB^2+AD^2-2\cdot AB\cdot AD\cos 120^\circ
=1+1-2\cdot 1\cdot 1\cdot\left(-\frac12\right)=3,
\]
pa je
\[
BD=\sqrt3.
\]

\subsection*{2) Trougao \(BCD\)}
Po\v sto je \v cetvorougao cikli\v can, suprotni uglovi su suplementarni:
\[
\angle BAD+\angle BCD=180^\circ \Rightarrow \angle BCD=60^\circ.
\]
Imamo i \(\angle CBD=90^\circ\) (jer \(BD\perp BC\)), pa je trougao \(BCD\) pravougli sa uglovima
\[
\angle B=90^\circ,\quad \angle C=60^\circ,\quad \angle D=30^\circ.
\]
U trouglu \(30^\circ\!-\!60^\circ\!-\!90^\circ\), stranica naspram \(60^\circ\) je \(\frac{\sqrt3}{2}\) puta hipotenuza.
Ovde je \(BD\) naspram \(60^\circ\), a \(CD\) je hipotenuza, pa va\v zi
\[
BD=\frac{\sqrt3}{2}CD.
\]
Po\v sto je \(BD=\sqrt3\), dobijamo
\[
\sqrt3=\frac{\sqrt3}{2}CD \Rightarrow CD=2.
\]

\subsection*{Proizvod}
\[
BD\cdot CD=\sqrt3\cdot 2=2\sqrt3\ \text{cm}^2.
\]

\subsection*{Odgovor}
\[
\boxed{2\sqrt3\ \text{cm}^2}\qquad\text{(C)}
\]

\end{document}

