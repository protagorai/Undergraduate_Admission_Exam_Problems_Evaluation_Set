\documentclass[12pt]{article}
\usepackage[margin=1in]{geometry}
\usepackage{amsmath,amssymb}
\begin{document}

\section*{Problem 3}
Vrednost izraza $(4+\sqrt{15})(\sqrt{10}-\sqrt{6})\sqrt{4-\sqrt{15}}$ jednaka je:
\[
(A)\ \frac{1}{2}\quad (B)\ 1\quad (C)\ 2\quad (D)\ \sqrt{2}\quad (E)\ \frac{\sqrt{2}}{2}\quad (N)\ \text{Ne znam}
\]

\subsection*{Solution}
Posmatrajmo treći činilac $\sqrt{4-\sqrt{15}}$. Možemo ga zapisati kao:
\[
\sqrt{4-\sqrt{15}} = \sqrt{\frac{8-2\sqrt{15}}{2}} = \frac{\sqrt{5-2\sqrt{15}+3}}{\sqrt{2}} = \frac{\sqrt{(\sqrt{5}-\sqrt{3})^2}}{\sqrt{2}} = \frac{\sqrt{5}-\sqrt{3}}{\sqrt{2}}
\]
Drugi činilac je:
\[
\sqrt{10}-\sqrt{6} = \sqrt{2}(\sqrt{5}-\sqrt{3})
\]
Množenjem drugog i trećeg činioca:
\[
(\sqrt{10}-\sqrt{6})\sqrt{4-\sqrt{15}} = \sqrt{2}(\sqrt{5}-\sqrt{3}) \cdot \frac{\sqrt{5}-\sqrt{3}}{\sqrt{2}} = (\sqrt{5}-\sqrt{3})^2 = 5 - 2\sqrt{15} + 3 = 8 - 2\sqrt{15}
\]
Ceo izraz postaje:
\[
(4+\sqrt{15})(8-2\sqrt{15}) = 2(4+\sqrt{15})(4-\sqrt{15}) = 2(16-15) = 2
\]

\subsection*{Answer}
$2$ (option \textbf{C}).

\end{document}