\documentclass[12pt]{article}
\usepackage[margin=1in]{geometry}
\usepackage{amsmath,amssymb}
\begin{document}

\section*{Problem 17}
Skup rešenja jednačine $(\log_{\sin x} 2)(\log_{\sin^2 x} \frac{4}{3}) = 2 \log_{\frac{4}{3}} 2$ ima tačno tri zajednička elementa sa skupom $S$, ako je:
\[
(A)\ S=\{\frac{3\pi}{4}, \frac{5\pi}{4}, \frac{7\pi}{4}, \frac{11\pi}{4}, \frac{15\pi}{4}\}
\]
\[
(B)\ S=\{\frac{\pi}{6}, \frac{5\pi}{6}, \frac{\pi}{3}, \frac{2\pi}{3}, \frac{7\pi}{6}, \frac{11\pi}{6}\}
\]
\[
(C)\ S=\{\frac{\pi}{3}, \frac{2\pi}{3}, -\frac{\pi}{3}, -\frac{2\pi}{3}, \frac{4\pi}{3}, \frac{5\pi}{3}\}
\]
\[
(D)\ S=\{\frac{\pi}{3}, \frac{7\pi}{3}, \frac{13\pi}{3}, \frac{19\pi}{3}, \frac{25\pi}{3}, \frac{31\pi}{3}\}
\]
\[
(E)\ S=\{\frac{2\pi}{3}, -\frac{4\pi}{3}, -\frac{5\pi}{3}, -\frac{2\pi}{3}, \frac{4\pi}{3}, \frac{5\pi}{3}\}
\]

\subsection*{Solution}
Uslov definisanosti logaritma: $\sin x > 0$ i $\sin x \ne 1$.
Transformišimo jednačinu. Koristimo $\log_{a^2} b = \frac{1}{2} \log_a b$.
Leva strana:
\[
L = \log_{\sin x} 2 \cdot \frac{1}{2} \log_{\sin x} \frac{4}{3} = \frac{1}{2} \log_{\sin x} 2 \cdot \log_{\sin x} \frac{4}{3}
\]
Desna strana:
\[
D = 2 \log_{\frac{4}{3}} 2 = \frac{2}{\log_2 \frac{4}{3}}
\]
Neka je $A = \log_{\sin x} 2$ i $B = \log_{\sin x} \frac{4}{3}$.
Veza između $A$ i $B$: $A/B = \log_{4/3} 2 = k$. Desna strana je $2k$.
Jednačina je $\frac{1}{2} A B = 2k$.
Zamenimo $A = kB$:
\[
\frac{1}{2} k B^2 = 2k \implies B^2 = 4 \implies B = \pm 2
\]
Slučaj 1: $\log_{\sin x} \frac{4}{3} = 2 \implies \sin^2 x = \frac{4}{3}$. Ovo nije moguće jer je $\sin^2 x \le 1$.
Slučaj 2: $\log_{\sin x} \frac{4}{3} = -2 \implies \sin^{-2} x = \frac{4}{3} \implies \sin^2 x = \frac{3}{4}$.
Rešenja su $\sin x = \frac{\sqrt{3}}{2}$ ili $\sin x = -\frac{\sqrt{3}}{2}$.
Zbog uslova $\sin x > 0$, imamo samo $\sin x = \frac{\sqrt{3}}{2}$.
Rešenja su oblika $x \in \{\frac{\pi}{3} + 2k\pi, \frac{2\pi}{3} + 2k\pi\}$.
Skup svih rešenja je $R = \{ \frac{\pi}{3}, \frac{2\pi}{3}, \frac{7\pi}{3}, \frac{8\pi}{3}, \dots, -\frac{4\pi}{3}, -\frac{5\pi}{3}, \dots \}$.
Proverimo presek sa ponuđenim skupovima $S$:
Skup (E) sadrži:
$\frac{2\pi}{3}$ (rešenje)
$-\frac{4\pi}{3} = \frac{2\pi}{3} - 2\pi$ (rešenje)
$-\frac{5\pi}{3} = \frac{\pi}{3} - 2\pi$ (rešenje)
$-\frac{2\pi}{3}$ (nije rešenje, sinus je negativan)
$\frac{4\pi}{3}$ (nije rešenje, sinus je negativan)
$\frac{5\pi}{3}$ (nije rešenje, sinus je negativan)
Presek sadrži tačno 3 elementa: $\frac{2\pi}{3}, -\frac{4\pi}{3}, -\frac{5\pi}{3}$.

\subsection*{Answer}
$S=\{\frac{2\pi}{3}, -\frac{4\pi}{3}, -\frac{5\pi}{3}, -\frac{2\pi}{3}, \frac{4\pi}{3}, \frac{5\pi}{3}\}$ (option \textbf{E}).

\end{document}