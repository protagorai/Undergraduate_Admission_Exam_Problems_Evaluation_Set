\documentclass[12pt]{article}
\usepackage[margin=1in]{geometry}
\usepackage{amsmath,amssymb}
\begin{document}

\section*{Problem 20}
Vrednost izraza $\text{tg} 10^\circ - \text{tg} 50^\circ + \text{tg} 70^\circ$ jeste:
\[
(A)\ \frac{\sqrt{3}}{3}\quad (B)\ 3\sqrt{3}\quad (C)\ 3\quad (D)\ -1\quad (E)\ \sqrt{3}\quad (N)\ \text{Ne znam}
\]

\subsection*{Solution}
Koristimo identitet $\text{tg}(60^\circ-x) - \text{tg}(60^\circ+x) + \text{tg} x$ ili slične veze.
Poznat je identitet $\text{tg} 3x = \text{tg} x \cdot \text{tg}(60^\circ-x) \cdot \text{tg}(60^\circ+x)$.
Ovde imamo zbir/razliku.
Posmatrajmo $\text{tg} 70^\circ - \text{tg} 50^\circ$:
\[
\frac{\sin(70^\circ-50^\circ)}{\cos 70^\circ \cos 50^\circ} = \frac{\sin 20^\circ}{\sin 20^\circ \cos 50^\circ} = \frac{1}{\cos 50^\circ}
\]
(koristili smo $\cos 70^\circ = \sin 20^\circ$).
Izraz postaje: $\text{tg} 10^\circ + \frac{1}{\cos 50^\circ} = \frac{\sin 10^\circ}{\cos 10^\circ} + \frac{1}{\sin 40^\circ}$.
\[
\frac{\sin 10^\circ \sin 40^\circ + \cos 10^\circ}{\cos 10^\circ \sin 40^\circ}
\]
Brojilac: $\frac{1}{2}(\cos 30^\circ - \cos 50^\circ) + \cos 10^\circ = \frac{\sqrt{3}}{4} - \frac{1}{2}\sin 40^\circ + \sin 80^\circ$.
Ovaj put je komplikovan.
Probajmo identitet $\text{tg} 3x = \text{tg} x - \text{tg}(60^\circ-x) + \text{tg}(60^\circ+x)$? Ne, to nije tačno.
Ali znamo da je $\text{tg} 30^\circ = \text{tg} 10^\circ - \text{tg} 50^\circ + \text{tg} 70^\circ$ netačno.
Tačan identitet je:
\[
\text{tg}(3x) = \frac{\text{tg} x - \text{tg}^3 x}{1 - 3\text{tg}^2 x}
\]
Postoji identitet: $3 \text{tg} 3x = \text{tg} x + \text{tg}(60^\circ+x) - \text{tg}(60^\circ-x)$? Proverimo za $x=10$.
$3 \text{tg} 30^\circ = 3 \frac{\sqrt{3}}{3} = \sqrt{3}$.
Desna strana: $\text{tg} 10^\circ + \text{tg} 70^\circ - \text{tg} 50^\circ$.
Ovo je upravo naš izraz.
Dokaz identiteta:
\[
\text{tg}(60+x) - \text{tg}(60-x) = \frac{\sin(2x)}{\cos(60+x)\cos(60-x)} = \frac{\sin 2x}{\frac{1}{2}(\cos 2x - \frac{1}{2})} = \frac{4\sin 2x}{2\cos 2x - 1}
\]
Dodamo $\text{tg} x = \frac{\sin x}{\cos x}$:
\[
\frac{\sin x(2\cos 2x - 1) + 4\sin 2x \cos x}{\cos x(2\cos 2x - 1)}
\]
Brojilac: $2\sin x(1-2\sin^2 x) - \sin x + 8\sin x \cos^2 x = 2\sin x - 4\sin^3 x - \sin x + 8\sin x(1-\sin^2 x) = \sin x - 4\sin^3 x + 8\sin x - 8\sin^3 x = 9\sin x - 12\sin^3 x$.
Imenilac: $\cos x(2(2\cos^2 x - 1) - 1) = \cos x(4\cos^2 x - 3) = 4\cos^3 x - 3\cos x$.
Odnos: $\frac{3(3\sin x - 4\sin^3 x)}{4\cos^3 x - 3\cos x} = \frac{3 \sin 3x}{\cos 3x} = 3 \text{tg} 3x$.
Dakle, identitet važi.
Vrednost izraza je $3 \text{tg} 30^\circ = 3 \frac{\sqrt{3}}{3} = \sqrt{3}$.

\subsection*{Answer}
$\sqrt{3}$ (option \textbf{E}).

\end{document}