\documentclass[12pt]{article}
\usepackage[margin=1in]{geometry}
\usepackage{amsmath,amssymb}
\begin{document}

\section*{Problem 12}
The side length of square $ABCD$ is 8 cm. A circle passes through vertices $A$ and $D$ and touches side $BC$. Find the area of the circle.

\subsection*{Solution}
Let's set up a coordinate system with $A$ at the origin, $B$ at $(8,0)$, $C$ at $(8,8)$, and $D$ at $(0,8)$.

The circle passes through $A(0,0)$ and $D(0,8)$, and touches side $BC$ (the line $x = 8$).

Since the circle passes through $A$ and $D$, its center lies on the perpendicular bisector of segment $AD$. The midpoint of $AD$ is $(0,4)$, and since $AD$ is vertical, the perpendicular bisector is the horizontal line $y = 4$.

So the center of the circle is at $(h, 4)$ for some value $h$.

The radius of the circle is the distance from the center $(h, 4)$ to point $A(0, 0)$:
\[
r = \sqrt{h^2 + 16}
\]

Since the circle touches the line $x = 8$, the distance from the center to this line equals the radius:
\[
|8 - h| = r = \sqrt{h^2 + 16}
\]

Since the circle passes through points on the left side of the square and touches the right side, we expect $h < 8$, so $|8 - h| = 8 - h$.

Therefore:
\[
8 - h = \sqrt{h^2 + 16}
\]

Squaring both sides:
\[
(8 - h)^2 = h^2 + 16
\]

\[
64 - 16h + h^2 = h^2 + 16
\]

\[
64 - 16h = 16
\]

\[
48 = 16h
\]

\[
h = 3
\]

So the center is at $(3, 4)$ and the radius is:
\[
r = \sqrt{3^2 + 4^2} = \sqrt{9 + 16} = \sqrt{25} = 5
\]

The area of the circle is:
\[
A = \pi r^2 = \pi \cdot 5^2 = 25\pi \text{ cm}^2
\]

\subsection*{Answer}
$25\pi$ cm² (option \textbf{E}).

\end{document}