\documentclass[12pt]{article}
\usepackage[margin=1in]{geometry}
\usepackage{amsmath,amssymb}
\begin{document}

\section*{Problem 11}
Solve the equation $x + \log_{21}(3^x + 1) = x \log_{21} 7 + \log_{21} 756$ and find the interval containing the solution.

\subsection*{Solution}
Let's rearrange the equation:
\[
x + \log_{21}(3^x + 1) = x \log_{21} 7 + \log_{21} 756
\]

\[
x - x \log_{21} 7 = \log_{21} 756 - \log_{21}(3^x + 1)
\]

\[
x(1 - \log_{21} 7) = \log_{21} \frac{756}{3^x + 1}
\]

Since $\log_{21} 7 = \log_{21} \frac{21}{3} = \log_{21} 21 - \log_{21} 3 = 1 - \log_{21} 3$:

\[
x \log_{21} 3 = \log_{21} \frac{756}{3^x + 1}
\]

\[
\log_{21} 3^x = \log_{21} \frac{756}{3^x + 1}
\]

Therefore:
\[
3^x = \frac{756}{3^x + 1}
\]

\[
3^x(3^x + 1) = 756
\]

\[
(3^x)^2 + 3^x = 756
\]

Let $y = 3^x$. Then:
\[
y^2 + y - 756 = 0
\]

Using the quadratic formula:
\[
y = \frac{-1 \pm \sqrt{1 + 4 \cdot 756}}{2} = \frac{-1 \pm \sqrt{3025}}{2} = \frac{-1 \pm 55}{2}
\]

So $y = 27$ or $y = -28$.

Since $y = 3^x > 0$, we have $y = 27$.

Therefore: $3^x = 27 = 3^3$, so $x = 3$.

Let's verify: $756 = 27 \cdot 28 = 27(27 + 1) = 3^3(3^3 + 1)$ ✓

Let's also verify in the original equation:
- Left side: $3 + \log_{21}(27 + 1) = 3 + \log_{21} 28$
- Right side: $3 \log_{21} 7 + \log_{21} 756 = 3 \log_{21} 7 + \log_{21}(27 \cdot 28) = 3 \log_{21} 7 + \log_{21} 27 + \log_{21} 28$

Since $27 = 3^3$ and $21 = 3 \cdot 7$:
$\log_{21} 27 = 3 \log_{21} 3$ and $\log_{21} 7 + \log_{21} 3 = \log_{21} 21 = 1$

So $\log_{21} 7 = 1 - \log_{21} 3$ and $\log_{21} 27 = 3 \log_{21} 3$.

The verification becomes complex, but our solution $x = 3$ is correct.

Since $x = 3$, it belongs to the interval $(0, 21)$.

\subsection*{Answer}
$(0, 21)$ (option \textbf{C}).

\end{document}