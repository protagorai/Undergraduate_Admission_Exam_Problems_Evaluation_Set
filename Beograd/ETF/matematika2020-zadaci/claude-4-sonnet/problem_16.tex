\documentclass[12pt]{article}
\usepackage[margin=1in]{geometry}
\usepackage{amsmath,amssymb}
\begin{document}

\section*{Problem 16}
If points $M$, $N$, $P$, and $Q$ are the midpoints of edges $DA$, $AB$, $BC$, and $CD$ of regular tetrahedron $ABCD$, then the surface area of quadrilateral $MNPQ$ is:

\subsection*{Solution}
Let's place the regular tetrahedron in a coordinate system. For a regular tetrahedron with edge length $a$, we can use coordinates:
\begin{align}
A &= (1, 1, 1)\\
B &= (1, -1, -1)\\
C &= (-1, 1, -1)\\
D &= (-1, -1, 1)
\end{align}

These coordinates give a regular tetrahedron with edge length $a = 2\sqrt{2}$.

The midpoints are:
\begin{align}
M &= \frac{D + A}{2} = (0, 0, 1)\\
N &= \frac{A + B}{2} = (1, 0, 0)\\
P &= \frac{B + C}{2} = (0, 0, -1)\\
Q &= \frac{C + D}{2} = (-1, 0, 0)
\end{align}

We can see that $MNPQ$ forms a square in the plane $y = 0$ with vertices at $(0, 0, 1)$, $(1, 0, 0)$, $(0, 0, -1)$, and $(-1, 0, 0)$.

The side length of this square is:
\[
|MN| = |(0,0,1) - (1,0,0)| = |(-1,0,1)| = \sqrt{1 + 0 + 1} = \sqrt{2}
\]

For a regular tetrahedron with edge length $a$, the edge length becomes $a = 2\sqrt{2}$, so the side length of the square $MNPQ$ is $\frac{a}{\sqrt{2}} = \frac{2\sqrt{2}}{\sqrt{2}} = 2$.

Actually, let me recalculate more carefully. For a regular tetrahedron with edge length $a$, the midpoints of the edges form a square with side length $\frac{a}{\sqrt{2}}$.

The area of this square is:
\[
\text{Area} = \left(\frac{a}{\sqrt{2}}\right)^2 = \frac{a^2}{2}
\]

For the given answer choices, if we assume the tetrahedron has edge length $a$, then:
- Option (A): $\frac{a^2}{4}$
- Option (B): $\frac{a^2\sqrt{3}}{4}$  
- Option (C): $\frac{a^2\sqrt{2}}{2}$
- Option (D): $\frac{a^2}{2}$
- Option (E): $\frac{a^2\sqrt{3}}{3}$

The correct answer is $\frac{a^2}{2}$.

\subsection*{Answer}
$\frac{a^2}{2}$ (option \textbf{D}).

\end{document}