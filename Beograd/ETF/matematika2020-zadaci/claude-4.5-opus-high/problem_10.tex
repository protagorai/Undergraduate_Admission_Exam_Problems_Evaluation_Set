\documentclass[12pt]{article}
\usepackage[margin=1in]{geometry}
\usepackage{amsmath,amssymb}
\begin{document}

\section*{Problem 10}
Find the number of real solutions of the equation $2x^2 - 6x + \sqrt{x^2 - 3x + 6} + 2 = 0$.

\subsection*{Solution}
Let's denote $u = \sqrt{x^2 - 3x + 6}$. Note that $u \geq 0$.

First, let's check if $x^2 - 3x + 6$ is always positive:
\[
x^2 - 3x + 6 = \left(x - \frac{3}{2}\right)^2 + 6 - \frac{9}{4} = \left(x - \frac{3}{2}\right)^2 + \frac{15}{4} > 0
\]

So $u = \sqrt{x^2 - 3x + 6}$ is always well-defined and positive.

The equation becomes:
\[
2x^2 - 6x + u + 2 = 0
\]

Note that $2x^2 - 6x = 2(x^2 - 3x)$. Also, $u^2 = x^2 - 3x + 6$, so $x^2 - 3x = u^2 - 6$.

Substituting:
\[
2(u^2 - 6) + u + 2 = 0
\]
\[
2u^2 - 12 + u + 2 = 0
\]
\[
2u^2 + u - 10 = 0
\]

Using the quadratic formula:
\[
u = \frac{-1 \pm \sqrt{1 + 80}}{4} = \frac{-1 \pm \sqrt{81}}{4} = \frac{-1 \pm 9}{4}
\]

So $u = \frac{8}{4} = 2$ or $u = \frac{-10}{4} = -\frac{5}{2}$.

Since $u \geq 0$, we have $u = 2$.

Now solve $\sqrt{x^2 - 3x + 6} = 2$:
\[
x^2 - 3x + 6 = 4
\]
\[
x^2 - 3x + 2 = 0
\]
\[
(x - 1)(x - 2) = 0
\]

So $x = 1$ or $x = 2$.

Let's verify both solutions in the original equation:

For $x = 1$:
\[
2(1)^2 - 6(1) + \sqrt{1 - 3 + 6} + 2 = 2 - 6 + \sqrt{4} + 2 = 2 - 6 + 2 + 2 = 0 \checkmark
\]

For $x = 2$:
\[
2(4) - 6(2) + \sqrt{4 - 6 + 6} + 2 = 8 - 12 + \sqrt{4} + 2 = 8 - 12 + 2 + 2 = 0 \checkmark
\]

Both solutions are valid.

\subsection*{Answer}
$2$ (option \textbf{A}).

\end{document}
