\documentclass[12pt]{article}
\usepackage[margin=1in]{geometry}
\usepackage{amsmath,amssymb}
\begin{document}

\section*{Problem 12}
The side length of square $ABCD$ is 8 cm. A circle passes through vertices $A$ and $D$ and is tangent to side $BC$. Find the area of this circle.

\subsection*{Solution}
Let's set up a coordinate system with $A$ at the origin:
\begin{itemize}
    \item $A = (0, 0)$
    \item $B = (8, 0)$
    \item $C = (8, 8)$
    \item $D = (0, 8)$
\end{itemize}

Let the center of the circle be at $(h, k)$ with radius $r$.

Since the circle passes through $A(0, 0)$ and $D(0, 8)$:
\[
h^2 + k^2 = r^2 \quad \text{(distance from center to } A \text{)}
\]
\[
h^2 + (k - 8)^2 = r^2 \quad \text{(distance from center to } D \text{)}
\]

From these two equations:
\[
h^2 + k^2 = h^2 + (k - 8)^2
\]
\[
k^2 = k^2 - 16k + 64
\]
\[
16k = 64 \implies k = 4
\]

So the center has $y$-coordinate 4 (which makes sense by symmetry since $A$ and $D$ are symmetric about $y = 4$).

The circle is tangent to line $BC$, which is the vertical line $x = 8$.

The distance from the center $(h, 4)$ to the line $x = 8$ equals the radius:
\[
|8 - h| = r
\]

Since the circle passes through $A(0, 0)$ and has center $(h, 4)$:
\[
r^2 = h^2 + 16
\]

From $|8 - h| = r$, and since the center must be to the left of $x = 8$ for the circle to be tangent (not intersecting), we have $h < 8$, so $r = 8 - h$.

Substituting into $r^2 = h^2 + 16$:
\[
(8 - h)^2 = h^2 + 16
\]
\[
64 - 16h + h^2 = h^2 + 16
\]
\[
64 - 16h = 16
\]
\[
48 = 16h \implies h = 3
\]

Therefore $r = 8 - 3 = 5$.

The area of the circle is:
\[
\pi r^2 = \pi \cdot 25 = 25\pi \text{ cm}^2
\]

\subsection*{Answer}
$25\pi$ cm$^2$ (option \textbf{B}).

\end{document}
