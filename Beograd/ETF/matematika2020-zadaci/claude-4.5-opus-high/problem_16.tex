\documentclass[12pt]{article}
\usepackage[margin=1in]{geometry}
\usepackage{amsmath,amssymb}
\begin{document}

\section*{Problem 16}
If points $M$, $N$, $P$, and $Q$ are the midpoints of edges $DA$, $AB$, $BC$, and $CD$ of a regular tetrahedron $ABCD$ with edge length $a$, find the area of quadrilateral $MNPQ$.

\subsection*{Solution}
In a regular tetrahedron, all edges have length $a$. Let's find the shape and area of quadrilateral $MNPQ$.

The midpoints are:
\begin{itemize}
    \item $M$ = midpoint of $DA$
    \item $N$ = midpoint of $AB$
    \item $P$ = midpoint of $BC$
    \item $Q$ = midpoint of $CD$
\end{itemize}

By the midpoint theorem, in any triangle, the segment connecting midpoints of two sides is parallel to the third side and has half its length.

In triangle $DAB$:
\begin{itemize}
    \item $MN$ connects midpoints of $DA$ and $AB$
    \item $MN \parallel DB$ and $MN = \frac{1}{2}DB = \frac{a}{2}$ (since $DB = a$)
\end{itemize}

In triangle $BCD$:
\begin{itemize}
    \item $PQ$ connects midpoints of $BC$ and $CD$
    \item $PQ \parallel DB$ and $PQ = \frac{1}{2}DB = \frac{a}{2}$
\end{itemize}

So $MN \parallel PQ$ and $MN = PQ = \frac{a}{2}$.

In triangle $ABC$:
\begin{itemize}
    \item $NP$ connects midpoints of $AB$ and $BC$
    \item $NP \parallel AC$ and $NP = \frac{1}{2}AC = \frac{a}{2}$
\end{itemize}

In triangle $DCA$:
\begin{itemize}
    \item $QM$ connects midpoints of $CD$ and $DA$
    \item $QM \parallel AC$ and $QM = \frac{1}{2}AC = \frac{a}{2}$
\end{itemize}

So $NP \parallel QM$ and $NP = QM = \frac{a}{2}$.

Therefore, $MNPQ$ is a parallelogram with all sides equal to $\frac{a}{2}$, making it a rhombus.

To find the area, we need the diagonals $MP$ and $NQ$.

$MP$ connects midpoints of $DA$ and $BC$. In the tetrahedron, $DA$ and $BC$ are opposite edges (they don't share a vertex). The distance between midpoints of opposite edges in a regular tetrahedron with edge $a$ is $\frac{a}{\sqrt{2}}$.

Similarly, $NQ$ connects midpoints of $AB$ and $CD$ (another pair of opposite edges), so $NQ = \frac{a}{\sqrt{2}}$.

Since $MP = NQ = \frac{a}{\sqrt{2}}$, and the diagonals of the rhombus bisect each other at right angles (which can be verified by symmetry of the tetrahedron), the area is:
\[
\text{Area} = \frac{1}{2} \cdot MP \cdot NQ = \frac{1}{2} \cdot \frac{a}{\sqrt{2}} \cdot \frac{a}{\sqrt{2}} = \frac{1}{2} \cdot \frac{a^2}{2} = \frac{a^2}{4}
\]

\subsection*{Answer}
$\frac{a^2}{4}$ (option \textbf{A}).

\end{document}
