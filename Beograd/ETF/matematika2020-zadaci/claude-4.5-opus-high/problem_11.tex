\documentclass[12pt]{article}
\usepackage[margin=1in]{geometry}
\usepackage{amsmath,amssymb}
\begin{document}

\section*{Problem 11}
The real solution of the equation $x + \log_{21}(3^x + 1) = x \log_{21} 7 + \log_{21} 756$ belongs to the interval:

\subsection*{Solution}
Starting with the equation:
\[
x + \log_{21}(3^x + 1) = x \log_{21} 7 + \log_{21} 756
\]

Note that $\log_{21} 21 = 1$, so $x = x \cdot \log_{21} 21 = \log_{21} 21^x$.

The equation becomes:
\[
\log_{21} 21^x + \log_{21}(3^x + 1) = \log_{21} 7^x + \log_{21} 756
\]

Using logarithm properties:
\[
\log_{21}[21^x(3^x + 1)] = \log_{21}[756 \cdot 7^x]
\]

Therefore:
\[
21^x(3^x + 1) = 756 \cdot 7^x
\]

Since $21 = 3 \cdot 7$, we have $21^x = 3^x \cdot 7^x$:
\[
3^x \cdot 7^x(3^x + 1) = 756 \cdot 7^x
\]

Dividing both sides by $7^x$ (which is always positive):
\[
3^x(3^x + 1) = 756
\]

Let $t = 3^x$, where $t > 0$:
\[
t(t + 1) = 756
\]
\[
t^2 + t - 756 = 0
\]

Using the quadratic formula:
\[
t = \frac{-1 \pm \sqrt{1 + 3024}}{2} = \frac{-1 \pm \sqrt{3025}}{2} = \frac{-1 \pm 55}{2}
\]

So $t = \frac{54}{2} = 27$ or $t = \frac{-56}{2} = -28$.

Since $t = 3^x > 0$, we have $t = 27 = 3^3$.

Therefore $3^x = 3^3$, which gives $x = 3$.

Let's verify: $x = 3$ should belong to one of the intervals.

Checking the intervals:
\begin{itemize}
    \item (A) $(-\infty, -21]$ - No
    \item (B) $(-21, 0]$ - No
    \item (C) $(0, 21)$ - Yes, $3 \in (0, 21)$
    \item (D) $[21, 42)$ - No
    \item (E) $[42, +\infty)$ - No
\end{itemize}

\subsection*{Answer}
$(0, 21)$ (option \textbf{C}).

\end{document}
