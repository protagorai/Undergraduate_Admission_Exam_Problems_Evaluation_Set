\documentclass[12pt]{article}
\usepackage[margin=1in]{geometry}
\usepackage{amsmath,amssymb}
\begin{document}

\section*{Problem 17}
U jednakokrakom trouglu $ABC$ je $AB = BC = b$, $AC = a$ i $\angle ABC = 20°$. Tada je izraz $\dfrac{a^2}{b^2} + \dfrac{b}{a}$ jednak:

\subsection*{Solution}
In isosceles triangle $ABC$ with $AB = BC = b$, $AC = a$, and $\angle ABC = 20°$.

The base angles are equal: $\angle BAC = \angle BCA = \dfrac{180° - 20°}{2} = 80°$.

Using the Law of Cosines:
\[
a^2 = b^2 + b^2 - 2b^2\cos(20°) = 2b^2(1 - \cos 20°)
\]

So:
\[
\frac{a^2}{b^2} = 2(1 - \cos 20°) = 4\sin^2(10°)
\]

Using the Law of Sines:
\[
\frac{a}{\sin 20°} = \frac{b}{\sin 80°}
\]
\[
\frac{a}{b} = \frac{\sin 20°}{\sin 80°} = \frac{\sin 20°}{\cos 10°}
\]

Therefore:
\[
\frac{b}{a} = \frac{\cos 10°}{\sin 20°} = \frac{\cos 10°}{2\sin 10° \cos 10°} = \frac{1}{2\sin 10°}
\]

Now compute:
\[
\frac{a^2}{b^2} + \frac{b}{a} = 4\sin^2(10°) + \frac{1}{2\sin 10°}
\]

Let $s = \sin 10°$. Then:
\[
4s^2 + \frac{1}{2s} = \frac{8s^3 + 1}{2s}
\]

Using the identity $8s^3 = 6s - 2\sin 30° = 6s - 1$ (from $\sin 30° = 3\sin 10° - 4\sin^3 10°$):

Actually, $\sin 3\theta = 3\sin\theta - 4\sin^3\theta$. For $\theta = 10°$:
\[
\sin 30° = 3\sin 10° - 4\sin^3 10°
\]
\[
\frac{1}{2} = 3s - 4s^3
\]
\[
4s^3 = 3s - \frac{1}{2}
\]
\[
8s^3 = 6s - 1
\]

So:
\[
8s^3 + 1 = 6s - 1 + 1 = 6s
\]

Therefore:
\[
\frac{8s^3 + 1}{2s} = \frac{6s}{2s} = 3
\]

\subsection*{Answer}
$3$ (option \textbf{C}).

\end{document}
