\documentclass[12pt]{article}
\usepackage[margin=1in]{geometry}
\usepackage{amsmath,amssymb}
\begin{document}

\section*{Problem 13}
Ako je $N$ broj šestocifrenih brojeva koji u svom zapisu sadrže cifru 1 bar na jednom mestu, tada $N$ pripada intervalu:

\subsection*{Solution}
Ukupan broj šestocifrenih brojeva je $9 \cdot 10^5 = 900.000$ (prva cifra ne može biti 0).
Izračunajmo broj onih koji NE sadrže cifru 1.
Cifre na raspolaganju su $\{0, 2, 3, 4, 5, 6, 7, 8, 9\}$.
Prva cifra (ne sme biti 0 i ne 1): 8 mogućnosti ($2 \dots 9$).
Ostale 5 cifre: 9 mogućnosti (sve osim 1).
Broj takvih brojeva je $8 \cdot 9^5$.
$9^5 = 59049$.
$8 \cdot 59049 = 472392$.
Broj brojeva koji sadrže bar jednu jedinicu je $N = 900000 - 472392 = 427608$.
Ovaj broj pripada intervalu $[4 \cdot 10^5, 5 \cdot 10^5]$.

\subsection*{Answer}
$[4 \cdot 10^5, 5 \cdot 10^5]$ (option \textbf{D}).

\end{document}