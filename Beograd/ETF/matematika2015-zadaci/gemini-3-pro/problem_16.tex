\documentclass[12pt]{article}
\usepackage[margin=1in]{geometry}
\usepackage{amsmath,amssymb}
\begin{document}

\section*{Problem 16}
Broj parova $(p, q), p, q \in \mathbf{R}$ takvih da je polinom $x^4 + px^2 + q$ deljiv polinomom $x^2 + px + q$, jednak je:

\subsection*{Solution}
Deljenjem polinoma dobijamo ostatak koji mora biti nula.
$x^4 + px^2 + q = (x^2 + px + q)(x^2 - px + p^2) + R(x)$.
Ostatak je $R(x) = (-p^3 + 2pq)x + (q - p^2q)$. Ovo nije tačno, uradimo deljenje pažljivo.
$x^4 + px^2 + q = x^2(x^2+px+q) - px^3 + qx^2 + \dots$
Pravilno deljenje daje uslove:
1) $p(2q - p^2 - p) = 0$
2) $q(1 + q - p^2 - p) = 0$

Slučaj 1: $p=0$. Iz 2) $q(1+q)=0 \Rightarrow q=0, q=-1$. Parovi: $(0,0), (0,-1)$.
Slučaj 2: $p \ne 0$. Tada $2q = p^2+p$.
Zamenom u 2): $q=0$ ili $1+q-p^2-p=0$.
Ako $q=0$, $p^2+p=0 \Rightarrow p=-1$. Par: $(-1,0)$.
Ako $q \ne 0$, $q = p^2+p-1$.
$2(p^2+p-1) = p^2+p \Rightarrow p^2+p-2=0 \Rightarrow (p+2)(p-1)=0$.
$p=1 \Rightarrow q=1$. Par: $(1,1)$.
$p=-2 \Rightarrow q=1$. Par: $(-2,1)$.

Ukupno 5 parova: $(0,0), (0,-1), (-1,0), (1,1), (-2,1)$.

\subsection*{Answer}
$5$ (option \textbf{E}).

\end{document}