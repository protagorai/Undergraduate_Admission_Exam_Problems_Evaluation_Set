\documentclass[12pt]{article}
\usepackage[margin=1in]{geometry}
\usepackage{amsmath,amssymb}
\begin{document}

\section*{Problem 5}
Ako za dijagonale romba važi jednakost $d_1 = (2-\sqrt{3})d_2$, tada je oštar ugao romba jednak:

\subsection*{Solution}
Neka su $d_1$ i $d_2$ dijagonale romba. One se seku pod pravim uglom i polove uglove romba.
Ako je $\alpha$ oštar ugao romba, tada je $\frac{d_1}{2} / \frac{d_2}{2} = \tan(\frac{\alpha}{2})$.
Dakle, $\tan(\frac{\alpha}{2}) = \frac{d_1}{d_2} = 2-\sqrt{3}$.

Znamo da je $\tan(15^\circ) = \tan(45^\circ - 30^\circ) = \frac{\tan 45^\circ - \tan 30^\circ}{1 + \tan 45^\circ \tan 30^\circ} = \frac{1 - \frac{\sqrt{3}}{3}}{1 + \frac{\sqrt{3}}{3}} = \frac{3-\sqrt{3}}{3+\sqrt{3}}$.
Racionalizacijom:
$\frac{3-\sqrt{3}}{3+\sqrt{3}} \cdot \frac{3-\sqrt{3}}{3-\sqrt{3}} = \frac{9 - 6\sqrt{3} + 3}{9-3} = \frac{12-6\sqrt{3}}{6} = 2-\sqrt{3}$.

Dakle, $\frac{\alpha}{2} = 15^\circ \Rightarrow \alpha = 30^\circ$.

\subsection*{Answer}
$30^\circ$ (option \textbf{B}).

\end{document}