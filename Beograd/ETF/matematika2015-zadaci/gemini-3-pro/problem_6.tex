\documentclass[12pt]{article}
\usepackage[margin=1in]{geometry}
\usepackage{amsmath,amssymb}
\begin{document}

\section*{Problem 6}
Prav valjak i prava kupa imaju zajedničku osnovu. Vrh kupe je centar druge osnove valjka. Ako je odnos visine valjka i izvodnice kupe 4:5, tada je odnos površina valjka i kupe jednak:

\subsection*{Solution}
Neka je $R$ poluprečnik osnove, $H$ visina valjka, $h$ visina kupe, i $s$ izvodnica kupe.
Kako je vrh kupe u centru druge osnove valjka, visina kupe je jednaka visini valjka: $h = H$.
Dat je odnos $H : s = 4 : 5$.
Iz Pitagorine teoreme za kupu: $R^2 + H^2 = s^2$.
Neka je $H = 4k$, $s = 5k$. Tada je $R^2 = (5k)^2 - (4k)^2 = 9k^2 \Rightarrow R = 3k$.

Površina valjka $P_V = 2\pi R^2 + 2\pi RH = 2\pi(9k^2) + 2\pi(3k)(4k) = 18\pi k^2 + 24\pi k^2 = 42\pi k^2$.
Površina kupe $P_K = \pi R^2 + \pi Rs = \pi(9k^2) + \pi(3k)(5k) = 9\pi k^2 + 15\pi k^2 = 24\pi k^2$.

Odnos površina:
$P_V : P_K = 42\pi k^2 : 24\pi k^2 = 42 : 24 = 7 : 4$.

\subsection*{Answer}
$7:4$ (option \textbf{E}).

\end{document}