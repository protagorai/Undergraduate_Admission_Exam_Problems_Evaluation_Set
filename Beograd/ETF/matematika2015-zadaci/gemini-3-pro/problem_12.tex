\documentclass[12pt]{article}
\usepackage[margin=1in]{geometry}
\usepackage{amsmath,amssymb}
\begin{document}

\section*{Problem 12}
Ako se zna da $\frac{14}{9}$ binomnog koeficijenta trećeg člana, binomni koeficijent četvrtog člana i binomni koeficijent petog člana u razvoju binoma $\left(\sqrt[3]{x} + \frac{1}{\sqrt{x}}\right)^n$ ($n \in \mathbf{N}, x > 0$), čine geometrijsku progresiju, tada je binomni koeficijent uz $\sqrt{x}$ jednak:

\subsection*{Solution}
Binomni koeficijenti su $\binom{n}{k}$.
Treći član: $\binom{n}{2}$. Četvrti: $\binom{n}{3}$. Peti: $\binom{n}{4}$.
Uslov geometrijske progresije: $\left(\binom{n}{3}\right)^2 = \frac{14}{9} \binom{n}{2} \binom{n}{4}$.
\[
\left(\frac{n(n-1)(n-2)}{6}\right)^2 = \frac{14}{9} \cdot \frac{n(n-1)}{2} \cdot \frac{n(n-1)(n-2)(n-3)}{24}
\]
Skratimo $n^2(n-1)^2$ i $(n-2)$:
\[
\frac{(n-2)}{36} = \frac{14}{9 \cdot 2 \cdot 24} (n-3) = \frac{7}{216} (n-3)
\]
Množenjem sa 216:
$6(n-2) = 7(n-3)$
$6n - 12 = 7n - 21 \Rightarrow n = 9$.

Tražimo koeficijent uz $\sqrt{x} = x^{1/2}$.
Opšti član razvoja: $T_{k+1} = \binom{n}{k} (\sqrt[3]{x})^{n-k} (x^{-1/2})^k = \binom{9}{k} x^{\frac{9-k}{3} - \frac{k}{2}}$.
Izjednačimo eksponent sa $1/2$:
$\frac{9-k}{3} - \frac{k}{2} = \frac{1}{2} \Rightarrow \frac{18-2k-3k}{6} = \frac{3}{6} \Rightarrow 18-5k=3 \Rightarrow 5k=15 \Rightarrow k=3$.
Koeficijent je $\binom{9}{3} = \frac{9 \cdot 8 \cdot 7}{3 \cdot 2 \cdot 1} = 3 \cdot 4 \cdot 7 = 84$.

\subsection*{Answer}
$84$ (option \textbf{C}).

\end{document}