\documentclass[12pt]{article}
\usepackage[margin=1in]{geometry}
\usepackage{amsmath,amssymb}
\begin{document}

\section*{Problem 14}
Data je aritmeticka progresija $a_1,a_2,\dots$ cija je razlika $d=1$, a zbir prvih $98$ clanova je
\[
a_1+a_2+\cdots +a_{98}=137.
\]
Odrediti zbir $a_2+a_4+a_6+\cdots+a_{98}$.

\subsection*{Solution}
Posto je $d=1$, vazi
\[
a_n=a_1+(n-1).
\]
Zbir prvih $98$ clanova je
\[
S_{98}=\frac{98}{2}\bigl(2a_1+97\bigr)=49(2a_1+97)=137,
\]
pa je
\[
2a_1+97=\frac{137}{49}
\quad\Longrightarrow\quad
a_1=\frac{1}{2}\left(\frac{137}{49}-97\right)= -\frac{2308}{49}.
\]
Trazimo
\[
S_{\text{par}}=\sum_{k=1}^{49} a_{2k}
 =\sum_{k=1}^{49}\bigl(a_1+(2k-1)\bigr)
 =49a_1+\sum_{k=1}^{49}(2k-1).
\]
Zbir prvih $49$ neparnih brojeva je $49^2=2401$, pa
\[
S_{\text{par}}=49a_1+2401=49\cdot\left(-\frac{2308}{49}\right)+2401=-2308+2401=93.
\]

\subsection*{Answer}
$93$ (opcija \textbf{B}).

\end{document}

