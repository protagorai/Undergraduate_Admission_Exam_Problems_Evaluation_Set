\documentclass[12pt]{article}
\usepackage[margin=1in]{geometry}
\usepackage{amsmath,amssymb}
\begin{document}

\section*{Problem 18}
Tangenta krive $y=e^{-x}$ (za $x>-1$) sece koordinatne ose u tackama $A$ i $B$.
Ako je $O$ koordinatni pocetak, maksimalna povrsina trougla $OAB$ iznosi:

\subsection*{Solution}
Neka je dodirna tacka $T(t,e^{-t})$, gde je $t>-1$.
Izvod je $y'=-e^{-x}$, pa je nagib tangente u $t$ jednak $-e^{-t}$.
Jednacina tangente:
\[
y-e^{-t}=-e^{-t}(x-t)
\quad\Longrightarrow\quad
y=-e^{-t}x+e^{-t}(t+1).
\]
Preseci sa osama:
\begin{itemize}
\item sa $x$-osom ($y=0$): $x=t+1$;
\item sa $y$-osom ($x=0$): $y=e^{-t}(t+1)$.
\end{itemize}
Povrsina trougla je
\[
S(t)=\frac12\,(t+1)\cdot e^{-t}(t+1)=\frac12 (t+1)^2 e^{-t}.
\]
Neka je $u=t+1>0$. Tada
\[
S(u)=\frac12 u^2 e^{-(u-1)}=\frac{e}{2}\,u^2 e^{-u}.
\]
Maksimizujemo $u^2e^{-u}$:
\[
\frac{d}{du}\left(u^2e^{-u}\right)=e^{-u}(2u-u^2)=e^{-u}u(2-u),
\]
pa maksimum nastaje za $u=2$.
Tada je
\[
S_{\max}=\frac{e}{2}\cdot 4\cdot e^{-2}=\frac{2}{e}.
\]

\subsection*{Answer}
$\dfrac{2}{e}$ (opcija \textbf{B}).

\end{document}

