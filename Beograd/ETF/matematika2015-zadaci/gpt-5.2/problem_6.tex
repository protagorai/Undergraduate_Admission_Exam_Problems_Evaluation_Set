\documentclass[12pt]{article}
\usepackage[margin=1in]{geometry}
\usepackage{amsmath,amssymb}
\begin{document}

\section*{Problem 6}
Prav valjak i prava kupa imaju zajednicku osnovu. Vrh kupe je centar druge osnove valjka.
Ako je odnos visine valjka i izvodnice kupe $4:5$, tada je odnos povrsina valjka i kupe:

\subsection*{Solution}
Neka je visina valjka $h$ i poluprecnik zajednicke osnove $r$.
Kako je vrh kupe u centru druge osnove valjka, visina kupe je takodje $h$.
Dat je odnos $h:l=4:5$, pa je $l=\frac{5}{4}h$.

Iz Pitagorine teoreme za kupu:
\[
l^2=r^2+h^2
\quad\Longrightarrow\quad
r^2=\left(\frac{5}{4}h\right)^2-h^2=\frac{25-16}{16}h^2=\frac{9}{16}h^2,
\]
odakle je $r=\frac{3}{4}h$.

Povrsina valjka (ukupna) je
\[
S_v=2\pi r^2+2\pi rh
=2\pi\cdot\frac{9}{16}h^2+2\pi\cdot\frac{3}{4}h\cdot h
=\left(\frac{9}{8}+\frac{3}{2}\right)\pi h^2=\frac{21}{8}\pi h^2.
\]
Povrsina kupe (ukupna) je
\[
S_k=\pi r^2+\pi rl
=\pi\cdot\frac{9}{16}h^2+\pi\cdot\frac{3}{4}h\cdot\frac{5}{4}h
=\left(\frac{9}{16}+\frac{15}{16}\right)\pi h^2=\frac{3}{2}\pi h^2.
\]
Zato je
\[
\frac{S_v}{S_k}=\frac{21/8}{3/2}=\frac{21}{8}\cdot\frac{2}{3}=\frac{7}{4}.
\]

\subsection*{Answer}
$7:4$ (opcija \textbf{E}).

\end{document}

