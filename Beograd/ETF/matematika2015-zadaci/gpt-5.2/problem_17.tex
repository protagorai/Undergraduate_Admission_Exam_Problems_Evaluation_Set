\documentclass[12pt]{article}
\usepackage[margin=1in]{geometry}
\usepackage{amsmath,amssymb}
\begin{document}

\section*{Problem 17}
U jednakokrakom trouglu $ABC$ je $AB=BC=b$, $AC=a$ i $\angle ABC=20^\circ$.
Odrediti vrednost izraza
\[
\frac{a^2}{b^2}+\frac{b}{a}.
\]

\subsection*{Solution}
Po kosinusnoj teoremi na uglu $B$:
\[
a^2=b^2+b^2-2b^2\cos 20^\circ=2b^2(1-\cos 20^\circ)=4b^2\sin^2 10^\circ.
\]
Dakle,
\[
\frac{a^2}{b^2}=4\sin^2 10^\circ,\qquad \frac{b}{a}=\frac{1}{2\sin 10^\circ}.
\]
Neka je $s=\sin 10^\circ$. Tada trazimo
\[
4s^2+\frac{1}{2s}.
\]
Koristimo identitet za sinus trostrukog ugla:
\[
\sin 30^\circ=\sin(3\cdot 10^\circ)=3s-4s^3.
\]
Posto je $\sin 30^\circ=\frac12$, dobijamo
\[
3s-4s^3=\frac12 \quad\Longrightarrow\quad 8s^3-6s+1=0.
\]
Pomnozimo trazeni izraz sa $2s$:
\[
2s\left(4s^2+\frac{1}{2s}\right)=8s^3+1.
\]
Iz jedna\v cine $8s^3-6s+1=0$ sledi $8s^3+1=6s$, pa je
\[
2s\left(4s^2+\frac{1}{2s}\right)=6s \quad\Longrightarrow\quad 4s^2+\frac{1}{2s}=3.
\]

\subsection*{Answer}
$3$ (opcija \textbf{C}).

\end{document}

