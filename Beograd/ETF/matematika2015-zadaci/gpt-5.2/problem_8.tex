\documentclass[12pt]{article}
\usepackage[margin=1in]{geometry}
\usepackage{amsmath,amssymb}
\begin{document}

\section*{Problem 8}
Znajuci da je
\[
\cos\!\left(x-\frac{3\pi}{2}\right)=-\frac{4}{5},\qquad \frac{\pi}{2}<x<\pi,
\]
odrediti vrednost izraza $\sin\frac{x}{2}\cos\frac{5x}{2}$.

\subsection*{Solution}
Koristimo identitet $\cos(\alpha+\frac{\pi}{2})=-\sin\alpha$ i periodicnost kosinusa:
\[
\cos\!\left(x-\frac{3\pi}{2}\right)=\cos\!\left(x+\frac{\pi}{2}\right)=-\sin x.
\]
Dakle, $-\sin x=-\frac45$, pa je $\sin x=\frac45$.
Kako je $x\in\left(\frac{\pi}{2},\pi\right)$, dobijamo $\cos x=-\frac35$.

Sada:
\[
\sin\frac{x}{2}\cos\frac{5x}{2}
=\frac12\left(\sin\left(3x\right)+\sin\left(-2x\right)\right)
\;=\;\frac12\left(\sin 3x-\sin 2x\right).
\]
Racunamo
\[
\sin 2x=2\sin x\cos x=2\cdot\frac45\cdot\left(-\frac35\right)=-\frac{24}{25},
\]
a
\[
\sin 3x=3\sin x-4\sin^3 x
=3\cdot\frac45-4\cdot\left(\frac45\right)^3
=\frac{12}{5}-\frac{256}{125}
=\frac{44}{125}.
\]
Zato je
\[
\sin\frac{x}{2}\cos\frac{5x}{2}
=\frac12\left(\frac{44}{125}-\left(-\frac{24}{25}\right)\right)
=\frac12\left(\frac{44}{125}+\frac{120}{125}\right)
=\frac{82}{125}.
\]

\subsection*{Answer}
$\dfrac{82}{125}$ (opcija \textbf{B}).

\end{document}

