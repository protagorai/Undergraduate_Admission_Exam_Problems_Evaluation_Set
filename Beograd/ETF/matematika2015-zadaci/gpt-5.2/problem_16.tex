\documentclass[12pt]{article}
\usepackage[margin=1in]{geometry}
\usepackage{amsmath,amssymb}
\begin{document}

\section*{Problem 16}
Broj parova $(p,q)$, $p,q\in\mathbb R$ takvih da je polinom
\[
x^4+px^2+q
\]
deljiv polinomom $x^2+px+q$ jednak je:

\subsection*{Solution}
Ako je deljivo, postoji kvadratni kolicnik $x^2+ax+b$ tako da
\[
(x^2+px+q)(x^2+ax+b)=x^4+px^2+q.
\]
Mnozenjem:
\[
x^4+(a+p)x^3+(b+ap+q)x^2+(bp+aq)x+bq.
\]
Poredjenjem koeficijenata sa $x^4+0\cdot x^3+px^2+0\cdot x+q$ dobijamo sistem:
\[
a+p=0,\qquad b+ap+q=p,\qquad bp+aq=0,\qquad bq=q.
\]
Iz $a+p=0$ je $a=-p$.
Iz $bq=q$ sledi $q(b-1)=0$, tj. $q=0$ ili $b=1$.
Uslov $bp+aq=0$ uz $a=-p$ daje $p(b-q)=0$, pa ili $p=0$ ili $b=q$.

\medskip
\noindent\textbf{1) $p=0$.}
Tada je $a=0$ i iz $b+q=0$ sledi $b=-q$.
Uslov $q(b-1)=0$ daje:
\begin{itemize}
\item $q=0 \Rightarrow b=0$ daje $(p,q)=(0,0)$;
\item $b=1 \Rightarrow -q=1 \Rightarrow q=-1$ daje $(p,q)=(0,-1)$.
\end{itemize}

\medskip
\noindent\textbf{2) $p\ne 0$.}
Tada mora vaziti $b=q$.
Uslov $q(b-1)=0$ postaje $q(q-1)=0$, pa je $q=0$ ili $q=1$.
\begin{itemize}
\item $q=0$: iz $b+ap+q=p$ dobijamo $0+(-p)p+0=p$, tj. $p^2+p=0$, pa (uz $p\ne 0$) $p=-1$.
\item $q=1$: iz $1+(-p)p+1=p$ dobijamo $2-p^2=p$, tj. $p^2+p-2=0$, pa $p=1$ ili $p=-2$.
\end{itemize}

Ukupno dobijamo $5$ parova:
\[
(0,0),\ (0,-1),\ (-1,0),\ (1,1),\ (-2,1).
\]

\subsection*{Answer}
$5$ (opcija \textbf{E}).

\end{document}

