\documentclass[12pt]{article}
\usepackage[margin=1in]{geometry}
\usepackage{amsmath,amssymb}
\begin{document}

\section*{Problem 13}
Ako je $N$ broj sestocifrenih brojeva koji u svom zapisu sadrze cifru $1$ bar na jednom mestu, odrediti intervalu kome pripada $N$.

\subsection*{Solution}
Ukupan broj sestocifrenih brojeva je
\[
900000 \quad (100000\text{ do }999999).
\]
Prebrojimo one koji \emph{ne sadrze} cifru $1$.
Prva cifra moze biti bilo koja od $2,3,\dots,9$ (ne sme biti $0$ niti $1$), dakle $8$ izbora.
Svaka od preostalih 5 cifara moze biti bilo koja osim $1$, dakle $9$ izbora po mestu.
Zato je broj brojeva bez cifre $1$ jednak
\[
8\cdot 9^5=8\cdot 59049=472392.
\]
Otuda
\[
N=900000-472392=427608.
\]
Jasno je $4\cdot 10^5\le 427608<5\cdot 10^5$.

\subsection*{Answer}
$N\in[4\cdot 10^5,\,5\cdot 10^5)$ (opcija \textbf{D}).

\end{document}

