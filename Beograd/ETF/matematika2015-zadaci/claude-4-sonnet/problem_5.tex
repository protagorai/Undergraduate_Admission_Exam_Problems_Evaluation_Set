\documentclass[12pt]{article}
\usepackage[margin=1in]{geometry}
\usepackage{amsmath,amssymb}
\begin{document}

\section*{Problem 5}
If the diagonals of a rhombus satisfy $d_1 = (2-\sqrt{3})d_2$, then the acute angle of the rhombus is:

\subsection*{Solution}
In a rhombus, the diagonals are perpendicular and bisect each other. Let's denote the side length as $a$ and the acute angle as $\alpha$.

If we place the rhombus with center at origin and diagonals along the coordinate axes, the vertices are at:
$(\pm \frac{d_1}{2}, 0)$ and $(0, \pm \frac{d_2}{2})$

The side length can be found using the Pythagorean theorem:
\[a^2 = \left(\frac{d_1}{2}\right)^2 + \left(\frac{d_2}{2}\right)^2 = \frac{d_1^2 + d_2^2}{4}\]

Also, in a rhombus with acute angle $\alpha$, the relationship between diagonals and the angle is:
\[\tan\left(\frac{\alpha}{2}\right) = \frac{d_1}{d_2}\]

Given that $d_1 = (2-\sqrt{3})d_2$, we have:
\[\tan\left(\frac{\alpha}{2}\right) = 2-\sqrt{3}\]

We can recognize that $2-\sqrt{3} = \tan(15°)$ because:
\[\tan(15°) = \tan(45° - 30°) = \frac{\tan(45°) - \tan(30°)}{1 + \tan(45°)\tan(30°)} = \frac{1 - \frac{1}{\sqrt{3}}}{1 + \frac{1}{\sqrt{3}}} = \frac{\sqrt{3} - 1}{\sqrt{3} + 1}\]

Rationalizing:
\[\frac{\sqrt{3} - 1}{\sqrt{3} + 1} \cdot \frac{\sqrt{3} - 1}{\sqrt{3} - 1} = \frac{(\sqrt{3} - 1)^2}{3 - 1} = \frac{3 - 2\sqrt{3} + 1}{2} = \frac{4 - 2\sqrt{3}}{2} = 2 - \sqrt{3}\]

Therefore, $\frac{\alpha}{2} = 15°$, which means $\alpha = 30°$.

\subsection*{Answer}
$30°$ (option \textbf{B}).

\end{document}