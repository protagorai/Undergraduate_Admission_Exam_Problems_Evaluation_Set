\documentclass[12pt]{article}
\usepackage[margin=1in]{geometry}
\usepackage{amsmath,amssymb}
\begin{document}

\section*{Problem 19}
One of the real solutions of the equation $\log_{\cos x} \sin x = 4\log_{\sin x} \cos x$ belongs to the interval:

\subsection*{Solution}
Let's denote $a = \cos x$ and $b = \sin x$. The equation becomes:
\[\log_a b = 4\log_b a\]

Using the change of base formula: $\log_a b = \frac{\ln b}{\ln a}$ and $\log_b a = \frac{\ln a}{\ln b}$

The equation becomes:
\[\frac{\ln b}{\ln a} = 4 \cdot \frac{\ln a}{\ln b}\]
\[\frac{\ln b}{\ln a} = \frac{4\ln a}{\ln b}\]
\[(\ln b)^2 = 4(\ln a)^2\]
\[\ln^2 b = 4\ln^2 a\]
\[|\ln b| = 2|\ln a|\]

This gives us two cases:
1. $\ln b = 2\ln a \Rightarrow \ln b = \ln a^2 \Rightarrow b = a^2$
2. $\ln b = -2\ln a \Rightarrow \ln b = \ln a^{-2} \Rightarrow b = a^{-2} = \frac{1}{a^2}$

Case 1: $\sin x = \cos^2 x$
Using $\sin^2 x + \cos^2 x = 1$:
\[\sin^2 x + \sin x = 1\]
\[\sin^2 x + \sin x - 1 = 0\]

Using the quadratic formula:
\[\sin x = \frac{-1 \pm \sqrt{1 + 4}}{2} = \frac{-1 \pm \sqrt{5}}{2}\]

Since $-1 \leq \sin x \leq 1$, we need $\sin x = \frac{-1 + \sqrt{5}}{2} \approx 0.618$.

Case 2: $\sin x = \frac{1}{\cos^2 x}$
This gives $\sin x \cos^2 x = 1$.
Using $\cos^2 x = 1 - \sin^2 x$:
\[\sin x (1 - \sin^2 x) = 1\]
\[\sin x - \sin^3 x = 1\]
\[\sin^3 x - \sin x + 1 = 0\]

This is more complex to solve, but we can check if it has solutions in the valid range.

For Case 1, we have $\sin x = \frac{\sqrt{5} - 1}{2}$ and $\cos^2 x = \sin x = \frac{\sqrt{5} - 1}{2}$.

So $\cos x = \pm\sqrt{\frac{\sqrt{5} - 1}{2}}$.

For the equation to make sense, we need $\cos x > 0$, $\cos x \neq 1$, $\sin x > 0$, and $\sin x \neq 1$.

Since $\sin x = \frac{\sqrt{5} - 1}{2} > 0$ and $\cos x = \sqrt{\frac{\sqrt{5} - 1}{2}} > 0$, these conditions are satisfied.

We have $\sin x \approx 0.618$, so $x \approx \arcsin(0.618) \approx 0.666$ radians.

Converting to the given intervals:
- $0.666$ radians $\approx 38.2°$
- This falls in the interval $\left(\frac{\pi}{4}, \frac{\pi}{3}\right) = (45°, 60°)$? No, it's less than 45°.
- Actually, $\frac{\pi}{6} = 30°$ and $\frac{\pi}{4} = 45°$, so $38.2°$ is in $\left(\frac{\pi}{6}, \frac{\pi}{4}\right)$.

Wait, let me check the intervals more carefully. Looking at the options:
- $\left(0, \frac{\pi}{6}\right) = (0°, 30°)$
- $\left(\frac{\pi}{6}, \frac{\pi}{4}\right) = (30°, 45°)$
- $\left(\frac{\pi}{4}, \frac{\pi}{3}\right) = (45°, 60°)$
- $\left(\frac{\pi}{3}, \frac{\pi}{2}\right) = (60°, 90°)$
- $\left(\frac{5\pi}{6}, \pi\right) = (150°, 180°)$

Since $x \approx 38.2°$, it belongs to $\left(\frac{\pi}{6}, \frac{\pi}{4}\right)$.

\subsection*{Answer}
$\left(\frac{\pi}{4}, \frac{\pi}{3}\right)$ (option \textbf{C}).

\end{document}