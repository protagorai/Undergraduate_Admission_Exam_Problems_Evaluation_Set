\documentclass[12pt]{article}
\usepackage[margin=1in]{geometry}
\usepackage{amsmath,amssymb}
\begin{document}

\section*{Problem 13}
If $N$ is the number of six-digit numbers that contain exactly one digit 1 in their decimal representation, then $N$ belongs to the interval:

\subsection*{Solution}
We need to count six-digit numbers that contain exactly one digit 1.

A six-digit number has the form $d_1d_2d_3d_4d_5d_6$ where $d_1 \in \{1,2,3,4,5,6,7,8,9\}$ (first digit cannot be 0) and $d_i \in \{0,1,2,3,4,5,6,7,8,9\}$ for $i = 2,3,4,5,6$.

We consider two cases:

\textbf{Case 1:} The digit 1 is in the first position.
Then $d_1 = 1$ and each of $d_2, d_3, d_4, d_5, d_6$ must be chosen from $\{0,2,3,4,5,6,7,8,9\}$ (9 choices each).
Number of such numbers: $1 \times 9^5 = 9^5 = 59049$

\textbf{Case 2:} The digit 1 is not in the first position.
Then $d_1 \in \{2,3,4,5,6,7,8,9\}$ (8 choices), exactly one of $d_2, d_3, d_4, d_5, d_6$ equals 1, and the remaining four digits are chosen from $\{0,2,3,4,5,6,7,8,9\}$ (9 choices each).

We have 5 positions to place the digit 1 (positions 2, 3, 4, 5, or 6).
Number of such numbers: $8 \times 5 \times 9^4 = 8 \times 5 \times 6561 = 40 \times 6561 = 262440$

Total: $N = 59049 + 262440 = 321489$

Let's check which interval this belongs to:
- $[10^5, 2 \times 10^5) = [100000, 200000)$: No
- $[2 \times 10^5, 3 \times 10^5) = [200000, 300000)$: No  
- $[3 \times 10^5, 4 \times 10^5) = [300000, 400000)$: Yes
- $[4 \times 10^5, 5 \times 10^5) = [400000, 500000)$: No
- $[5 \times 10^5, 6 \times 10^5) = [500000, 600000)$: No

Since $321489 \in [300000, 400000) = [3 \times 10^5, 4 \times 10^5)$, the answer corresponds to option (C).

\subsection*{Answer}
$[3 \times 10^5, 4 \times 10^5)$ (option \textbf{C}).

\end{document}