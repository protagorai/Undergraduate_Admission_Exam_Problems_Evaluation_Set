\documentclass[12pt]{article}
\usepackage[margin=1in]{geometry}
\usepackage{amsmath,amssymb}
\begin{document}

\section*{Problem 2}
The shortest distance between the lines $\sqrt{2}x + y = 1$ and $2x + \sqrt{2}y = 3\sqrt{2}$ is:

\subsection*{Solution}
First, let's rewrite both lines in standard form $ax + by + c = 0$:
\begin{align}
L_1: \sqrt{2}x + y - 1 &= 0\\
L_2: 2x + \sqrt{2}y - 3\sqrt{2} &= 0
\end{align}

To find if the lines are parallel, we check if their direction vectors are proportional.
For $L_1$: direction vector is $(-1, \sqrt{2})$
For $L_2$: direction vector is $(-\sqrt{2}, 2)$

Check proportionality: $\frac{-1}{-\sqrt{2}} = \frac{1}{\sqrt{2}}$ and $\frac{\sqrt{2}}{2} = \frac{\sqrt{2}}{2} = \frac{1}{\sqrt{2}}$

Since the ratios are equal, the lines are parallel.

For parallel lines $a_1x + b_1y + c_1 = 0$ and $a_2x + b_2y + c_2 = 0$, the distance is:
\[d = \frac{|c_1 - c_2 \cdot \frac{\sqrt{a_1^2 + b_1^2}}{\sqrt{a_2^2 + b_2^2}}|}{\sqrt{a_1^2 + b_1^2}}\]

Let's normalize the second equation to have the same coefficients ratio:
$L_2: 2x + \sqrt{2}y - 3\sqrt{2} = 0$

Divide by $\sqrt{2}$: $\sqrt{2}x + y - 3 = 0$

Now both lines have the form $\sqrt{2}x + y + c = 0$:
- $L_1$: $c_1 = -1$  
- $L_2$: $c_2 = -3$

The distance between parallel lines is:
\[d = \frac{|c_1 - c_2|}{\sqrt{(\sqrt{2})^2 + 1^2}} = \frac{|-1 - (-3)|}{\sqrt{2 + 1}} = \frac{2}{\sqrt{3}} = \frac{2\sqrt{3}}{3}\]

\subsection*{Answer}
$\frac{2\sqrt{3}}{3}$ (option \textbf{D}).

\end{document}