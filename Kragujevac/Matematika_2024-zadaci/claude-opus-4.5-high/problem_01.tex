\documentclass[12pt]{article}
\usepackage[utf8]{inputenc}
\usepackage[T2A]{fontenc}
\usepackage[serbian]{babel}
\usepackage{amsmath,amssymb}
\usepackage{geometry}
\geometry{a4paper, margin=2.5cm}

\begin{document}

\section*{Problem 1 - Solution}

\textbf{Problem:} For $x = \sqrt{(-2)^2} - \sqrt{2^2} + \sqrt{|-2|}$ and $y = \sqrt[3]{-1} - \sqrt{2}$, find the value of:
\[
\frac{x^3 - y^3}{x^2 - 2xy + y^2} : \frac{x^2 + xy + y^2}{x^2 - y^2}
\]

\textbf{Solution:}

First, let's calculate $x$ and $y$:
\begin{align*}
x &= \sqrt{(-2)^2} - \sqrt{2^2} + \sqrt{|-2|} \\
&= \sqrt{4} - \sqrt{4} + \sqrt{2} \\
&= 2 - 2 + \sqrt{2} = \sqrt{2}
\end{align*}

\begin{align*}
y &= \sqrt[3]{-1} - \sqrt{2} = -1 - \sqrt{2}
\end{align*}

Now let's simplify the expression using algebraic identities:
\begin{itemize}
\item $x^3 - y^3 = (x-y)(x^2 + xy + y^2)$
\item $x^2 - 2xy + y^2 = (x-y)^2$
\item $x^2 - y^2 = (x-y)(x+y)$
\end{itemize}

The expression becomes:
\[
\frac{(x-y)(x^2 + xy + y^2)}{(x-y)^2} \cdot \frac{(x-y)(x+y)}{x^2 + xy + y^2} = \frac{(x-y)(x+y)}{(x-y)} = x + y
\]

Therefore:
\[
x + y = \sqrt{2} + (-1 - \sqrt{2}) = -1
\]

\textbf{Answer: Г) $-1$}

\end{document}

