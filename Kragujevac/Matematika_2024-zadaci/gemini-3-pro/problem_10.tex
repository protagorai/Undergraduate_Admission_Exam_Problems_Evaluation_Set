\documentclass[12pt]{article}
\usepackage[margin=1in]{geometry}
\usepackage{amsmath,amssymb}
\usepackage[utf8]{inputenc}
\begin{document}

\section*{Problem 10}
The base of a quadrilateral pyramid is a rectangle with diagonal length $4\,\text{cm}$ and angle $60^\circ$ between diagonals. Each lateral edge makes an angle of $30^\circ$ with the base. Find the volume of the pyramid.

\subsection*{Solution}
Let the base be a rectangle with diagonals $d=4$. Since the diagonals bisect each other, the segments from the center to the vertices have length $R=d/2=2$.
The angle between diagonals is $60^\circ$.
The area of a quadrilateral with diagonals $d_1, d_2$ and angle $\phi$ between them is $S = \frac{1}{2}d_1 d_2 \sin \phi$.
Here $d_1=d_2=4$ and $\phi=60^\circ$, so the area of the base is
\[
B = \frac{1}{2} \cdot 4 \cdot 4 \cdot \sin 60^\circ = 8 \cdot \frac{\sqrt{3}}{2} = 4\sqrt{3}.
\]
Let $h$ be the height of the pyramid. Since all lateral edges make the same angle $\alpha=30^\circ$ with the base, the apex projects to the center of the circumcircle of the base (which is the intersection of diagonals for a rectangle).
From the right triangle formed by the height, a lateral edge, and the semi-diagonal on the base:
\[
\tan \alpha = \frac{h}{R} \implies h = R \tan 30^\circ = 2 \cdot \frac{\sqrt{3}}{3} = \frac{2\sqrt{3}}{3}.
\]
The volume of the pyramid is
\[
V = \frac{1}{3} B h = \frac{1}{3} \cdot (4\sqrt{3}) \cdot \frac{2\sqrt{3}}{3} = \frac{8 \cdot 3}{9} = \frac{8}{3}.
\]

\subsection*{Answer}
$\frac{8}{3}$ (option \textbf{A}).

\end{document}






