\documentclass[12pt]{article}
\usepackage[margin=1in]{geometry}
\usepackage{amsmath,amssymb}
\begin{document}

\section*{Problem 1}
Given
\[
x=\sqrt{(-2)^2}-\sqrt{2^2}+\sqrt{|{-2}|},\qquad
y=\sqrt[3]{-1}-\sqrt2,
\]
compute the value of
\[
\frac{x^3-y^3}{x^2-2xy+y^2}:\frac{x^2+xy+y^2}{x^2-y^2}.
\]

\subsection*{Solution}
First,
\[
\sqrt{(-2)^2}=2,\quad \sqrt{2^2}=2,\quad \sqrt{|{-2}|}=\sqrt2
\;\Rightarrow\; x=\sqrt2.
\]
Also, $\sqrt[3]{-1}=-1$, hence $y=-1-\sqrt2$.

Use identities:
\[
x^3-y^3=(x-y)(x^2+xy+y^2),\qquad x^2-2xy+y^2=(x-y)^2,\qquad x^2-y^2=(x-y)(x+y).
\]
Then
\[
\frac{x^3-y^3}{x^2-2xy+y^2}
=\frac{(x-y)(x^2+xy+y^2)}{(x-y)^2}
=\frac{x^2+xy+y^2}{x-y},
\]
so the whole expression equals
\[
\frac{x^2+xy+y^2}{x-y}\cdot \frac{x^2-y^2}{x^2+xy+y^2}
=\frac{(x-y)(x+y)}{x-y}=x+y.
\]
Therefore
\[
x+y=\sqrt2+(-1-\sqrt2)=-1.
\]

\subsection*{Answer}
$-1$ (option \textbf{Г}).

\end{document}










